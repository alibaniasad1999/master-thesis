\documentclass[a4paper]{article}
%To  colorize table
\usepackage[table]{xcolor}
\usepackage[a4paper, total={7in, 11.0in},includefoot, includehead, headsep=24pt, headheight=4cm]{geometry}
\usepackage{layout}%No need for this package at production.
\usepackage{setspace}%For switch between line spacing.
\usepackage{graphicx}
\usepackage{multirow}
%For drawing lines and arbitrary shapes:
\usepackage{tikz}
%To set fixed length for table: 
\usepackage{array}
\usepackage{ifthen}
%To draw squares:
\usepackage{amssymb}
%To trim strings:
\usepackage{trimspaces}
\usepackage{ifthen}
%To draw dynamic multi-column table:
\usepackage{etoolbox}
\usepackage{float}

%\usepackage{biblatex}
%\addbibresource{refs.bib} % with extension
%\setcitestyle{authoryear,open={(},close={)}}
%\bibliographystyle{ieeetr-fa.bst}
%For references:
% \usepackage[style=plain]{biblatex}
%\bibliographystyle{ieeetr}



% \usepackage{biblatex}
% \addbibresource{references.bib}
%\usepackage[style=numeric]{biblatex}




%\renewcommand{\bibsection}{}{\rl{{مراجع}\hfill}} 


%\addbibresource{references.bib}


%List of needed counters:
\newcounter{length}
\newcounter{itemCount}


\title{Proposal}
\author{Name}
\date{\today}

\newcommand{\blankField}{ $\ldots\ldots\ldots\ldots\ldots$}
\makeatletter
\newcommand{\trim}[1]{\trim@spaces@noexp{#1}}
\makeatother
\newcolumntype{P}[1]{>{\centering\arraybackslash}p{#1}}


\usepackage{fancyhdr}
\pagestyle{fancy}
\fancyhf{}
\usepackage{xepersian}
\settextfont[Scale=1.2]{Yas}
\setlatintextfont[Scale=1]{Times New Roman}
%\bibliographystyle{ieeetr-fa.bst}
%\bibliographystyle{your-bst-file} % Replace 'your-bst-file' with the actual BST file name (without the .bst extension)

\bibliography{references} 
\rhead{
    \begin{tabular}{c}
         \includegraphics[width=3cm, height=3cm]{./logo-fa-IR.png}\\
    \end{tabular}
}
\chead{
    \setstretch{2}
    بسمه تعالی \\
    دانشکده مهندسی هوافضا \\
    \textbf{\large فرم تعریف پروژه کارشناسی ارشد}
}

\lhead{
    \ifthenelse{\isodd{\value{page}}}
        {
            \setstretch{1.5}
            \begin{tabular}{c c}
                تاریخ: & \blankField \\
                شماره: & \blankField\\
                پیوست: & \blankField                
                % \blankField & تاریخ: \\
                % \blankField & شماره:\\
                % \blankField & پیوست:
            \end{tabular}
        }
        {
            \setstretch{1.5}
            \begin{tabular}{c c}
                تاریخ: & \blankField \\
                شماره: & \blankField\\
                پیوست: & \blankField
                
                % \blankField & تاریخ: \\
                % \blankField & شماره:\\
                % \blankField & پیوست:
            \end{tabular}
        }
}
% cite

%Improvement: take advantages of the optional parameters! Currently I don't have any time to learn and implement it.
\newcommand{\Information}[9]{
{
    \setstretch{1.5}
    \small
    \bfseries
    \begin{tabular}{r r r}
         نام و نام خانوادگی : #1
         &
         شماره دانشجویی: #2
         &
         معدل: #3
         \\
         گرایش: #4
         &
         تعداد واحدهای گذرانده: #5
         &
         استاد راهنما: #6
         \\
         استاد راهنمای همکار: #7
         &
         تعداد واحد پروژه: #8
         &
         استاد ممتحن: #9
    \end{tabular}
}
}
\usepackage{array} % Add this package to use 'm' column specifier

\newcommand{\Title}[3]{
	{
		\setstretch{1.5}
		\bfseries
		\noindent
		عنوان کامل پروژه:
		
		فارسی:
        
        #1
		
		انگلیسی:


		\lr{#2}\\
		
		\noindent
		نوع پروژه:\;
		\begin{tabular}{m{10em} m{10em} m{10em}}
			کاربردی:
			\ifthenelse{#3 = 0 \or #3 = 3 \or #3 = 4}{
				$\blacksquare$
			}{
				$\square$
			}
			&
			بنیادی:
			\ifthenelse{#3 = 1 \or #3 = 3}{
				$\blacksquare$
			}{
				$\square$
			}
			&
			توسعه‌ای:
			\ifthenelse{#3 = 2 \or #3 = 3 \or #3 = 4}{
				$\blacksquare$
			}{
				$\square$
			} \\
		\end{tabular}
	}
}

% \newcounter{length}
% \newcounter{itemCount}

\newcommand{\Description}[4]{
    \setstretch{1.5}
    \noindent
    \textbf{#1:}
    \noindent
    
    #2

    \vspace{20pt}
    
    \setcounter{length}{0}
    \setcounter{itemCount}{0}
    %First obtain the length of field
    \foreach\x in#4{%
        \addtocounter{length}{1}
    }
    
    \ifthenelse{#3 = 1}{
    \textbf{کلمات کلیدی:}\;
    \foreach\x in#4{%
        \addtocounter{itemCount}{1}
        \trim{\x}
        \ifthenelse{\thelength = \theitemCount}{}{-}
    }
    \vspace{1em}
    }
}


\makeatletter
\newcommand{\ProgressTable}[1]{
{
    \noindent
    \textbf{مراحل انجام پروژه و زمان‌بندی آن:}
    \\
    
    \def\tableData{}
    
    \foreach \x\y\z in #1{
        \protected@xappto\tabledata{ \x & \y & \z\\ \noexpand\hline}
    }
    
    %Add space to each row in table:
    \renewcommand{\arraystretch}{1.5}
    \begin{center}
    \begin{tabular}{| c | m{0.8\textwidth} | c |}
        \hline
        \tabledata
    \end{tabular}
    %Add space to each row in table:
        
    \end{center}
    \renewcommand{\arraystretch}{1}
}
    
}
\makeatother

%\newcommand{\References}{
%{
%    \noindent
%    \nocitep*}
%    \textbf{مراجع:}
%    \begin{latin}
%        \printbibliography[heading=none]
%    \end{latin}
%}
%}
\newcommand{\Courses}[1]{
{
    \noindent
    \textbf{دروس مورد نیاز:}\\
    
    %Not completed yet.
    \centering
    %Add space to each row in table:
    \renewcommand{\arraystretch}{1.5}
    \begin{center}
    \begin{tabular}{| >{\centering}m{6em} | >{\centering}m{3em} | >{\centering}m{6em} | >{\centering}m{6em} | >{\centering}m{3em} | >{\centering\arraybackslash}m{6em} |}
         \hline
         \rowcolor{lightgray}
         \multicolumn{3}{|c|}{جبرانی}
         &
         \multicolumn{3}{|c|}{تخصصی
            \tiny
            (ارتباط موضوع پروژه با دروسی که دانشجو گذرانده یا باید بگذراند)
         }
         \\
         \hline
         \rowcolor{lightgray}
         گذرانده
         &
         نمره
         &
         باید بگذراند
         &
         گذرانده
         &
         نمره
         &
         باید بگذراند
         \\
         \hline
         &
         &
         &
         &
         &
         \\
         \hline
         &
         &
         &
         &
         &
         \\
         \hline
         &
         &
         &
         &
         &
         \\
         \hline
    \end{tabular}
    \end{center}
    %Add space to each row in table:
    \renewcommand{\arraystretch}{1}
}
}
\newcommand{\Signatures}{
{
    %Add space to each row in table:
    \renewcommand{\arraystretch}{1.5}
    
    \begin{center}
        \scriptsize
        \begin{tabular}{|m{15em}|m{15em}|m{15em}|}
            \hline
            استاد راهنما:
            &
            نظر گروه:
            &
            نظر کمیته تحصیلات تکمیلی دانشکده:
            \\
            تاریخ تحویل فرم به مدیر گروه:
            &
            &
            \\
            امضای استاد راهنما:
            &
            تاریخ جلسه گروه:
            &
            تاریخ جلسه کمیته:
            \\ 
            &
            امضای مدیر گروه:
            &
            امضای معاون تحصیلات تکمیلی:
            \\
            &
            &
            \\
            &
            &
            \\
            & &\\
            & &\\
            \hline
        \end{tabular}
    \end{center}
    \footnotesize
    توجه: فرم تعریف پروژه بایستی یک روز قبل از جلسه گروه توسط استاد راهنما تحویل مدیر گروه شود.
    %Add space to each row in table:
    \renewcommand{\arraystretch}{1}
}
}

%\renewcommand{\headrulewidth}{0pt}
%\renewcommand{\footrulewidth}{0pt}


\begin{document}

%First parameter: Name of student(Your name)
%Second parameter: Student Number
%Third parameter: Student's GPA
%Fourth parameter: Student's Major
%Fifth parameter: Student's passed courses credits.
%Sixth parameter: Project Supervisor.
%Seventh parameter: Project Co-Supervisor
%Eighth parameter: Project Credits.
%Ninth parameter: Project Examiner
\Information{
    علی بنی‌اسد
}{
    401209244
}{
    16/34
}{
    فضا
}{
    15
}{
    دکتر نوبهاری
}{}{
    6
}{
%    نام استاد ممتحن
}

\noindent

%First parameter: Title of your project in Persian.
%Second parameter: Title of your project in English.
%Third parameter: Type of your project =>
%   0: Theoretical.
%   1: Practical.
%   2: Both practical & theoretical
\Title{
    هدایت بازی دیفرانسیلی با استفاده از یادگیری تقویتی مقاوم در محیط‌های پویا چندجسمی با پیشران کم}{
    Robust Reinforcement Learning Differential Game Guidance in Low-Thrust, Multi-Body Dynamical Environments
}{
    4
}

\noindent

\Description{معرفی موضوع}{
این پژوهش یک منطق هدایت مقاوم با استفاده از یادگیری تقویتی بازی‌های دیفرانسیلی 
\LTRfootnote{Differential Game Reinforcement Learning (DGRL)}
 را ارائه می‌دهد. این منطق هدایت در بستر شبکه عصبی
 \LTRfootnote{Neural Network (NN)}
  برای راهنمایی خودکار و حلقه بسته فضاپیماهایی با پیشران کم در محیط‌های چند جسمی متغیر مانند منظومه زمین-ماه، ارائه می‌شود.
}{0}{
    {
        کلمه کلیدی نمونه
        ,
        کلمه کلیدی نمونه
        ,
        کلمه کلیدی نمونه
    }
}

\Description{اهمیت موضوع}{
مطالعه و توسعه این الگوریتم هدایت جدید با استفاده از الگوریتم‌های یادگیری تقویتی و یادگیری بازی‌های دیفرانسیلی دارای اهمیت ویژه‌ای است. این الگوریتم هدایت، به عنوان یک مدل مقاوم، قابلیت مقابله با نوسانات و متغیرهای محیطی را داراست. علاوه بر این، ساختار جدید پردازنده‌های کامپیوتر پرواز، با امکانات به‌روز و محاسبات ماتریسی بهبود یافته، اجرای بهینه‌تر الگوریتم‌های یادگیری ماتریسی را آسان کرده و امکان اجرای موثر آنها را فراهم کرده است.

با توجه به اینکه ایستگاه زمینی در بیشتر زمان ماموریت فاصله زمانی قابل توجهی از محل اجرای دستورات و مکان اعمال دستورات دارد، استفاده از این الگوریتم هدایت می‌تواند به عنوان یک تکنولوژی بسیار مهم در کاهش تاخیرهای ناشی از این فواصل زمانی و افزایش پایداری و دقت سیستم‌های کامپیوتر پرواز تلقی شود. توانایی این الگوریتم هدایت در انجام محاسبات پیچیده و تصمیم‌گیری‌های برنامه‌ریزی در زمان واقعی، اهمیت ویژه‌ای برای امنیت و پایداری پروازهای فضایی دارد و می‌تواند تاثیر مثبتی در اجرای موثر ماموریت‌های فضایی آینده داشته باشد.
}{0}{{}}

\Description{کاربردها}{
 کاربردهای الگوریتم هدایت در  ماموریت‌های مختلف فضایی در ادامه آورده شده است. این الگوریتم هدایت می‌تواند در ماموریت‌های فضایی مختلفی مانند ماموریت‌های ماهواره‌ای، ماموریت‌های سفر به ماه و ماموریت‌های نزدیک به زمین مورد استفاده قرار گیرد.

\begin{enumerate}
    \item \textbf{هدایت خودکار فضاپیماها:} یکی از کاربردهای اصلی این پژوهش، هدایت خودکار فضاپیماها در محیط‌های پویا و پیچیده فضایی است. این پژوهش به ایجاد یک سیستم هدایت مبتنی بر یادگیری تقویتی که توانایی انجام مانورها، تعقیب مسیر‌های مرجع بین مدارهای مختلف و بهبود اشتباهات را دارد، می‌پردازد.

    \item \textbf{حل مسائل پیچیده هدایت:} این پژوهش برای حل مسائل پیچیده هدایت فضایی که نیازمند تصمیم‌گیری در زمان واقعی و مطابق با متغیرهای محیطی می‌باشند، استفاده می‌شود. از جمله این مسائل می‌توان به تغییر مسیر در محیط‌های پیچیده، تنظیمات نیرویه پیشرانه، و اصلاح وضعیت نسبت به مدارهای مرجع اشاره کرد.

    \item \textbf{هدایت در محیط‌های نامطمئن:} از طریق استفاده از الگوریتم‌های یادگیری تقویتی، این پژوهش به هدایت در محیط‌های نامطمئن و بدون نیاز به مدل دقیق محیط فضایی امکان می‌دهد. این موضوع در مواقعی که مدل‌سازی دقیق مدل‌های دینامیکی مشکل باشد، قابل اجرا است.

    \item \textbf{استفاده در ماموریت‌های متنوع:} این پژوهش امکان استفاده در ماموریت‌های مختلف فضایی را فراهم می‌کند. از جمله ماموریت‌های مختلف می‌توان به ماموریت‌های ماهواره‌ای، تعمیر و نگهداری ماموریت فضایی، تعقیب اهداف در مدارهای مختلف، و انجام مانور‌های پیچیده اشاره کرد.
\end{enumerate}
}{0}{{}}






\Description{تعریف دقیق مسئله}{
	در پژوهش حاضر، مسأله هدایت فضاپیما برای سیستم زمین-ماه با استفاده از مدل دینامیکی سه‌جسم محدود دایره‌ای\LTRfootnote{Circular Restricted Three-Body Problem (CR3BP)} مطرح می‌شود. در این مسأله، یک فضاپیما با سیستم پیشران کم به یک مسیر مرجع در سیستم \lr{CR3BP} منتقل می‌شود. شرایط اولیه فضاپیما از مسیر مرجع انحراف‌های تصادفی دارند. هدف از این پژوهش، توسعه یک الگوریتم هدایت حلقه بسته با استفاده از اصول یادگیری تقویتی بر مبنای بازی دیفرانسیلی است که به فضاپیما اجازه می‌دهد با فرض بدترین اغتشاش به محیط مرجع بازگردد، مسیر مرجع را دنبال کرده و به مدار مقصد برسد. بازی دیفرانسیلی موجب می‌شود الگوریتم هدایت نه تنها بهینه باشد بلکه مقاوم نیز باشد.
}{0}{{}}


\Description{فرضیات مسئله}{
\begin{itemize}
	\item مدل دینامیکی مسأله، مسأله سه‌جسم محدود دایره‌ای
	سیستم زمین-ماه است. این فرضیه یک سیستم دوجسمی ساده با فضاپیما به عنوان جسم سوم بسیار کوچک را فرض می‌کند.
	\item حرکت فضاپیما تنها در صفحه اتفاق می‌افتد و تنها نیروهای گرانشی در نظر گرفته می‌شوند. سایر انحراف‌های مانند فشار تابشی خورشید نادیده گرفته می‌شوند.
	\item فضاپیما به یک سیستم پیشران کم مجهز است که شعاع تاثیر مشخصی دارد. میزان و جهت پیشران به صورت مستمر تغییر کرده و قابل تنظیم است.
	\item مسأله هدایت شامل انتقال بین مدارهای دوره‌ای (مدارهای لیاپانوف) در مدل \lr{CR3BP} زمین-ماه است.
	\item  توانایی موتور دارای محدودیت است. موتور قادر به تولید پیشران در هر جهت و مقدار نیست.
	\item وضعیت فضاپیما به طور کامل قابل مشاهده و در دسترس الگوریتم هدایت در هر مرحله زمانی است.
	\item محیط یادگیری تقویتی ویژگی‌های مارکوف را دارد، به این معنا که وضعیت فعلی تمام اطلاعات لازم برای پیش‌بینی وضعیت‌های آینده را فراهم می‌کند.
	\item آموزش در یک محیط چرخه‌ای \LTRfootnote{episodic}
	 انجام می‌شود.
	\item تابع پاداش به صورت بازی دیفرانسیلی طراحی شده  است که فضاپیما در مسیر نزدیک به مرجع با فرض بدترین اغتشاش  بماند و به مدار مقصد برسد.
	\item هدف اصلی تنها  اجرا پذیری مسأله   نیست و دیگر اهداف مانند مصرف کمتر سوخت یا زمان انتقال با استفاده از نظریه بازی در نظر گرفته می‌شوند.
	\item این رویکرد نیاز به دسترسی به منابع محاسباتی با عملکرد بالا برای آموزش یادگیری تقویتی قبل از اجرای آن در رایانه پرواز فضاپیما را دارد.
\end{itemize}

	
}{0}{{}}

\Description{روش انجام کار}{
	% مرحله 1: ایجاد محیط دینامیکی
% مراحل کلی
% مراحل کلی
\begin{itemize}
	\item \textbf{مرحله 1: ایجاد محیط دینامیکی}
	\begin{itemize}
		\item پیاده‌سازی مدل دینامیکی محیط مورد نظر انجام می‌شود. این مدل باید شامل تمام جوانب مهم محیط از جمله قوانین حرکت، ابعاد فضایی، و وضعیت اولیه باشد.
		\item در صورت پیچیدگی مدل دینامیکی محیط، از کتابخانه‌ها و ابزارهای مهندسی نرم‌افزاری مانند NumPy و SciPy در پایتون برای پیاده‌سازی آن استفاده می‌شود.
		\item مدل دینامیکی محیط  با ویژگی اینکه به درستی کار کند و اطلاعات مورد نیاز برای یادگیری تقویتی (مانند وضعیت فعلی) در دسترس باشند، ایجاد می‌شود.
	\end{itemize}
	
	\item \textbf{مرحله 2: ایجاد بازیگر هدایت}
	\begin{itemize}
		\item در این مرحله، بازیگری برای انجام هدایت در محیط خود ایجاد می‌شود. این بازیگر مسئول انتخاب اعمال (اعمال کم پیشران) بر اساس وضعیت فعلی محیط است.
		\item از الگوریتم‌های یادگیری تقویتی مختلف مانند \lr{Q-Learning}، \lr{Deep Q-Networks (DQN)}، یا \lr{Proximal Policy Optimization (PPO)} برای ایجاد بازیگر هدایت استفاده می‌شود. انتخاب الگوریتم بستگی به  محیط و اهداف آموزش دارد.
	\end{itemize}
	
	\item \textbf{مرحله 3: استفاده از منابع \lr{GPU} در \lr{Google Colab}}
	\begin{itemize}
		\item \lr{Google Colab} به عنوان یک محیط آموزش آنلاین به کار می‌رود که اجازه می‌دهد کدهای پایتون در یک محیط مبتنی بر مرورگر اجرا شوند. از این مزیت این محیط می‌توان به دسترسی به منابع \lr{GPU} برای آموزش مدل‌های عمیق اشاره کرد.
		\item برای استفاده از \lr{GPU} در \lr{Google Colab}، می‌توان از کتابخانه‌های معروفی مانند \lr{TensorFlow} یا \lr{PyTorch} که از \lr{GPU} پشتیبانی می‌کنند، استفاده می‌شود. به این ترتیب، آموزش مدل‌های یادگیری تقویتی بسیار سریع‌تر و رایگان انجام می‌شود.
	\end{itemize}
\end{itemize}

	
}{0}{{}}

\Description{پیشینه‌ی موضوع}{
	رایج‌ترین تکنیک‌های هدایت معمولاً با استفاده از ایستگاه‌های کنترل مستقر در زمین انجام می‌شوند. با این حال، حساسیت به شکست ارتباطات، تاخیرهای زمانی، محدودیت‌های انتقال داده، پیچیدگی سنسورها و هزینه‌های عملیاتی همگی از دلایلی هستند که هدایت، مسیریابی و کنترل\LTRfootnote{Guidance, Navigation and Control (GNC)}
	 را از کامپیوتر ایستگاه زمینی به کامپیوتر پرواز منتقل می‌کنند. 
در حالی که در طراحی مسیر از افزایش قدرت سخت‌افزارهای کامپیوتر بهره‌برداری می‌کنند، تعداد کمی از آن‌ها برای اجرای خودکار درون‌سفینه عملی هستند. این به دلیل محدودیت‌های منابع محاسباتی است که در فضاپیماها وجود دارد.

در فرآیند طراحی مسیر، معمولاً یک مسیر بهینه و تاریخچه فرمان کنترلی طراحی می‌شود که با معیارهای ماموریت، مصرف سوخت و زمان پرواز، مطابقت داشته باشد. این روش قبل از پرواز انجام می‌شود و می‌تواند از استراتژی‌های متعددی برای هدایت بهینه با پیشران کم استفاده کند، از جمله تکنیک‌های بهینه‌سازی جهانی \cite{vavrina2017global} و برنامه‌نویسی غیرخطی \cite{ocampo2004finite}.
با این حال، توانایی سریع محاسبه مجدد مسیر مرجع و تاریخچه فرمان کنترلی در داخل فضاپیما در هنگام پرواز بسیار مهم است. برخی از پژوهش‌ها تلاش می‌کنند که روش‌های بهینه‌سازی را در کامپیوتر سفینه اجرا کنند، برای مثال از دینامیک‌های ساده‌تری که جسم سوم در آن نیست، استفاده می‌شود. با نگاه به هدایت درون‌سفینه‌ای از دیدگاه یادگیری ماشین، یک کنترل‌کننده شبکه عصبی حلقه بسته، امکان محاسبه سریع و خودکار تاریخچه کنترل را برای یک فضاپیما فراهم می‌کند. به علاوه، یادگیری تقویتی\LTRfootnote{Reinforcement Learning (RL)}
 یادگیری پیش از پرواز واقعی است و می‌تواند از سخت‌افزارهای سریع و قدرتمند و ارزان قیمت زمینی جهت یادگیری استفاده کند.
روش‌های هدایت و بهینه‌سازی مسیر فضاپیماها به طور کلی به راه‌حل‌های اولیه مناسب نیاز دارند. در مسائل چندجسمی، طراحان مسیر اغلب حدس‌های اولیه کم‌هزینه‌ای برای انتقال‌ها با استفاده از نظریه سیستم‌های دینامیکی و صفحات نامتعیض
\cite{2013AcAau, Haapala}
 ایجاد می‌کنند. 
روش‌های مبتنی بر سیستم‌های دینامیکی در بسیاری از کاربردهای گذشته مفید بوده‌اند و در ترکیب با اصلاحات دیفرانسیل و یا تکنیک‌های بهینه‌سازی، برای بسیاری از کاربردها راه‌حل‌های بهینه تولید می‌کنند. با این حال، این رویکرد محاسباتی، فشرده است و اغلب به تعاملات انسان در حلقه نیاز دارد. به عنوان یک جایگزین، تکنیک‌های بهینه‌سازی جهانی ابتکاری مانند جستجوی حوضچه و الگوریتم‌های تکاملی، نیاز به راه‌حل‌های راه‌اندازی دقیق را کاهش می‌دهند \cite{vavrina2017global}، اما پیچیدگی محاسباتی مربوطه آنها را برای استفاده درون‌سفینه ناممکن می‌سازد.


شبکه‌های عصبی ویژگی‌های جذابی را برای فعال‌سازی \lr{automatic guidance} در فضاپیما دارند. به عنوان مثال، شبکه‌های عصبی می‌توانند به‌طور مستقیم از تخمین‌های وضعیت به دستورهای پیشران کنترلی که با محدودیت‌های مأموریت سازگار است، برسند. عملکرد هدایت شبکه‌های عصبی در مطالعاتی مانند فرود بر سیارات \cite{gaudet2020six}، عملیات نزدیکی به سیارات \cite{gaudet2020terminal} و کنترل فضاپیما با پیشران ازدست رفته \cite{rubinsztejn2020neural} نشان داده شده است.
تازه‌ترین پیشرفت‌های تکنیک‌های یادگیری ماشینی در مسائل \lr{onboard automation} به‌طور گسترده‌ای مورد مطالعه قرار گرفته‌اند، از پژوهش‌های اولیه تا توانایی‌های پیاده‌سازی.
به‌عنوان مثال، الگوریتم‌های یادگیری ماشینی ابتدایی در فضاپیماهای مریخی برای کمک به شناسایی ویژگی‌های زمین‌شناسی تعبیه شده‌اند. الگوریتم \lr{AEGIS} توانایی انتخاب خودکار هدف \lr{ChemCam} در داخل فضاپیماهای \lr{Spirit}، \lr{Opportunity} و \lr{Curiosity} را فعال می‌کند
\cite{estlin2012aegis}.
  در کامپیوتر پرواز اصلی \lr{Curiosity}، فرآیند بهبود نقطه‌گیری\LTRfootnote{(refinement process) } نیاز به 94 تا 96 ثانیه دارد 
  \cite{francis2017aegis},
   که به طور قابل توجهی کمتر از زمان مورد نیاز برای ارسال تصاویر به زمین و انتظار برای انتخاب دستی توسط دانشمندان است.
   برنامه‌های آینده برای کاربردهای یادگیری ماشینی درون‌سفینه شامل توانایی‌های رباتیکی درون‌سفینه برای فضاپیمای \lr{Perseverance}
\cite{higa2019vision, rothrock2016spoc}
     و شناسایی عیب برای 
   \lr{Europa Clipper}
\cite{wagstaff2019enabling} می‌شوند. الگوریتم‌های یادگیری ماشینی پتانسیلی برای سهم مهمی در مأموریت‌های اتوماسیون آینده دارند.


علاوه بر رباتیک سیاره‌ای، تحقیقات مختلفی به تکنیک‌های مختلف یادگیری ماشینی برای مسائل \lr{astrodynamics} اختصاص داده‌اند. در طراحی مسیر، وظایف رگرسیون معمولاً مؤثرتر هستند. به عنوان مثال، \lr{Dachwald} در سال 2004 از یک شبکه عصبی کم‌عمق (\lr{NN}) در بهینه‌سازی مسیرهای رانشگر کم پیشران استفاده کرد 
\cite{dachwald2004evolutionary}.
 تحقیقات جدید شامل شناسایی انتقال‌های هتروکلینیک \cite{desmet2019identifying}، اصلاح مسیر رانشگر کم پیشران\cite{parrish2018lowthrust} و تجزیه و تحلیل مشکلات ازدست رفتن رانشگر\cite{rubinsztejn2020neural} می‌شوند.
 
تکنیک‌های یادگیری نظارتی می‌توانند نتایج مطلوبی تولید کنند، اما دارای محدودیت‌های قابل توجهی هستند. ابتدا، این رویکردها بر وجود دانش از پیش از فرآیند تصمیم‌گیری متکی هستند. کاربر با انتخاب نتایج مطلوب، فرض می‌کند که این دانش را دارد. این امر مستلزم دقیق بودن داده‌های تولید‌شده توسط کاربر برای نتایج مطلوب و همچنین وجود تکنیک‌های موجود برای حل مشکل کنونی و تولید داده است. در بخش‌هایی که چنین دانشی وجود ندارد، تکنیک‌های یادگیری نظارتی قابل استفاده نیستند.

%در سال‌های اخیر، \lr{RL} به اثبات مفید بودن خود در دستیابی به عملکرد بهترین حالت در دامنه‌هایی با ابهام محیطی قابل توجه رسیده است
%. هدایت فعال‌سازی‌شده توسط \lr{RL} به صورت گسترده‌ای بر اساس فاز پرواز دسته‌بندی می‌شوند. مسائل فرود \cite{furfaro2020adaptive, gaudet2020deep} 
%و عملیات در نزدیکی به اجسام کوچک
%\cite{gaudet2020terminal, gaudet2020six}
%از حوزه‌های مطالعاتی بهره‌برداری‌شده‌ای هستند. تحقیقات دیگر شامل مواجهه ،\cite{broida2019spacecraft} تداخل خارجی‌جوی
%\cite{gaudet2020reinforcement}
%، نگهداری ایستگاهی \cite{guzzetti2019reinforcement} و اجتناب از تشخیص \cite{reiter2020augmenting} هستند. مطالعاتی که فضاپیماهای رانشگر کم پیشرانرا در یک چارچوب دینامیکی چندبدنی با استفاده از \lr{RL} شامل طراحی انتقال با استفاده از
% \lr{Q-Learning}
% \cite{dasstuart2020rapid} و \lr{Proximal Policy Optimization }
%  \cite{miller2019lowthrust}، و همچنین هدایت نزدیکی مدار
%  \cite{sullivan2020using}، شده‌اند.
  
  در سال‌های اخیر، یادگیری تقویتی (\lr{RL}) در دستیابی به عملکرد بهینه در دامنه‌هایی با ابهام محیطی قابل توجه، به اثبات رسیده است  \cite{heess2017emergence, silver2017mastering}.
  هدایت انجام‌شده توسط \lr{RL} را می‌توان به صورت گسترده بر اساس فاز پرواز دسته‌بندی کرد.
  مسائل فرود \cite{furfaro2020adaptive, gaudet2020deep} 
   و عملیات در نزدیکی اجسام کوچک
   \cite{gaudet2020terminal, gaudet2020six}،
    از حوزه‌های پژوهشی هستند که از \lr{RL} استفاده می‌کنند.
  تحقیقات دیگر شامل مواجهه تداخل خارجی جوی \cite{gaudet2020reinforcement}،
   نگهداری ایستگاهی \cite{guzzetti2019reinforcement}  و جلوگیری از شناسایی \cite{reiter2020augmenting} است.
  مطالعاتی که فضاپیماهای رانشگر کم پیشران را در یک چارچوب دینامیکی چندبدنی با استفاده از RL اجام شده است شامل طراحی انتقال با استفاده از \lr{Q-learning} 
   \cite{dasstuart2020rapid}، \lr{Proximal Policy Optimization}
     \cite{miller2019lowthrust}
      و  هدایت نزدیکی مدار  \cite{sullivan2020using} است.
  

}{0}{{}}



\Description{اهداف پژوهش} {
\begin{itemize}
	\item  طراحی یک بازیکن جهت هدایت به کمک یادگیری تقویتی
	\item طراحی یک الگوریتم هدایت مقاوم به کمک بازی دیفرانسیلی
	\item بهبود عملکرد فضاپیما در محیط سه جسم
	\item عدم نیاز به ایستگاه زمینی جهت صدور فرمان هدایت 
	\item پیاده‌سازی الگوریتم درون‌سفینه‌ای در محیط شبیه‌سازی شده

\end{itemize}}{0}{{}}


\renewcommand{\arraystretch}{2.5}
\begin{table}[H]
	\centering
	\caption{جدول برنامه‌ریزی پروژه ناوبری اینرسی مشارکتی مبتنی بر هوش مصنوعی}
	\vspace{.2cm}
	\begin{tabular}{|P{3cm}|P{10cm}|}
		\hline
		\textbf{فصل} & \textbf{فعالیت‌ها} \\
		\hline
		\multirow{4}{*}{\textbf{تابستان سال اول}} & انجام مطالعه و مروری جامع در زمینه بازی دیفرانسیلی و یادگیری تقویتی.\\
		\cline{2-2}
		&
        بررسی و تعریف دقیق مسأله یادگیری تقویتی و الگوریتم هدایت به وسیله یادگیری تقویتی در فصای سه‌جسم.
          \\
		\cline{2-2}
		& طراحی ساختار محیط سه‌جسم و بازیکن یادگیری تقویتی و ارتباط بین محیط و بازیکن. \\
		\hline
		\multirow{3}{*}{\textbf{نیمسال اول سال دوم}} & طراحی مسیر انتقالی بهینه بین دو مدار \\
		\cline{2-2}
		& شبیه‌سازی دقیق محیط مسأله سه‌جسم جهت آموزش بازیکن یادگیری تقویتی \\
		\cline{2-2}
		&  طراحی و توسعه سیستم یادگیری تقویتی جهت محاسبه دستور \\
		\hline
		\multirow{2}{*}{\textbf{نیمسال دوم سال دوم}} & پیاده‌سازی هدایت فضاپیما با استفاده از الگوریتم‌های کلاسیک \\
		\cline{2-2}
		& آموزش الگوریتم‌های یادگیری تقویتی در محیط‌های شبیه‌سازی شده\\
		\hline
		\multirow{2}{*}{\textbf{تابستان سال دوم}}  
		&  اصلاح و بهبود عملکرد یادگیری تقویتی در هدایت فضاپیما \\ 
		\cline{2-2}
		\cline{2-2}
		& نتیجه‌گیری از تحقیقات و ارائه گزارش نهایی شامل معرفی مسئله، روش‌های استفاده‌شده، نتایج حاصل و پیشنهادات برای تحقیقات آتی \\
		\hline
	\end{tabular}
\end{table}





\Description{نتایج مورد انتظار} {
	\begin{itemize}
		\item  هدایت فضاپیما بدون نیاز به ایستگاه زمینی
		\item کاهش هزینه ساخت به دلیل عدم نیاز به سیستم‌های ارتباطی قوی
		\item  افزایش ایمنی ماموریت به دلیل استفاده از بازی دیفرانسیلی
		\item  کاهش مصرف سوخت و هزینه به دلیل بهینه بودن الگوریتم هدایت
		
\end{itemize}}{0}{{}}

\Description{روش صحت‌سنجی نتایج} {
	\begin{itemize}
		\item  مقایسه با سایر روشهای معتبر
		\item مدلسازی و شبیهسازی سیستم محیط و بازیگر
		
\end{itemize}}{0}{{}}

\Description{گلوگاه‌های پیش‌بینی‌‌شده} {
	\begin{itemize}
		\item  آموزش شبکه: به علت پیچیده بودن شبکه و محطی که طراحی شده است، محاسبات سنگین می‌شود.
		\item پایداری و کنترل: به علت غیرخطی بودن دینامیک سیسیتم و ناپایداری ذاتی سیستم، کنترل و پایداری آن مشکل است.
		
\end{itemize}}{0}{{}}


\Description{‌‌نوآوری‌ها} {
	\begin{itemize}
		\item  طراحی الگوریتم هدایت بهینه درون‌سفینه‌ای
		\item  طراحی الگوریتم هدایت مقاوم درون‌سفینه‌ای
		\item طراحی الگوریتم هدایت یادگیری تقویتی بر مبنای بازی دیفرانسیلی
		
\end{itemize}}{0}{{}}



%\bibliography{refs.bib}
%\input{refs.bib}
%%\bibliography{refs.bib}
%\bibliographystyle{unsrt}


%\printbibliography

%\bibliography{refs}


\noindent

% Add our references in references,bib
%\References{}
%
% -------------------------------------------------------
%  Bibliography
% -------------------------------------------------------


\begin{latin}
%\baselineskip=.8\baselineskip


% Uncomment next line to include uncited references
% \nocite{*}

\bibliography{bibs/full,bibs/refs}

\bibliographystyle{./styles/packages/unsrtabbrv}
 
\end{latin}
\newpage


%\begin{thebibliography}{1}
%	
% -------------------------------------------------------
%  Bibliography
% -------------------------------------------------------


\begin{latin}
%\baselineskip=.8\baselineskip


% Uncomment next line to include uncited references
% \nocite{*}

\bibliography{bibs/full,bibs/refs}

\bibliographystyle{./styles/packages/unsrtabbrv}
 
\end{latin}
\newpage

%\end{thebibliography}

\begin{latin}
	\begin{thebibliography}{1}
		
		
		\bibitem{vavrina2017global}
		Vavrina, Matthew A, Jacob A Englander, Sean M Phillips, and Kyle M Hughes.
		"Global, multi-objective trajectory optimization with parametric spreading."
		In AAS AIAA Astrodynamics Specialist Conference 2017, Tech. No. GSFC-E-DAA-TN45282, 2017.	
		
		
		\bibitem{ocampo2004finite}
		Ocampo, Cesar.
		"Finite Burn Maneuver Modeling for a Generalized Spacecraft Trajectory Design and Optimization System."
		Annals of the New York Academy of Sciences 1017 (2004): 210-233.
		doi: 10.1196/annals.1311.013.
		
		
		\bibitem{Haapala}
		Haapala, A. F., \& Howell, K. C. (2016).
		"A framework for constructing transfers linking periodic libration point orbits in the spatial circular restricted three-body problem."
		International Journal of Bifurcation and Chaos, 26(05), 1630013.
		
		
		\bibitem{2013AcAau}
		Marchand, B.~G. and Scarritt, S.~K. and Pavlak, T.~A. and Howell, K.~C.
		``A dynamical approach to precision entry in multi-body regimes: Dispersion manifolds.''
		Acta Astronautica 89 (2013): 107-120.
		doi: 10.1016/j.actaastro.2013.02.015.
		
		\bibitem{estlin2012aegis}
		T.A. Estlin, B.J. Bornstein, D.M. Gaines, R.C. Anderson, D.R. Thompson, M. Burl, R. Castaño, M. Judd.
		"Aegis automated science targeting for the MER Opportunity rover."
		ACM Trans. Intell. Syst. Technol. (TIST), 3, 1-19, 2012.
		
		\bibitem{francis2017aegis}
		R. Francis, T. Estlin, G. Doran, S. Johnstone, D. Gaines, V. Verma, M. Burl, J. Frydenvang, S. Montano, R. Wiens, S. Schaffer, O. Gasnault, L. Deflores, D. Blaney, B. Bornstein.
		"Aegis autonomous targeting for ChemCam on Mars Science Laboratory: Deployment and results of initial science team use."
		Science Robotics, 2, 2017.
		
		\bibitem{wagstaff2019enabling}
		K.L. Wagstaff, G. Doran, A. Davies, S. Anwar, S. Chakraborty, M. Cameron, I. Daubar, C. Phillips.
		"Enabling onboard detection of events of scientific interest for the Europa Clipper spacecraft,"
		in: 25th ACM SIGKDD International Conference on Knowledge Discovery \& Data Mining,
		Association for Computing Machinery, Anchorage, Alaska, 2019, pp. 2191--2201,
		doi: 10.1145/3292500.3330656.
		
		\bibitem{higa2019vision}
		S. Higa, Y. Iwashita, K. Otsu, M. Ono, O. Lamarre, A. Didier, M. Hoffmann.
		"Vision-based estimation of driving energy for planetary rovers using deep learning and terramechanics,"
		IEEE Robot. Autom. Lett. 4 (2019) 3876–3883.
		
		\bibitem{rothrock2016spoc}
		B. Rothrock, J. Papon, R. Kennedy, M. Ono, M. Heverly, C. Cunningham.
		"Spoc: Deep learning-based terrain classification for Mars rover missions,"
		in: AIAA Space and Astronautics Forum and Exposition, SPACE 2016,
		American Institute of Aeronautics and Astronautics Inc, AIAA, 2016, pp. 1–12.
		
		\bibitem{dachwald2004evolutionary}
		B. Dachwald.
		"Evolutionary neurocontrol: A smart method for global optimization of low-thrust trajectories,"
		in: AIAA/AAS Astrodynamics Specialist Conference and Exhibit,
		Providence, Rhode Island, 2004, pp. 1–16.
		
		\bibitem{desmet2019identifying}
		S. De Smet, D.J. Scheeres.
		"Identifying heteroclinic connections using artificial neural networks,"
		Acta Astronaut. 161 (2019) 192–199.
		
		\bibitem{parrish2018lowthrust}
		N.L.O. Parrish.
		"Low Thrust Trajectory Optimization in Cislunar and Translunar Space (Ph.D. thesis),"
		University of Colorado Boulder, 2018.
		
		\bibitem{rubinsztejn2020neural}
		A. Rubinsztejn, R. Sood, F.E. Laipert.
		"Neural network optimal control in astrodynamics: Application to the missed thrust problem,"
		Acta Astronaut. 176 (2020) 192–203.
		
		\bibitem{heess2017emergence}
		N. Heess, D. TB, S. Sriram, J. Lemmon, J. Merel, G. Wayne, Y. Tassa, T. Erez, Z. Wang, S.M.A. Eslami, M.A. Riedmiller, D. Silver.
		"Emergence of locomotion behaviours in rich environments," 2017.
		CoRR abs/1707.02286.
		arXiv:1707.02286.
		
		\bibitem{silver2017mastering}
		D. Silver, J. Schrittwieser, K. Simonyan, I. Antonoglou, A. Huang, A. Guez, T. Hubert, L. Baker, M. Lai, A. Bolton, Y. Chen, T. Lillicrap, F. Hui, L. Sifre, G. van den Driessche, T. Graepel, D. Hassabis.
		"Mastering the game of Go without human knowledge," Nature 550 (2017) Article.
		
		\bibitem{furfaro2020adaptive}
		R. Furfaro, A. Scorsoglio, R. Linares, M. Massari,
		"Adaptive generalized ZEM-ZEV feedback guidance for planetary landing via a deep reinforcement learning approach,"
		Acta Astronaut. 171 (2020) 156–171.
		
		\bibitem{gaudet2020deep}
		B. Gaudet, R. Linares, R. Furfaro,
		"Deep reinforcement learning for six degrees of freedom planetary landing,"
		Adv. Space Res. 65 (2020) 1723–1741.
		
        \bibitem{broida2019spacecraft}
		J. Broida, R. Linares,
		"Spacecraft rendezvous guidance in cluttered environments via reinforcement learning,"
		in: \textit{29th AAS/AIAA Space Flight Mechanics Meeting}, American Astronautical Society, Ka’anapali, Hawaii, 2019, pp. 1–15.
		
		\bibitem{gaudet2020reinforcement}
		B. Gaudet, R. Furfaro, R. Linares,
		"Reinforcement learning for angle-only intercept guidance of maneuvering targets,"
		\textit{Aerosp. Sci. Technol.} 99 (2020).
		
		\bibitem{guzzetti2019reinforcement}
		D. Guzzetti,
		"Reinforcement learning and topology of orbit manifolds for station-keeping of unstable symmetric periodic orbits,"
		in: \textit{AAS/AIAA Astrodynamics Specialist Conference}, American Astronautical Society, Portland, Maine, 2019, pp. 1–20.
		
		\bibitem{reiter2020augmenting}
		J.A. Reiter, D.B. Spencer,
		"Augmenting spacecraft maneuver strategy optimization for detection avoidance with competitive coevolution,"
		in: \textit{20th AIAA Scitech Forum}, AIAA, Orlando, Florida, 2020, pp. 1–11.
		
		\bibitem{dasstuart2020rapid}
		A. Das-Stuart, K.C. Howell, D.C. Folta,
		"Rapid trajectory design in complex environments enabled by reinforcement learning and graph search strategies,"
		\textit{Acta Astronaut.} 171 (2020) 172–195.
		
		\bibitem{miller2019lowthrust}
		D. Miller, R. Linares,
		"Low-thrust optimal control via reinforcement learning,"
		in: \textit{29th AAS/AIAA Space Flight Mechanics Meeting}, American Astronautical Society, Kaanapali, Hawaii, 2019, pp. 1–18.
		
		\bibitem{sullivan2020using}
		C.J. Sullivan, N. Bosanac,
		"Using reinforcement learning to design a low-thrust approach into a periodic orbit in a multi-body system,"
		in: \textit{20th AIAA Scitech Forum}, AIAA, Orlando, Florida, 2020, pp. 1–19.
		
		
		\bibitem{gaudet2020terminal}
		B. Gaudet, R. Linares, R. Furfaro,
		"Terminal adaptive guidance via reinforcement meta-learning: Applications to autonomous asteroid close-proximity operations,"
		\textit{Acta Astronaut.} 171 (2020) 1–13.
		
		
		\bibitem{gaudet2020six}
		B. Gaudet, R. Linares, R. Furfaro,
		"Six degree-of-freedom hovering over an asteroid with unknown environmental dynamics via reinforcement learning,"
		in: \textit{20th AIAA Scitech Forum}, AIAA, Orlando, Florida, 2020, pp. 1–15.
		
	
		
		
		
	\end{thebibliography}
\end{latin}


\noindent

% Change \newcommand{\Courses} in the previous lines
%\Courses{}

%\noindent

%\Signatures{}

\end{document}
