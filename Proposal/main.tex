\documentclass[a4paper]{article}
%To  colorize table
\usepackage[table]{xcolor}
\usepackage[a4paper, total={7in, 11.0in},includefoot, includehead, headsep=24pt, headheight=4cm]{geometry}
\usepackage{layout}%No need for this package at production.
\usepackage{setspace}%For switch between line spacing.
\usepackage{graphicx}
%For drawing lines and arbitrary shapes:
\usepackage{tikz}
%To set fixed length for table: 
\usepackage{array}
\usepackage{ifthen}
%To draw squares:
\usepackage{amssymb}
%To trim strings:
\usepackage{trimspaces}
\usepackage{ifthen}
%To draw dynamic multi-column table:
\usepackage{etoolbox}
%For references:
\usepackage[
    backend=biber,
    sorting=ydnt,
    style=ieee
]{biblatex}


\addbibresource{references.bib}

%List of needed counters:
\newcounter{length}
\newcounter{itemCount}


\title{Proposal}
\author{Name}
\date{\today}

\newcommand{\blankField}{ $\ldots\ldots\ldots\ldots\ldots$}
\makeatletter
\newcommand{\trim}[1]{\trim@spaces@noexp{#1}}
\makeatother

\usepackage{fancyhdr}
\pagestyle{fancy}
\fancyhf{}
\usepackage{xepersian}
\settextfont[Scale=1.2]{Yas}
\setlatintextfont[Scale=1]{Times New Roman}
\rhead{
    \begin{tabular}{c}
         \includegraphics[width=3cm, height=3cm]{./logo-fa-IR.png}\\
    \end{tabular}
}
\chead{
    \setstretch{2}
    بسمه تعالی \\
    دانشکده مهندسی هوافضا \\
    \textbf{\large فرم تعریف پروژه کارشناسی ارشد}
}

\lhead{
    \ifthenelse{\isodd{\value{page}}}
        {
            \setstretch{1.5}
            \begin{tabular}{c c}
                تاریخ: & \blankField \\
                شماره: & \blankField\\
                پیوست: & \blankField
                
                % \blankField & تاریخ: \\
                % \blankField & شماره:\\
                % \blankField & پیوست:
            \end{tabular}
        }
        {
            \setstretch{1.5}
            \begin{tabular}{c c}
                تاریخ: & \blankField \\
                شماره: & \blankField\\
                پیوست: & \blankField
                
                % \blankField & تاریخ: \\
                % \blankField & شماره:\\
                % \blankField & پیوست:
            \end{tabular}
        }
}

%Improvement: take advantages of the optional parameters! Currently I don't have any time to learn and implement it.
\newcommand{\Information}[9]{
{
    \setstretch{1.5}
    \small
    \bfseries
    \begin{tabular}{r r r}
         نام و نام خانوادگی : #1
         &
         شماره دانشجویی: #2
         &
         معدل: #3
         \\
         گرایش: #4
         &
         تعداد واحدهای گذرانده: #5
         &
         استاد راهنما: #6
         \\
         استاد راهنمای همکار: #7
         &
         تعداد واحد پروژه: #8
         &
         استاد ممتحن: #9
    \end{tabular}
}
}
\usepackage{array} % Add this package to use 'm' column specifier

\newcommand{\Title}[3]{
	{
		\setstretch{1.5}
		\bfseries
		\noindent
		عنوان کامل پروژه:
		
		فارسی:
        
        #1
		
		انگلیسی:


		\lr{#2}\\
		
		\noindent
		نوع پروژه:\;
		\begin{tabular}{m{10em} m{10em} m{10em}}
			کاربردی:
			\ifthenelse{#3 = 0 \or #3 = 3 \or #3 = 4}{
				$\blacksquare$
			}{
				$\square$
			}
			&
			بنیادی:
			\ifthenelse{#3 = 1 \or #3 = 3}{
				$\blacksquare$
			}{
				$\square$
			}
			&
			توسعه‌ای:
			\ifthenelse{#3 = 2 \or #3 = 3 \or #3 = 4}{
				$\blacksquare$
			}{
				$\square$
			} \\
		\end{tabular}
	}
}

% \newcounter{length}
% \newcounter{itemCount}

\newcommand{\Description}[4]{
    \setstretch{1.5}
    \noindent
    \textbf{#1:}
    \noindent
    
    #2

    \vspace{20pt}
    
    \setcounter{length}{0}
    \setcounter{itemCount}{0}
    %First obtain the length of field
    \foreach\x in#4{%
        \addtocounter{length}{1}
    }
    
    \ifthenelse{#3 = 1}{
    \textbf{کلمات کلیدی:}\;
    \foreach\x in#4{%
        \addtocounter{itemCount}{1}
        \trim{\x}
        \ifthenelse{\thelength = \theitemCount}{}{-}
    }
    \vspace{1em}
    }
}


\makeatletter
\newcommand{\ProgressTable}[1]{
{
    \noindent
    \textbf{مراحل انجام پروژه و زمان‌بندی آن:}
    \\
    
    \def\tableData{}
    
    \foreach \x\y\z in #1{
        \protected@xappto\tabledata{ \x & \y & \z\\ \noexpand\hline}
    }
    
    %Add space to each row in table:
    \renewcommand{\arraystretch}{1.5}
    \begin{center}
    \begin{tabular}{| c | m{0.8\textwidth} | c |}
        \hline
        \tabledata
    \end{tabular}
    %Add space to each row in table:
        
    \end{center}
    \renewcommand{\arraystretch}{1}
}
    
}
\makeatother

\newcommand{\References}{
{
    \noindent
    \nocite{*}
    \textbf{مراجع:}
    \begin{latin}
        \printbibliography[heading=none]
    \end{latin}
}
}
\newcommand{\Courses}[1]{
{
    \noindent
    \textbf{دروس مورد نیاز:}\\
    
    %Not completed yet.
    \centering
    %Add space to each row in table:
    \renewcommand{\arraystretch}{1.5}
    \begin{center}
    \begin{tabular}{| >{\centering}m{6em} | >{\centering}m{3em} | >{\centering}m{6em} | >{\centering}m{6em} | >{\centering}m{3em} | >{\centering\arraybackslash}m{6em} |}
         \hline
         \rowcolor{lightgray}
         \multicolumn{3}{|c|}{جبرانی}
         &
         \multicolumn{3}{|c|}{تخصصی
            \tiny
            (ارتباط موضوع پروژه با دروسی که دانشجو گذرانده یا باید بگذراند)
         }
         \\
         \hline
         \rowcolor{lightgray}
         گذرانده
         &
         نمره
         &
         باید بگذراند
         &
         گذرانده
         &
         نمره
         &
         باید بگذراند
         \\
         \hline
         &
         &
         &
         &
         &
         \\
         \hline
         &
         &
         &
         &
         &
         \\
         \hline
         &
         &
         &
         &
         &
         \\
         \hline
    \end{tabular}
    \end{center}
    %Add space to each row in table:
    \renewcommand{\arraystretch}{1}
}
}
\newcommand{\Signatures}{
{
    %Add space to each row in table:
    \renewcommand{\arraystretch}{1.5}
    
    \begin{center}
        \scriptsize
        \begin{tabular}{|m{15em}|m{15em}|m{15em}|}
            \hline
            استاد راهنما:
            &
            نظر گروه:
            &
            نظر کمیته تحصیلات تکمیلی دانشکده:
            \\
            تاریخ تحویل فرم به مدیر گروه:
            &
            &
            \\
            امضای استاد راهنما:
            &
            تاریخ جلسه گروه:
            &
            تاریخ جلسه کمیته:
            \\ 
            &
            امضای مدیر گروه:
            &
            امضای معاون تحصیلات تکمیلی:
            \\
            &
            &
            \\
            &
            &
            \\
            & &\\
            & &\\
            \hline
        \end{tabular}
    \end{center}
    \footnotesize
    توجه: فرم تعریف پروژه بایستی یک روز قبل از جلسه گروه توسط استاد راهنما تحویل مدیر گروه شود.
    %Add space to each row in table:
    \renewcommand{\arraystretch}{1}
}
}

%\renewcommand{\headrulewidth}{0pt}
%\renewcommand{\footrulewidth}{0pt}


\begin{document}

%First parameter: Name of student(Your name)
%Second parameter: Student Number
%Third parameter: Student's GPA
%Fourth parameter: Student's Major
%Fifth parameter: Student's passed courses credits.
%Sixth parameter: Project Supervisor.
%Seventh parameter: Project Co-Supervisor
%Eighth parameter: Project Credits.
%Ninth parameter: Project Examiner
\Information{
    علی بنی‌اسد
}{
    401209244
}{
    20
}{
    فضا
}{
    15
}{
    دکتر نوبهاری
}{}{
    6
}{
    نام استاد ممتحن
}

\noindent

%First parameter: Title of your project in Persian.
%Second parameter: Title of your project in English.
%Third parameter: Type of your project =>
%   0: Theoretical.
%   1: Practical.
%   2: Both practical & theoretical
\Title{
    هدایت بازی دیفرانسیلی با استفاده از یادگیری تقویتی مقاوم در محیط‌های دینامیکی چندجسمی با رانشگر کم پیشران}{
    Robust Reinforcement Learning Differential Game Guidance in Low-Thrust, Multi-Body Dynamical Environments
}{
    4
}

\noindent

\Description{معرفی موضوع}{
این پژوهش به منظور توسعه یک منطق هدایت مقاوم با استفاده از یادگیری تقویتی بازی‌های دیفرانسیلی 
\LTRfootnote{Differential Game Reinforcement Learning (DGRL)}
 را ارائه می‌دهد. این منطق هدایت در بستر شبکه عصبی
 \LTRfootnote{Neural Network (NN)}
  برای راهنمایی خودکار و حلقه بسته فضاپیماهایی با رانشگر کم‌پیشران در محیط‌های چند جسمی متغیر چالش‌برانگیز مانند منظومه زمین-ماه، ارائه می‌شود.
}{0}{
    {
        کلمه کلیدی نمونه
        ,
        کلمه کلیدی نمونه
        ,
        کلمه کلیدی نمونه
    }
}

\Description{اهمیت موضوع}{
مطالعه و توسعه این الگوریتم هدایت جدید با استفاده از الگوریتم‌های یادگیری تقویتی و یادگیری بازی‌های دیفرانسیلی دارای اهمیت ویژه‌ای است. این الگوریتم هدایت، به عنوان یک مدل مقاوم، قابلیت مقابله با نوسانات و متغیرهای محیطی را داراست. علاوه بر این، ساختار جدید پردازنده‌های کامپیوتر پرواز، با امکانات به‌روز و محاسبات ماتریسی بهبود یافته، اجرای بهینه‌تر الگوریتم‌های یادگیری ماتریسی را آسان کرده و توانایی اجرای موثر آنها را فراهم کرده است.

با توجه به اینکه ایستگاه زمینی در بیشتر زمان ماموریت فاصله زمانی قابل توجهی از محل اجرای دستورات تا مکان اعمال واقعی دستورات دارد، استفاده از این الگوریتم هدایت می‌تواند به عنوان یک تکنولوژی بسیار مهم در کاهش تاخیرهای ناشی از این فواصل زمانی و افزایش پایداری و دقت سیستم‌های کامپیوتر پرواز تلقی شود. توانایی این الگوریتم هدایت در انجام محاسبات پیچیده و تصمیم‌گیری‌های برنامه‌ریزی در زمان واقعی، اهمیت ویژه‌ای برای امنیت و پایداری پروازهای فضایی دارد و می‌تواند تاثیر مثبتی در اجرای موثر ماموریت‌های فضایی آینده داشته باشد.
}{0}{{}}

\Description{کاربردها}{
 کاربردهای الگوریتم هدایت در  ماموریت‌های مختلف فضایی در ادامه آورده شده است. این الگوریتم هدایت می‌تواند در ماموریت‌های فضایی مختلفی مانند ماموریت‌های ماهواره‌ای، ماموریت‌های سفر به ماه و ماموریت‌های نزدیک به زمین مورد استفاده قرار گیرد.

\begin{enumerate}
    \item \textbf{هدایت خودکار فضاپیماها:} یکی از کاربردهای اصلی این پژوهش، هدایت خودکار فضاپیماها در محیط‌های پویا و پیچیده فضایی است. این پژوهش به ایجاد یک سیستم هدایت مبتنی بر یادگیری تقویتی که توانایی انجام مانورها، تعقیب مسیر‌های مرجع بین مدارهای مختلف و بهبود اشتباهات را دارد، می‌پردازد.

    \item \textbf{حل مسائل پیچیده هدایت:} این پژوهش برای حل مسائل پیچیده هدایت فضایی که نیازمند تصمیم‌گیری در زمان واقعی و مطابق با متغیرهای محیطی می‌باشند، استفاده می‌شود. از جمله این مسائل می‌توان به تغییر مسیر در محیط‌های پیچیده، تنظیمات نیرویه پیشرانه، و اصلاح وضعیت نسبت به مدارهای مرجع اشاره کرد.

    \item \textbf{هدایت در محیط‌های نامطمئن:} از طریق استفاده از الگوریتم‌های یادگیری تقویتی، این پژوهش به هدایت در محیط‌های نامطمئن و بدون نیاز به مدل دقیق محیط فضایی امکان می‌دهد. این موضوع در مواقعی که مدل‌سازی دقیق مدل‌های دینامیکی مشکل باشد، قابل اجرا است.

    \item \textbf{استفاده در ماموریت‌های متنوع:} این پژوهش امکان استفاده در ماموریت‌های مختلف فضایی را فراهم می‌کند. از جمله ماموریت‌های مختلف می‌توان به ماموریت‌های ماهواره‌ای، تعمیر و نگهداری ماموریت فضایی، تعقیب اهداف در مدارهای مختلف، و انجام مانور‌های پیچیده اشاره کرد.
\end{enumerate}
}{0}{{}}






\Description{تعریف دقیق مسئله}{
}{0}{{}}


\Description{فرضیات مسئله}{
}{0}{{}}

\Description{روش انجام کار}{
}{0}{{}}

\Description{پژوهش‌های خارجی}{
}{0}{{}}


\Description{پژوهش‌های داخلی}{
}{0}{{}}


\Description{پژوهش‌های داخل دانشگاه و دانشکده}{
}{0}{{}}





%Table of progress:
%   Pass line number, works and their duration in a comma delimited list.
%   line number / work / duration ,
%   example: 1 / read X / 1 year ,

\ProgressTable{
    {
    1/پایداری و کنترل: برقراری پایداری و کنترل مناسب در حین ناوبری گروهی از پرندهها بسیاراهمیت دارد. هماهنگی در حرکت و جهتدهی به پرندهها بهمنظور جلوگیری از تصادف و تضاد در  حرکت ضروری است.
    \newline
    علی
    /۴ ماه,
    ۲/مرحله دوم/۳ ماه,
    ۳/مرحله سوم/۲ ماه,
    ۳/مرحله چهارم/۵ ماه,
    ۵/مرحله پنجم/۲ ماه
    }
}

\noindent

% Add our references in references,bib
\References{}

\noindent

% Change \newcommand{\Courses} in the previous lines
\Courses{}

\noindent

\Signatures{}

\end{document}
