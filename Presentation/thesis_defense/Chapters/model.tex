\section{Dynamical Model}

\begin{frame}
  \frametitle{Planar CRTBP Model}
  \vspace{-.4cm}
  \begin{figure}[H]
    \centering
        \hspace{-1.5cm}
        \begin{subfigure}{0.45\linewidth}
          \centering
          \resizebox{\linewidth}{!}{%
            \begin{tikzpicture}
              % Coordinates
              \coordinate (earth) at (1,2);
              \coordinate (moon) at (8,1);
              \coordinate (earth-point1) at ({\r*cos(\Theta)+1},{\r*sin(\Theta)+2});
              \coordinate (A) at (-.5,.5);
              \coordinate (B) at (8.5,-0.5);
              
              % Earth
              \draw[thick, fill=black!30, draw=black!30
              ] (earth) circle (\r);
              \node[inner sep=0pt] (Earth_c) at (earth) {\includegraphics[width=1.8cm]{../../Figure/TBP/Earth.png}};
              % Text
              \node[below, shift={(0,-0.8)}] at (earth) {$m_1$};
              \node (a) at (A) {Earth};
              
              % Moon
              \node[circle, inner sep=5.5pt, fill=black!30] (MOON) at (moon) {};
              \node[inner sep=0pt] (moon_c) at (moon) {\includegraphics[width=.5cm]{../../Figure/TBP/Moon.png}};
              % Text 
              \node[below, shift={(0,-0.4)}] at (MOON) {$m_2$};
              \node (b) at (B) {Moon};
              
              % Lines
              \draw[-stealth] (a) to[bend left=30] ({\r*cos(\Phi)+1},{\r*sin(\Phi)+2});
              \draw[-stealth] (b) to[bend left=-30] (MOON);
              \draw[dashed, black] (earth) -- (MOON.center);
              
              % center of mass 0.25 from earth
              \coordinate (center) at ($(earth)!0.3!(MOON)$);
              % small circle
              \draw[fill=black] (center) circle (1.5pt) node[below, shift={(0,-0.1)}] {Center of Mass};
              
              % Calculate direction from Earth to Moon
              \pgfmathsetmacro{\xDiff}{8 - 1} % X difference between Moon and Earth
              \pgfmathsetmacro{\yDiff}{1 - 2} % Y difference between Moon and Earth
              \pgfmathsetmacro{\angle}{atan2(\yDiff,\xDiff)} % Angle of the line
              
              % Add axes at center of mass
              \draw[->, thick] (center) -- ++(\angle:2) node[above, shift={(0,0.2)}] {$x$ axis};
              \draw[->, thick] (center) -- ++(\angle+90:2) node[above] {$y$ axis};
              
              % add satellite with shift
              \coordinate (satellite) at ($(center)!0.5!(MOON)+(0,2)$);
              \node (satellite) at (satellite) {\faSpaceShuttle};
              
              % connect earth to satellite r1
              \draw[-stealth] (earth) -- (satellite) node[pos=0.3, above] {$\vb{r}_1$};   
              % connect moon to satellite r2
              \draw[-stealth] (MOON) -- (satellite) node[pos=0.5, above] {$\vb{r}_2$};
              % connect center of mass to satellite r
              \draw[-stealth] (center) -- (satellite) node[pos=0.5, above] {$\vb{r}$};
              % add line to show satellite is in between
              \node (c) at ($(satellite)+(1.5,0.5)$) {Spacecraft};
              \draw[-stealth] (c) to[bend left=30] (satellite);
            \end{tikzpicture}
          }
          \caption{CRTBP Configuration}
        \end{subfigure}%
        % \hfill
        % \hspace{-3.5cm}
        \begin{subfigure}{0.4\linewidth}
        %   \raggedleft
          \resizebox{\linewidth}{!}{%
        \begin{tikzpicture}
            % Define radius of the orbit
            \def\orbitRadius{4cm}
            
            % Draw the orbit circle with arrows
            \draw[-{Stealth[length=3mm]}, thick] (12:\orbitRadius) arc (12:170:\orbitRadius);
            \draw[-{Stealth[length=3mm]}, thick] (180:\orbitRadius) arc (180:355:\orbitRadius);
            
            % Position for Earth (center)
            \node[inner sep=0pt] (Earth) at (0,0) {\includegraphics[width=1.8cm]{../../Figure/TBP/Earth.png}};
            \node[text=black] at (0,-1.3) {Earth};
            
            % Position for Moon (on the right side of the orbit)
            \node[inner sep=0pt] (Moon) at (0.95*\orbitRadius,0) {\includegraphics[width=0.6cm]{../../Figure/TBP/Moon.png}};
            \node[text=black] at (\orbitRadius,0.6) {Moon};
            
            % Lagrangian points
            \coordinate (L1) at (0.8*\orbitRadius,0);
            \coordinate (L2) at (1.2*\orbitRadius,0);
            \coordinate (L3) at (-1*\orbitRadius,0);
            \coordinate (L4) at (60:\orbitRadius);
            \coordinate (L5) at (300:\orbitRadius);
            
            % Draw Lagrangian points as red circles
            \foreach \point in {L1,L2,L3,L4,L5} {
                \fill[red] (\point) circle (0.1cm);
            }
            
            % Connect Lagrangian points with dashed lines
            \draw[dashed, blue, thick] (Earth) -- (L1);
            \draw[dashed, blue, thick] (Moon) -- (L1);
            \draw[dashed, blue, thick] (Moon) -- (L2);
            \draw[dashed, blue, thick] (Earth) -- (L3);
            \draw[dashed, blue, thick] (Earth) -- (L4);
            \draw[dashed, blue, thick] (Moon) -- (L4);
            \draw[dashed, blue, thick] (Earth) -- (L5);
            \draw[dashed, blue, thick] (Moon) -- (L5);
            
            % Labels for the Lagrangian points
            \node at ($(L1) + (0,0.5)$) {$L_1$};
            \node at ($(L2) + (0,0.5)$) {$L_2$};
            \node at ($(L3) + (0,0.5)$) {$L_3$};
            \node at ($(L4) + (0.5,0.3)$) {$L_4$};
            \node at ($(L5) + (0.5,-0.3)$) {$L_5$};
        \end{tikzpicture}
          }
        %   \raggedright
          \caption{Lagrangian points in the Earth-Moon system}
        \end{subfigure}
        \vspace{-0.2cm}
        \caption{CRTBP Model and Lagrangian Points}
  \end{figure}
\end{frame}
%           \draw[-stealth] (center) -- (satellite) node[pos=0.5, above] {$\vb{r}$};
%           % add line to show satellite is in between
%           \node (c) at ($(satellite)+(1.5,0.5)$) {Satellite};
%           \draw[-stealth] (c) to[bend left=30] (satellite);
%         \end{tikzpicture}
%       }
%       \caption{CRTBP Configuration}
%     \end{subfigure}
%     \caption{Duplicated CRTBP geometric schematic (side-by-side).}
%   \end{figure}
% \end{frame}
