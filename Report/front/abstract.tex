% -------------------------------------------------------
%  Abstract
% -------------------------------------------------------


\pagestyle{plain}

\begin{وسط‌چین}
\مهم{چکیده}
\end{وسط‌چین}

\بدون‌تورفتگی 
در این پژوهش، یک چارچوب هدایت مقاوم برای فضاپیماهای کم‌پیشران در محیط‌های دینامیکی چندجسمی (سامانه سه جسمی زمین–ماه) ارائه شده‌است. مسئله به‌صورت بازی دیفرانسیلی مجموع‌صفر بین عامل هدایت (فضاپیما) و عامل مزاحم (عدم قطعیت‌های محیطی) فرمول‌بندی شده و با رویکرد آموزش متمرکز و اجرای توزیع‌شده پیاده‌سازی گردیده است. در این راستا، چهار الگوریتم یادگیری تقویتی پیوسته \lr{DDPG}، \lr{TD3}، \lr{SAC} و \lr{PPO} به نسخه‌های چندعاملی مجموع‌صفر گسترش یافته‌اند (\lr{MA‑DDPG}، \lr{MA-TD3}، \lr{MA-SAC} و \lr{MA-PPO}) و 
جریان آموزش آن‌ها همراه با ساختار شبکه‌ها در قالب اطلاعات کامل تشریح شده‌است.
ارزیابی الگوریتم‌ها در سناریوهای متنوع عدم قطعیت شامل شرایط اولیه تصادفی، اغتشاش عملگر، نویز حسگر، تأخیر زمانی و عدم تطابق مدل روی مسیر مدار لیاپانوف زمین–ماه انجام گرفت. نتایج به‌وضوح نشان می‌دهد که نسخه‌های مجموع‌صفر در تمامی معیارهای ارزیابی بر نسخه‌های تک‌عاملی برتری دارند. به‌ویژه الگوریتم \lr{MA-TD3} با حفظ پایداری سیستم، کمترین انحراف مسیر و مصرف سوخت بهینه را حتی در سخت‌ترین سناریوهای آزمون از خود نشان داد.
در نهایت، چارچوب پیشنهادی نشان می‌دهد که یادگیری تقویتی چندعاملی مبتنی بر بازی دیفرانسیلی مجموع‌صفر می‌تواند بدون نیاز به مدل‌سازی دقیق، هدایت تطبیقی و مقاوم فضاپیماهای کم‌پیشران را در نواحی ناپایدار سیستم‌های سه‌جسمی تضمین کند.


\پرش‌بلند
\بدون‌تورفتگی \مهم{کلیدواژه‌ها}: 
 یادگیری تقویتی عمیق، بازی دیفرانسیلی، سامانه‌های چندعاملی، هدایت کم‌پیشران، بازی مجموع‌صفر، مسئله محدود سه‌جسمی، کنترل مقاوم.
\صفحه‌جدید
