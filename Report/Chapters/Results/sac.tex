\section{الگوریتم \lr{SAC}}
\label{sec:sac_results}

الگوریتم \lr{SAC}  از روش‌های نوین یادگیری تقویتی است که با استفاده از مفهوم آنتروپی، تعادل بهتری بین اکتشاف و بهره‌برداری ایجاد می‌کند. این الگوریتم در شرایط فضاهای پیوسته عملکرد قابل توجهی دارد.
\subsection{مسیر طی‌شده}
\begin{figure}[H]
	\centering
	
	% سطر اول
	\subfloat[\lr{SAC} استاندارد]{\includegraphics[width=.45\textwidth]{plots/sac/trajectory_force/plot_trajectory.pdf}}%
	\subfloat[\lr{SAC} بازی مجموع‌صفر]{\includegraphics[width=.45\textwidth]{plots/sac/trajectory_force/plot_trajectory_zs.pdf}}%
	
	\caption{مسیر طی‌شده فضاپیما با \lr{SAC} استاندارد و نسخه بازی مجموع‌صفر \lr{MA-SAC}.}
\end{figure}

\subsection{مسیر و فرمان پیشران}
\begin{figure}[H]
	\centering
	% سطر اول
	\subfloat[\lr{SAC} استاندارد]{\includegraphics[width=.45\textwidth]{plots/sac/trajectory_force/plot_trajectory_force.pdf}}%
	\subfloat[\lr{SAC} بازی مجموع‌صفر]{\includegraphics[width=.45\textwidth]{plots/sac/trajectory_force/plot_trajectory_force_zs.pdf}}%
	
	\caption{مسیر و فرمان پیشران فضاپیما در \lr{SAC} استاندارد و نسخه بازی مجموع‌صفر \lr{MA-SAC}.}
\end{figure}

الگوریتم \lr{SAC} در هر دو حالت عملکرد قابل قبولی ارائه می‌دهد. ویژگی خاص این الگوریتم در تنظیم خودکار پارامتر آنتروپی باعث می‌شود که بتواند تعادل مناسبی بین اکتشاف و بهره‌برداری ایجاد کند، اما نسخه بازی مجموع‌صفر آن در شرایط سخت‌تر مقاومت بیشتری نشان می‌دهد و در برخی شاخص‌ها عملکردی برابر با نسخه استاندارد دارد.
\subsubsection*{جمع‌بندی تحلیلی پایان‌نامه}
\begin{itemize}
	\item در سناریوهای شرایط اولیه تصادفی و عدم تطابق مدل، نسخه چندعامله با کاهش چشمگیر احتمال شکست به بازه زیر ۰.۵ (۰.۴۰، ۰.۴۵ و ۰.۴۲) و بهبود پاداش، نشان می‌دهد که کنترل آنتروپی مشترک میان عامل‌ها چطور می‌تواند به مدیریت ریسک ساختاری کمک کند.
	\item عملکرد ممتاز در مشاهده ناقص و تأخیر زمانی (کاهش خطای مسیر از ۱.۹۵ به ۰.۰۶ و از ۱.۸۷ به ۰.۰۱) نشان می‌دهد که سازوکار چندعامله به‌خوبی می‌تواند اطلاعات ناقص یا دیررس را همپوشانی کند؛ این نکته در تحلیل پایان‌نامه برای مأموریت‌های واقعی با لینک‌های ارتباطی محدود حیاتی است.
	\item در نویز حسگر، حفظ برتری پاداش با هزینه کنترلی کمتر بیان می‌کند که تنظیم آنتروپی در نسخه چندعامله به سمت رفتارهای ملایم‌تر سوق پیدا کرده و انرژی کمتری مصرف شده است.
	\item تساوی نسبی تلاش کنترلی در سناریوهای سخت‌تر و در عین حال بهبود پاداش، نشان می‌دهد که نسخه چندعامله تعادلی بین اکتشاف و بهره‌برداری برقرار کرده که در نسخه استاندارد تنها با هزینه بیشتر قابل دستیابی است؛ این مسئله مزیت سرمایه‌گذاری روی معماری چندعامله را توجیه می‌کند.
\end{itemize}


\subsection{توزیع پاداش تجمعی}
این بخش نمودارهای ویولن توزیع پاداش تجمعی را در سناریوهای مختلف برای \lr{SAC} و \lr{MA-SAC} نمایش می‌دهد.
\begin{figure}[H]
	\centering
	% سطر اول
	\subfloat[شرایط اولیه تصادفی]{\includegraphics[width=.33\textwidth]{plots/sac/violin_plot/initial_condition_shift.pdf}}%
	\subfloat[اغتشاش در عملگرها]{\includegraphics[width=.33\textwidth]{plots/sac/violin_plot/actuator_disturbance.pdf}}%
	\subfloat[عدم تطابق مدل]{\includegraphics[width=.33\textwidth]{plots/sac/violin_plot/model_mismatch.pdf}}\\[1ex]
	% سطر دوم
	\subfloat[مشاهده ناقص]{\includegraphics[width=.33\textwidth]{plots/sac/violin_plot/partial_observation.pdf}}%
	\subfloat[نویز حسگر]{\includegraphics[width=.33\textwidth]{plots/sac/violin_plot/sensor_noise.pdf}}%
	\subfloat[تأخیر زمانی]{\includegraphics[width=.33\textwidth]{plots/sac/violin_plot/time_delay.pdf}}
	\caption{مقایسه توزیع پاداش تجمعی در سناریوهای مختلف برای \lr{SAC} و \lr{MA-SAC}.}
	\label{fig:sac_robustness_violin}
\end{figure}


\subsection{مقایسه عددی}
این بخش شاخص‌های عددی را گزارش می‌کند؛ نتایج بر اساس 100 اجرای مستقل شبیه‌سازی برای هر سناریو به‌دست آمده‌اند.
\begin{table}[H]
	\centering
	\setlength{\tabcolsep}{3pt}
	\small
	\begin{tabular}{@{} R{3.2cm} *{8}{C{1.05cm}} @{}}
		\toprule
		\multirow{2}{*}{\makecell[r]{سناریو}}
		& \multicolumn{2}{c}{پاداش تجمعی} & \multicolumn{2}{c}{مجموع خطای مسیر}
		& \multicolumn{2}{c}{مجموع تلاش کنترلی} & \multicolumn{2}{c}{احتمال شکست} \\
		\cmidrule(lr){2-3}\cmidrule(lr){4-5}\cmidrule(lr){6-7}\cmidrule(lr){8-9}
		& {\rotatebox[origin=c]{90}{\lr{SAC}}} & {\rotatebox[origin=c]{90}{\lr{MA-SAC}}}
		& {\rotatebox[origin=c]{90}{\lr{SAC}}} & {\rotatebox[origin=c]{90}{\lr{MA-SAC}}}
		& {\rotatebox[origin=c]{90}{\lr{SAC}}} & {\rotatebox[origin=c]{90}{\lr{MA-SAC}}}
		& {\rotatebox[origin=c]{90}{\lr{SAC}}} & {\rotatebox[origin=c]{90}{\lr{MA-SAC}}} \\
		\midrule
		شرایط اولیه تصادفی
		&
		$-4.69$ & ${-2.50}$ & $0.29$ & ${0.22}$ & $2.15$ & ${1.20}$ & $1.00$ & ${0.40}$ \\
		اغتشاش در عملگرها
		&
		$-1.95$ & ${-1.60}$ & $8.02$ & ${6.90}$ & $3.26$ & ${2.80}$ & $1.00$ & ${0.45}$ \\
		عدم تطابق مدل
		&
		$-4.89$ & ${-3.50}$ & $0.38$ & ${0.22}$ & $1.99$ & ${1.00}$ & $1.00$ & ${0.42}$ \\
		مشاهده ناقص
		&
		$-3.63$ & ${-0.35}$ & $1.95$ & ${0.06}$ & $2.32$ & ${1.70}$ & $1.00$ & ${0.00}$ \\
		نویز حسگر
		&
		$-0.89$ & ${0.14}$ & $0.12$ & ${0.10}$ & $2.10$ & ${1.60}$ & $0.00$ & ${0.00}$ \\
		تأخیر زمانی
		&
		$-4.14$ & ${-0.04}$ & $1.87$ & ${0.01}$ & $2.22$ & ${1.10}$ & $1.00$ & ${0.00}$ \\
		\bottomrule
	\end{tabular}
	\caption{مقایسه عملکرد \lr{SAC} و \lr{MA-SAC} در سناریوهای مختلف مقاومت}
	\label{tab:sac_comparison}
\end{table}