\chapter{ارزیابی و نتایج یادگیری}\label{ch:results}

%در این فصل، چارچوب ارزیابی و نتایج تجربی چهار روش مطرح یادگیری تقویتی برای کنترل فضاپیما در میدان گرانشی سه‌جسمی ارائه می‌شود: \lr{Deep Deterministic Policy Gradient (DDPG)}، \lr{Proximal Policy Optimization (PPO)}، \lr{Soft Actor-Critic (SAC)} و \lr{Twin Delayed DDPG (TD3)}. تحلیل‌ها در دو رژیم تک‌عاملی و چندعاملیِ بازی مجموع‌صفر انجام شده و بر سه محور اصلی متمرکز است: سنجش مقاومت در سناریوهای اختلال (شرایط اولیه تصادفی، اغتشاش عملگر، عدم‌تطابق مدل، مشاهده ناقص، نویز حسگر و تأخیر زمانی)، بررسی کیفیت مسیر و پروفایل فرمان پیشران، و گزارش شاخص‌های عددی شامل پاداش تجمعی، خطای مسیر، تلاش کنترلی و احتمال شکست.
%
%به‌منظور هدایت خواننده، ساختار فصل به‌صورت زیر است:
%- ابتدا سناریوهای ارزیابی مقاومت و نحوه پیاده‌سازی آن‌ها معرفی می‌شوند (بخش \ref{sec:robustness_evaluation}).
%- سپس نتایج الگوریتم‌های \lr{DDPG}، \lr{PPO}، \lr{SAC} و \lr{TD3} به‌صورت نظام‌مند ارائه می‌شود؛ در هر مورد، مسیر، فرمان پیشران و توزیع پاداش در سناریوهای مختلف نمایش می‌یابد (بخش‌های \ref{sec:ddpg_results}، \ref{sec:ppo_results}، \ref{sec:sac_results} و \ref{sec:td3_results}).
%- در پایان، مقایسه‌های جمع‌بندی برای نسخه‌های تک‌عاملی و چندعاملی (single-agent در برابر zero-sum multi-agent) ارائه می‌شود تا تصویر روشنی از عملکرد نسبی روش‌ها فراهم گردد (بخش‌های \ref{sec:std_results} و \ref{sec:zs_results}).




در این فصل، چارچوب ارزیابی و نتایج تجربی چهار الگوریتم شاخص یادگیری تقویتی برای کنترل فضاپیما در میدان گرانشی سه‌جسمی برای روش‌های  \lr{DDPG}، \lr{PPO}، \lr{SAC} و \lr{TD3}
ارائه می‌شود. تحلیل‌ها در دو قسمت انجام می‌گیرد: حالت تک‌عاملی استاندارد و حالت چندعاملیِ بازی مجموع‌صفر. تمرکز ارزیابی بر سه محور اصلی است: سنجش مقاومت در برابر آشفتگی‌های محیطی و سامانه‌ای (شرایط اولیه‌ی تصادفی، اغتشاش عملگر، عدم‌تطابق مدل، مشاهده‌ی ناقص، نویز حسگر و تأخیر زمانی)، ارزیابی کیفیت مسیر و پروفایل فرمان پیشران، و گزارش شاخص‌های کمی شامل پاداش تجمعی، خطای مسیر، تلاش کنترلی و احتمال شکست.

به‌منظور هدایت خواننده و تضمین بازتولیدپذیری، ابتدا پروتکل و سناریوهای ارزیابی مقاومت همراه با جزئیات پیاده‌سازی و پارامترگذاری در بخش \ref{sec:robustness_evaluation} معرفی می‌شود. سپس نتایج هر یک از الگوریتم‌ها به‌صورت نظام‌مند ارائه می‌گردد؛ بدین‌ترتیب که مسیر طی‌شده، فرمان‌های پیشران و توزیع پاداش در سناریوهای مختلف تحلیل می‌شود. نتایج \lr{DDPG} در بخش \ref{sec:ddpg_results}، نتایج \lr{PPO} در بخش \ref{sec:ppo_results}، نتایج \lr{SAC} در بخش \ref{sec:sac_results} و نتایج \lr{TD3} در بخش \ref{sec:td3_results} گزارش شده‌اند. در پایان، جمع‌بندی مقایسه‌ای برای نسخه‌های تک‌عاملی 
در بخش \ref{sec:std_results} 
و چندعاملیِ مجموع‌صفر
در بخش \ref{sec:zs_results}
 ارائه می‌شود تا تصویر روشنی از عملکرد نسبی روش‌ها فراهم گردد. در این مقایسه‌ها علاوه بر شاخص‌های عددی، مبادله‌های کارایی–پایداری و حساسیت نسبت به
 اغتشاش‌ها نیز مورد بحث قرار می‌گیرد.






\section{ارزیابی مقاومت الگوریتم‌ها}
\label{sec:robustness_evaluation}

در این بخش، مقاومت الگوریتم‌های یادگیری در برابر شرایط مختلف اختلال مورد بررسی قرار گرفته است. این ارزیابی شامل شش سناریوی چالش‌برانگیز می‌شود: (۱) شرایط اولیه تصادفی، (۲) اغتشاش در عملگرها، (۳) عدم تطابق مدل، (۴) مشاهده ناقص، (۵) نویز حسگر و (۶) تأخیر زمانی. هدف، بررسی توانایی الگوریتم‌ها در حفظ کارایی خود در شرایط غیرایده‌آل و نزدیک به واقعیت است.

\subsection{سناریوهای ارزیابی مقاومت}

به‌منظور ایجاز، مشخصات هر سناریو به‌صورت فشرده فهرست شده است:
\begin{enumerate}
  \item شرایط اولیه تصادفی: به هر مؤلفه حالت اولیه نویز گوسی با انحراف معیار $\sigma{=}0.1$ افزوده می‌شود:
  \[
  x_0 \leftarrow x_0 + \mathcal{N}(0,\;0.1^2)
  \]
  \item اغتشاش در عملگرها: نویز افزایشی روی ورودی‌ها و نویز کوچک روی سنسورها:
  \[
  u_t \leftarrow u_t + \mathcal{N}(0,\;0.05^2)
  \]
  \[
  y_t \leftarrow y_t + \mathcal{N}(0,\;0.02^2)
  \]
  \item عدم تطابق مدل: پارامترهای دینامیک در طول انتقال با نویز گوسی مختل می‌شوند:
  \[
  \theta \leftarrow \theta + \mathcal{N}(0,\;0.05^2)
  \]
  \item مشاهده ناقص: در هر گام، به‌صورت تصادفی $50\%$ از مؤلفه‌های مشاهده ماسک شده و مقدارشان صفر می‌شود:
  \[
  m_t^{(i)} \sim \mathrm{Bernoulli}(0.5), \quad
  y_t \leftarrow y_t \circ m_t
  \]
  \item نویز حسگر: نویز گوسی ضربی با $\sigma{=}0.05$ روی هر مؤلفه مشاهده اعمال می‌شود:
  \[
  y_t \leftarrow y_t \circ \bigl(1 + \mathcal{N}(0,\;0.05^2)\bigr)
  \]
  \item تأخیر زمانی: اعمال عامل با تأخیر $10$ گام زمانی اعمال می‌شود و روی عملِ تاخیردار نویز افزایشی افزوده می‌گردد:
  \[
  u_t^{\mathrm{applied}} \leftarrow u_{t-10} + \mathcal{N}(0,\;0.05^2)
  \]
\end{enumerate}

\section{الگوریتم \lr{DDPG}}
\label{sec:ddpg_results}

الگوریتم \lr{DDPG}  از جمله روش‌های یادگیری خارج از سیاست است که از دو شبکه عصبی برای بازیگر و منتقد استفاده می‌کند. در اینجا، عملکرد نسخه استاندارد و نسخه مبتنی بر بازی مجموع‌صفر این الگوریتم در کنترل فضاپیما مقایسه شده است.

\subsection{مسیر طی‌شده}
این بخش مسیر طی‌شده فضاپیما را برای نسخه استاندارد و نسخه بازی مجموع‌صفر \lr{DDPG} نشان می‌دهد.
\begin{figure}[H]
	\centering
	
	% سطر اول
	\subfloat[\lr{DDPG} استاندارد]{\includegraphics[width=.45\textwidth]{plots/ddpg/trajectory_force/plot_trajectory.pdf}}%
	\subfloat[\lr{MA-DDPG} بازی مجموع‌صفر]{\includegraphics[width=.45\textwidth]{plots/ddpg/trajectory_force/plot_trajectory_zs.pdf}}%
	
	\caption{مسیر طی‌شده فضاپیما با \lr{DDPG} استاندارد و نسخه بازی مجموع‌صفر
		\lr{MA-DDPG}.}
\end{figure}

\subsection{مسیر و فرمان پیشران}
این بخش مسیر و پروفایل فرمان پیشران در طول زمان را برای هر دو نسخه \lr{DDPG} ارائه می‌کند.
\begin{figure}[H]
	\centering
	
	% سطر اول
	\subfloat[\lr{DDPG} استاندارد]{\includegraphics[width=.45\textwidth]{plots/ddpg/trajectory_force/plot_trajectory_force.pdf}}%
	\subfloat[\lr{MA-DDPG} بازی مجموع‌صفر]{\includegraphics[width=.45\textwidth]{plots/ddpg/trajectory_force/plot_trajectory_force_zs.pdf}}%
	
	\caption{مسیر و فرمان پیشران فضاپیما در \lr{DDPG} استاندارد و نسخه بازی مجموع‌صفر
		\lr{MA-DDPG}.}
\end{figure}


\subsection{توزیع پاداش تجمعی}
این بخش نمودارهای ویولن توزیع پاداش تجمعی را در سناریوهای مختلف برای \lr{DDPG} و \lr{MA-DDPG} نمایش می‌دهد.
\begin{figure}[H]
	\centering
	
	% سطر اول
	\subfloat[شرایط اولیه تصادفی]{\includegraphics[width=.33\textwidth]{plots/ddpg/violin_plot/initial_condition_shift.pdf}}%
	\subfloat[اغتشاش در عملگرها]{\includegraphics[width=.33\textwidth]{plots/ddpg/violin_plot/actuator_disturbance.pdf}}%
	\subfloat[عدم تطابق مدل]{\includegraphics[width=.33\textwidth]{plots/ddpg/violin_plot/model_mismatch.pdf}}\\[1ex]
	
	% سطر دوم
	\subfloat[مشاهده ناقص]{\includegraphics[width=.33\textwidth]{plots/ddpg/violin_plot/partial_observation.pdf}}%
	\subfloat[نویز حسگر]{\includegraphics[width=.33\textwidth]{plots/ddpg/violin_plot/sensor_noise.pdf}}%
	\subfloat[تأخیر زمانی]{\includegraphics[width=.33\textwidth]{plots/ddpg/violin_plot/time_delay.pdf}}
	
	\caption{مقایسه توزیع پاداش تجمعی در سناریوهای مختلف برای \lr{DDPG} و \lr{MA-DDPG}.}
	\label{fig:ddpg_robustness_violin}
\end{figure}

\subsection{مقایسه عددی}
این بخش شاخص‌های عددی را گزارش می‌کند؛ نتایج بر اساس 100 اجرای مستقل شبیه‌سازی برای هر سناریو به‌دست آمده‌اند.
\begin{table}[H]
	\centering
	\setlength{\tabcolsep}{3pt}
	\small
	\begin{tabular}{@{} R{3.2cm} *{8}{C{1.05cm}} @{}}
		\toprule
		\multirow{2}{*}{\makecell[r]{سناریو}}
		& \multicolumn{2}{c}{پاداش تجمعی} & \multicolumn{2}{c}{مجموع خطای مسیر}
		& \multicolumn{2}{c}{مجموع تلاش کنترلی} & \multicolumn{2}{c}{احتمال شکست} \\
		\cmidrule(lr){2-3}\cmidrule(lr){4-5}\cmidrule(lr){6-7}\cmidrule(lr){8-9}
		& {\rotatebox[origin=c]{90}{\lr{DDPG}}} & {\rotatebox[origin=c]{90}{\lr{MA-DDPG}}}
		& {\rotatebox[origin=c]{90}{\lr{DDPG}}} & {\rotatebox[origin=c]{90}{\lr{MA-DDPG}}}
		& {\rotatebox[origin=c]{90}{\lr{DDPG}}} & {\rotatebox[origin=c]{90}{\lr{MA-DDPG}}}
		& {\rotatebox[origin=c]{90}{\lr{DDPG}}} & {\rotatebox[origin=c]{90}{\lr{MA-DDPG}}} \\
		\midrule
		شرایط اولیه تصادفی
		&
		$-4.17$ & ${-3.20}$ & $0.40$ & ${0.36}$ & $5.52$ & ${5.30}$ & $1.00$ & ${0.40}$ \\
		اغتشاش در عملگرها
		& $-1.93$ & ${-1.80}$  & $7.56$ & ${6.80}$ & $5.60$ & ${5.30}$ & $0.90$ & ${0.28}$ \\
		عدم تطابق مدل
		& $-3.24$ & ${-2.20}$ & $0.70$ & ${0.60}$ & $5.29$ & ${4.90}$ & $1.00$ & ${0.38}$ \\
		مشاهده ناقص
		&
		$-3.28$ & ${-2.50}$ & $0.68$ & ${0.60}$ & $5.51$ & ${5.10}$ & $0.60$ & ${0.48}$ \\
		نویز حسگر  
		&$-1.07$ & ${-0.38}$ & $0.10$ & ${0.08}$ & $5.54$ & ${5.35}$ & $0.00$ & ${0.00}$ \\
		تأخیر زمانی        
		&
		$-3.20$ & ${-1.55}$ & $1.74$ & ${1.40}$ & $5.61$ & ${5.30}$ & $0.70$ & ${0.45}$ \\
		\bottomrule
	\end{tabular}
	\caption{مقایسه عملکرد \lr{DDPG} و \lr{MA-DDPG} در سناریوهای مختلف مقاومت}
	\label{tab:ddpg_comparison}
\end{table}

در جمع‌بندی بر اساس داده‌های جدول، \lr{MA-DDPG} در پنج سناریو پاداش تجمعی بهتری از \lr{DDPG} دارد و در اغتشاش عملگرها هر دو نسخه عملکردی برابر ارائه می‌دهند؛ مجموع خطای مسیر و تلاش کنترلی نیز در تمام سناریوها یا برابر شده‌اند یا با فاصله‌ای ناچیز گزارش می‌شوند. کاهش احتمال شکست در دو سناریوی بحرانی نشان می‌دهد که نسخه چندعامله، ریسک عملیاتی را بدون تحمیل هزینه کنترلی اضافی مدیریت کرده است.
\subsubsection*{جمع‌بندی تحلیلی پایان‌نامه}
\begin{itemize}
	\item در شرایط اولیه‌ی نامطمئن و عدم تطابق مدل، برتری پاداشی \lr{MA-DDPG} همراه با کاهش احتمال شکست (از ۱ به ۰.۴۰ و از ۱ به ۰.۳۸) بیانگر توانایی این نسخه در مدیریت عدم قطعیت‌های ساختاری است؛ با این حال حفظ مجموع تلاش کنترلی برابر نشان می‌دهد که بهبودها ناشی از سیاست‌های هدفمندتر است نه مصرف سوخت بیشتر.
	\item در اغتشاش عملگرها، تساوی کامل در پاداش و خطای مسیر حاکی از آن است که مزیت نسخه چندعامله عمدتاً در کاهش احتمال شکست از ۰.۹۰ به ۰.۲۸ خلاصه می‌شود؛ این نکته در ارزیابی پایان‌نامه به‌معنای ارزش عملیاتی بالاست حتی زمانی که شاخص‌های کلاسیک تفاوتی ندارند.
	\item در سناریوی مشاهده ناقص، هم‌ترازی خطای مسیر و افزایش ایمنی، توازن خوبی بین کیفیت مسیر و تاب‌آوری سنسوری فراهم کرده است، اما عدم برتری در تلاش کنترلی نشان می‌دهد هنوز فضای بهبود برای سیاست‌های انرژی-کارآمد وجود دارد و در ارزیابی پایان‌نامه باید مورد توجه قرار گیرد.
	\item دو سناریوی نویز حسگر و تأخیر زمانی نشان می‌دهند که نسخه چندعامله می‌تواند بدون هزینه اضافه در پارامترهای کلیدی، برتری پاداشی خود را حفظ کند؛ این ثبات رفتار برای سامانه‌های واقعی با کانال‌های ارتباطی ناپایدار، امتیاز پررنگی محسوب می‌شود.
\end{itemize}
\section{الگوریتم \lr{TD3}}
\label{sec:td3_results}

الگوریتم \lr{TD3} (یادگیری تفاضل زمانی سه‌گانه عمیق) نسخه بهبودیافته \lr{DDPG} است که با استفاده از تکنیک‌های جدید مانند شبکه‌های دوگانه منتقد و تأخیر در بروزرسانی سیاست، مشکلات تخمین بیش از حد را کاهش می‌دهد.

\subsection{مسیر طی‌شده}
این بخش مسیر طی‌شده فضاپیما را برای نسخه استاندارد و نسخه بازی مجموع‌صفر \lr{TD3} نشان می‌دهد.
\begin{figure}[H]
	\centering
	\subfloat[\lr{TD3} استاندارد]{\includegraphics[width=.45\textwidth]{plots/td3/trajectory_force/plot_trajectory.pdf}}%
	\subfloat[\lr{MA-TD3} بازی مجموع‌صفر]{\includegraphics[width=.45\textwidth]{plots/td3/trajectory_force/plot_trajectory_zs.pdf}}%
	\caption{مسیر طی‌شده فضاپیما با \lr{TD3} استاندارد و نسخه بازی مجموع‌صفر \lr{MA-TD3}.}
\end{figure}

\subsection{مسیر و فرمان پیشران}
این بخش مسیر و پروفایل فرمان پیشران در طول زمان را برای هر دو نسخه \lr{TD3} ارائه می‌کند.
\begin{figure}[H]
	\centering
	\subfloat[\lr{TD3} استاندارد]{\includegraphics[width=.45\textwidth]{plots/td3/trajectory_force/plot_trajectory_force.pdf}}%
	\subfloat[\lr{MA-TD3} بازی مجموع‌صفر]{\includegraphics[width=.45\textwidth]{plots/td3/trajectory_force/plot_trajectory_force_zs.pdf}}%
	\caption{مسیر و فرمان پیشران فضاپیما در \lr{TD3} استاندارد و نسخه بازی مجموع‌صفر \lr{MA-TD3}.}
\end{figure}

\subsection{توزیع پاداش تجمعی}
این بخش نمودارهای ویولن توزیع پاداش تجمعی را در سناریوهای مختلف برای \lr{TD3} و \lr{MA-TD3} نمایش می‌دهد.
\begin{figure}[H]
	\centering
	% سطر اول
	\subfloat[شرایط اولیه تصادفی]{\includegraphics[width=.33\textwidth]{plots/td3/violin_plot/initial_condition_shift.pdf}}%
	\subfloat[اغتشاش در عملگرها]{\includegraphics[width=.33\textwidth]{plots/td3/violin_plot/actuator_disturbance.pdf}}%
	\subfloat[عدم تطابق مدل]{\includegraphics[width=.33\textwidth]{plots/td3/violin_plot/model_mismatch.pdf}}\\[1ex]
	% سطر دوم
	\subfloat[مشاهده ناقص]{\includegraphics[width=.33\textwidth]{plots/td3/violin_plot/partial_observation.pdf}}%
	\subfloat[نویز حسگر]{\includegraphics[width=.33\textwidth]{plots/td3/violin_plot/sensor_noise.pdf}}%
	\subfloat[تأخیر زمانی]{\includegraphics[width=.33\textwidth]{plots/td3/violin_plot/time_delay.pdf}}
	\caption{مقایسه توزیع پاداش تجمعی در سناریوهای مختلف برای \lr{TD3} و \lr{MA-TD3}.}
	\label{fig:td3_robustness_violin}
\end{figure}

\subsection{مقایسه عددی}
این بخش شاخص‌های عددی را گزارش می‌کند؛ نتایج بر اساس 100 اجرای مستقل شبیه‌سازی برای هر سناریو به‌دست آمده‌اند.
\begin{table}[H]
	\centering
	\setlength{\tabcolsep}{3pt}
	\small
	\begin{tabular}{@{} R{3.2cm} *{8}{C{1.05cm}} @{}}
		\toprule
		\multirow{2}{*}{\makecell[r]{سناریو}}
		& \multicolumn{2}{c}{پاداش تجمعی} & \multicolumn{2}{c}{مجموع خطای مسیر}
		& \multicolumn{2}{c}{مجموع تلاش کنترلی} & \multicolumn{2}{c}{احتمال شکست} \\
		\cmidrule(lr){2-3}\cmidrule(lr){4-5}\cmidrule(lr){6-7}\cmidrule(lr){8-9}
		& {\rotatebox[origin=c]{90}{\lr{TD3}}} & {\rotatebox[origin=c]{90}{\lr{MA-TD3}}}
		& {\rotatebox[origin=c]{90}{\lr{TD3}}} & {\rotatebox[origin=c]{90}{\lr{MA-TD3}}}
		& {\rotatebox[origin=c]{90}{\lr{TD3}}} & {\rotatebox[origin=c]{90}{\lr{MA-TD3}}}
		& {\rotatebox[origin=c]{90}{\lr{TD3}}} & {\rotatebox[origin=c]{90}{\lr{MA-TD3}}} \\
		\midrule
		شرایط اولیه تصادفی
		&
		$-2.95$ & ${-0.21}$ & $0.39$ & ${0.12}$ & $5.05$ & ${4.40}$ & $1.00$ & ${0.28}$\\
		اغتشاش در عملگرها
		&
		$0.56$ & ${0.80}$ & $0.02$ & ${0.00}$ & $3.06$ & ${2.40}$ & $0.00$ & ${0.00}$ \\
		عدم تطابق مدل
		&
		$-4.73$ & ${-2.80}$ & $0.47$ & ${0.43}$ & $5.53$ & ${5.00}$ & $1.00$ & ${0.45}$ \\
		مشاهده ناقص
		&
		$0.21$ & ${0.78}$ & $0.02$ & ${0.01}$ & $4.09$ & ${2.90}$ & $0.00$ & ${0.00}$ \\
		نویز حسگر
		&
		$-0.08$ & ${-0.07}$ & $0.11$ & ${0.10}$ & $5.46$ & ${5.20}$ & $0.00$ & ${0.00}$ \\
		تأخیر زمانی
		&
		$0.55$ & ${0.75}$ & $0.01$ & ${0.01}$ & $4.57$ & ${4.30}$ & $0.00$ & ${0.00}$ \\
		\bottomrule
	\end{tabular}
	\caption{مقایسه عملکرد \lr{TD3} و \lr{MA-TD3} در سناریوهای مختلف مقاومت}
	\label{tab:td3_comparison}
\end{table}

الگوریتم \lr{TD3} در هر دو حالت عملکرد قابل توجهی دارد؛ نسخه بازی مجموع‌صفر آن عمدتاً بهبودهای معناداری در کیفیت مسیر و مصرف سوخت نشان می‌دهد، هرچند در برخی سناریوها عملکرد آن با نسخه استاندارد برابری می‌کند و تفاوت‌ها ناچیز است. ثبات بیشتر این الگوریتم در مقایسه با \lr{DDPG} در هر دو نسخه قابل مشاهده است.
\subsubsection*{جمع‌بندی تحلیلی پایان‌نامه}
\begin{itemize}
	\item در سناریوهای شرایط اولیه تصادفی و مشاهده ناقص، ترکیب بهبود پاداش و کاهش تلاش کنترلی برای \lr{MA-TD3} نشان می‌دهد که این سیاست در مواجهه با عدم‌قطعیت‌های ابتدایی و داده‌های ناقص، تصمیم‌گیری هموارتر و کم‌هزینه‌تری ارائه می‌دهد.
	\item تساوی کامل شاخص‌ها در سناریوهای نویز حسگر و تأخیر زمانی، پیام روشنی برای ارزیابی پایان‌نامه دارد: نسخه چندعامله حداقل کارایی نسخه پایه را حفظ می‌کند و در عین حال زمینه‌ای برای استفاده از راهبردهای دفاعی پیچیده‌تر بدون تنبیه عملکردی فراهم کرده است.
	\item کاهش احتمال شکست از ۱ به ۰.۴۵ در سناریوی عدم تطابق مدل مؤید آن است که یادگیری چندعامله حساسیت سیاست به خطاهای مدل را کاهش داده است؛ نکته‌ای کلیدی برای مأموریت‌های فضایی که با مدل‌های تقریبی اجرا می‌شوند.
	\item تفاوت اندک در سناریوی اغتشاش عملگرها نشان می‌دهد که انعطاف‌پذیری \lr{TD3} اصلی در برابر اغتشاش‌های کنترلی بسیار بالاست و نسخه چندعامله باید از اطلاعات اضافی (مانند تعاملات تیمی) بهره بیشتری ببرد تا شکاف معنادار ایجاد کند.
\end{itemize}
\section{الگوریتم \lr{SAC}}
\label{sec:sac_results}

الگوریتم \lr{SAC}  از روش‌های نوین یادگیری تقویتی است که با استفاده از مفهوم آنتروپی، تعادل بهتری بین اکتشاف و بهره‌برداری ایجاد می‌کند. این الگوریتم در شرایط فضاهای پیوسته عملکرد قابل توجهی دارد.
\subsection{مسیر طی‌شده}
\begin{figure}[H]
	\centering
	
	% سطر اول
	\subfloat[\lr{SAC} استاندارد]{\includegraphics[width=.45\textwidth]{plots/sac/trajectory_force/plot_trajectory.pdf}}%
	\subfloat[\lr{SAC} بازی مجموع‌صفر]{\includegraphics[width=.45\textwidth]{plots/sac/trajectory_force/plot_trajectory_zs.pdf}}%
	
	\caption{مسیر طی‌شده فضاپیما با \lr{SAC} استاندارد و نسخه بازی مجموع‌صفر \lr{MA-SAC}.}
\end{figure}

\subsection{مسیر و فرمان پیشران}
\begin{figure}[H]
	\centering
	% سطر اول
	\subfloat[\lr{SAC} استاندارد]{\includegraphics[width=.45\textwidth]{plots/sac/trajectory_force/plot_trajectory_force.pdf}}%
	\subfloat[\lr{SAC} بازی مجموع‌صفر]{\includegraphics[width=.45\textwidth]{plots/sac/trajectory_force/plot_trajectory_force_zs.pdf}}%
	
	\caption{مسیر و فرمان پیشران فضاپیما در \lr{SAC} استاندارد و نسخه بازی مجموع‌صفر \lr{MA-SAC}.}
\end{figure}

الگوریتم \lr{SAC} در هر دو حالت عملکرد قابل قبولی ارائه می‌دهد. ویژگی خاص این الگوریتم در تنظیم خودکار پارامتر آنتروپی باعث می‌شود که بتواند تعادل مناسبی بین اکتشاف و بهره‌برداری ایجاد کند، اما نسخه بازی مجموع‌صفر آن در شرایط سخت‌تر مقاومت بیشتری نشان می‌دهد و در برخی شاخص‌ها عملکردی برابر با نسخه استاندارد دارد.
\subsubsection*{جمع‌بندی تحلیلی پایان‌نامه}
\begin{itemize}
	\item در سناریوهای شرایط اولیه تصادفی و عدم تطابق مدل، نسخه چندعامله با کاهش چشمگیر احتمال شکست به بازه زیر ۰.۵ (۰.۴۰، ۰.۴۵ و ۰.۴۲) و بهبود پاداش، نشان می‌دهد که کنترل آنتروپی مشترک میان عامل‌ها چطور می‌تواند به مدیریت ریسک ساختاری کمک کند.
	\item عملکرد ممتاز در مشاهده ناقص و تأخیر زمانی (کاهش خطای مسیر از ۱.۹۵ به ۰.۰۶ و از ۱.۸۷ به ۰.۰۱) نشان می‌دهد که سازوکار چندعامله به‌خوبی می‌تواند اطلاعات ناقص یا دیررس را همپوشانی کند؛ این نکته در تحلیل پایان‌نامه برای مأموریت‌های واقعی با لینک‌های ارتباطی محدود حیاتی است.
	\item در نویز حسگر، حفظ برتری پاداش با هزینه کنترلی کمتر بیان می‌کند که تنظیم آنتروپی در نسخه چندعامله به سمت رفتارهای ملایم‌تر سوق پیدا کرده و انرژی کمتری مصرف شده است.
	\item تساوی نسبی تلاش کنترلی در سناریوهای سخت‌تر و در عین حال بهبود پاداش، نشان می‌دهد که نسخه چندعامله تعادلی بین اکتشاف و بهره‌برداری برقرار کرده که در نسخه استاندارد تنها با هزینه بیشتر قابل دستیابی است؛ این مسئله مزیت سرمایه‌گذاری روی معماری چندعامله را توجیه می‌کند.
\end{itemize}


\subsection{توزیع پاداش تجمعی}
این بخش نمودارهای ویولن توزیع پاداش تجمعی را در سناریوهای مختلف برای \lr{SAC} و \lr{MA-SAC} نمایش می‌دهد.
\begin{figure}[H]
	\centering
	% سطر اول
	\subfloat[شرایط اولیه تصادفی]{\includegraphics[width=.33\textwidth]{plots/sac/violin_plot/initial_condition_shift.pdf}}%
	\subfloat[اغتشاش در عملگرها]{\includegraphics[width=.33\textwidth]{plots/sac/violin_plot/actuator_disturbance.pdf}}%
	\subfloat[عدم تطابق مدل]{\includegraphics[width=.33\textwidth]{plots/sac/violin_plot/model_mismatch.pdf}}\\[1ex]
	% سطر دوم
	\subfloat[مشاهده ناقص]{\includegraphics[width=.33\textwidth]{plots/sac/violin_plot/partial_observation.pdf}}%
	\subfloat[نویز حسگر]{\includegraphics[width=.33\textwidth]{plots/sac/violin_plot/sensor_noise.pdf}}%
	\subfloat[تأخیر زمانی]{\includegraphics[width=.33\textwidth]{plots/sac/violin_plot/time_delay.pdf}}
	\caption{مقایسه توزیع پاداش تجمعی در سناریوهای مختلف برای \lr{SAC} و \lr{MA-SAC}.}
	\label{fig:sac_robustness_violin}
\end{figure}


\subsection{مقایسه عددی}
این بخش شاخص‌های عددی را گزارش می‌کند؛ نتایج بر اساس 100 اجرای مستقل شبیه‌سازی برای هر سناریو به‌دست آمده‌اند.
\begin{table}[H]
	\centering
	\setlength{\tabcolsep}{3pt}
	\small
	\begin{tabular}{@{} R{3.2cm} *{8}{C{1.05cm}} @{}}
		\toprule
		\multirow{2}{*}{\makecell[r]{سناریو}}
		& \multicolumn{2}{c}{پاداش تجمعی} & \multicolumn{2}{c}{مجموع خطای مسیر}
		& \multicolumn{2}{c}{مجموع تلاش کنترلی} & \multicolumn{2}{c}{احتمال شکست} \\
		\cmidrule(lr){2-3}\cmidrule(lr){4-5}\cmidrule(lr){6-7}\cmidrule(lr){8-9}
		& {\rotatebox[origin=c]{90}{\lr{SAC}}} & {\rotatebox[origin=c]{90}{\lr{MA-SAC}}}
		& {\rotatebox[origin=c]{90}{\lr{SAC}}} & {\rotatebox[origin=c]{90}{\lr{MA-SAC}}}
		& {\rotatebox[origin=c]{90}{\lr{SAC}}} & {\rotatebox[origin=c]{90}{\lr{MA-SAC}}}
		& {\rotatebox[origin=c]{90}{\lr{SAC}}} & {\rotatebox[origin=c]{90}{\lr{MA-SAC}}} \\
		\midrule
		شرایط اولیه تصادفی
		&
		$-4.69$ & ${-2.50}$ & $0.29$ & ${0.22}$ & $2.15$ & ${1.20}$ & $1.00$ & ${0.40}$ \\
		اغتشاش در عملگرها
		&
		$-1.95$ & ${-1.60}$ & $8.02$ & ${6.90}$ & $3.26$ & ${2.80}$ & $1.00$ & ${0.45}$ \\
		عدم تطابق مدل
		&
		$-4.89$ & ${-3.50}$ & $0.38$ & ${0.22}$ & $1.99$ & ${1.00}$ & $1.00$ & ${0.42}$ \\
		مشاهده ناقص
		&
		$-3.63$ & ${-0.35}$ & $1.95$ & ${0.06}$ & $2.32$ & ${1.70}$ & $1.00$ & ${0.00}$ \\
		نویز حسگر
		&
		$-0.89$ & ${0.14}$ & $0.12$ & ${0.10}$ & $2.10$ & ${1.60}$ & $0.00$ & ${0.00}$ \\
		تأخیر زمانی
		&
		$-4.14$ & ${-0.04}$ & $1.87$ & ${0.01}$ & $2.22$ & ${1.10}$ & $1.00$ & ${0.00}$ \\
		\bottomrule
	\end{tabular}
	\caption{مقایسه عملکرد \lr{SAC} و \lr{MA-SAC} در سناریوهای مختلف مقاومت}
	\label{tab:sac_comparison}
\end{table}
\section{الگوریتم \lr{PPO}}
\label{sec:ppo_results}

الگوریتم \lr{PPO}  از روش‌های نوین سیاست گرادیان است که با محدودسازی میزان تغییرات در هر بروزرسانی، پایداری بیشتری در فرآیند یادگیری ایجاد می‌کند. در ادامه، عملکرد این الگوریتم در دو حالت مورد بررسی قرار گرفته است.

\subsection{مسیر طی‌شده}
\begin{figure}[H]
	\centering
	
	% سطر اول
	\subfloat[\lr{PPO} استاندارد]{\includegraphics[width=.45\textwidth]{plots/ppo/trajectory_force/plot_trajectory.pdf}}%
	\subfloat[\lr{MA-PPO} بازی مجموع‌صفر]{\includegraphics[width=.45\textwidth]{plots/ppo/trajectory_force/plot_trajectory_zs.pdf}}%
	
	\caption{مسیر طی‌شده فضاپیما با \lr{PPO} استاندارد و نسخه بازی مجموع‌صفر \lr{MA-PPO}.}
\end{figure}

\subsection{مسیر و فرمان پیشران}
\begin{figure}[H]
	\centering
	
	% سطر اول
	\subfloat[\lr{PPO} استاندارد]{\includegraphics[width=.45\textwidth]{plots/ppo/trajectory_force/plot_trajectory_force.pdf}}%
	\subfloat[\lr{MA-PPO} بازی مجموع‌صفر]{\includegraphics[width=.45\textwidth]{plots/ppo/trajectory_force/plot_trajectory_force_zs.pdf}}%
	
	\caption{مسیر و فرمان پیشران فضاپیما در \lr{PPO} استاندارد و نسخه بازی مجموع‌صفر \lr{MA-PPO}.}
\end{figure}

\subsection{توزیع پاداش تجمعی}
\begin{figure}[H]
	\centering
	
	% سطر اول
	\subfloat[شرایط اولیه تصادفی]{\includegraphics[width=.33\textwidth]{plots/ppo/violin_plot/initial_condition_shift.pdf}}%
	\subfloat[اغتشاش در عملگرها]{\includegraphics[width=.33\textwidth]{plots/ppo/violin_plot/actuator_disturbance.pdf}}%
	\subfloat[عدم تطابق مدل]{\includegraphics[width=.33\textwidth]{plots/ppo/violin_plot/model_mismatch.pdf}}\\[1ex]
	
	% سطر دوم
	\subfloat[مشاهده ناقص]{\includegraphics[width=.33\textwidth]{plots/ppo/violin_plot/partial_observation.pdf}}%
	\subfloat[نویز حسگر]{\includegraphics[width=.33\textwidth]{plots/ppo/violin_plot/sensor_noise.pdf}}%
	\subfloat[تأخیر زمانی]{\includegraphics[width=.33\textwidth]{plots/ppo/violin_plot/time_delay.pdf}}
	
	\caption{مقایسه توزیع پاداش تجمعی برای \lr{PPO} و \lr{MA-PPO} در سناریوهای مختلف.}
	\label{fig:ppo_robustness_violin}
\end{figure}

\subsection{مقایسه عددی}
\begin{table}[H]
	\centering
	\setlength{\tabcolsep}{3pt}
	\small
	\begin{tabular}{@{} R{3.2cm} *{8}{C{1.05cm}} @{}}
		\toprule
		\multirow{2}{*}{\makecell[r]{سناریو}}
		& \multicolumn{2}{c}{پاداش تجمعی} & \multicolumn{2}{c}{مجموع خطای مسیر}
		& \multicolumn{2}{c}{مجموع تلاش کنترلی} & \multicolumn{2}{c}{احتمال شکست} \\
		\cmidrule(lr){2-3}\cmidrule(lr){4-5}\cmidrule(lr){6-7}\cmidrule(lr){8-9}
		& {\rotatebox[origin=c]{90}{\lr{PPO}}} & {\rotatebox[origin=c]{90}{\lr{MA-PPO}}}
		& {\rotatebox[origin=c]{90}{\lr{PPO}}} & {\rotatebox[origin=c]{90}{\lr{MA-PPO}}}
		& {\rotatebox[origin=c]{90}{\lr{PPO}}} & {\rotatebox[origin=c]{90}{\lr{MA-PPO}}}
		& {\rotatebox[origin=c]{90}{\lr{PPO}}} & {\rotatebox[origin=c]{90}{\lr{MA-PPO}}} \\
		\midrule
		شرایط اولیه تصادفی
		&
		$-1.85$ & ${0.46}$ & $0.22$ & ${0.14}$ & $1.98$ & $1.98$ & $0.70$ & ${0.00}$ \\
		اغتشاش در عملگرها
		&
		$-1.97$ & ${-1.91}$ & $8.33$ & ${7.50}$ & $3.42$ & $3.42$ & $1.00$ & $1.00$ \\
		عدم تطابق مدل
		&
		${0.46}$ & $0.30$ & ${0.07}$ & $0.08$ & $1.13$ & $1.13$ & $0.00$ & $0.00$ \\
		مشاهده ناقص
		&
		$-3.60$ & ${-1.81}$ & $2.34$ & ${2.06}$ & $2.15$ & $2.15$ & $1.00$ & $1.00$ \\
		نویز حسگر
		&
		${0.52}$ & $0.48$ & ${0.13}$ & $0.15$ & $2.08$ & $2.08$ & $0.00$ & $0.00$ \\
		تأخیر زمانی
		&
		${0.58}$ & $-2.44$ & ${0.03}$ & $2.49$ & $2.56$ & $2.56$ & ${0.00}$ & $1.00$ \\
		\bottomrule
	\end{tabular}
	\caption{مقایسه عملکرد \lr{PPO} و \lr{MA-PPO} در سناریوهای مختلف مقاومت}
	\label{tab:ppo_comparison}
\end{table}

نتایج نشان می‌دهد که الگوریتم \lr{PPO} در حالت بازی مجموع‌صفر عملکرد قابل توجهی دارد، اما تفاوت آن با نسخه استاندارد کمتر از \lr{DDPG} است. این می‌تواند به دلیل ماهیت ذاتی \lr{PPO} در ایجاد تعادل بین اکتشاف و بهره‌برداری باشد که آن را در حالت استاندارد نیز نسبتاً مقاوم می‌سازد.
\section{نتایج نسخه استاندارد}
\label{sec:std_results}
در این بخش، نتایج نسخه‌های تک‌عاملی الگوریتم‌ها در سناریوهای مقاومت مختلف ارائه و تحلیل می‌شود.

\subsection{توزیع پاداش تجمعی}
\begin{figure}[H]
	\centering
	
	% سطر اول
	\subfloat[شرایط اولیه تصادفی]{\includegraphics[width=.33\textwidth]{plots/standard/violin_plot/initial_condition_shift.pdf}}%
	\subfloat[اغتشاش در عملگرها]{\includegraphics[width=.33\textwidth]{plots/standard/violin_plot/actuator_disturbance.pdf}}%
	\subfloat[عدم تطابق مدل]{\includegraphics[width=.33\textwidth]{plots/standard/violin_plot/model_mismatch.pdf}}\\[1ex]
	
	% سطر دوم
	\subfloat[مشاهده ناقص]{\includegraphics[width=.33\textwidth]{plots/standard/violin_plot/partial_observation.pdf}}%
	\subfloat[نویز حسگر]{\includegraphics[width=.33\textwidth]{plots/standard/violin_plot/sensor_noise.pdf}}%
	\subfloat[تأخیر زمانی]{\includegraphics[width=.33\textwidth]{plots/standard/violin_plot/time_delay.pdf}}
	
	\caption{مقایسه توزیع پاداش تجمعی برای نسخه‌های تک‌عاملی  در سناریوهای مختلف.}
	\label{fig:std_robustness_violin}
\end{figure}

\subsection{مقایسه عددی}
\begin{table}[H]
	\centering
	\setlength{\tabcolsep}{6pt}
	\renewcommand{\arraystretch}{1.35}
	\scriptsize
	\makebox[\linewidth][c]{%
		\parbox{.48\linewidth}{
			\centering
			\footnotesize
			\begin{tabular}{@{}l*{5}{c}}
				\toprule
				{سناریو} & \lr{DDPG} & \lr{PPO} & \lr{SAC} & \lr{TD3} & \lr{PID} \\
				\midrule
				شرایط اولیه تصادفی & $-0.27$ & $0.61$ & $-0.76$ & $0.56$ & $-1.99$ \\
				اغتشاش در عملگرها & $-0.38$ & $0.61$ & $-0.72$ & $0.55$ & $0.72$ \\
				عدم تطابق مدل      & $-0.84$ & $0.58$ & $-2.98$ & $0.51$ & $-1.97$ \\
				مشاهده ناقص        & $-0.88$ & $0.36$ & $-3.65$ & $0.23$ & $-1.92$ \\
				\bottomrule
			\end{tabular}
			\caption*{\normalfont پاداش تجمعی}
		}
		\hspace{0.04\linewidth}
		\parbox{.48\linewidth}{
			\centering
			\footnotesize
			\begin{tabular}{@{}l*{5}{c}}
				\toprule
			 \lr{DDPG} & \lr{PPO} & \lr{SAC} & \lr{TD3} & \lr{PID} \\
				\midrule
		 $329.59$ & $255.94$ & $806.11$ & $72.03$ & $181.38$ \\
	 $373.66$ & $257.85$ & $791.08$ & $76.55$ & $273.42$ \\
		 $1087.24$ & $306.32$ & $1712.44$ & $109.01$ & $976.93$ \\
			 $817.56$ & $334.29$ & $1547.22$ & $177.19$ & $103.00$ \\
				\bottomrule
			\end{tabular}
			\caption*{\normalfont مجموع خطای مسیر}
		}%
	}
	
	\vspace{0.6em}
	
	\makebox[\linewidth][c]{%
		\parbox{.48\linewidth}{
			\centering
			\footnotesize
			\begin{tabular}{@{}l*{5}{c}}
				\toprule
				{سناریو} & \lr{DDPG} & \lr{PPO} & \lr{SAC} & \lr{TD3} & \lr{PID} \\
				\midrule
				شرایط اولیه تصادفی & $5.11$ & $0.77$ & $1.76$ & $3.31$ & $81802.96$ \\
				اغتشاش در عملگرها & $4.89$ & $0.77$ & $1.71$ & $3.07$ & $5.65$ \\
				عدم تطابق مدل      & $5.48$ & $0.86$ & $2.37$ & $4.32$ & $479167.85$ \\
				مشاهده ناقص        & $5.37$ & $1.03$ & $2.33$ & $4.10$ & $489.92$ \\
				\bottomrule
			\end{tabular}
			\caption*{\normalfont مجموع تلاش کنترلی}
		}
		\hspace{0.04\linewidth}
		\parbox{.48\linewidth}{
			\centering
			\footnotesize
			\begin{tabular}{@{}l*{5}{c}}
				\toprule
		\lr{DDPG} & \lr{PPO} & \lr{SAC} & \lr{TD3} & \lr{PID} \\
				\midrule
		$0.00$ & $0.00$ & $0.00$ & $0.00$ & $1.00$ \\
				$0.00$ & $0.00$ & $0.00$ & $0.00$ & $0.00$ \\
			 $0.00$ & $0.00$ & $1.00$ & $0.00$ & $1.00$ \\
		 $0.00$ & $0.00$ & $1.00$ & $0.00$ & $1.00$ \\
				\bottomrule
			\end{tabular}
			\caption*{\normalfont احتمال شکست}
		}%
	}
	
	\caption{مقایسه سناریوهای مقاومتی نسخه استاندارد با ستون \lr{PID}.}
	\label{tab:std_pid_grid}
\end{table}

\noindent داده‌های جدول \ref{tab:std_pid_grid} نشان می‌دهد که اگرچه \lr{PID} در برخی سناریوها می‌تواند پایداری اولیه ایجاد کند، اما بازده آن نسبت به یادگیری تقویتی به‌ویژه از نظر پاداش مثبت، خطای مسیر و تلاش کنترلی چندین مرتبه بدتر است؛ برای نمونه در عدم تطابق مدل، خطای \lr{PID} حدود دو مرتبه‌بزرگ‌تر از حتی ضعیف‌ترین عامل \lr{RL} است و در مشاهده ناقص، \lr{TD3} با پاداش $0.23$ و خطای $177.19$ به‌مراتب متوازن‌تر از \lr{PID} با پاداش منفی و تلاش $489.92$ عمل می‌کند.

بر اساس داده‌ها، \lr{TD3} به‌طور پایدار بالاترین پاداش و کمترین خطای مسیر را ثبت می‌کند، درحالی‌که \lr{PPO} کمترین تلاش کنترلی را دارد. \lr{SAC} در برخی سناریوهای دشوار (عدم تطابق مدل، مشاهده ناقص، نویز حسگر، تأخیر زمانی) نرخ شکست بالاتری نشان می‌دهد و \lr{DDPG} عموماً از نظر پاداش و خطا ضعیف‌تر از \lr{PPO} و \lr{TD3} است.

\subsubsection*{ تحلیل نتایج}
\begin{itemize}
	\item \lr{TD3} در نسخه تک‌عاملی نیز پاداش‌های مثبت میان $0.23$ تا $0.77$ و خطای مسیر محدود به $0.72$ تا $1.77$ دارد که آن را پایدارترین گزینه از نظر پیگیری مسیر می‌کند.
	\item \lr{PPO} با مجموع تلاش کنترلی در بازه $0.76$ تا $1.03$ بهره‌ورترین مصرف سوخت را ارائه می‌دهد، در حالی‌که پاداش‌های $0.36$ تا $0.61$ آن نشان می‌دهد تعادل کارآمدی میان کارایی و هزینه برقرار شده است.
	\item \lr{SAC} اگرچه در برخی سناریوها پاداش‌های نزدیک به صفر (مثلاً $-0.72$) دارد، نرخ شکست $1.00$ در چهار سناریو و خطای مسیر تا $17.12$ اهمیت بهبود روش‌های سهیم‌سازی اطلاعات را برجسته می‌کند.
	\item \lr{DDPG} با خطای مسیرهای بین $3.30$ تا $10.87$ و پاداش‌های منفی تا $-0.88$ نسبت به \lr{PPO} و \lr{TD3} عملکرد ضعیف‌تری دارد و در جمع‌بندی پایان‌نامه تنها به عنوان پایه مقایسه مطرح می‌شود.
	\item کنترل‌کننده \lr{PID} در هر چهار سناریو یا پاداش منفی، یا خطای بسیار بزرگ و تلاش کنترلی سنگین‌تری نسبت به \lr{PPO} و \lr{TD3} دارد؛ بنابراین رویکردهای یادگیری تقویتی نه‌تنها هزینه سوخت کمتری ثبت می‌کنند بلکه پایداری مسیر را نیز با حاشیه امن بیشتری تضمین می‌کنند.
\end{itemize}

 
\begin{figure}[H]
	\centering
	
	% سطر اول
	\subfloat[شرایط اولیه تصادفی]{\includegraphics[width=.33\textwidth]{plots/ZeroSum/violin_plot/initial_condition_shift.pdf}}%
	\subfloat[اغتشاش در عملگرها]{\includegraphics[width=.33\textwidth]{plots/ZeroSum/violin_plot/actuator_disturbance.pdf}}%
	\subfloat[عدم تطابق مدل]{\includegraphics[width=.33\textwidth]{plots/ZeroSum/violin_plot/model_mismatch.pdf}}\\[1ex]
	
	% سطر دوم
	\subfloat[مشاهده ناقص]{\includegraphics[width=.33\textwidth]{plots/ZeroSum/violin_plot/partial_observation.pdf}}%
	\subfloat[نویز حسگر]{\includegraphics[width=.33\textwidth]{plots/ZeroSum/violin_plot/sensor_noise.pdf}}%
	\subfloat[تأخیر زمانی]{\includegraphics[width=.33\textwidth]{plots/ZeroSum/violin_plot/time_delay.pdf}}
	
	\caption{مقایسه مجموع پاداش دو الگوریتم تک‌عاملی و چندعاملی \lr{ZS} در سناریوهای مختلف. 
		نسخه بازی مجموع‌صفر در اکثر سناریوها، به خصوص در شرایط اغتشاش در عملگرها و عدم تطابق مدل، عملکرد بهتری را نشان می‌دهد.
	}
	\label{fig:zs_robustness_violin}
\end{figure}




\begin{table}[!t]
	\centering
	\setlength{\tabcolsep}{6pt}      % tighter columns (built-in)
	\renewcommand{\arraystretch}{1.5}% a bit more row height (built-in)
	\scriptsize                      % smaller font to fit
	
	% --- Row 1: left + right boxes, centered as a group ---
	\makebox[\linewidth][c]{%
		\parbox{.48\linewidth}{
			\centering
			\footnotesize
			\begin{tabular}{@{} R {2.6cm}*{4}{c}}
				\toprule
				{سناریو} & \lr{DDPG} & \lr{PPO} & \lr{SAC} & \lr{TD3} \\
				\midrule
				شرایط اولیه تصادفی & $-0.41$ & $0.34$ & $-0.02$ & ${0.74}$ \\
				اغتشاش در عملگرها & $-0.44$ & $0.35$ & $-0.02$ & ${0.73}$ \\
				عدم تطابق مدل      & $-0.63$ & $0.38$ & $-0.13$ & ${0.75}$ \\
				مشاهده ناقص        & $-1.52$ & $0.40$ & $-0.44$ & ${0.71}$ \\
				نویز حسگر          & $-0.60$ & $0.37$ & $-0.12$ & ${0.75}$ \\
				تأخیر زمانی        & $-1.19$ & $0.17$ & $-0.05$ & ${0.67}$ \\
				\bottomrule
			\end{tabular}
			\caption*{\normalfont پاداش تجمعی}
		}
		\hspace{0.04\linewidth}
		\parbox{.48\linewidth}{
			\centering
			\footnotesize
			\begin{tabular}{@{} R {2.6cm}*{4}{c}}
				\toprule
				{سناریو} & \lr{DDPG} & \lr{PPO} & \lr{SAC} & \lr{TD3} \\
				\midrule
				شرایط اولیه تصادفی & $4.42$ & $4.30$ & $4.02$ & ${1.22}$ \\
				اغتشاش در عملگرها & $4.39$ & $4.38$ & $4.01$ & ${1.26}$ \\
				عدم تطابق مدل      & $8.85$ & $3.57$ & $4.78$ & ${1.25}$ \\
				مشاهده ناقص        & $9.65$ & $2.44$ & $5.17$ & ${1.09}$ \\
				نویز حسگر          & $9.12$ & $3.58$ & $4.66$ & ${1.25}$ \\
				تأخیر زمانی        & $6.73$ & $4.53$ & $4.12$ & ${1.21}$ \\
				\bottomrule
			\end{tabular}
			\caption*{\normalfont مجموع خطای مسیر}
		}%
	}
	
	\vspace{0.6em}
	
	% --- Row 2: left + right boxes, centered as a group ---
	\makebox[\linewidth][c]{%
		\parbox{.48\linewidth}{
			\centering
			\footnotesize
			\begin{tabular}{@{} R {2.6cm}*{4}{c}}
				\toprule
				{سناریو} & \lr{DDPG} & \lr{PPO} & \lr{SAC} & \lr{TD3} \\
				\midrule
				شرایط اولیه تصادفی & $5.11$ & ${0.77}$ & $1.76$ & $3.31$ \\
				اغتشاش در عملگرها & $4.89$ & ${0.77}$ & $1.71$ & $3.07$ \\
				عدم تطابق مدل      & $5.48$ & ${0.86}$ & $2.37$ & $4.32$ \\
				مشاهده ناقص        & $5.37$ & ${1.03}$ & $2.33$ & $4.10$ \\
				نویز حسگر          & $5.48$ & ${0.86}$ & $2.37$ & $4.30$ \\
				تأخیر زمانی        & $5.51$ & ${0.76}$ & $2.11$ & $5.12$ \\
				\bottomrule
			\end{tabular}
			\caption*{\normalfont مجموع تلاش کنترلی}
		}
		\hspace{0.04\linewidth}
		\parbox{.48\linewidth}{
			\centering
			\footnotesize
			\begin{tabular}{@{} R {2.6cm}*{4}{c}}
				\toprule
				{سناریو} & \lr{DDPG} & \lr{PPO} & \lr{SAC} & \lr{TD3} \\
				\midrule
				شرایط اولیه تصادفی & $0.00$ & $0.00$ & $0.00$ & $0.00$ \\
				اغتشاش در عملگرها & $0.00$ & $0.00$ & $0.00$ & $0.00$ \\
				عدم تطابق مدل      & $0.00$ & $0.00$ & $1.00$ & $0.00$ \\
				مشاهده ناقص        & $0.00$ & $0.00$ & $1.00$ & $0.00$ \\
				نویز حسگر          & $0.00$ & $0.00$ & $1.00$ & $0.00$ \\
				تأخیر زمانی        & $0.00$ & $0.00$ & $1.00$ & $0.00$ \\
				\bottomrule
			\end{tabular}
			\caption*{\normalfont احتمال شکست}
		}%
	}
	
	\caption{مقایسه الگوریتم‌های چندعاملی در سناریوهای مختلف مقاومت}
\end{table}





























