\section{عامل گرادیان سیاست عمیق قطعی}\label{sec:DDPG}

گرادیان سیاست عمیق قطعی\LTRfootnote{Deep Deterministic Policy Gradient (DDPG)}
الگوریتمی است که همزمان یک تابع \lr{Q} و یک سیاست را یاد می‌گیرد. این الگوريتم برای یادگیری تابع \lr{Q} از داده‌های غیرسیاست محور\LTRfootnote{Off-Policy}
 و معادله بلمن استفاده می‌کند. این الگوریتم برای یادگیری سیاست نیز از تابع \lr{Q} استفاده می‌کند.
 
 این رویکرد وابستگی نزدیکی به یادگیری \lr{Q} دارد. اگر تابع ارزش-عمل بهینه 
% \lr{(optimal action-value function)}
 مشخص باشد، در هر حالت داده‌شده عمل بهینه
%  \lr{(optimal action)} 
  را می‌توان با حل
 معادله
 \eqref{eq:optimal_action_value_function}
  به دست آورد.
\begin{equation}
	\label{eq:optimal_action_value_function}
	a^*(s) = \arg \max_a Q^*(s,a)
\end{equation}
الگوریتم
\lr{DDPG}
ترکیبی از یادگیری تقریبی برای $ Q^*(s,a)$ و یادگیری تقریبی برای
 $a^*(s)$ 
 است و به‌صورتی طراحی شده‌است که برای محیط‌هایی با فضاهای عمل پیوسته مناسب باشد. 
% اما چه معنی دارد که DDPG به طور خاص برای محیط‌هایی با فضاهای عمل پیوسته مناسب است؟ 
% این به روش محاسبه بیشینه عمل‌ها در
%  $\max_a Q^*(s,a)$
%   مربوط است.
%    در این الگوریتم
%    آن را برای فضای پیوسته مناسب می‌کند.
    آنچه این الگوریتم را برای فضای عمل پیوسته مناسب می‌کند، 
       روش محاسبه 
    $a^*(s)$
    است.
    فرض می‌شود که تابع
     $Q^*(s,a)$
      نسبت به آرگومان عمل مشتق‌پذیر است. مشتق‌پذیری این امکان را می‌دهد که یک روش یادگیری مبتنی بر گرادیان برای سیاست
       $\mu(s)$
        استفاده شود. سپس، به‌جای اجرای یک بهینه‌سازی زمان‌بر در هر بار محاسبه
        $\max_a Q(s,a)$،
       می‌توان آن را با رابطه
        \(
        \max_a Q(s,a) \approx Q(s,\mu(s))
        \)
         تقریب زد.
         
         
\subsection{ یادگیری
\lr{Q}
در \lr{DDPG}
}
معادله بلمن که تابع ارزش عمل بهینه
 $(Q^*(s,a))$
  را توصیف می‌کند، در پایین آورده شده‌است.
\begin{equation}
	Q^*(s,a) = r(s,a) +  \underset{s' \sim P}{{\mathrm E}}\left[ \gamma \max_{a'} Q^*(s', a')\right]
\end{equation}
عبارت 
$s' \sim P$ 
به این معنی است که وضعیت بعدی یعنی $s'$ از توزیع احتمال
 $P(\cdot| s,a)$
 نمونه ‌گرفته می‌شود.
 در معادله بلمن نقطه شروع برای یادگیری
  $Q^*(s,a)$
 یک مقداردهی تقریبی
   است. پارامترهای شبکه عصبی
    $Q_{\phi}(s,a)$
    با علامت
    $\phi$ 
    نشان داده شده‌است.
    مجموعه
     ${\mathcal D}$
      شامل اطلاعات  جمع‌آوری شده تغییر از یک حالت به حالت دیگر
       $(s,a,r,s',d)$
       (که $d$ نشان می‌دهد که آیا وضعیت $s'$ پایانی است یا خیر) است.
       در بهینه‌سازی از تابع خطای میانگین مربعات بلمن\LTRfootnote{Mean Squared Bellman Error}
        (\lr{MSBE}) استفاده شده‌است که معیاری برای نزدیکی
         $Q_{\phi}$
         به حالت بهینه
          برای برآورده کردن معادله بلمن است.
          
          \begin{equation}
          	L(\phi, {\mathcal D}) = \underset{(s,a,r,s',d) \sim {\mathcal D}}{{\mathrm E}}\left[
          	\Bigg( Q_{\phi}(s,a) - \left(r + \gamma (1 - d) \max_{a'} Q_{\phi}(s',a') \right) \Bigg)^2
          	\right]
          \end{equation}
          
          
          
          
          در الگوریتم \lr{DDPG} دو ترفند برای عمکرد بهتر استفاده شده‌است که در ادامه به بررسی آن پرداخته شده‌است.
          \begin{itemize}
          	\item بافرهای تکرار بازی
          	
          	 الگوریتم‌های یادگیری تقویتی جهت آموزش یک شبکه عصبی عمیق برای تقریب 
          	 $Q^*(s,a)$
          	  از  بافرهای تکرار بازی\LTRfootnote{Replay Buffers}
          	   تجربه‌شده استفاده می‌کنند. 
          	           این مجموعه 
          	   ${\mathcal D}$
          	   شامل تجربیات قبلی عامل است. برای داشتن رفتار پایدار در الگوریتم، بافر تکرار بازی باید به اندازه کافی بزرگ باشد تا شامل یک دامنه گسترده از تجربیات شود. انتخاب داده‌های بافر به دقت انجام شده‌است چرا که اگر فقط از داده‌های بسیار جدید استفاده شود، بیش‌برازش\LTRfootnote{Overfit}
          	   رخ می‌دهید و اگر از تجربه بیش از حد استفاده شود، ممکن است فرآیند یادگیری کند شود.
          	\item شبکه‌های هدف
          	
          	الگوریتم‌های یادگیری \lr{Q} از شبکه‌های هدف استفاده می‌کنند. اصطلاح زیر به‌عنوان هدف شناخته می‌شود.
          	\begin{equation}\label{eq:target}
          		r + \gamma (1 - d) \max_{a'} Q_{\phi}(s',a')
          	\end{equation}
          	در هنگام کمینه کردن تابع خطای میانگین مربعات بلمن، سعی شده‌است تا تابع \lr{Q} شبیه‌تر به هدف یعنی رابطه \eqref{eq:target} شود. اما مشکل این است که هدف بستگی به پارامترهای در حال آموزش $\phi$ دارد.
 این باعث ایجاد ناپایداری در کمینه کردن تابع خطای میانگین مربعات بلمن می‌شود. راه حل آن استفاده از یک مجموعه پارامترهایی است که با تأخیر زمانی به 
          	$\phi$
          	 نزدیک می‌شوند. به عبارت دیگر، یک شبکه دوم  ایجاد می‌شود که به آن شبکه هدف گفته می‌شود. شبکه هدف 
          	  پارامترهای شبکه اول را با تاخیر دنبال می‌کند. پارامترهای شبکه هدف با نشان 
          	​$\phi_{\text{targ}}$
          	نشان داده می‌شوند.
          	در الگوریتم \lr{DDPG}، شبکه هدف در هر به‌روزرسانی شبکه اصلی، با میانگین‌گیری پولیاک\LTRfootnote{Polyak Averaging}
       به‌صورت زیر به‌روزرسانی می‌شود.
          	 \begin{equation}
          	 	\phi_{\text{targ}} \leftarrow \rho \phi_{\text{targ}} + (1 - \rho) \phi
          	 \end{equation}
          	 در رابطه بالا $\rho$ یک ابرپارامتر\LTRfootnote{Hyperparameter} است که بین صفر و یک انتخاب می‌شود. در این پژوهش این مقدار نزدیک به یک درنظر گرفته شده‌است.
          \end{itemize}
          
          
          الگوریتم
          \lr{DDPG}
           نیاز به یک شبکه سیاست هدف
           $(\mu_{\theta_{\text{targ}}})$
            برای محاسبه عمل‌هایی که به‌طور تقریبی بیشینه
            $Q_{\phi_{\text{targ}}}$
             را حاصل کند، را دارد. برای رسیدن به این شبکه سیاست هدف
              از همان روشی که تابع \lr{Q} به دست می‌آید یعنی با میانگین‌گیری پولیاک از پارامترهای سیاست در طول زمان آموزش استفاده می‌شود.
          
          
          با درنظرگرفتن موارد اشاره‌شده، یادگیری \lr{Q} در \lr{DDPG} با کمینه‌کردن تابع خطای میانگین مربعات بلمن (\lr{MSBE}) یعنی معادله \eqref{eq:msbe} با استفاده از کاهش گرادیان تصادفی\LTRfootnote{Stochastic Gradient Descent}
           انجام می‌شود.
          \begin{equation}\label{eq:msbe}
          	L(\phi, {\mathcal D}) = \underset{(s,a,r,s',d) \sim {\mathcal D}}{{\mathrm E}}\left[
          	\Bigg( Q_{\phi}(s,a) - \left(r + \gamma (1 - d) Q_{\phi_{\text{targ}}}(s', \mu_{\theta_{\text{targ}}}(s')) \right) \Bigg)^2
          	\right]
          \end{equation}
          
          
          \subsection{ سیاست
          	در \lr{DDPG}
          }
  در این بخش یک سیاست تعیین‌شده 
  $\mu_{\theta}(s)$
   یاد گرفته می‌شود تا عملی را انجام می‌دهد که بیشینه 
   $Q_{\phi}(s,a)$
    رخ دهد. از آنجا که فضای عمل پیوسته است و فرض شده‌است که تابع \lr{Q} نسبت به عمل مشتق‌پذیر است، رابطه زیر با استفاده از صعود گرادیان\LTRfootnote{Gradient Ascent}
     (تنها نسبت به پارامترهای سیاست) بیشینه می‌شود.
     \begin{equation}
     	\max_{\theta} \underset{s \sim {\mathcal D}}{{\mathrm E}}\left[ Q_{\phi}(s, \mu_{\theta}(s)) \right]
     \end{equation}
     
     
     
     \subsection{اکتشاف و بهره‌برداری در 
     \lr{DDPG}
     }
       برای بهبود اکتشاف\LTRfootnote{Exploration}
        در سیاست‌های \lr{DDPG}، در زمان آموزش نویز به عمل‌ها اضافه می‌شود. نویسندگان مقاله
          \lr{DDPG}
          \cite{silver2014deterministic}
           توصیه کرده‌اند که نویز 
        \lr{OU}\LTRfootnote{Ornstein–Uhlenbeck} 
        با هم‌بندی زمانی\LTRfootnote{Time-Correlated} اضافه شود.
%         اما نتایج به‌روزتر نشان می‌دهد که نویز گوسی بدون هم‌ارتباط\LTRfootnote{Uncorrelated} و میانگین صفر کاملاً موثر عمل می‌کند. از آنجا که نویز گوسی با میانگین صفر ساده‌تر است، در این پژوهش از این روش استفاده شده‌است.
        در زمان بهره‌برداری\LTRfootnote{Exploitation} سیاست، از آنچه یاد گرفته است، نویز به عمل‌ها اضافه نمی‌شود.
        
        
        \subsection{شبه‌کد
        \lr{DDPG}}
        
       در این بخش، شبه‌کد الگوریتم
       \lr{DDPG}
        پیاده‌سازی‌شده آورده شده‌است. در این پژوهش الگوریتم~\رجوع{alg:DDPG} در محیط پایتون با استفاده از کتابخانه
         \lr{TensorFlow} \cite{tensorflow2015-whitepaper}
          پیاده‌سازی شده‌است.
        
%        \شروع{الگوریتم}{گرادیان سیاست عمیق قطعی}
%        \ورودی پارامترهای اولیه سیاست
%         $(\theta)$، پارامترهای تابع
%         \lr{Q}
%         $(\phi)$، بافر بازی خالی $(\mathcal{D})$
%%        \خروجی یک پوشش رأسی از $G$
%\State پارامترهای هدف را برابر با پارامترهای اصلی قرار دهید
%$\theta_{\text{targ}} \leftarrow \theta$, $\phi_{\text{targ}} \leftarrow \phi$
%        
%%        \دستور قرار بده $C = \emptyset$  % \توضیحات{مقداردهی اولیه}
%\While{همگرایی رخ دهد}
%\State 
%وضعیت $(s)$ را مشاهده کرده و عمل $a = \text{clip}(\mu_{\theta}(s) + \epsilon, a_{\text{Low}}, a_{\text{High}})$ را انتخاب کنید، به‌طوری که $\epsilon \sim \mathcal{N}$ است.
%\State عمل $a$ را در محیط اجرا کنید.
%\State وضعیت بعدی $s'$، پاداش $r$ و سیگنال پایان $d$ را مشاهده کنید تا نشان دهد آیا $s'$ پایانی است یا خیر.
%\State اگر $s'$ پایانی است، وضعیت محیط را بازنشانی کنید.
%\If{زمان به‌روزرسانی فرا رسیده است}
%\For{هر تعداد به‌روزرسانی}
%\State یک دسته تصادفی گذر از ‌یک حالت به حالت دیگر، $B = \{ (s,a,r,s',d) \}$، از $\mathcal{D}$ نمونه‌گیری شود.
%\State
% اهداف را محاسبه کنید:
%\[ y(r,s',d) = r + \gamma (1-d) Q_{\phi_{\text{targ}}}(s', \mu_{\theta_{\text{targ}}}(s')) \]
%\State تابع \lr{Q} را با یک مرحله از نزول گرادیان با استفاده از رابطه زیر به‌روزرسانی کنید:
%\[ \nabla_{\phi} \frac{1}{|B|}\sum_{(s,a,r,s',d) \in B} \left( Q_{\phi}(s,a) - y(r,s',d) \right)^2 \]
%\State سیاست را با یک مرحله از صعود گرادیان با استفاده از رابطه زیر به‌روزرسانی کنید:
%\[ \nabla_{\theta} \frac{1}{|B|}\sum_{s \in B}Q_{\phi}(s, \mu_{\theta}(s)) \]
%\State شبکه‌های هدف را با استفاده از معادلات زیر به‌روزرسانی کنید:
%\begin{align*}
%	\phi_{\text{targ}} &\leftarrow \rho \phi_{\text{targ}} + (1-\rho) \phi \\
%	\theta_{\text{targ}} &\leftarrow \rho \theta_{\text{targ}} + (1-\rho) \theta
%\end{align*}
%\EndFor
%\EndIf
%\EndWhile
%%        \تاوقتی{$E$ تهی نیست}
%%        %\اگر{$|E| > 0$}
%%        %	\دستور{یک کاری انجام بده}
%%        %\پایان‌اگر
%%        \If{$i\geq 5$} 
%%        \State $i \gets i-1$ \do
%%        \Else
%%        \If{$i\leq 3$}
%%        \State $i \gets i+2$
%%        \EndIf
%%        \EndIf
%%        \دستور یال دل‌‌خواه $uv \in E$ را انتخاب کن
%%        \دستور رأس‌های $u$ و $v$ را به $C$ اضافه کن
%%        \دستور تمام یال‌های واقع بر $u$ یا $v$ را از $E$ حذف کن
%%        \پایان‌تاوقتی
%%        \دستور $C$ را برگردان
%        \پایان{الگوریتم}
        
        
%         \STATE Update target networks with \begin{align*} \phi_{\text{targ}} &\leftarrow \rho \phi_{\text{targ}} + (1-\rho) \phi \\ \theta_{\text{targ}} &\leftarrow \rho \theta_{\text{targ}} + (1-\rho) \theta \end{align*} \ENDFOR \ENDIF \UNTIL{convergence} \end{algorithmic} \end{algorithm}
%        
        
        
        \begin{algorithm}[H]
        	\caption{گرادیان سیاست عمیق قطعی}\label{alg:DDPG}
        	\begin{algorithmic}[1]
  \ورودی پارامترهای اولیه سیاست
$(\theta)$، پارامترهای تابع
\lr{Q}
$(\phi)$، بافر تکرار بازی خالی $(\mathcal{D})$
%        \خروجی یک پوشش رأسی از $G$
\State پارامترهای هدف را برابر با پارامترهای اصلی قرار دهید
$\theta_{\text{targ}} \leftarrow \theta$, $\phi_{\text{targ}} \leftarrow \phi$

%        \دستور قرار بده $C = \emptyset$  % \توضیحات{مقداردهی اولیه}
\While{همگرایی رخ دهد}
%\State 
%وضعیت 
%\(s\)
% را مشاهده کرده و عمل $a = \text{clip}(\mu_{\theta}(s) + \epsilon, a_{\text{Low}}, a_{\text{High}})$ را انتخاب کنید به‌طوری که $\epsilon \sim \mathcal{N}$ است.
   \State \parbox[t]{\dimexpr\linewidth-\algorithmicindent}{
 	وضعیت 
 	\(s\)
 	را مشاهده کرده و عمل $a = \text{clip}(\mu_{\theta}(s) + \epsilon, a_{\text{Low}}, a_{\text{High}})$ را انتخاب کنید به‌طوری که $\epsilon \sim \mathcal{N}$ است.
 	\strut}
\State عمل $a$ را در محیط اجرا کنید.
%\State
% وضعیت بعدی $s'$، پاداش $r$ و سیگنال پایان $d$ را مشاهده کنید تا نشان دهد آیا $s'$ پایانی است یا خیر.
  \State \parbox[t]{\dimexpr\linewidth-\algorithmicindent}{
	 وضعیت بعدی $s'$، پاداش $r$ و سیگنال پایان $d$ را مشاهده کنید تا نشان دهد آیا $s'$ پایانی است یا خیر.
	\strut}
\State اگر $s'$ پایانی است، وضعیت محیط را بازنشانی کنید.
\If{زمان به‌روزرسانی فرا رسیده است}
\For{هر تعداد به‌روزرسانی}
\State یک دسته تصادفی گذر از ‌یک حالت به حالت دیگر، $B = \{ (s,a,r,s',d) \}$، از $\mathcal{D}$ نمونه‌گیری شود.
\State
هدف را محاسبه کنید:
\[ y(r,s',d) = r + \gamma (1-d) Q_{\phi_{\text{targ}}}(s', \mu_{\theta_{\text{targ}}}(s')) \]
\State تابع \lr{Q} را با یک مرحله از نزول گرادیان با استفاده از رابطه زیر به‌روزرسانی کنید:
\[ \nabla_{\phi} \frac{1}{|B|}\sum_{(s,a,r,s',d) \in B} \left( Q_{\phi}(s,a) - y(r,s',d) \right)^2 \]
\State سیاست را با یک مرحله از صعود گرادیان با استفاده از رابطه زیر به‌روزرسانی کنید:
\[ \nabla_{\theta} \frac{1}{|B|}\sum_{s \in B}Q_{\phi}(s, \mu_{\theta}(s)) \]
\State شبکه‌های هدف را با استفاده از معادلات زیر به‌روزرسانی کنید:
\begin{align*}
	\phi_{\text{targ}} &\leftarrow \rho \phi_{\text{targ}} + (1-\rho) \phi \\
	\theta_{\text{targ}} &\leftarrow \rho \theta_{\text{targ}} + (1-\rho) \theta
\end{align*}
\EndFor
\EndIf
\EndWhile
%        \تاوقتی{$E$ تهی نیست}
%        %\اگر{$|E| > 0$}
%        %	\دستور{یک کاری انجام بده}
%        %\پایان‌اگر
%        \If{$i\geq 5$} 
%        \State $i \gets i-1$ \do
%        \Else
%        \If{$i\leq 3$}
%        \State $i \gets i+2$
%        \EndIf
%        \EndIf
%        \دستور یال دل‌‌خواه $uv \in E$ را انتخاب کن
%        \دستور رأس‌های $u$ و $v$ را به $C$ اضافه کن
%        \دستور تمام یال‌های واقع بر $u$ یا $v$ را از $E$ حذف کن
%        \پایان‌تاوقتی
%        \دستور $C$ را برگردان
        	\end{algorithmic}
        \end{algorithm}
        
     


