\chapter{پیشینه پژوهش}

\section{ماموریت‌های بین مداری}
\section{بازی دیفزانسیلی}
\section{یادگیری تقویتی}


از نخستین صورت‌بندی‌های فرایند تصمیم‌گیری مارکُفی در یادگیری تقویتی، پژوهش بر آن بوده‌است که عامل بتواند با اجرای عمل‌ها و دریافت پاداش، سیاستی برای بیشینه‌سازی برگشت بیاموزد. تبیین جامع این چارچوب و الگوریتم‌های بنیادین در ویرایش دوم کتاب سوتون و بارتو به‌مثابه مرجع کلاسیک این حوزه ارائه شده و همچنان مبنای بسیاری از آثار معاصر است \cite{SuttonBarto2018}. % 


دهه‌‌ی ۱۹۹۰ ملادی شاهد شکل‌گیری روش‌هایی بر پایه‌ی ارزش\LTRfootnote{Value}
 نظیر \lr{Q-learning} و نخستین رویکردهای گرادیانِ سیاست بود؛ با وجود این، محدودیت توان محاسباتی و فقدان داده‌ی فراوان، سرعت رشد را کند می‌کرد. ورود شبکه‌های عصبی عمیق نقطه‌ی عطفی بود: مقاله‌ی معروف دیپ‌مایند\LTRfootnote{DeepMind}
  نشان داد که شبکه‌ی \lr{Q} عمیق (\lr{DQN}) می‌تواند صرفاً از پیکسل‌های بازی آتاری سیاستی نزدیک به انسان بیاموزد \cite{Mnih2015}. % 


موفقیت \lr{DQN} نگاه‌ها را به‌سوی گرادیانِ سیاستِ مقیاس‌پذیر معطوف ساخت. بهینه‌سازی ناحیه‌ی اطمینان\LTRfootnote{Trust Region Policy Optimization (TRPO)}
  تضمین بهبود یکنواخت سیاست را فراهم کرد \cite{Schulman2015TRPO}, و روش \lr{A3C} با موازی‌سازی بازیگران، سرعت یادگیری را چند برابر افزایش داد \cite{Mnih2016A3C}. % :contentReference[oaicite:2]{index=2}
کمی بعد، \lr{DDPG} اولین بار گرادیان سیاست قطعی را به فضاهای عمل پیوسته وارد کرد \cite{lillicrap2019continuouscontroldeepreinforcement}. % 
سپس \lr{PPO} با ساده‌سازی قیود \lr{TRPO} و کاهش پارامترهای حساس، به انتخاب پیش‌فرض بسیاری از کاربردهای مهندسی بدل شد \cite{Schulman2017PPO}. % 


با گسترش دامنه‌ی مسائل، پایداری و کاراییِ داده به چالش اصلی بدل گشت. \lr{TD3} نشان داد که کمینه‌کردن میان دو منتقد می‌تواند برآورد بیش‌از‌حد \lr{Q} را مهار کند \cite{Fujimoto2018TD3}, و \lr{SAC} با افزودن بند آنتروپی، هم‌زمان اکتشاف و بازده را بهبود داد \cite{Haarnoja2018SAC}. % :contentReference[oaicite:5]{index=5}


%برای کاهش هزینه‌ی تعامل، موج نوینی از یادگیری تقویتی مدل‌مبنا شکل گرفت. منابع نظام‌مند از این خط پژوهش را بررسی جامع مورلاند و همکاران عرضه می‌کند \cite{Moerland2020MBRLsurvey}، درحالی‌که عامل Dreamer نشان داد می‌توان سیاست را تنها با «تخیل نهفته» در مدل آموخته‌شده به‌روزرسانی کرد \cite{Hafner2019Dreamer}. % :contentReference[oaicite:6]{index=6}


در محیط‌های پرخطر یا گران، جمع‌آوری داده‌ی برخط ناممکن است؛ ازاین‌رو \lr{RL} آفلاین مطرح شد. روش \lr{CQL} با برقراری کران محافظه‌کارانه بر \lr{Q-value} از گرایشِ خارج از توزیع جلوگیری می‌کند \cite{Kumar2020CQL}، و مرور اخیر پراودِنسیو و همکاران طبقه‌بندی جامعی از چالش‌های باز این حوزه ارائه داده است \cite{Prudencio2022OfflineSurvey}. % 


هم‌زمان، دغدغه‌ی ایمنی و تبیین در سامانه‌های واقعی پررنگ شد. مرور سال ۲۰۲۲ نشان می‌دهد که ترکیب قیدهای سخت، توابع جریمه‌ی ریسک و شبیه‌سازی محیط‌های بدبینانه سه خط اصلی ایمنی در \lr{RL} هستند \cite{JMLR:v16:garcia15a}. سلسله‌مراتب نیز با هدف انتقال دانش و تسریع یادگیری مورد توجه قرار گرفت و یک مطالعه‌ی جامع در 
\lr{ACM Computing Surveys}
 چهار چالش کشف زیرکار، یادگیری اشتراک‌پذیر، انتقال و مقیاس‌پذیری را برجسته می‌کند \cite{Ghazalpour2021HRLsurvey}. % 


وقتی چند عامل به‌طور هم‌زمان یاد می‌گیرند، پویایی محیط از دید هر عامل غیرایستا می‌شود. مرور جامع ۲۰۲۴ نشان می‌دهد که چارچوب ناظر متمرکز ـ بازیگر توزیع‌شده\LTRfootnote{Centralized Training with Decentralized Execution (CTDE)}
 راهکاری موثر برای این چالش است و مباحثی چون تخصیص اعتبار جمعی و کشف تعادل را معرفی می‌کند \cite{Song2024MARLsurvey}. % 


پیشرفت‌های یادشده در نهایت به دستاوردهای نمادینی چون \lr{AlphaGo} \cite{Silver2016AlphaGo} و \lr{AlphaStar} \cite{Vinyals2019AlphaStar} انجامیدند که در بازی‌های \lr{Go} و \lr{StarCraft II} از انسان پیشی گرفتند، و معماری توزیع‌شده‌ی \lr{IMPALA} نشان داد که چگونه می‌توان هزاران شبیه‌ساز را با به‌روزرسانی وزن‌های مهم ادغام کرد \cite{Espeholt2018IMPALA}. % 


به‌رغم این جهش‌ها، سه شکاف اساسی پابرجا مانده است: ۱) تضمین ایمنی سخت‌گیرانه در سناریوهای نزدیک‌برخورد، ۲) کاهش وابستگی به داده‌ی پرهزینه یا نایاب از طریق روش‌های مدل‌مبنا و آفلاین، و ۳) مقیاس‌پذیری یادگیری چندعاملی برای سامانه‌های رباتیکی یا فضاپیمای چندگانه. 
%پژوهش حاضر در پی آن است که با تلفیق یادگیری تقویتی مقاوم و چندعاملی در چارچوب سه‌جسمی مداری، به این خلأ پاسخ دهد.

%\section{یادگیری تقویتی چندعاملی}
\section{پیشینهٔ پژوهش یادگیری تقویتی چندعاملی}\label{sec:marl_lit}

امروز یادگیری تقویتی چندعاملی (MARL) به‌عنوان ستون فقرات سامانه‌های هوشمند مشارکتی شناخته می‌شود؛ مسیری که از آزمون‌های سادهٔ دو‌عاملی در دههٔ ۱۹۹۰ آغاز شد و اکنون به معماری‌های توزیع‌شده‌ی در مقیاس هزاران بازیگر رسیده است. این بخش، بدون ورود به زیرشاخه‌های ریز، روایت تاریخیِ پیوسته‌ای ارائه می‌کند که نشان می‌دهد چگونه ایدهٔ \emph{آموزش متمرکز ـ اجرای توزیع‌شده} (CTDE) به پاسخ غالب برای چالش‌های غیرایستایی و انفجار بُعدی بدل شد و چه گام‌هایی هنوز برای ایمنی، ناهمگونی و مقیاس‌پذیری باقی مانده است.

دههٔ ۱۹۹۰ با مقالهٔ \cite{Tan1993} آغاز شد؛ جایی که برای نخستین‌بار مقایسهٔ «عامل‌های مستقل» با «عامل‌های همکار» انجام شد و سود ارتباط و اشتراک تجربه به‌صورت تجربی نشان داده شد. در میانهٔ دههٔ بعد، مرور جامع پانایت و لوک \cite{Panait2005} چشم‌اندازی از مسائل تخصیص اعتبار و غیرایستایی ترسیم کرد و دو مکتب «یادگیری تیمی» و «یادگیری هم‌زمان» را صورت‌بندی نمود.  هم‌زمان، بوشونیـو و همکاران \cite{Busoniu2008} ادبیات متکثر MARL را در قالب اهداف «پایداری دینامیک یادگیری» و «انطباق با رفتار سایر عامل‌ها» جمع‌بندی کردند و راه را برای تحلیل‌های بازی‌محور هموار ساختند. 

ورود شبکه‌های عمیق در سال‌های ۲۰۱۶ـ۲۰۱۷ نقطهٔ عطف بعدی بود؛ «منتقد متمرکز ـ بازیگر توزیع‌شده» در \emph{MADDPG} \cite{Lowe2017} نشان داد که می‌توان از حالت سراسری در فاز آموزش بهره برد، اما سیاست نهایی را صرفاً بر اساس مشاهدات محلی اجرا کرد.  در همان سال، \emph{Value‐Decomposition Networks} یا VDN \cite{Sunehag2017} ایدهٔ تجزیهٔ خطی پاداش را برای تیم‌های کاملاً تعاونی مطرح کرد و راه را برای فاکتوربندی‌های پیش‌رفته گشود. 

۲۰۱۸ شاهد جهش مهمی با \emph{QMIX} بود؛ این روش با اعمال قید تک‌نوا بر ترکیب مقادیر منفرد، هم امکان بهینه‌سازی آف‌پالیسی را فراهم کرد و هم تضمین سازگاری سیاست‌های محلی با ارزش مشترک را برقرار ساخت \cite{Rashid2018}. 

سال ۲۰۱۹ به گسترش بسترهای آزمایش اختصاص یافت. چالش استاندارد \emph{SMAC} بر مبنای StarCraft II معرفی شد و معیار مشترکی برای مقایسهٔ الگوریتم‌ها مهیا کرد \cite{Samvelyan2019SMAC}.هم‌زمان، \emph{QTRAN} \cite{Son2019QTRAN} نشان داد که می‌توان بدون قید خطی یا تک‌نوا، تابع ارزش مشترک را به فضای قابل تجزیه تبدیل کرد. از سوی دیگر، \emph{MAVEN} با افزودن متغیر نهفتهٔ مشترک، کاوش هماهنگ و سلسله‌مراتبی را امکان‌پذیر ساخت \cite{Mahajan2019MAVEN}.  نقطهٔ اوج همان سال، سامانهٔ \emph{AlphaStar} بود که نشان داد ترکیب خودبازی و معماری توزیع‌شده می‌تواند به رتبهٔ گرندمسترِ انسان برساند \cite{Vinyals2019AlphaStar}. 

در ۲۰۲۰ مفهوم «نقش‌های در حال ظهور» با \emph{ROMA} \cite{Wang2020ROMA} معرفی شد تا عامل‌ها بر اساس شباهت رفتاری به‌طور خودکار خوشه‌بندی و اشتراک دانش کنند؛ رویکردی که در نقشه‌های پرتراکم SMAC برتری محسوسی نشان داد. :contentReference[oaicite:10]{index=10} پژوهش‌های متا در ۲۰۲۱، از مرور نظری زانگ و بشار \cite{Zhang2021Survey} تا بنچ‌مارک تطبیقی پاپوداکیس \emph{et al.}، شکاف‌های باقی‌مانده در تضمین همگرایی و مقیاس را فهرست کردند. 

آخرین موج مطالعات (۲۰۲۲ به بعد) بر ناهمگونی و ایمنی تمرکز دارد. مرور جامع \cite{Yu2022Heterogeneous} نشان می‌دهد که تفاوت در قابلیت‌ها و اطلاعات عامل‌ها، مسائلی نظیر تخصیص اعتبار و تعادل را پیچیده‌تر می‌سازد و به الگوریتم‌های سازگار با نقش‌های پویا نیاز دارد.

به‌طور خلاصه، مسیر تاریخی MARL از الگوهای مستقل دههٔ ۱۹۹۰ به سامانه‌های توزیع‌شدهٔ امروزی، همواره با سه دغدغهٔ اصلی هدایت شده است: کنترل انفجار بُعدی توابع ارزش، مقابله با غیرایستایی ناشی از یادگیری هم‌زمان، و انتقال مؤثر تجربه میان عامل‌ها. علی‌رغم پیشرفت‌های شتابان، تضمین ایمنی سخت‌گیرانه در محیط‌های شکست‌پذیر، مدیریت نقش‌های پویا در تیم‌های ناهمگون و کاهش نیاز به دادهٔ شبیه‌سازی پرهزینه همچنان چالش‌های باز باقی می‌مانند؛ چالش‌هایی که در این رساله با رویکرد ترکیبی مدل‌مبنا، مقاوم و چندعاملی پیگیری می‌شوند.





\subsection{ماموریت‌های بین‌مداری}
\subsubsection{تعریف ماموریت‌های بین‌مداری}
اصول طراحی و بهینه‌سازی ماموریت‌ها و چالش‌های ارتباطی و کنترلی.
\subsubsection{مدل‌سازی مسیرهای بین‌مداری}
رویکردهای عددی برای بهینه‌سازی مسیر و استفاده از یادگیری ماشینی در برنامه‌ریزی مسیر.
\subsubsection{ایمنی در ماموریت‌های بین‌مداری}
تضمین پایداری مسیرها در حضور عدم قطعیت و مدیریت ریسک برخورد.

\subsection{بازی دیفرانسیلی}
\subsubsection{اصول بازی دیفرانسیلی}
تعریف بازی دیفرانسیلی، مفاهیم پایه و کاربردهای آن در تعاملات چندعاملی.
\subsubsection{بازی دیفرانسیلی با جمع صفر}
مدل‌سازی مسائل با جمع صفر و کاربردهای آن در مسائل دفاعی و نظارتی.
\subsubsection{تکنیک‌های حل بازی‌های دیفرانسیلی}
روش‌های تحلیلی و استفاده از الگوریتم‌های یادگیری تقویتی.

\subsection{یادگیری تقویتی}
\subsubsection{مفاهیم یادگیری تقویتی}
سیاست، پاداش، و به‌روزرسانی ارزش‌ها.
\subsubsection{الگوریتم‌های پیشرفته یادگیری تقویتی}
الگوریتم‌های Monte Carlo، Temporal-Difference و Deep Q-Network.
\subsubsection{ایمنی در یادگیری تقویتی}
یادگیری ایمن، محدودیت‌ها و تضمین‌ها در سیاست‌های یادگیری.

\subsection{یادگیری تقویتی چندعاملی}
\subsubsection{تعریف و اهمیت یادگیری تقویتی چندعاملی}
مفهوم همکاری و رقابت در محیط‌های چندعاملی و اهمیت ایمنی.
\subsubsection{بازی‌های چندعاملی در یادگیری تقویتی}
بازی‌های جمع صفر، تعادل نش و کاربردهای آن‌ها.
\subsubsection{چالش‌های یادگیری تقویتی چندعاملی}
مدیریت تعارضات، تضمین پایداری و مشکلات همگرایی.
\subsubsection{ایمنی در یادگیری تقویتی چندعاملی}
تکنیک‌های تضمین ایمنی و کاربرد ایمنی در تعاملات حساس.
\subsubsection{الگوریتم‌های یادگیری تقویتی چندعاملی}
الگوریتم‌های همکاری مانند MADDPG و یادگیری توزیع‌شده با تضمین ایمنی.
\subsubsection{کاربرد MARL در مسائل نظری}
حل مسائل بازی‌های جمع صفر و مدل‌سازی مسائل رقابتی و همکاری.
\subsubsection{کاربرد MARL در ماموریت‌های فضایی}
هماهنگی میان ماهواره‌ها، تخصیص منابع و حل مسائل ترکیبی در کاوش سیارات.

\section{یادگیری تقویتی چندعاملی}

\subsection{تعریف و اهمیت یادگیری تقویتی چندعاملی}
\subsubsection{تعریف یادگیری تقویتی چندعاملی}
% محتوای مربوط به تعریف MARL

\subsubsection{اهمیت یادگیری تقویتی چندعاملی}
% محتوای مربوط به اهمیت MARL

\subsection{بازی‌های چندعاملی در یادگیری تقویتی}
\subsubsection{بازی‌های جمع صفر}
% توضیح درباره بازی‌های جمع صفر

\subsubsection{تعادل نش و کاربردهای آن}
% توضیح درباره تعادل نش

\subsection{چالش‌های یادگیری تقویتی چندعاملی}
\subsubsection{مدیریت تعارضات}
% توضیح درباره مدیریت تعارضات

\subsubsection{تضمین پایداری}
% توضیح درباره تضمین پایداری

\subsubsection{مشکلات همگرایی}
% توضیح درباره مشکلات همگرایی

\subsection{ایمنی در یادگیری تقویتی چندعاملی}
\subsubsection{تکنیک‌های تضمین ایمنی}
% توضیح درباره تکنیک‌های ایمنی

\subsubsection{کاربرد ایمنی در تعاملات حساس}
% توضیح درباره کاربرد ایمنی

\subsection{الگوریتم‌های یادگیری تقویتی چندعاملی}
\subsubsection{الگوریتم‌های همکاری مانند MADDPG}
% توضیح درباره MADDPG و الگوریتم‌های همکاری

\subsubsection{یادگیری توزیع‌شده با تضمین ایمنی}
% توضیح درباره یادگیری توزیع‌شده

\subsection{کاربرد MARL در مسائل نظری}
\subsubsection{حل مسائل بازی‌های جمع صفر}
% توضیح درباره حل بازی‌های جمع صفر

\subsubsection{مدل‌سازی مسائل رقابتی و همکاری}
% توضیح درباره مدل‌سازی مسائل

\subsection{کاربرد MARL در ماموریت‌های فضایی}
\subsubsection{هماهنگی میان ماهواره‌ها}
% توضیح درباره هماهنگی ماهواره‌ها

\subsubsection{تخصیص منابع}
% توضیح درباره تخصیص منابع

\subsubsection{حل مسائل ترکیبی در کاوش سیارات}
% توضیح درباره حل مسائل ترکیبی


\section{کاربرد مولتی‌اِجنت در بازی‌ها}

در این بخش به بررسی کاربردهای مولتی‌اِجنت در حوزه بازی‌ها پرداخته و نشان می‌دهیم که الگوریتم پیشنهادی ما نه تنها بهینه بلکه ایمن و قابل اعتماد است.

\section{کاربرد چند عاملی در بازی‌ها}

در این بخش به بررسی کاربردهای یادگیری تقویتی چند عاملی در حوزه بازی‌ها پرداخته و نشان می‌دهیم که الگوریتم پیشنهادی ما نه تنها بهینه بلکه ایمن و قابل اعتماد است.

\subsection{تعریف بازی‌های چند عاملی}
\subsubsection{مفاهیم پایه در بازی‌های چند عاملی}
بازی‌های چند عاملی شامل محیط‌هایی هستند که در آن چندین عامل به صورت همزمان و مستقل به تعامل می‌پردازند. این تعاملات می‌تواند شامل همکاری، رقابت یا ترکیبی از هر دو باشد. در چنین محیط‌هایی، هر عامل با هدف خود به حداکثر رساندن پاداش یا دستیابی به اهداف مشخص، استراتژی‌های خود را توسعه می‌دهد.

\subsubsection{نمونه‌های بازی‌های چند عاملی}
از جمله نمونه‌های معروف بازی‌های چند عاملی می‌توان به بازی‌های استراتژیک مانند \lr{StarCraft}، بازی‌های ورزشی چند نفره و بازی‌های تخته‌ای مانند شطرنج چند عاملی اشاره کرد. این بازی‌ها به دلیل پیچیدگی‌های بالای تعاملات میان عوامل، محیط‌های مناسبی برای آزمایش و ارزیابی الگوریتم‌های یادگیری تقویتی چند عاملی فراهم می‌کنند.

\subsection{الگوریتم بهینه در بازی‌های چند عاملی}
\subsubsection{معرفی الگوریتم پیشنهادی}
الگوریتم پیشنهادی ما بر اساس ترکیبی از یادگیری عمیق و تکنیک‌های بهینه‌سازی چند عاملی طراحی شده است. این الگوریتم با استفاده از شبکه‌های عصبی عمیق، استراتژی‌های بهینه برای هر عامل را در محیط‌های پیچیده بازی‌های چند عاملی یاد می‌گیرد. علاوه بر این، با استفاده از مکانیزم‌های تعاملی، هماهنگی و همکاری میان عوامل بهبود می‌یابد.

\subsubsection{بهینه بودن الگوریتم}
الگوریتم ما با هدف کاهش زمان همگرایی و افزایش کارایی در محیط‌های دینامیک بهینه شده است. با استفاده از تکنیک‌های پیشرفته مانند یادگیری انتقالی و تنظیم خودکار نرخ یادگیری، الگوریتم قادر است به سرعت به تعادل‌های مطلوب برسد و عملکرد بهینه‌ای در بازی‌ها ارائه دهد.

\subsection{ایمنی و قابلیت اطمینان الگوریتم در بازی‌ها}
\subsubsection{تضمین ایمنی در تعاملات میان عوامل}
ایمنی در تعاملات چند عاملی به معنای جلوگیری از رفتارهای غیرمنتظره و تضمین هماهنگی میان عوامل است. در الگوریتم ما، از مکانیزم‌های نظارتی و محدودکننده استفاده شده است که اطمینان حاصل می‌کنند تعاملات میان عوامل منجر به نتیجه‌ای نامطلوب نمی‌شود. این شامل محدود کردن فضای عملیاتی و اعمال قیود بر سیاست‌های یادگیری هر عامل می‌باشد.

\subsubsection{قابلیت اطمینان و مقاومت در برابر خطاها}
الگوریتم پیشنهادی با هدف افزایش قابلیت اطمینان و مقاومت در برابر خطاها طراحی شده است. با استفاده از تکنیک‌های افزونگی و یادگیری توزیع‌شده، الگوریتم قادر است در صورت بروز خطا یا نقص در برخی از عوامل، به عملکرد مطلوب خود ادامه دهد. این ویژگی‌ها الگوریتم را برای استفاده در محیط‌های حساس و پویا مناسب می‌سازد.

\subsection{مطالعات موردی و نتایج تجربی}
\subsubsection{پیاده‌سازی در بازی‌های مشخص}
الگوریتم ما در بازی‌های مختلفی مانند بازی استراتژیک \lr{StarCraft} و بازی‌های ورزشی چند نفره پیاده‌سازی شده است. در هر یک از این بازی‌ها، عملکرد الگوریتم ما با توجه به معیارهای بهینه بودن و ایمنی ارزیابی شده است.

\subsubsection{تحلیل عملکرد و مقایسه با الگوریتم‌های دیگر}
نتایج تجربی نشان می‌دهد که الگوریتم پیشنهادی ما نسبت به الگوریتم‌های موجود در زمینه یادگیری تقویتی چند عاملی از نظر سرعت همگرایی و کارایی بهتری دارد. علاوه بر این، با استفاده از مکانیزم‌های تضمین ایمنی، رفتارهای ناخواسته و خطرناک در تعاملات میان عوامل به طور قابل توجهی کاهش یافته است.

\subsection{نتیجه‌گیری}
در این بخش، به بررسی و تحلیل نتایج به دست آمده از پیاده‌سازی الگوریتم پیشنهادی در بازی‌های چند عاملی پرداخته شد. مشاهده شد که الگوریتم ما توانسته است با ارائه راه‌حل‌های بهینه و ایمن، عملکرد بهتری نسبت به الگوریتم‌های موجود داشته باشد. این امر نشان‌دهنده قابلیت‌های بالای الگوریتم در مدیریت تعاملات پیچیده و تضمین ایمنی در محیط‌های چند عاملی است.

\subsection{پیشنهادات برای تحقیقات آینده}
با توجه به نتایج حاصل شده، پیشنهاد می‌شود که تحقیقات آینده بر روی بهبود بیشتر ایمنی و قابلیت اطمینان الگوریتم‌های چند عاملی متمرکز شود. همچنین، توسعه الگوریتم‌های سازگار با محیط‌های پویا و پیچیده‌تر و ارزیابی آن‌ها در بازی‌ها و محیط‌های واقعی‌تر از جمله مسیرهای پیشنهادی برای ادامه تحقیقات می‌باشد.

