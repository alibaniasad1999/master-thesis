\chapter{پیشینه پژوهش}

\section{ماموریت‌های بین‌مداری}\label{sec:inter_orbital}
هدایت فضاپیماها معمولاً با استفاده از ایستگاه‌های زمینی
% هدایت، مسیریابی و کنترل\LTRfootnote{Guidance, Navigation and Control (GNC)}  
انجام می‌شود. با این حال، این تکنیک‌ها دارای محدودیت‌هایی از جمله حساسیت به قطع ارتباطات، تأخیرهای زمانی و محدودیت‌های منابع محاسباتی هستند. الگوریتم‌های یادگیری تقویتی و بازی‌های دیفرانسیلی می‌توانند برای بهبود قابلیت‌های هدایت فضاپیماها، از جمله مقاومت در برابر تغییرات محیطی، کاهش تأخیرهای ناشی از ارتباطات زمینی و افزایش کارایی محاسباتی، مورد استفاده قرار گیرند.


%در فرآیند طراحی مسیر، معمولاً یک مسیر بهینه و تاریخچه فرمان کنترلی طراحی می‌شود که با معیارهای ماموریت، مصرف سوخت و زمان پرواز، مطابقت داشته باشد. این روش قبل از پرواز انجام می‌شود و می‌تواند از استراتژی‌های متعددی برای هدایت بهینه با پیشران کم استفاده کند، از جمله تکنیک‌های بهینه‌سازی جهانی \cite{vavrina2017global} و برنامه‌نویسی غیرخطی \cite{ocampo2004finite}.
هدایت فضاپیماها معمولاً  پیش از پرواز انجام می‌شود. این روش‌ها می‌توانند از تکنیک‌های بهینه‌سازی فراگیر  \cite{vavrina2017global} یا برنامه‌نویسی غیرخطی برای تولید مسیرها و فرمان‌های کنترلی بهینه استفاده کنند. با این حال، این روش‌ها معمولاً حجم محاسباتی زیادی دارند و برای استفاده درون‌سفینه‌ای نامناسب هستند \cite{ocampo2004finite}.
%با این حال، توانایی سریع محاسبه مجدد مسیر مرجع و تاریخچه فرمان کنترلی در داخل فضاپیما در هنگام پرواز بسیار مهم است. با نگاه به هدایت درون‌سفینه‌ای از دیدگاه یادگیری ماشین، یک کنترل‌کننده شبکه عصبی حلقه بسته، امکان محاسبه سریع و خودکار تاریخچه کنترل را برای یک فضاپیما فراهم می‌کند. به علاوه، یادگیری تقویتی\LTRfootnote{Reinforcement Learning (RL)}
% یادگیری پیش از پرواز واقعی است و می‌تواند از سخت‌افزارهای سریع و قدرتمند و ارزان قیمت زمینی جهت یادگیری استفاده کند.
% هدایت فضاپیماها معمولاً با استفاده از روش‌های طراحی مسیر پیش‌پرواز انجام می‌شود. با این حال، این روش‌ها محاسباتی فشرده هستند و برای استفاده درون‌سفینه نامناسب هستند.
یادگیری ماشین می‌تواند برای بهبود قابلیت‌های هدایت فضاپیماها استفاده شود. کنترل‌کننده شبکه عصبی حلقه‌‌بسته می‌تواند برای محاسبه سریع و خودکار تاریخچه کنترل استفاده شود. یادگیری تقویتی نیز می‌تواند برای یادگیری رفتارهای هدایت بهینه استفاده شود.

روش‌های هدایت و بهینه‌سازی مسیر فضاپیماها به‌طور کلی به راه‌حل‌های اولیه مناسب نیاز دارند. در مسائل چند جسمی، طراحان مسیر اغلب حدس‌های اولیه کم‌هزینه‌ای برای انتقال‌ها با استفاده از نظریه سیستم‌های دینامیکی و منیفولدهای ثابت
\cite{2013AcAau, haapala2016framework}
ایجاد می‌کنند. 
%روش‌های مبتنی بر سیستم‌های دینامیکی در بسیاری از کاربردهای گذشته مفید بوده‌اند و در ترکیب با اصلاحات دیفرانسیل و یا تکنیک‌های بهینه‌سازی، برای بسیاری از کاربردها راه‌حل‌های بهینه تولید می‌کنند. با این حال، این رویکرد محاسباتی، فشرده است و اغلب به تعاملات انسان در حلقه نیاز دارد. به عنوان یک جایگزین، تکنیک‌های بهینه‌سازی جهانی ابتکاری مانند جستجوی حوضچه و الگوریتم‌های تکاملی، نیاز به راه‌حل‌های راه‌اندازی دقیق را کاهش می‌دهند 
%\cite{vavrina2017global}، اما پیچیدگی محاسباتی مربوطه آنها را برای استفاده درون‌سفینه ناممکن می‌سازد.

%روش‌های مبتنی بر سیستم‌های دینامیکی می‌توانند راه‌حل‌های بهینه تولید کنند، اما محاسباتی فشرده و نیازمند تعاملات انسان هستند. تکنیک‌های بهینه‌سازی جهانی می‌توانند نیاز به راه‌حل‌های راه‌اندازی دقیق \cite{vavrina2017global}
%را کاهش دهند، اما محاسباتی پیچیده هستند.



شبکه‌های عصبی قابلیت‌های منحصر به فردی برای انجام هدایت در فضاپیما دارند. به‌عنوان مثال، شبکه‌های عصبی می‌توانند به‌طور مستقیم از تخمین‌های وضعیت به دستورهای پیشران کنترلی که با محدودیت‌های مأموریت سازگار است، برسند. عملکرد مناسب هدایت شبکه‌های عصبی در مطالعاتی مانند فرود بر سیارات \cite{gaudet2020six}، عملیات نزدیکی به سیارات \cite{gaudet2020terminal} و کنترل فضاپیما با پیشران از دست‌رفته \cite{rubinsztejn2020neural} نشان داده شده‌است.
تازه‌ترین پیشرفت‌های تکنیک‌های یادگیری ماشین در مسائل خودکارسازی درونی به‌طور گسترده‌ای مورد مطالعه قرار گرفته‌اند؛ از پژوهش‌های اولیه تا توانایی‌های پیاده‌سازی.
به‌عنوان مثال، الگوریتم‌های یادگیری ماشین ابتدایی در فضاپیماهای مریخی‌نورد برای کمک به شناسایی ویژگی‌های زمین‌شناسی تعبیه شده‌اند. الگوریتم \lr{AEGIS} توانایی انتخاب خودکار هدف توسط یک دوربین در داخل فضاپیماهای \lr{Spirit}، \lr{Opportunity} و \lr{Curiosity} را دارد
\cite{estlin2012aegis}.
در کامپیوتر پرواز اصلی، فرآیند دقت افزایی\LTRfootnote{Refinement Process} نیاز به 94 تا 96 ثانیه دارد 
\cite{francis2017aegis},
که به طور قابل توجهی کمتر از زمان مورد نیاز برای ارسال تصاویر به زمین و انتظار برای انتخاب دستی توسط دانشمندان است.
برنامه‌های آینده برای کاربردهای یادگیری ماشین درون‌سفینه شامل توانایی‌های رباتیکی درون‌سفینه برای فضاپیمای \lr{Perseverance}
\cite{higa2019vision, rothrock2016spoc}
و شناسایی عیب برای 
\lr{Europa Clipper}
\cite{wagstaff2019enabling} می‌شود. 
الگوریتم‌های یادگیری ماشین پتانسیل انجام نقش مهمی در مأموریت‌های خودکار آینده را دارند.


علاوه بر رباتیک سیاره‌ای، پژوهش‌های مختلفی به استفاده از تکنیک‌های مختلف یادگیری ماشین در مسائل نجومی پرداخته‌اند. در طراحی مسیر عملکرد رگرسیون معمولاً مؤثرتر هست. به عنوان مثال، از یک شبکه عصبی\LTRfootnote{Neural Network}
  در بهینه‌سازی مسیرهای رانشگر کم‌پیشران استفاده شده‌است 
\cite{dachwald2004evolutionary}.
پژوهش‌های جدید شامل شناسایی انتقال‌های هتروکلینیک \cite{desmet2019identifying}، اصلاح مسیر رانشگر کم‌پیشران \cite{parrish2018lowthrust} و تجزیه و تحلیل مشکلات ازدست‌رفتن رانشگر \cite{rubinsztejn2020neural} می‌شود.

%تکنیک‌های یادگیری نظارتی می‌توانند نتایج مطلوبی تولید کنند، اما دارای محدودیت‌های قابل توجهی هستند. ابتدا، این رویکردها بر وجود دانش از پیش از فرآیند تصمیم‌گیری متکی هستند. کاربر با انتخاب نتایج مطلوب، فرض می‌کند که این دانش را دارد. این امر مستلزم دقیق بودن داده‌های تولید‌شده توسط کاربر برای نتایج مطلوب و همچنین وجود تکنیک‌های موجود برای حل مشکل کنونی و تولید داده است. در بخش‌هایی که چنین دانشی وجود ندارد، تکنیک‌های یادگیری نظارتی قابل استفاده نیستند.
تکنیک‌های یادگیری نظارتی می‌توانند نتایج مطلوبی تولید کنند؛ اما، دارای محدودیت‌های قابل توجهی هستند. یکی از این محدودیت‌ها این است که این رویکردها بر وجود دانش پیش از فرآیند تصمیم‌گیری متکی هستند. این امر مستلزم دقیق‌بودن داده‌های تولید‌شده توسط کاربر برای نتایج مطلوب و همچنین وجود تکنیک‌های موجود برای حل مشکل کنونی و تولید داده است.

%در سال‌های اخیر، \lr{RL} به اثبات مفید بودن خود در دستیابی به عملکرد بهترین حالت در دامنه‌هایی با ابهام محیطی قابل توجه رسیده است
%. هدایت فعال‌سازی‌شده توسط \lr{RL} به‌صورت گسترده‌ای بر اساس فاز پرواز دسته‌بندی می‌شوند. مسائل فرود \cite{furfaro2020adaptive, gaudet2020deep} 
%و عملیات در نزدیکی به اجسام کوچک
%\cite{gaudet2020terminal, gaudet2020six}
%از حوزه‌های مطالعاتی بهره‌برداری‌شده‌ای هستند. تحقیقات دیگر شامل مواجهه ،\cite{broida2019spacecraft} تداخل خارجی‌جوی
%\cite{gaudet2020reinforcement}
%، نگهداری ایستگاهی \cite{guzzetti2019reinforcement} و اجتناب از تشخیص \cite{reiter2020augmenting} هستند. مطالعاتی که فضاپیماهای رانشگر کم پیشرانرا در یک چارچوب دینامیکی چندبدنی با استفاده از \lr{RL} شامل طراحی انتقال با استفاده از
% \lr{Q-Learning}
% \cite{dasstuart2020rapid} و \lr{Proximal Policy Optimization }
%  \cite{miller2019lowthrust} و همچنین هدایت نزدیکی مدار
%  \cite{sullivan2020using}، شده‌اند.

در سال‌های اخیر، قابلیت یادگیری تقویتی\LTRfootnote{Reinforcement Learning (RL)}
 در دستیابی به عملکرد بهینه در بخش‌هایی با ابهام محیطی قابل توجه، به اثبات رسیده است  \cite{heess2017emergence, silver2017mastering}.
هدایت انجام‌شده توسط
یادگیری تقویتی
  را می‌توان به‌صورت گسترده بر اساس فاز پرواز دسته‌بندی کرد.
مسائل فرود \cite{furfaro2020adaptive, gaudet2020deep} 
و عملیات در نزدیکی اجسام کوچک
\cite{gaudet2020terminal, gaudet2020six}،
از حوزه‌های پژوهشی هستند که از یادگیری تقویتی استفاده می‌کنند.
تحقیقات دیگر شامل مواجهه تداخل خارجی جوی \cite{gaudet2020reinforcement}،
نگهداری ایستگاهی \cite{guzzetti2019reinforcement}  و هدایت به‌صورت جلوگیری از شناسایی \cite{reiter2020augmenting} است.
مطالعاتی که فضاپیماهای رانشگر کم‌پیشران را در یک چارچوب دینامیکی چندبدنی با استفاده از یادگیری تقویتی انجام شده است، شامل طراحی انتقال با استفاده از \lr{Q-learning}
\cite{dasstuart2020rapid}، \lr{Proximal Policy Optimization}
\cite{miller2019lowthrust}
و  هدایت نزدیکی مدار  \cite{sullivan2020using} است.
\section{بازی دیفرانسیلی}
بازی دیفرانسیلی زیر مجموعه‌ای از نظریه بازی است.
نظریه بازی با استفاده از مدل‌های ریاضی به تحلیل روش‌های همکاری یا رقابت موجودات منطقی و هوشمند می‌پردازد. نظریه بازی، شاخه‌ای از ریاضیات کاربردی است که در علوم اجتماعی و به ویژه در اقتصاد، زیست‌شناسی، مهندسی، علوم سیاسی، روابط بین‌الملل، علوم رایانه، بازاریابی و فلسفه مورد استفاده قرار می‌گیرد. نظریه بازی در تلاش است تا به وسیله‌ی ریاضیات، رفتار را در شرایط راهبردی یا در یک بازی که در آن موفقیت فرد در انتخاب کردن، وابسته به انتخاب دیگران می‌باشد، برآورد کند..
%در سال ۱۹۲۱ یک ریاضی‌دان فرانسوی به نام اِمیل بُرِل برای نخستین بار به مطالعهٔ تعدادی از بازی‌های رایج در قمارخانه‌ها پرداخت و چند مقاله در موردِ آن‌ها نوشت. او در این مقاله‌ها بر قابل پیش‌بینی بودنِ نتایجِ این نوع بازی‌ها از راه‌های منطقی، تأکید کرده بود. 
%کوتاه است یا نه؟؟؟؟؟؟؟؟؟؟؟
در سال ۱۹۹۴ جان فوربز نش به همراه جان هارسانی و راینهارد سیلتن به خاطر مطالعات خلاقانه‌ی خود در زمینه‌ی نظریه بازی، برنده‌ی جایزه نوبل اقتصاد شدند. در سال‌های پس از آن نیز بسیاری از برندگان جایزه‌ی نوبل اقتصاد از میان متخصصین نظریه بازی انتخاب شدند. آخرین آن‌ها، ژان تیرول فرانسوی است که در سال ۲۰۱۴ این جایزه را کسب کرد \cite{nobel}.


پژوهش‌ها در این زمینه اغلب بر مجموعه‌ای از راهبردهای شناخته شده به عنوان تعادل در بازی‌ها استوار است. این راهبردها به‌طور معمول از قواعد عقلانی به نتیجه می‌رسند. مشهورترین تعادل‌ها، تعادل نش است. 
تعادل نش در بازی‌هایی کاربرد دارد  در آن فرض شده‌است که هر بازیکن به راهبرد تعادل دیگر بازیکنان آگاه است.
%در نظریه بازی، تعادل نش (به نام جان فوربز نش، که آن را پیشنهاد کرد) راه حلی از نظریه بازی است که شامل دو یا چند بازیکن است، که در آن فرض بر آگاهی هر بازیکن به راهبرد تعادل دیگر بازیکنان است. 
بر اساس نظریه‌ی تعادل نش، در یک بازی که هر بازیکن امکان انتخاب‌های گوناگون دارد اگر بازیکنان به روش منطقی  راهبردهای خود را انتخاب کنند و به دنبال حداکثر سود در بازی باشند، دست کم یک راهبرد برای به دست آوردن بهترین نتیجه برای هر بازیکن وجود دارد و چنانچه بازیکن راهکار دیگری را انتخاب کند، نتیجه‌ی بهتری به دست نخواهد آورد.
\section{یادگیری تقویتی}


از نخستین صورت‌بندی‌های فرایند تصمیم‌گیری مارکُفی، پژوهش در «یادگیری تقویتی» (RL) بر آن بوده‌است که عامل بتواند با اجرای کنش‌ها و دریافت پاداش، سیاستی برای بیشینه‌سازی بازده بیاموزد. تبیین جامع این چارچوب و الگوریتم‌های بنیادین در ویرایش دوم کتاب سوتون و بارتو به‌مثابه مرجع کلاسیک این حوزه ارائه شده و همچنان مبنای بسیاری از آثار معاصر است \cite{SuttonBarto2018}. % 

\noindent
دههٔ ۱۳۹۰ (۱۹۹۰s میلادی) شاهد شکل‌گیری روش‌های «پایه‌ارزش» نظیر Q-learning و نخستین رویکردهای گرادیانِ سیاست بود؛ با وجود این، محدودیت توان محاسباتی و فقدان دادهٔ حجیم، سرعت رشد را کند می‌کرد. ورود شبکه‌های عصبی عمیق در اوایل دههٔ ۱۳۹۰ خورشیدی (۲۰۱۰s) نقطهٔ عطفی بود: مقالهٔ معروف دیپ‌مایند نشان داد که شبکهٔ Q عمیق (DQN) می‌تواند صرفاً از پیکسل‌های بازی آتاری سیاستی نزدیک به انسان بیاموزد \cite{Mnih2015}. % 

\noindent
موفقیت DQN نگاه‌ها را به‌سوی گرادیانِ سیاستِ مقیاس‌پذیر معطوف ساخت. «بهینه‌سازی ناحیهٔ اطمینان» یا TRPO تضمین بهبود یکنواخت سیاست را فراهم کرد \cite{Schulman2015TRPO}, و روش A3C با موازی‌سازی بازیگران، سرعت یادگیری را چند برابر افزایش داد \cite{Mnih2016A3C}. % :contentReference[oaicite:2]{index=2}
کمی بعد، DDPG اولین بار گرادیان سیاست قطعی را به فضاهای عمل پیوسته وارد کرد \cite{Lillicrap2015DDPG}. % 
سپس PPO با ساده‌سازی قیود TRPO و کاهش پارامترهای حساس، به انتخاب پیش‌فرض بسیاری از کاربردهای مهندسی بدل شد \cite{Schulman2017PPO}. % 

\noindent
با گسترش دامنهٔ مسائل، پایداری و کاراییِ داده به چالش اصلی بدل گشت. TD3 نشان داد که «کمینه‌کردن» میان دو منتقد می‌تواند برآورد بیش‌از‌حد Q را مهار کند \cite{Fujimoto2018TD3}, و SAC با افزودن بند آنتروپی، هم‌زمان اکتشاف و بازده را بهبود داد \cite{Haarnoja2018SAC}. % :contentReference[oaicite:5]{index=5}

\noindent
برای کاهش هزینهٔ تعامل، موج نوینی از «یادگیری تقویتی مدل‌مبنا» شکل گرفت. مأخذی نظام‌مند از این خط پژوهش را بررسی جامع مورلاند و همکاران عرضه می‌کند \cite{Moerland2020MBRLsurvey}، درحالی‌که عامل Dreamer نشان داد می‌توان سیاست را تنها با «تخیل نهفته» در مدل آموخته‌شده به‌روزرسانی کرد \cite{Hafner2019Dreamer}. % :contentReference[oaicite:6]{index=6}

\noindent
در محیط‌های پرخطر یا گران، جمع‌آوری دادهٔ برخط ناممکن است؛ ازاین‌رو RL «آفلاین» مطرح شد. روش CQL با برقراری کران محافظه‌کارانه بر Q-value از گرایشِ خارج از توزیع جلوگیری می‌کند \cite{Kumar2020CQL}، و مرور اخیر پراودِنسیو و همکاران طبقه‌بندی جامعی از چالش‌های باز این حوزه ارائه داده است \cite{Prudencio2022OfflineSurvey}. % 

\noindent
هم‌زمان، دغدغهٔ ایمنی و تبیین در سامانه‌های واقعی پررنگ شد. مرور سال ۲۰۲۲ نشان می‌دهد که ترکیب قیدهای سخت، توابع جریمهٔ ریسک و شبیه‌سازی محیط‌های بدبینانه سه خط اصلی ایمنی در RL هستند \cite{Garcia2022SafeSurvey}. سلسله‌مراتب نیز با هدف انتقال دانش و تسریع یادگیری مورد توجه قرار گرفت و یک مطالعهٔ جامع در {\it ACM Computing Surveys} چهار چالش کشف زیرکار، یادگیری اشتراک‌پذیر، انتقال و مقیاس‌پذیری را برجسته می‌کند \cite{Ghazalpour2021HRLsurvey}. % 

\noindent
وقتی چند عامل به‌طور هم‌زمان یاد می‌گیرند، پویایی محیط از دید هر عامل غیرایستا می‌شود. مرور جامع ۲۰۲۴ نشان می‌دهد که چارچوب «ناظر متمرکز ـ بازیگر توزیع‌شده» (CTDE) راهکاری موثر برای این چالش است و مباحثی چون تخصیص اعتبار جمعی و کشف تعادل را معرفی می‌کند \cite{Song2024MARLsurvey}. % 

\noindent
پیشرفت‌های یادشده در نهایت به دستاوردهای نمادینی چون AlphaGo \cite{Silver2016AlphaGo} و AlphaStar \cite{Vinyals2019AlphaStar} انجامیدند که در بازی‌های Go و StarCraft II از انسان پیشی گرفتند، و معماری توزیع‌شدهٔ IMPALA نشان داد که چگونه می‌توان هزاران شبیه‌ساز را با به‌روزرسانی وزن‌های مهم ادغام کرد \cite{Espeholt2018IMPALA}. % 

\noindent
به‌رغم این جهش‌ها، سه شکاف اساسی پابرجا مانده است: ۱) تضمین ایمنی سخت‌گیرانه در سناریوهای نزدیک‌برخورد، ۲) کاهش وابستگی به دادهٔ پرهزینه یا نایاب از طریق روش‌های مدل‌مبنا و آفلاین، و ۳) مقیاس‌پذیری یادگیری چندعاملی برای سامانه‌های رباتیکی یا فضاپیمای چندگانه. پژوهش حاضر در پی آن است که با تلفیق یادگیری تقویتی مقاوم و چندعاملی در چارچوب سه‌جسمی مداری، به این خلأ پاسخ دهد.

%\section{یادگیری تقویتی چندعاملی}
\section{پیشینه‌ی پژوهش یادگیری تقویتی چندعاملی}\label{sec:marl_lit}

امروز یادگیری تقویتی چندعاملی\LTRfootnote{Multi-Agent Reinforcement Learning (MARL)} به‌عنوان بنیاد اصلی سامانه‌های هوشمند مشارکتی شناخته می‌شود؛ مسیری که از آزمون‌های ساده‌ی دو‌عاملی در دهه‌ی ۱۹۹۰ آغاز شد و اکنون به معماری‌های توزیع‌شده‌ی در مقیاس هزاران بازیگر رسیده است. این بخش، به بررسی اینکه چگونه ایده‌ی {آموزش متمرکز ـ اجرای توزیع‌شده} (\lr{CTDE}) به پاسخ غالب برای چالش‌های غیرایستایی و انفجار بُعدی\LTRfootnote{Curse of Dimensionality}
 بدل شد و چه گام‌هایی هنوز برای ایمنی، ناهمگونی و مقیاس‌پذیری باقی مانده است.

دهه‌ی ۱۹۹۰ با مقاله‌ی \cite{Tan1993} آغاز شد؛ جایی که برای نخستین‌بار مقایسه‌ی عامل‌های مستقل با عامل‌های همکار انجام شد و سود ارتباط و اشتراک تجربه به‌صورت تجربی نشان داده شد. در میانه‌ی دهه‌ی بعد، مرور جامع پانایت و لوک \cite{Panait2005} چشم‌اندازی از مسائل تخصیص اعتبار و غیرایستایی ترسیم کرد و دو موضوع یادگیری تیمی و یادگیری هم‌زمان را صورت‌بندی نمود.  هم‌زمان، بوشونیـو و همکاران \cite{Busoniu2008} ادبیات \lr{MARL} را در قالب اهداف پایداری دینامیک یادگیری و انطباق با رفتار سایر عامل‌ها جمع‌بندی کردند و راه را برای تحلیل‌های بازی‌محور هموار ساختند. 

ورود شبکه‌های عمیق در سال‌های ۲۰۱۶ و ۲۰۱۷ نقطه‌ی عطف بعدی بود؛ منتقد متمرکز ـ بازیگر توزیع‌شده در \lr{MA-DDPG} \cite{Lowe2017} نشان داد که می‌توان از حالت سراسری در فاز آموزش بهره برد، اما سیاست نهایی را صرفاً بر اساس مشاهدات محلی اجرا کرد.  در همان سال، \lr{Value‐Decomposition Networks}
\cite{Sunehag2017} ایده‌ی تجزیه‌ی خطی پاداش را برای همکاری عامل‌ها مطرح کرد و راه را برای تقسیم بندی‌های پیش‌رفته پاداش گشود. 

۲۰۱۸ شاهد جهش مهمی با \lr{QMIX} بود؛ این روش با اعمال قید تک‌نوا\LTRfootnote{Monotonic}
 بر ترکیب مقادیر منفرد، هم امکان بهینه‌سازی غیرسیاست‌محور را فراهم کرد و هم تضمین سازگاری سیاست‌های محلی با ارزش مشترک را برقرار ساخت \cite{Rashid2018}. 

سال ۲۰۱۹ به گسترش بسترهای آزمایش اختصاص یافت. چالش استاندارد
 \lr{StarCraft Multi-Agent Challenge (SMAC)} بر مبنای \lr{StarCraft II} معرفی شد و معیار مشترکی برای مقایسه‌ی الگوریتم‌ها را مهیا کرد
  \cite{Samvelyan2019SMAC}.
  هم‌زمان، \lr{QTRAN} \cite{Son2019QTRAN} نشان داد که می‌توان بدون قید خطی یا تک‌نوا، تابع ارزش مشترک را به فضای قابل تجزیه تبدیل کرد. از سوی دیگر، \lr{MAVEN} با افزودن متغیر نهفته‌ی مشترک، کاوش هماهنگ و سلسله‌مراتبی را امکان‌پذیر ساخت \cite{Mahajan2019MAVEN}.  نقطه‌ی اوج همان سال، سامانه‌ی \lr{AlphaStar} بود که نشان داد ترکیب خودبازی و معماری توزیع‌شده می‌تواند به رتبه‌ی استاد بزرگ\LTRfootnote{Grandmaster}
   انسان برساند \cite{Vinyals2019AlphaStar}. 

در ۲۰۲۰ مفهوم نقش‌های در حال ظهور با \lr{ROMA} \cite{Wang2020ROMA} معرفی شد تا عامل‌ها بر اساس شباهت رفتاری به‌طور خودکار خوشه‌بندی و اشتراک دانش کنند؛ رویکردی که در نقشه‌های پرتراکم \lr{SMAC} برتری محسوسی نشان داد. پژوهش‌های متا در ۲۰۲۱، از مرور نظری زانگ و بشار \cite{Zhang2021Survey} تا 
محک\LTRfootnote{Benchmark}
 تطبیقی پاپوداکیس و همکاران
\cite{9583665}،
 شکاف‌های باقی‌مانده در تضمین همگرایی و مقیاس را فهرست کردند. 

آخرین موج مطالعات بر ناهمگونی و ایمنی تمرکز دارد. مرور جامع \cite{Yu2022Heterogeneous} نشان می‌دهد که تفاوت در قابلیت‌ها و اطلاعات عامل‌ها، مسائلی نظیر تخصیص اعتبار و تعادل را پیچیده‌تر می‌سازد و به الگوریتم‌های سازگار با نقش‌های پویا نیاز دارد.

به‌طور خلاصه، مسیر تاریخی \lr{MARL} از الگوهای مستقل دهه‌ی ۱۹۹۰ به سامانه‌های توزیع‌شده‌ی امروزی، همواره با سه دغدغه‌ی اصلی هدایت شده است: کنترل انفجار بُعدی توابع ارزش، مقابله با غیرایستایی ناشی از یادگیری هم‌زمان، و انتقال مؤثر تجربه میان عامل‌ها. علی‌رغم پیشرفت‌های شتابان، تضمین ایمنی سخت‌گیرانه در محیط‌های شکست‌پذیر، مدیریت نقش‌های پویا در تیم‌های ناهمگون و کاهش نیاز به داده‌ی شبیه‌سازی پرهزینه همچنان چالش‌های باز باقی می‌مانند؛ چالش‌هایی که در این پژوهش با رویکرد ترکیبی مدل‌مبنا، مقاوم و چندعاملی پیگیری می‌شوند.





\subsection{ماموریت‌های بین‌مداری}
\subsubsection{تعریف ماموریت‌های بین‌مداری}
اصول طراحی و بهینه‌سازی ماموریت‌ها و چالش‌های ارتباطی و کنترلی.
\subsubsection{مدل‌سازی مسیرهای بین‌مداری}
رویکردهای عددی برای بهینه‌سازی مسیر و استفاده از یادگیری ماشینی در برنامه‌ریزی مسیر.
\subsubsection{ایمنی در ماموریت‌های بین‌مداری}
تضمین پایداری مسیرها در حضور عدم قطعیت و مدیریت ریسک برخورد.

\subsection{بازی دیفرانسیلی}
\subsubsection{اصول بازی دیفرانسیلی}
تعریف بازی دیفرانسیلی، مفاهیم پایه و کاربردهای آن در تعاملات چندعاملی.
\subsubsection{بازی دیفرانسیلی با جمع صفر}
مدل‌سازی مسائل با جمع صفر و کاربردهای آن در مسائل دفاعی و نظارتی.
\subsubsection{تکنیک‌های حل بازی‌های دیفرانسیلی}
روش‌های تحلیلی و استفاده از الگوریتم‌های یادگیری تقویتی.

\subsection{یادگیری تقویتی}
\subsubsection{مفاهیم یادگیری تقویتی}
سیاست، پاداش، و به‌روزرسانی ارزش‌ها.
\subsubsection{الگوریتم‌های پیشرفته یادگیری تقویتی}
الگوریتم‌های Monte Carlo، Temporal-Difference و Deep Q-Network.
\subsubsection{ایمنی در یادگیری تقویتی}
یادگیری ایمن، محدودیت‌ها و تضمین‌ها در سیاست‌های یادگیری.

\subsection{یادگیری تقویتی چندعاملی}
\subsubsection{تعریف و اهمیت یادگیری تقویتی چندعاملی}
مفهوم همکاری و رقابت در محیط‌های چندعاملی و اهمیت ایمنی.
\subsubsection{بازی‌های چندعاملی در یادگیری تقویتی}
بازی‌های جمع صفر، تعادل نش و کاربردهای آن‌ها.
\subsubsection{چالش‌های یادگیری تقویتی چندعاملی}
مدیریت تعارضات، تضمین پایداری و مشکلات همگرایی.
\subsubsection{ایمنی در یادگیری تقویتی چندعاملی}
تکنیک‌های تضمین ایمنی و کاربرد ایمنی در تعاملات حساس.
\subsubsection{الگوریتم‌های یادگیری تقویتی چندعاملی}
الگوریتم‌های همکاری مانند MADDPG و یادگیری توزیع‌شده با تضمین ایمنی.
\subsubsection{کاربرد MARL در مسائل نظری}
حل مسائل بازی‌های جمع صفر و مدل‌سازی مسائل رقابتی و همکاری.
\subsubsection{کاربرد MARL در ماموریت‌های فضایی}
هماهنگی میان ماهواره‌ها، تخصیص منابع و حل مسائل ترکیبی در کاوش سیارات.

\section{یادگیری تقویتی چندعاملی}

\subsection{تعریف و اهمیت یادگیری تقویتی چندعاملی}
\subsubsection{تعریف یادگیری تقویتی چندعاملی}
% محتوای مربوط به تعریف MARL

\subsubsection{اهمیت یادگیری تقویتی چندعاملی}
% محتوای مربوط به اهمیت MARL

\subsection{بازی‌های چندعاملی در یادگیری تقویتی}
\subsubsection{بازی‌های جمع صفر}
% توضیح درباره بازی‌های جمع صفر

\subsubsection{تعادل نش و کاربردهای آن}
% توضیح درباره تعادل نش

\subsection{چالش‌های یادگیری تقویتی چندعاملی}
\subsubsection{مدیریت تعارضات}
% توضیح درباره مدیریت تعارضات

\subsubsection{تضمین پایداری}
% توضیح درباره تضمین پایداری

\subsubsection{مشکلات همگرایی}
% توضیح درباره مشکلات همگرایی

\subsection{ایمنی در یادگیری تقویتی چندعاملی}
\subsubsection{تکنیک‌های تضمین ایمنی}
% توضیح درباره تکنیک‌های ایمنی

\subsubsection{کاربرد ایمنی در تعاملات حساس}
% توضیح درباره کاربرد ایمنی

\subsection{الگوریتم‌های یادگیری تقویتی چندعاملی}
\subsubsection{الگوریتم‌های همکاری مانند MADDPG}
% توضیح درباره MADDPG و الگوریتم‌های همکاری

\subsubsection{یادگیری توزیع‌شده با تضمین ایمنی}
% توضیح درباره یادگیری توزیع‌شده

\subsection{کاربرد MARL در مسائل نظری}
\subsubsection{حل مسائل بازی‌های جمع صفر}
% توضیح درباره حل بازی‌های جمع صفر

\subsubsection{مدل‌سازی مسائل رقابتی و همکاری}
% توضیح درباره مدل‌سازی مسائل

\subsection{کاربرد MARL در ماموریت‌های فضایی}
\subsubsection{هماهنگی میان ماهواره‌ها}
% توضیح درباره هماهنگی ماهواره‌ها

\subsubsection{تخصیص منابع}
% توضیح درباره تخصیص منابع

\subsubsection{حل مسائل ترکیبی در کاوش سیارات}
% توضیح درباره حل مسائل ترکیبی


\section{کاربرد مولتی‌اِجنت در بازی‌ها}

در این بخش به بررسی کاربردهای مولتی‌اِجنت در حوزه بازی‌ها پرداخته و نشان می‌دهیم که الگوریتم پیشنهادی ما نه تنها بهینه بلکه ایمن و قابل اعتماد است.

\section{کاربرد چند عاملی در بازی‌ها}

در این بخش به بررسی کاربردهای یادگیری تقویتی چند عاملی در حوزه بازی‌ها پرداخته و نشان می‌دهیم که الگوریتم پیشنهادی ما نه تنها بهینه بلکه ایمن و قابل اعتماد است.

\subsection{تعریف بازی‌های چند عاملی}
\subsubsection{مفاهیم پایه در بازی‌های چند عاملی}
بازی‌های چند عاملی شامل محیط‌هایی هستند که در آن چندین عامل به صورت همزمان و مستقل به تعامل می‌پردازند. این تعاملات می‌تواند شامل همکاری، رقابت یا ترکیبی از هر دو باشد. در چنین محیط‌هایی، هر عامل با هدف خود به حداکثر رساندن پاداش یا دستیابی به اهداف مشخص، استراتژی‌های خود را توسعه می‌دهد.

\subsubsection{نمونه‌های بازی‌های چند عاملی}
از جمله نمونه‌های معروف بازی‌های چند عاملی می‌توان به بازی‌های استراتژیک مانند \lr{StarCraft}، بازی‌های ورزشی چند نفره و بازی‌های تخته‌ای مانند شطرنج چند عاملی اشاره کرد. این بازی‌ها به دلیل پیچیدگی‌های بالای تعاملات میان عوامل، محیط‌های مناسبی برای آزمایش و ارزیابی الگوریتم‌های یادگیری تقویتی چند عاملی فراهم می‌کنند.

\subsection{الگوریتم بهینه در بازی‌های چند عاملی}
\subsubsection{معرفی الگوریتم پیشنهادی}
الگوریتم پیشنهادی ما بر اساس ترکیبی از یادگیری عمیق و تکنیک‌های بهینه‌سازی چند عاملی طراحی شده است. این الگوریتم با استفاده از شبکه‌های عصبی عمیق، استراتژی‌های بهینه برای هر عامل را در محیط‌های پیچیده بازی‌های چند عاملی یاد می‌گیرد. علاوه بر این، با استفاده از مکانیزم‌های تعاملی، هماهنگی و همکاری میان عوامل بهبود می‌یابد.

\subsubsection{بهینه بودن الگوریتم}
الگوریتم ما با هدف کاهش زمان همگرایی و افزایش کارایی در محیط‌های دینامیک بهینه شده است. با استفاده از تکنیک‌های پیشرفته مانند یادگیری انتقالی و تنظیم خودکار نرخ یادگیری، الگوریتم قادر است به سرعت به تعادل‌های مطلوب برسد و عملکرد بهینه‌ای در بازی‌ها ارائه دهد.

\subsection{ایمنی و قابلیت اطمینان الگوریتم در بازی‌ها}
\subsubsection{تضمین ایمنی در تعاملات میان عوامل}
ایمنی در تعاملات چند عاملی به معنای جلوگیری از رفتارهای غیرمنتظره و تضمین هماهنگی میان عوامل است. در الگوریتم ما، از مکانیزم‌های نظارتی و محدودکننده استفاده شده است که اطمینان حاصل می‌کنند تعاملات میان عوامل منجر به نتیجه‌ای نامطلوب نمی‌شود. این شامل محدود کردن فضای عملیاتی و اعمال قیود بر سیاست‌های یادگیری هر عامل می‌باشد.

\subsubsection{قابلیت اطمینان و مقاومت در برابر خطاها}
الگوریتم پیشنهادی با هدف افزایش قابلیت اطمینان و مقاومت در برابر خطاها طراحی شده است. با استفاده از تکنیک‌های افزونگی و یادگیری توزیع‌شده، الگوریتم قادر است در صورت بروز خطا یا نقص در برخی از عوامل، به عملکرد مطلوب خود ادامه دهد. این ویژگی‌ها الگوریتم را برای استفاده در محیط‌های حساس و پویا مناسب می‌سازد.

\subsection{مطالعات موردی و نتایج تجربی}
\subsubsection{پیاده‌سازی در بازی‌های مشخص}
الگوریتم ما در بازی‌های مختلفی مانند بازی استراتژیک \lr{StarCraft} و بازی‌های ورزشی چند نفره پیاده‌سازی شده است. در هر یک از این بازی‌ها، عملکرد الگوریتم ما با توجه به معیارهای بهینه بودن و ایمنی ارزیابی شده است.

\subsubsection{تحلیل عملکرد و مقایسه با الگوریتم‌های دیگر}
نتایج تجربی نشان می‌دهد که الگوریتم پیشنهادی ما نسبت به الگوریتم‌های موجود در زمینه یادگیری تقویتی چند عاملی از نظر سرعت همگرایی و کارایی بهتری دارد. علاوه بر این، با استفاده از مکانیزم‌های تضمین ایمنی، رفتارهای ناخواسته و خطرناک در تعاملات میان عوامل به طور قابل توجهی کاهش یافته است.

\subsection{نتیجه‌گیری}
در این بخش، به بررسی و تحلیل نتایج به دست آمده از پیاده‌سازی الگوریتم پیشنهادی در بازی‌های چند عاملی پرداخته شد. مشاهده شد که الگوریتم ما توانسته است با ارائه راه‌حل‌های بهینه و ایمن، عملکرد بهتری نسبت به الگوریتم‌های موجود داشته باشد. این امر نشان‌دهنده قابلیت‌های بالای الگوریتم در مدیریت تعاملات پیچیده و تضمین ایمنی در محیط‌های چند عاملی است.

\subsection{پیشنهادات برای تحقیقات آینده}
با توجه به نتایج حاصل شده، پیشنهاد می‌شود که تحقیقات آینده بر روی بهبود بیشتر ایمنی و قابلیت اطمینان الگوریتم‌های چند عاملی متمرکز شود. همچنین، توسعه الگوریتم‌های سازگار با محیط‌های پویا و پیچیده‌تر و ارزیابی آن‌ها در بازی‌ها و محیط‌های واقعی‌تر از جمله مسیرهای پیشنهادی برای ادامه تحقیقات می‌باشد.

