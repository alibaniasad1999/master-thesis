\section{بازی دیفرانسیلی}
بازی دیفرانسیلی زیر مجموعه‌ای از نظریه بازی است.
نظریه بازی با استفاده از مدل‌های ریاضی به تحلیل روش‌های همکاری یا رقابت موجودات منطقی و هوشمند می‌پردازد. نظریه بازی، شاخه‌ای از ریاضیات کاربردی است که در علوم اجتماعی و به ویژه در اقتصاد، زیست‌شناسی، مهندسی، علوم سیاسی، روابط بین‌الملل، علوم رایانه، بازاریابی و فلسفه مورد استفاده قرار می‌گیرد. نظریه بازی در تلاش است تا به وسیله‌ی ریاضیات، رفتار را در شرایط راهبردی یا در یک بازی که در آن موفقیت فرد در انتخاب کردن، وابسته به انتخاب دیگران می‌باشد، برآورد کند..
%در سال ۱۹۲۱ یک ریاضی‌دان فرانسوی به نام اِمیل بُرِل برای نخستین بار به مطالعهٔ تعدادی از بازی‌های رایج در قمارخانه‌ها پرداخت و چند مقاله در موردِ آن‌ها نوشت. او در این مقاله‌ها بر قابل پیش‌بینی بودنِ نتایجِ این نوع بازی‌ها از راه‌های منطقی، تأکید کرده بود. 
%کوتاه است یا نه؟؟؟؟؟؟؟؟؟؟؟
در سال ۱۹۹۴ جان فوربز نش به همراه جان هارسانی و راینهارد سیلتن به خاطر مطالعات خلاقانه‌ی خود در زمینه‌ی نظریه بازی، برنده‌ی جایزه نوبل اقتصاد شدند. در سال‌های پس از آن نیز بسیاری از برندگان جایزه‌ی نوبل اقتصاد از میان متخصصین نظریه بازی انتخاب شدند. آخرین آن‌ها، ژان تیرول فرانسوی است که در سال ۲۰۱۴ این جایزه را کسب کرد \cite{nobel}.


پژوهش‌ها در این زمینه اغلب بر مجموعه‌ای از راهبردهای شناخته شده به عنوان تعادل در بازی‌ها استوار است. این راهبردها به‌طور معمول از قواعد عقلانی به نتیجه می‌رسند. مشهورترین تعادل‌ها، تعادل نش است. 
تعادل نش در بازی‌هایی کاربرد دارد  در آن فرض شده‌است که هر بازیکن به راهبرد تعادل دیگر بازیکنان آگاه است.
%در نظریه بازی، تعادل نش (به نام جان فوربز نش، که آن را پیشنهاد کرد) راه حلی از نظریه بازی است که شامل دو یا چند بازیکن است، که در آن فرض بر آگاهی هر بازیکن به راهبرد تعادل دیگر بازیکنان است. 
بر اساس نظریه‌ی تعادل نش، در یک بازی که هر بازیکن امکان انتخاب‌های گوناگون دارد اگر بازیکنان به روش منطقی  راهبردهای خود را انتخاب کنند و به دنبال حداکثر سود در بازی باشند، دست کم یک راهبرد برای به دست آوردن بهترین نتیجه برای هر بازیکن وجود دارد و چنانچه بازیکن راهکار دیگری را انتخاب کند، نتیجه‌ی بهتری به دست نخواهد آورد.