%\section{یادگیری تقویتی چندعاملی}
\section{پیشینهٔ پژوهش یادگیری تقویتی چندعاملی}\label{sec:marl_lit}

امروز یادگیری تقویتی چندعاملی (MARL) به‌عنوان ستون فقرات سامانه‌های هوشمند مشارکتی شناخته می‌شود؛ مسیری که از آزمون‌های سادهٔ دو‌عاملی در دههٔ ۱۹۹۰ آغاز شد و اکنون به معماری‌های توزیع‌شده‌ی در مقیاس هزاران بازیگر رسیده است. این بخش، بدون ورود به زیرشاخه‌های ریز، روایت تاریخیِ پیوسته‌ای ارائه می‌کند که نشان می‌دهد چگونه ایدهٔ \emph{آموزش متمرکز ـ اجرای توزیع‌شده} (CTDE) به پاسخ غالب برای چالش‌های غیرایستایی و انفجار بُعدی بدل شد و چه گام‌هایی هنوز برای ایمنی، ناهمگونی و مقیاس‌پذیری باقی مانده است.

دههٔ ۱۹۹۰ با مقالهٔ \cite{Tan1993} آغاز شد؛ جایی که برای نخستین‌بار مقایسهٔ «عامل‌های مستقل» با «عامل‌های همکار» انجام شد و سود ارتباط و اشتراک تجربه به‌صورت تجربی نشان داده شد. در میانهٔ دههٔ بعد، مرور جامع پانایت و لوک \cite{Panait2005} چشم‌اندازی از مسائل تخصیص اعتبار و غیرایستایی ترسیم کرد و دو مکتب «یادگیری تیمی» و «یادگیری هم‌زمان» را صورت‌بندی نمود.  هم‌زمان، بوشونیـو و همکاران \cite{Busoniu2008} ادبیات متکثر MARL را در قالب اهداف «پایداری دینامیک یادگیری» و «انطباق با رفتار سایر عامل‌ها» جمع‌بندی کردند و راه را برای تحلیل‌های بازی‌محور هموار ساختند. 

ورود شبکه‌های عمیق در سال‌های ۲۰۱۶ـ۲۰۱۷ نقطهٔ عطف بعدی بود؛ «منتقد متمرکز ـ بازیگر توزیع‌شده» در \emph{MADDPG} \cite{Lowe2017} نشان داد که می‌توان از حالت سراسری در فاز آموزش بهره برد، اما سیاست نهایی را صرفاً بر اساس مشاهدات محلی اجرا کرد.  در همان سال، \emph{Value‐Decomposition Networks} یا VDN \cite{Sunehag2017} ایدهٔ تجزیهٔ خطی پاداش را برای تیم‌های کاملاً تعاونی مطرح کرد و راه را برای فاکتوربندی‌های پیش‌رفته گشود. 

۲۰۱۸ شاهد جهش مهمی با \emph{QMIX} بود؛ این روش با اعمال قید تک‌نوا بر ترکیب مقادیر منفرد، هم امکان بهینه‌سازی آف‌پالیسی را فراهم کرد و هم تضمین سازگاری سیاست‌های محلی با ارزش مشترک را برقرار ساخت \cite{Rashid2018}. 

سال ۲۰۱۹ به گسترش بسترهای آزمایش اختصاص یافت. چالش استاندارد \emph{SMAC} بر مبنای StarCraft II معرفی شد و معیار مشترکی برای مقایسهٔ الگوریتم‌ها مهیا کرد \cite{Samvelyan2019SMAC}.هم‌زمان، \emph{QTRAN} \cite{Son2019QTRAN} نشان داد که می‌توان بدون قید خطی یا تک‌نوا، تابع ارزش مشترک را به فضای قابل تجزیه تبدیل کرد. از سوی دیگر، \emph{MAVEN} با افزودن متغیر نهفتهٔ مشترک، کاوش هماهنگ و سلسله‌مراتبی را امکان‌پذیر ساخت \cite{Mahajan2019MAVEN}.  نقطهٔ اوج همان سال، سامانهٔ \emph{AlphaStar} بود که نشان داد ترکیب خودبازی و معماری توزیع‌شده می‌تواند به رتبهٔ گرندمسترِ انسان برساند \cite{Vinyals2019AlphaStar}. 

در ۲۰۲۰ مفهوم «نقش‌های در حال ظهور» با \emph{ROMA} \cite{Wang2020ROMA} معرفی شد تا عامل‌ها بر اساس شباهت رفتاری به‌طور خودکار خوشه‌بندی و اشتراک دانش کنند؛ رویکردی که در نقشه‌های پرتراکم SMAC برتری محسوسی نشان داد. :contentReference[oaicite:10]{index=10} پژوهش‌های متا در ۲۰۲۱، از مرور نظری زانگ و بشار \cite{Zhang2021Survey} تا بنچ‌مارک تطبیقی پاپوداکیس \emph{et al.}، شکاف‌های باقی‌مانده در تضمین همگرایی و مقیاس را فهرست کردند. 

آخرین موج مطالعات (۲۰۲۲ به بعد) بر ناهمگونی و ایمنی تمرکز دارد. مرور جامع \cite{Yu2022Heterogeneous} نشان می‌دهد که تفاوت در قابلیت‌ها و اطلاعات عامل‌ها، مسائلی نظیر تخصیص اعتبار و تعادل را پیچیده‌تر می‌سازد و به الگوریتم‌های سازگار با نقش‌های پویا نیاز دارد.

به‌طور خلاصه، مسیر تاریخی MARL از الگوهای مستقل دههٔ ۱۹۹۰ به سامانه‌های توزیع‌شدهٔ امروزی، همواره با سه دغدغهٔ اصلی هدایت شده است: کنترل انفجار بُعدی توابع ارزش، مقابله با غیرایستایی ناشی از یادگیری هم‌زمان، و انتقال مؤثر تجربه میان عامل‌ها. علی‌رغم پیشرفت‌های شتابان، تضمین ایمنی سخت‌گیرانه در محیط‌های شکست‌پذیر، مدیریت نقش‌های پویا در تیم‌های ناهمگون و کاهش نیاز به دادهٔ شبیه‌سازی پرهزینه همچنان چالش‌های باز باقی می‌مانند؛ چالش‌هایی که در این رساله با رویکرد ترکیبی مدل‌مبنا، مقاوم و چندعاملی پیگیری می‌شوند.
