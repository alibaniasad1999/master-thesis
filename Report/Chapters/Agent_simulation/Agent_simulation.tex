\chapter{شبیه‌سازی عامل درمحیط سه جسمی}


\section{آموزش عامل در یادگیری توقیتی}

\subsection{تعریف عامل و فرآیند آموزش}
در یادگیری توقیتی، عامل (\lr{Agent}) موجودیتی است که در یک محیط قرار دارد و با آن تعامل می‌کند تا به هدف خاصی برسد. هدف عامل در این مدل، یادگیری استراتژی‌هایی است که باعث افزایش پاداش (\lr{reward}) یا کاهش هزینه (\lr{cost}) در درازمدت می‌شود. این فرآیند شامل دریافت اطلاعات از محیط، انتخاب اقدام‌ها (\lr{actions}) و به‌روزرسانی سیاست‌ها یا مدل‌های تصمیم‌گیری است.

فرآیند یادگیری عامل به‌طور کلی شامل سه مرحله اصلی است:
\begin{itemize}
	\item \textbf{انتخاب استراتژی‌ها (\lr{Policy Selection})}: انتخاب بهترین استراتژی برای انجام اقدامات در هر وضعیت.
	\item \textbf{دریافت پاداش (\lr{Reward Reception})}: دریافت پاداش از محیط برای هر اقدام انجام‌شده.
	\item \textbf{بهبود سیاست‌ها (\lr{Policy Improvement})}: با توجه به پاداش‌ها و تجربیات، به‌روزرسانی سیاست‌ها برای انتخاب بهتر اقدامات در آینده.
\end{itemize}

در این بخش، به معرفی مهم‌ترین الگوریتم‌های یادگیری توقیتی پرداخته می‌شود که برای آموزش عامل‌ها در این زمینه استفاده می‌شوند:

\subsubsection{الگوریتم‌ها}
\begin{itemize}
	\item \textbf{DDPG (\lr{Deep Deterministic Policy Gradient})}: این الگوریتم یک روش یادگیری تقویتی مبتنی بر سیاست است که از یک شبکه عصبی برای تخمین سیاست استفاده می‌کند. این الگوریتم برای حل مسائل با فضای حالت و اقدام پیوسته کاربرد دارد.
	\item \textbf{TD3 (\lr{Twin Delayed Deep Deterministic Policy Gradient})}: الگوریتم TD3 نسخه بهبود یافته‌ای از DDPG است که با استفاده از استراتژی‌هایی مانند تاخیر در به‌روزرسانی سیاست و استفاده از دو شبکه عمل می‌کند تا از پدیده‌های نوسانی در یادگیری جلوگیری کند.
	\item \textbf{SAC (\lr{Soft Actor-Critic})}: الگوریتم SAC یک روش مبتنی بر سیاست است که به‌طور همزمان به بهینه‌سازی سیاست و تابع ارزش پرداخته و از مفهوم انتروپی نرم برای بهبود بهره‌وری یادگیری استفاده می‌کند. این الگوریتم به‌ویژه در مسائل با فضای حالت و اقدام پیوسته موثر است.
	\item \textbf{PPO (\lr{Proximal Policy Optimization})}: الگوریتم PPO یکی از الگوریتم‌های یادگیری تقویتی مبتنی بر سیاست است که از روش‌های بهینه‌سازی موقتی برای بهبود سیاست‌های عامل استفاده می‌کند. این الگوریتم به‌ویژه برای مسائل با فضای حالت و اقدام گسسته و پیوسته مناسب است.
\end{itemize}

این الگوریتم‌ها از طریق بهینه‌سازی تدریجی سیاست عامل‌ها، قادر به یادگیری و بهبود استراتژی‌های خود برای رسیدن به پاداش بیشتر در طول زمان هستند. انتخاب هرکدام از این الگوریتم‌ها به ویژگی‌های خاص محیط و چالش‌های یادگیری بستگی دارد.

\subsection{زمان و نوع پردازنده در آموزش عامل}
آموزش عامل‌ها معمولاً به شدت به زمان و منابع پردازشی نیاز دارد. بسته به نوع الگوریتم و پیچیدگی محیط، زمان آموزش می‌تواند متفاوت باشد. در این بخش، دو موضوع کلیدی بررسی می‌شود: \textbf{تأثیر زمان بر الگوریتم‌ها} و \textbf{تأثیر نوع پردازنده‌ها}.

\subsubsection{تأثیر زمان بر الگوریتم‌ها}
زمان یادگیری به پیچیدگی محیط و الگوریتم‌های انتخابی بستگی دارد. برای مثال:
\begin{itemize}
	\item در \lr{DDPG}، که یک الگوریتم مبتنی بر سیاست است، زمان یادگیری معمولاً به تعداد وضعیت‌ها و اقدامات پیوسته بستگی دارد.
	\item در \lr{TD3}، که نسخه بهبود یافته‌ای از \lr{DDPG} است، زمان یادگیری ممکن است بیشتر باشد زیرا دو شبکه برای به‌روزرسانی استفاده می‌شود.
	\item \lr{SAC} زمان یادگیری بیشتری نیاز دارد زیرا همزمان از دو فرآیند یادگیری سیاست و تابع ارزش استفاده می‌کند.
	\item در \lr{PPO}، زمان یادگیری به‌طور معمول بستگی به تعداد تکرارها و حجم داده‌های ورودی دارد.
\end{itemize}

\subsubsection{تأثیر نوع پردازنده‌ها}
- \textbf{پردازنده‌های مرکزی (\lr{CPU})}: برای الگوریتم‌های ساده‌تر مانند \lr{DDPG} و \lr{TD3} که به پردازش موازی کمتری نیاز دارند، استفاده از \lr{CPU} کافی است.
- \textbf{پردازنده‌های گرافیکی (\lr{GPU})}: الگوریتم‌های پیچیده‌تری مانند \lr{SAC} و \lr{PPO} که نیاز به پردازش موازی دارند، می‌توانند از \lr{GPU} بهره‌برداری بیشتری کنند. این پردازنده‌ها می‌توانند زمان یادگیری را به‌طور چشمگیری کاهش دهند.
- \textbf{پردازنده‌های مخصوص یادگیری (\lr{TPU})}: در برخی شبیه‌سازی‌های پیچیده که به محاسبات بیشتری نیاز دارند، \lr{TPU} می‌تواند زمان آموزش را به طور قابل توجهی کاهش دهد.

\subsection{نتیجه‌گیری}
انتخاب الگوریتم‌های یادگیری توقیتی مناسب و پردازنده‌های بهینه، می‌تواند تأثیر زیادی بر کارایی آموزش عامل‌ها در محیط‌های پیچیده مانند محیط سه جسمی داشته باشد. الگوریتم‌هایی مانند \lr{DDPG}، \lr{TD3}، \lr{SAC} و \lr{PPO} هرکدام بسته به نوع چالش‌های محیط، مزایا و معایب خاص خود را دارند و انتخاب پردازنده مناسب می‌تواند به‌طور قابل توجهی زمان یادگیری و عملکرد عامل‌ها را بهبود بخشد.
