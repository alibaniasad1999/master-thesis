\section{شبیه‌سازی عامل}\label{sec:agent_sim}

در این زیربخش، فرآیند شبیه‌سازی و آموزش عامل با استفاده از الگوریتم‌های یادگیری تقویتی پیشرفته ارائه شده‌است. تمرکز بر طراحی شبکه‌ها، منطق انتخاب الگوریتم‌ها، فراپارامترهای کلیدی و ملاحظات پایداری در حین آموزش است تا تکرارپذیری و دقت نتایج تضمین شود.

\subsection{پارامترهای یادگیری و منطق انتخاب الگوریتم‌ها}
الگوریتم‌های \lr{DDPG}، \lr{TD3}، \lr{SAC} و \lr{PPO} به دلیل کارایی در فضاهای کنش پیوسته و عملکرد پایدار در محیط‌های پیچیده انتخاب شده‌اند. به‌طور خلاصه:
\begin{itemize}
  \item \lr{DDPG}: سیاست قطعی با شبکه‌های هدف و میانگین پلیاک؛ مناسب محیط‌های پیوسته با هزینه محاسباتی پایین‌تر، اما حساس به نویز.
  \begin{table}[H]
  	\caption{جدول پارامترها و مقادیر پیش‌فرض الگوریتم \lr{DDPG}
  		\cite{SpinningUp2018}}
  	\centering
  	\setlength{\tabcolsep}{8pt}
  	\renewcommand{\arraystretch}{0.95}
  	\begin{RTL}
  		\begin{tabular}{|c|c|c|c|}
  			\hline
  			\textbf{نام پارامتر} & \textbf{مقدار} & \textbf{نام پارامتر} & \textbf{مقدار} \\
  			\hline
  			گام در هر دوره یادگیری & $30\,000$ & تعداد دوره‌های یادگیری & $100$ \\
  			اندازه‌ی مخزنِ تجربه & $10^{6}$ &	ضریب تنزیل \((\gamma)\)& $0.99$ \\
  			ضریب میانگین پلیاک & $0.995$ & نرخِ یادگیریِ سیاست & $10^{-3}$ \\
  			نرخِ یادگیریِ \lr{Q} & $10^{-3}$ & اندازه‌ی دسته & $1024$ \\
  			گام‌ شروعِ استفاده از سیاست & $5\,000$ & گام شروعِ به‌روزرسانی& $1\,000$ \\
  			فاصله‌ی به‌روزرسانی & $2\,000$ & نویز عمل & $0.1$ \\
  			حداکثر طولِ رخداد & $6\,000$ & دستگاه & \lr{Cuda} \\
  			اندازه شبکه‌ی \lr{Actor}
  			& \((2^5, 2^5) \)  & تابع فعال‌سازی  \lr{Actor} & \lr{ReLU} \\
  			اندازه شبکه‌ی \lr{Critic}
  			& \( (2^5, 2^5) \)  & تابع فعال‌سازی  \lr{Critic} & \lr{ReLU} \\
  			\hline
  		\end{tabular}
  	\end{RTL}
  \end{table}
  \item \lr{TD3}: بهبود \lr{DDPG} با دو \lr{Critic}، هموارسازی سیاست هدف و به‌روزرسانی تأخیری سیاست؛ کاهش بیش‌براوردی \lr{Q} و پایداری بیشتر.
  \begin{table}[H]
  	\caption{جدول پارامترها و مقادیر پیش‌فرض الگوریتم \lr{TD3}
  		\cite{SpinningUp2018}}
  	\centering
  	\setlength{\tabcolsep}{8pt}
  	\renewcommand{\arraystretch}{0.95}
  	\begin{RTL}
  		\begin{tabular}{|c|c|c|c|}
  			\hline
  			\textbf{نام پارامتر} & \textbf{مقدار} & \textbf{نام پارامتر} & \textbf{مقدار} \\
  			\hline
  			گام در هر دوره یادگیری & $30\,000$ & تعداد دوره‌های یادگیری & $100$ \\
  			اندازه‌ی مخزنِ تجربه & $10^{6}$ &	ضریب تنزیل \((\gamma)\)& $0.99$ \\
  			ضریب میانگین پلیاک & $0.995$ & نرخِ یادگیریِ سیاست & $10^{-3}$ \\
  			نرخِ یادگیریِ \lr{Q} & $10^{-3}$ & اندازه‌ی دسته & $1024$ \\
  			گام‌ شروعِ استفاده از سیاست & $5\,000$ & گام شروعِ به‌روزرسانی & $1\,000$ \\
  			فاصله‌ی به‌روزرسانی & $2\,000$ & نویز عمل & $0.1$ \\
  			نویز هدف & $0.2$ & برش نویز & $0.5$ \\
  			تأخیر در به‌روزرسانی سیاست & $2$ &	حداکثر طولِ رخداد & $30\,000$  \\
  			اندازه شبکه‌ی \lr{Actor} & \( (2^5, 2^5) \) & تابع فعال‌سازی \lr{Actor} & \lr{ReLU} \\
  			اندازه شبکه‌ی \lr{Critic} & \( (2^5, 2^5) \) & تابع فعال‌سازی \lr{Critic} & \lr{ReLU} \\
  			\hline
  		\end{tabular}
  	\end{RTL}
  \end{table}
  \item \lr{SAC}: سیاست تصادفی بیشینه‌ساز آنتروپی با دمای \(\alpha\)؛ کاوش مؤثرتر و همگرایی پایدارتر در محیط‌های نویزی.
  \begin{table}[H]
  	\caption{جدول پارامترها و مقادیر پیش‌فرض الگوریتم \lr{SAC} 
  		\cite{SpinningUp2018}}
  	\centering
  	\setlength{\tabcolsep}{8pt}
  	\renewcommand{\arraystretch}{0.95}
  	\begin{RTL}
  		\begin{tabular}{|c|c|c|c|}
  			\hline
  			\textbf{نام پارامتر} & \textbf{مقدار} & \textbf{نام پارامتر} & \textbf{مقدار} \\
  			\hline
  			گام در هر دوره یادگیری & $30\,000$ & تعداد دوره‌های یادگیری & $100$ \\
  			اندازه‌ی مخزنِ تجربه & $10^{6}$ &	ضریب تنزیل \((\gamma)\)& $0.99$ \\
  			ضریب میانگین پلیاک & $0.995$ & نرخِ یادگیری & $10^{-3}$ \\
  			نرخ دمای آلفا & $0.2$ & اندازه‌ی دسته & $1024$ \\
  			گام‌ شروعِ استفاده از سیاست & $5\,000$ & گام شروعِ به‌روزرسانی & $1\,000$ \\
  			تعداد به‌روزرسانی در هر مرحله & $10$ & فاصله‌ی به‌روزرسانی & $2\,000$ \\
  			تعداد اپیزودهای آزمون & $10$ & حداکثر طولِ رخداد & $30\,000$ \\
  			اندازه شبکه‌ی \lr{Actor} & \( (2^5, 2^5) \) & تابع فعال‌سازی \lr{Actor} & \lr{ReLU} \\
  			اندازه شبکه‌ی \lr{Critic} & \( (2^5, 2^5) \) & تابع فعال‌سازی \lr{Critic} & \lr{ReLU} \\
  			\hline
  		\end{tabular}
  	\end{RTL}
  \end{table}
  \item \lr{PPO}: روش مبتنی بر سیاست با برش نسبت احتمال؛ به‌روزرسانی‌های ایمن و پیاده‌سازی ساده با کارایی تجربی بالا.
  \begin{table}[H]
  	\caption{جدول پارامترها و مقادیر پیش‌فرض الگوریتم \lr{PPO}
  		\cite{SpinningUp2018}}
  	\centering
  	\setlength{\tabcolsep}{8pt}
  	\renewcommand{\arraystretch}{0.95}
  	\begin{RTL}
  		\begin{tabular}{|c|c|c|c|}
  			\hline
  			\textbf{نام پارامتر} & \textbf{مقدار} & \textbf{نام پارامتر} & \textbf{مقدار} \\
  			\hline
  			گام در هر دوره یادگیری & $30\,000$ & تعداد دوره‌های یادگیری & $100$ \\
  			ضریب تنزیل \((\gamma)\) & $0.99$ & ضریب برش \(\text{clip ratio}\) & $0.2$ \\
  			نرخِ یادگیریِ سیاست & $\!3\times\!10^{-4}$ & نرخِ یادگیریِ تابع ارزش & $10^{-3}$ \\
  			تعداد تکرار آموزش سیاست & $80$ & تعداد تکرار آموزش ارزش & $80$ \\
  			اندازه شبکه‌ی \lr{Actor} & \( (2^5, 2^5) \) & تابع فعال‌سازی \lr{Actor} & \lr{ReLU} \\
  			اندازه شبکه‌ی \lr{Critic} & \( (2^5, 2^5) \) & تابع فعال‌سازی \lr{Critic} & \lr{ReLU} \\
  			\hline
  		\end{tabular}
  	\end{RTL}
  \end{table}
\end{itemize}
این الگوریتم‌ها به دلیل توانایی در مدیریت فضاهای پیوسته و عملکرد مؤثر در محیط‌های پیچیده انتخاب شده‌اند.

در شکل‌های \ref{fig:actor_nn} و \ref{fig:critic_nn} ساختار شبکه‌های \lr{Actor} و \lr{Critic} آورده شده‌است.
\begin{figure}[H]
	\centering
	\begin{tikzpicture}[x=2.4cm,y=1.2cm]
		\readlist\Nnod{4,5,5,2} % array of number of nodes per layer
		\readlist\Nstr{n,32,k} % array of string number of nodes per layer
		\readlist\Cstr{x,h^{(\prev)},u} % array of coefficient symbol per layer
		\def\yshift{0.55} % shift last node for dots
		% LOOP over LAYERS
		% LOOP over LAYERS
		\foreachitem \N \in \Nnod{
			\def\lay{\Ncnt} % alias of index of current layer
			\pgfmathsetmacro\prev{int(\Ncnt-1)} % number of previous layer
			\foreach \i [evaluate={\c=int(\i==\N);
				\layercheck=\ifnum\Ncnt=1 0 \else \ifnum\Ncnt=\Nnodlen 0 \else \yshift \fi \fi;
				\y=\N/2-\i*1.2-\c*\layercheck;
				\x=\lay; \n=\nstyle;
				\index=(\i<\N?int(\i):"\Nstr[\n]");}] in {1,...,\N}{ % loop over nodes
				% NODES
				\ifnum \lay=1
				\ifnum \i=1
				\node[node \n] (N\lay-\i) at (\x,\y) {$\delta x$};
				\fi
				\ifnum \i=2
				\node[node \n] (N\lay-\i) at (\x,\y) {$\delta y$};
				\fi
				\ifnum \i=3
				\node[node \n] (N\lay-\i) at (\x,\y) {$\delta {\dot{x}}$};
				\fi
				\ifnum \i=4
				\node[node \n] (N\lay-\i) at (\x,\y) {$\delta {\dot{y}}$};
				\fi
				\else \ifnum \lay=\Nnodlen
				\ifnum \i=1
				\node[node \n] (N\lay-\i) at (\x,\y) {$u_x$};
				\fi
				\ifnum \i=2
				\node[node \n] (N\lay-\i) at (\x,\y) {$u_y$};
				\fi
				\else
				\node[node \n] (N\lay-\i) at (\x,\y) {$\strut\Cstr[\n]_{\index}$};
				\fi \fi
				% CONNECTIONS
				\ifnumcomp{\lay}{>}{1}{ % connect to previous layer
					\foreach \j in {1,...,\Nnod[\prev]}{ % loop over nodes in previous layer
						\draw[white,line width=1.2,shorten >=1] (N\prev-\j) -- (N\lay-\i);
						\draw[connect] (N\prev-\j) -- (N\lay-\i);
					}
					%   \ifnum \lay=\Nnodlen
					%     \draw[connect] (N\lay-\i) --++ (0.5,0); % arrows out
					%   \fi
				}{
					%   \draw[connect] (0.5,\y) -- (N\lay-\i); % arrows in
				}
			}
			% Dots (skip first and last layers)
			\ifnum \lay>1 \ifnum \lay<\Nnodlen
			\path (N\lay-\N) --++ (0,1+\yshift) node[midway,scale=1.6] {$\vdots$}; % dots
			\fi \fi
		}
		% LABELS
		\node[above=.1,align=center,mydarkgreen] at (N1-1.90) {{لایه}\\[-0.6em]{ورودی}};
		\node[above=.1,align=center,mydarkblue] at (N2-1.90) {{لایه}\\[-0.6em]{پنهان}};
		\node[above=.1,align=center,mydarkblue] at (N3-1.90) {{لایه}\\[-0.6em]{پنهان}};
		\node[above=.1,align=center,mydarkred] at (N\Nnodlen-1.90) {{لایه}\\[-0.6em]{خروجی}};
	\end{tikzpicture}
	\caption{ساختار شبکه عصبی سیاست}
	\label{fig:actor_nn}
\end{figure}
\begin{figure}[H]
	\centering
\begin{tikzpicture}[x=2.8cm,y=1.5cm]
	\readlist\Nnod{6,7,7,1} % array of number of nodes per layer
	\readlist\Nstr{n,32,k} % array of string number of nodes per layer
	\readlist\Cstr{x,h^{(\prev)},u} % array of coefficient symbol per layer
	\def\yshift{0.55} % shift last node for dots
	
	% LOOP over LAYERS
	% LOOP over LAYERS
	\foreachitem \N \in \Nnod{
		\def\lay{\Ncnt} % alias of index of current layer
		\pgfmathsetmacro\prev{int(\Ncnt-1)} % number of previous layer
		\foreach \i [evaluate={\c=int(\i==\N); 
			\layercheck=\ifnum\Ncnt=1 0 \else \ifnum\Ncnt=\Nnodlen 0 \else \yshift \fi \fi;
			\y=\N/2-\i-\c*\layercheck;
			\x=\lay; \n=\nstyle;
			\index=(\i<\N?int(\i):"\Nstr[\n]");}] in {1,...,\N}{ % loop over nodes
			% NODES
			\ifnum \lay=1
			\ifnum \i=1
			\node[node \n] (N\lay-\i) at (\x,\y) {$\delta x$};
			\fi
			\ifnum \i=2
			\node[node \n] (N\lay-\i) at (\x,\y) {$\delta y$};
			\fi
			\ifnum \i=3
			\node[node \n] (N\lay-\i) at (\x,\y) {$\delta {\dot{x}}$};
			\fi
			\ifnum \i=4
			\node[node \n] (N\lay-\i) at (\x,\y) {$\delta {\dot{y}}$};
			\fi
			\ifnum \i=5
			\node[node \n] (N\lay-\i) at (\x,\y) {$u_x$};
			\fi
			\ifnum \i=6
			\node[node \n] (N\lay-\i) at (\x,\y) {$u_y$};
			\fi
			\else \ifnum \lay=\Nnodlen
			\ifnum \i=1
			\node[node \n] (N\lay-\i) at (\x,\y) {$Q$};
			\fi
			\ifnum \i=2
			\node[node \n] (N\lay-\i) at (\x,\y) {$u_y$};
			\fi
			\else
			\node[node \n] (N\lay-\i) at (\x,\y) {$\strut\Cstr[\n]_{\index}$};
			\fi \fi
			% CONNECTIONS
			\ifnumcomp{\lay}{>}{1}{ % connect to previous layer
				\foreach \j in {1,...,\Nnod[\prev]}{ % loop over nodes in previous layer
					\draw[white,line width=1.2,shorten >=1] (N\prev-\j) -- (N\lay-\i);
					\draw[connect] (N\prev-\j) -- (N\lay-\i);
				}
				%   \ifnum \lay=\Nnodlen
				%     \draw[connect] (N\lay-\i) --++ (0.5,0); % arrows out
				%   \fi
			}{
				%   \draw[connect] (0.5,\y) -- (N\lay-\i); % arrows in
			}
		}
		
		% Dots (skip first and last layers)
		\ifnum \lay>1 \ifnum \lay<\Nnodlen
		\path (N\lay-\N) --++ (0,1+\yshift) node[midway,scale=1.6] {$\vdots$}; % dots
		\fi \fi
	}
	
	
	
	% LABELS
	\node[above=.1,align=center,mydarkgreen] at (N1-1.90) {\lr{Input}\\[-0.6em]\lr{layer}};
	\node[above=.1,align=center,mydarkblue] at (N2-1.90) {\lr{Hidden}\\[-0.6em]\lr{layers}};
	\node[above=.1,align=center,mydarkblue] at (N3-1.90) {\lr{Hidden}\\[-0.6em]\lr{layers}};
	\node[above=.1,align=center,mydarkred] at (N\Nnodlen-1.90) {\lr{Output}\\[-0.6em]\lr{layer}};
\end{tikzpicture}
	\caption{ساختار شبکه عصبی نقاد}
	\label{fig:actor_nn}
\end{figure}

% جلوگیری از عبور شناورها از این نقطه (در صورت نبودِ بسته، خنثی است)
\providecommand{\FloatBarrier}{}
\FloatBarrier

% --- جداول در جای مرتبط با متن، با عنوان در بالا ---

%\subsubsection*{پارامترهای الگوریتم \lr{DDPG}}


%\subsubsection*{پارامترهای الگوریتم \lr{TD3}}


%\subsubsection*{پارامترهای الگوریتم \lr{SAC}}


%\subsubsection*{پارامترهای الگوریتم \lr{PPO}}


\subsection{فرآیند آموزش}

رویه آموزش با \lr{PyTorch} و اجرای \lr{Cuda} به‌صورت زیر انجام شده‌است:
\begin{enumerate}
  \item گردآوری تجربه‌ی اولیه با سیاست تصادفی تا رسیدن به گام شروع به‌روزرسانی برای پرشدن اولیه‌ی مخزن تجربه.
  \item حلقه‌ی یادگیری: در هر گام، اجرای کنش، ذخیره‌ی چهار‌تایی‌ها \((s,a,r,s')\) (و در صورت نیاز \(d\) برای پایان اپیزود) در مخزن تجربه با ظرفیت \(10^6\).
  \item نمونه‌گیری دسته داده  و به‌روزرسانی \lr{Critic}ها با هدف‌های حاوی شبکه‌های هدف و میانگین پلیاک؛ در \lr{TD3}
   استفاده از دو شبکه \lr{Q} مستقل و هدف‌های کمینه‌شده.
  \item به‌روزرسانی \lr{Actor}: در \lr{DDPG}/\lr{TD3} بیشینه‌سازی \(\mathbb{E}_s[Q(s,\pi_\theta(s))]\) و در \lr{SAC} بیشینه‌سازی بازگشت انتروپی‌دار؛ در \lr{PPO} به‌روزرسانی برش‌خورده با نسبت احتمال.
  \item تکنیک‌های پایداری: \lr{Target networks} با پلیاک، \lr{reward/observation normalization}، هموارسازی هدف \lr{TD3}، \lr{gradient clipping} در صورت نیاز، و بذردهی ثابت برای تکرارپذیری.
  \item ارزیابی دوره‌ای: اجرای چند اپیزود آزمون بدون نویز کنش و ثبت بازگشت، نرخ موفقیت و واریانس.
\end{enumerate}

برای جلوگیری از بیش‌برازش و همگرایی زودرس، از نویز کاوش کنش و هموارسازی سیاست هدف (در \lr{TD3}) استفاده شده‌است. معیار توقف زمانی فعال می‌شود که نرخ موفقیت آزمون در چند پنجره‌ی پیاپی از 90٪ عبور کند و واریانس بازگشت کاهش یابد.

\subsubsection*{بهینه‌سازی و پس‌انتشار گرادیان}
محاسبه‌ی گرادیان‌ها با \lr{autograd} انجام شده‌است. به‌روزرسانی پارامترها با \lr{Adam}  
\cite{kingma2017adammethodstochasticoptimization}
بوده است که در عمل نسبت به گرادیان نزولی ساده پایدارتر است:
\begin{align}
  g_t &= \nabla_{\!w} L_t,\quad
  m_t = \beta_1 m_{t-1} + (1-\beta_1) g_t,\quad
  v_t = \beta_2 v_{t-1} + (1-\beta_2) g_t^2 \nonumber\\
  \hat{m}_t &= \frac{m_t}{1-\beta_1^t},\quad
  \hat{v}_t = \frac{v_t}{1-\beta_2^t},\quad
  w_{t+1} = w_t - \eta \frac{\hat{m}_t}{\sqrt{\hat{v}_t} + \epsilon}
\end{align}
که در آن \(\eta\) نرخ یادگیری، \(\beta_1,\beta_2\) ضرایب مومنتوم (\(0.9,\,0.999\)) و \(\epsilon\) برای پایدارسازی عددی است. به‌صورت مفهومی، زنجیره گرادیان نیز برقرار است:
\begin{equation}
  \nabla_{\!w} L = \frac{\partial L}{\partial y}\,\frac{\partial y}{\partial w}
\end{equation}

%\paragraph{معیارهای ارزیابی.}
%\begin{itemize}
%  \item بازگشت تجمعی میانگین و صدک‌ها در اپیزودهای آزمون بدون نویز.
%  \item نرخ موفقیت سناریوی مأموریت‌-محور (تعریف دامنه‌محور کار)، با پنجره‌ی میانگین‌گیری متحرک.
%  \item پایداری سیاست: واریانس بازگشت و تعداد نقض قیود (در صورت وجود قیود).
%\end{itemize}





%\subsubsection*{بهینه‌سازی و پس‌انتشار گرادیان}
%محاسبه‌ی گرادیان‌ها با استفاده از \lr{autograd} صورت می‌گیرد که به‌صورت خودکار قاعده‌ی زنجیره‌ای مشتق‌گیری را اعمال می‌کند. به‌روزرسانی پارامترها بر پایه‌ی الگوریتم \lr{Adam} انجام شده‌است که یک روش تطبیقی و پایدارتر نسبت به گرادیان نزولی ساده به‌شمار می‌رود. فرمول‌های به‌روزرسانی در Adam به‌صورت زیر است:
%\begin{align}
%	g_t &= \nabla_{\!w} L_t,\quad
%	m_t = \beta_1 m_{t-1} + (1-\beta_1) g_t,\quad
%	v_t = \beta_2 v_{t-1} + (1-\beta_2) g_t^2 \nonumber\\
%	\hat{m}_t &= \frac{m_t}{1-\beta_1^t},\quad
%	\hat{v}_t = \frac{v_t}{1-\beta_2^t},\quad
%	w_{t+1} = w_t - \eta \frac{\hat{m}_t}{\sqrt{\hat{v}_t} + \epsilon}
%\end{align}

در این رابطه:
\begin{itemize}
	\item $L_t$: مقدار تابع هزینه (\lr{Loss}) در گام زمانی $t$.
	\item $w_t$: بردار وزن‌ها یا پارامترهای مدل در گام $t$.
	\item $g_t = \nabla_{\!w} L_t$: گرادیان تابع هزینه نسبت به پارامترها در زمان $t$.
	\item $m_t$: میانگین نمایی گرادیان‌ها (مومنتوم مرتبه اول) که حافظه‌ای از جهت گرادیان‌ها ایجاد می‌کند.
	\item $v_t$: میانگین نمایی مربعات گرادیان‌ها (مومنتوم مرتبه دوم) که بزرگی تغییرات گرادیان را ثبت می‌کند.
	\item $\hat{m}_t,\,\hat{v}_t$: نسخه‌های اصلاح‌شده‌ی بایاس برای $m_t$ و $v_t$ به‌منظور پایدارسازی در مراحل اولیه.
	\item $\eta$: نرخ یادگیری (\lr{Learning Rate}) که اندازه‌ی گام به‌روزرسانی وزن‌ها را مشخص می‌کند.
	\item $\beta_1,\,\beta_2$: ضرایب کاهش (\lr{Decay Rates}) برای میانگین‌گیری نمایی؛ مقادیر معمول آن‌ها به‌ترتیب $0.9$ و $0.999$ است.
	\item $\epsilon$: یک مقدار بسیار کوچک (معمولاً $10^{-8}$) برای جلوگیری از تقسیم بر صفر و افزایش پایداری عددی.
\end{itemize}

الگوریتم \lr{Adam} به این صورت عمل می‌کند که همزمان از میانگین مرتبه‌ی اول ($m_t$) برای جهت حرکت و از میانگین مرتبه‌ی دوم ($v_t$) برای تنظیم نرخ یادگیری هر پارامتر استفاده می‌کند. در نتیجه هم از نوسانات شدید جلوگیری می‌شود و هم فرآیند همگرایی سرعت می‌گیرد.  

از دیدگاه محاسبه‌ی گرادیان، زنجیره‌ی مشتق‌گیری (قاعده‌ی زنجیره‌ای) نیز برقرار است:
\begin{equation}
	\nabla_{\!w} L = \frac{\partial L}{\partial y}\,\frac{\partial y}{\partial w}
\end{equation}
که در آن $y$ خروجی لایه یا شبکه است. این فرمول مبنای پس‌انتشار خطا (\lr{Backpropagation}) در شبکه‌های عصبی محسوب می‌شود و باعث می‌گردد که گرادیان تابع هزینه نسبت به تمامی پارامترها به‌صورت کارآمد محاسبه شود.

