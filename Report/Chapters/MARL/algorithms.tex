\subsection{الگوریتم‌های یادگیری تقویتی چندعاملی}

در حوزه یادگیری تقویتی چندعاملی (\lr{Multi-Agent Reinforcement Learning - MARL}), توسعه الگوریتم‌های خاص برای تعامل و همکاری میان عوامل مختلف از اهمیت بالایی برخوردار است. الگوریتم‌های MARL باید بتوانند با پیچیدگی‌ها و چالش‌های ناشی از تعاملات متعدد میان عوامل، عملکرد بهینه‌ای را ارائه دهند. در این بخش، به بررسی چند الگوریتم پیشرفته چندعاملی پرداخته می‌شود که هر یک ویژگی‌ها و مزایای خاص خود را دارند.

\subsubsection{الگوریتم Multi-Agent Deep Deterministic Policy Gradient (MADDPG)}

الگوریتم MADDPG (\lr{Multi-Agent Deep Deterministic Policy Gradient}) به عنوان یکی از الگوریتم‌های پیشرفته در MARL معرفی شده است که برای تعاملات همکاری میان عوامل طراحی گردیده است. این الگوریتم بر پایه الگوریتم DDPG توسعه یافته و از شبکه‌های عصبی عمیق برای یادگیری سیاست‌های بهینه استفاده می‌نماید. ویژگی اصلی MADDPG استفاده از اطلاعات دیگر عوامل در فرآیند یادگیری است که بهبود هماهنگی و همکاری میان عوامل را ممکن می‌سازد.

\paragraph{تعریف ریاضی}
در MADDPG، برای هر عامل \( i \) سیاست (\lr{Policy}) و تابع ارزش (\lr{Critic}) جداگانه‌ای تعریف شده است. فرمول به‌روزرسانی سیاست برای هر عامل به صورت زیر است:
\[
\nabla_{\theta_i} J(\theta_i) = \mathbb{E}_{s,a \sim \mathcal{D}} \left[ \nabla_{\theta_i} Q_i^{\mu}(s,a) \big|_{a=\mu_i(s|\theta_i)} \cdot \nabla_{\theta_i} \mu_i(s|\theta_i) \right]
\]
که در آن \( Q_i^{\mu}(s,a) \) تابع ارزش عملیات برای عامل \( i \) و \( \mu_i(s|\theta_i) \) سیاست عامل \( i \) می‌باشد.

\paragraph{ویژگی‌ها و مزایا}
\begin{itemize}
	\item \textbf{هماهنگی و همکاری بهتر:} استفاده از اطلاعات دیگر عوامل به بهبود هماهنگی و همکاری میان عوامل کمک می‌نماید.
	\item \textbf{قابلیت مقیاس‌پذیری:} توانایی کارکرد در محیط‌های با تعداد زیادی عامل.
	\item \textbf{پشتیبانی از سیاست‌های پیوسته:} مناسب برای مسائل با فضای عمل پیوسته.
\end{itemize}

\subsubsection{الگوریتم Multi-Agent Soft Actor-Critic (MASAC)}

الگوریتم MASAC (\lr{Multi-Agent Soft Actor-Critic}) یکی از الگوریتم‌های پیشرفته در MARL است که بر پایه SAC توسعه یافته و برای تعاملات همکاری و رقابتی میان عوامل طراحی گردیده است. این الگوریتم با بهینه‌سازی انتروپی به همراه تابع ارزش، تعادل بهتری بین اکتشاف و بهره‌برداری ایجاد می‌نماید.

\paragraph{تعریف ریاضی}
در MASAC، تابع هدف برای هر عامل به صورت زیر تعریف می‌گردد:
\[
J(\pi_i) = \mathbb{E}_{(s,a) \sim \rho_{\pi}} \left[ \alpha \mathcal{H}(\pi_i(\cdot|s)) - Q_i^{\phi}(s,a) \right]
\]
که در آن \( \mathcal{H}(\pi_i(\cdot|s)) \) انتروپی سیاست و \( \alpha \) ضریب تنظیم تعادل بین انتروپی و تابع ارزش است.

\paragraph{ویژگی‌ها و مزایا}
\begin{itemize}
	\item \textbf{تعادل بین اکتشاف و بهره‌برداری:} با بهینه‌سازی انتروپی، اکتشاف در فضای عمل افزایش می‌یابد.
	\item \textbf{پایداری بیشتر:} استفاده از تکنیک‌های مختلف برای افزایش پایداری فرآیند یادگیری.
	\item \textbf{پشتیبانی از محیط‌های پیچیده:} قابلیت یادگیری در محیط‌های با تعاملات پیچیده میان عوامل.
\end{itemize}

\subsubsection{الگوریتم Multi-Agent Proximal Policy Optimization (MA-PPO)}

الگوریتم MA-PPO (\lr{Multi-Agent Proximal Policy Optimization}) به عنوان یک توسعه‌یافته از PPO برای محیط‌های چندعاملی معرفی شده است. این الگوریتم با استفاده از محدودیت‌های نزدیک به سیاست جدید، کارایی و پایداری بالایی در تعاملات میان عوامل فراهم می‌آورد.

\paragraph{تعریف ریاضی}
تابع هدف MA-PPO به صورت زیر تعریف می‌گردد:
\[
L^{CLIP}(\theta_i) = \mathbb{E}_t \left[ \min \left( r_t(\theta_i) \hat{A}_t, \text{clip}(r_t(\theta_i), 1-\epsilon, 1+\epsilon) \hat{A}_t \right) \right]
\]
که در آن \( r_t(\theta_i) \) نسبت احتمالات سیاست جدید به سیاست قدیمی و \( \hat{A}_t \) برآورد مزایای عامل در زمان \( t \) می‌باشد.

\paragraph{ویژگی‌ها و مزایا}
\begin{itemize}
	\item \textbf{سادگی در پیاده‌سازی:} نسبت به الگوریتم‌های پیچیده‌تر، پیاده‌سازی آسان‌تری دارد.
	\item \textbf{پایداری بالا:} با محدود کردن تغییرات سیاست، از نوسانات بزرگ در فرآیند یادگیری جلوگیری می‌شود.
	\item \textbf{کارایی بالا:} عملکرد قابل توجهی در مسائل مختلف MARL دارد.
\end{itemize}

\subsubsection{الگوریتم Multi-Agent Twin Delayed Deep Deterministic Policy Gradient (MA-TD3)}

الگوریتم MA-TD3 (\lr{Multi-Agent Twin Delayed Deep Deterministic Policy Gradient}) به عنوان یک توسعه‌یافته از TD3 برای محیط‌های چندعاملی معرفی شده است. این الگوریتم با استفاده از دو شبکه ارزش مجزا و به‌روزرسانی تأخیری سیاست، کارایی و پایداری بالایی را در تعاملات میان عوامل فراهم می‌آورد.

\paragraph{تعریف ریاضی}
تابع هدف MA-TD3 به صورت زیر تعریف می‌گردد:
\[
\mathcal{L}(\phi_i) = \mathbb{E}_{(s,a,r,s') \sim \mathcal{D}} \left[ \left( Q_i(s,a|\theta_i^Q) - \left( r + \gamma \min_{j=1,2} Q_j(s', \mu_i(s'|\theta_i^\mu)|\theta_j^{Q'}) \right) \right)^2 \right]
\]
که در آن \( i = 1, 2 \).

\paragraph{ویژگی‌ها و مزایا}
\begin{itemize}
	\item \textbf{کاهش نوسانات تخمین تابع ارزش:} استفاده از دو شبکه ارزش برای کاهش تخمین‌های بیش از حد.
	\item \textbf{به‌روزرسانی تأخیری شبکه سیاست:} افزایش پایداری و کاهش نوسانات در فرآیند یادگیری.
	\item \textbf{پشتیبانی از فضای عمل پیوسته:} حفظ قابلیت کارکرد در مسائل با فضای عمل پیوسته.
\end{itemize}

\paragraph{ویژگی‌های مشترک}
الگوریتم‌های MADDPG، MASAC، MA-PPO و MA-TD3 هر کدام با توجه به ویژگی‌های خاص خود، توانسته‌اند در محیط‌های پیچیده و پویا با تعاملات متنوع میان عوامل، عملکرد قابل توجهی را ارائه دهند. این الگوریتم‌ها با استفاده از شبکه‌های عصبی عمیق، تعادل بین اکتشاف و بهره‌برداری را بهبود داده و قابلیت‌های مقاوم‌سازی سیستم‌ها را افزایش داده‌اند.

\paragraph{کاربردها}
الگوریتم‌های یادگیری تقویتی چندعاملی در حوزه‌های مختلفی مانند سامانه‌های خودران، مدیریت انرژی در شبکه‌های هوشمند، ربات‌های همکاری‌کننده و محیط‌های صنعتی مورد استفاده قرار می‌گیرند. این الگوریتم‌ها با بهینه‌سازی سیاست‌ها و کاهش نوسانات در فرآیند یادگیری، امکان همکاری و تعامل موثر میان عوامل را فراهم می‌آورند.

\paragraph{چالش‌ها و فرصت‌ها}
هر کدام از این الگوریتم‌ها با چالش‌های خاص خود مواجه هستند، از جمله نیاز به تنظیم دقیق پارامترها، محاسبه توابع ارزش پیچیده‌تر و افزایش هزینه محاسباتی. با این حال، فرصت‌های زیادی برای بهبود و توسعه این الگوریتم‌ها وجود دارد که می‌تواند به افزایش کارایی و قابلیت‌های یادگیری تقویتی چندعاملی کمک کند.

\subsubsection{نتیجه‌گیری}

الگوریتم‌های یادگیری تقویتی چندعاملی نقش مهمی در توسعه سیستم‌های هوشمند و خودکار ایفا می‌نمایند. الگوریتم‌های مختلف مانند MADDPG، MASAC، MA-PPO و MA-TD3 با ویژگی‌ها و مزایای منحصر به فرد خود، توانسته‌اند در محیط‌های پیچیده و پویا عملکرد قابل توجهی داشته باشند. این الگوریتم‌ها با بهینه‌سازی سیاست‌ها و کاهش نوسانات در فرآیند یادگیری، امکان همکاری و تعامل موثر میان عوامل را فراهم می‌آورند. با ادامه تحقیقات و بهبود این الگوریتم‌ها، انتظار می‌رود که MARL نقش کلیدی‌تری در پیشرفت تکنولوژی‌های هوشمند و خودکار ایفا نماید.
