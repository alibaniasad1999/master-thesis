\section{تعریف یادگیری تقویتی چندعاملی}
یادگیری تقویتی چندعاملی (\lr{Multi-Agent Reinforcement Learning - MARL}) شاخه‌ای از یادگیری ماشین است که به مطالعه و توسعه الگوریتم‌هایی می‌پردازد که در آن‌ها چندین عامل به طور همزمان در یک محیط تعامل می‌کنند. هر عامل با هدف به حداکثر رساندن پاداش خود، سیاست‌های خود را بهبود می‌بخشد. در \lr{MARL}، تعاملات میان عامل‌ها می‌تواند شامل همکاری، رقابت یا هر دو باشد که این امر چالش‌ها و فرصت‌های جدیدی را به همراه دارد.

\subsection{مفاهیم پایه در یادگیری تقویتی چندعاملی}
یادگیری تقویتی چندعاملی ترکیبی از یادگیری تقویتی و سیستم‌های چندعاملی است. در اینجا به برخی از مفاهیم پایه پرداخته می‌شود:

\subsubsection{عامل‌ها و محیط}
در \lr{MARL}، چندین عامل (\lr{Agents}) در یک محیط مشترک فعالیت می‌کنند. هر عامل در هر لحظه زمانی، وضعیت محیط (\lr{State}) را مشاهده کرده و بر اساس آن اقدام (\lr{Action})ی انجام می‌دهد. محیط سپس بازخوردی به صورت پاداش (\lr{Reward}) و وضعیت جدید ارائه می‌دهد.

\subsubsection{سیاست‌ها و اهداف}
هر عامل دارای یک سیاست (\lr{Policy}) است که نقشه‌ای از وضعیت‌ها به اقدامات را فراهم می‌کند. هدف هر عامل در \lr{MARL} به حداکثر رساندن مجموع پاداش‌های دریافتی در طول زمان است. این اهداف می‌توانند همزمان شامل اهداف فردی و گروهی باشند.

\subsection{تعاملات میان عامل‌ها}
در \lr{MARL}، تعاملات میان عامل‌ها نقش مهمی در یادگیری و عملکرد سیستم دارند. این تعاملات می‌توانند به صورت‌های مختلفی ظاهر شوند:

\subsubsection{همکاری}
در تعاملات همکاری، عامل‌ها به منظور دستیابی به اهداف مشترک با یکدیگر همکاری می‌کنند. این نوع تعاملات معمولاً در مسائلی که نیاز به هماهنگی و اشتراک‌گذاری اطلاعات دارند، مشاهده می‌شود. به عنوان مثال، در یک تیم رباتیک که باید با هم کار کنند تا یک شیء را جابجا کنند.

\subsubsection{رقابت}
در تعاملات رقابتی، عامل‌ها با یکدیگر به رقابت می‌پردازند تا منابع محدود یا اهداف مشخص را به دست آورند. این نوع تعاملات معمولاً در بازی‌های استراتژیک یا سیستم‌های بازار مشاهده می‌شود.

\subsubsection{ترکیبی از همکاری و رقابت}
در بسیاری از موارد، تعاملات میان عامل‌ها ترکیبی از همکاری و رقابت هستند. به عنوان مثال، در یک بازی تیمی، اعضای هر تیم با یکدیگر همکاری می‌کنند در حالی که با تیم‌های دیگر رقابت می‌نمایند.

\subsection{تفاوت‌های MARL با یادگیری تقویتی تک عاملی}
یادگیری تقویتی چندعاملی تفاوت‌های مهمی با یادگیری تقویتی تک عاملی دارد که در زیر به برخی از آن‌ها اشاره می‌شود:

\subsubsection{دینامیک محیط}
در \lr{MARL}، محیط به طور همزمان توسط چندین عامل تحت تأثیر قرار می‌گیرد، که این امر دینامیک محیط را پیچیده‌تر می‌کند. در مقابل، در یادگیری تقویتی تک عاملی، تنها یک عامل وجود دارد که محیط را تحت تأثیر قرار می‌دهد.

\subsubsection{تداخل و ناپایداری}
وجود چندین عامل می‌تواند منجر به تداخل و ناپایداری در یادگیری شود، زیرا هر عامل ممکن است سیاست‌های خود را تغییر دهد که تأثیر مستقیم بر سایر عامل‌ها دارد. این امر به ایجاد یک محیط غیر ثابت و چالش‌برانگیز برای یادگیری منجر می‌شود.

\subsubsection{هماهنگی و ارتباطات}
در \lr{MARL}، عامل‌ها ممکن است نیاز به هماهنگی و ارتباط با یکدیگر داشته باشند تا به اهداف مشترک دست یابند. این امر نیازمند مکانیزم‌های ارتباطی و هماهنگی موثر بین عامل‌ها است که در یادگیری تقویتی تک عاملی وجود ندارد.

\subsection{چالش‌های یادگیری تقویتی چندعاملی}
یادگیری تقویتی چندعاملی با چالش‌های خاصی مواجه است که باید برای رسیدن به عملکرد مطلوب، آن‌ها را مدیریت کرد:

\subsubsection{مسئله همگرایی}
همگرایی الگوریتم‌های \lr{MARL} پیچیده‌تر از الگوریتم‌های تک عاملی است زیرا تعاملات میان عامل‌ها می‌تواند منجر به رفتارهای غیرپیش‌بینی‌پذیر و ناپایدار شود. تضمین همگرایی به بهینه یا حداقل مطلوب بودن سیاست‌ها یکی از چالش‌های اصلی است.

\subsubsection{مدیریت تعارضات}
در محیط‌های رقابتی، تعارضات میان عامل‌ها می‌تواند به کاهش کارایی سیستم منجر شود. طراحی مکانیزم‌هایی برای مدیریت تعارضات و ایجاد تعادل میان اهداف فردی و گروهی ضروری است.

\subsubsection{پایداری و مقیاس‌پذیری}
با افزایش تعداد عامل‌ها، حفظ پایداری و مقیاس‌پذیری سیستم یک چالش مهم است. الگوریتم‌های \lr{MARL} باید بتوانند به طور موثری با افزایش تعداد عامل‌ها سازگار شوند بدون اینکه عملکرد سیستم کاهش یابد.

\subsubsection{اطلاعات ناقص و عدم قطعیت}
در بسیاری از سناریوهای \lr{MARL}، عامل‌ها ممکن است با اطلاعات ناقص یا عدم قطعیت در مورد سیاست‌های دیگر عامل‌ها مواجه شوند. طراحی الگوریتم‌هایی که بتوانند در شرایط عدم قطعیت به خوبی عمل کنند، بسیار مهم است.

\subsection{کاربردهای یادگیری تقویتی چندعاملی}
یادگیری تقویتی چندعاملی در حوزه‌های متعددی کاربرد دارد که در زیر به برخی از آن‌ها اشاره می‌شود:

\subsubsection{سیستم‌های رباتیک}
در سیستم‌های رباتیک چندعاملی، ربات‌ها برای انجام وظایف پیچیده به صورت هماهنگ با یکدیگر کار می‌کنند. این وظایف می‌تواند شامل جست‌وجو و نجات، حمل و نقل مواد، یا عملیات هماهنگ در محیط‌های غیرقابل پیش‌بینی باشد.

\subsubsection{مدیریت منابع در شبکه‌های ارتباطی}
در شبکه‌های ارتباطی، تخصیص بهینه منابع مانند پهنای باند و انرژی به عامل‌های مختلف (مانند دستگاه‌های کاربر) می‌تواند با استفاده از \lr{MARL} بهبود یابد. این الگوریتم‌ها می‌توانند به طور پویا و خودکار تخصیص منابع را بهینه کنند.

\subsubsection{بازی‌ها و شبیه‌سازی‌های اقتصادی}
در بازی‌های چندعاملی و شبیه‌سازی‌های اقتصادی، \lr{MARL} می‌تواند به مدل‌سازی و تحلیل رفتارهای بازار و تصمیم‌گیری‌های اقتصادی کمک کند. این کاربردها به پیش‌بینی دقیق‌تر روندهای اقتصادی و بهبود سیاست‌گذاری‌های مالی منجر می‌شوند.

\subsection{مبنای نظریه بازی در MARL}
یکی از مبانی نظری \lr{MARL}، نظریه بازی‌ها است که به تحلیل تعاملات میان عامل‌ها در محیط‌های رقابتی و همکاری می‌پردازد. استفاده از مفاهیم تعادل نش (\lr{Nash Equilibrium}) و دیگر تعادل‌های بازی‌ها، به طراحی الگوریتم‌های \lr{MARL} کمک می‌کند تا رفتارهای پایدار و بهینه‌ای را در تعاملات میان عامل‌ها تضمین نمایند.

%\subsection{نتیجه‌گیری}
%یادگیری تقویتی چندعاملی به عنوان یک حوزه پیشرفته در یادگیری ماشین، قابلیت‌های بالایی در مدل‌سازی و حل مسائل پیچیده و پویا ارائه می‌دهد. با وجود چالش‌های متعدد، توسعه الگوریتم‌های موثر و مدیریت تعاملات میان عامل‌ها می‌تواند به بهبود عملکرد سیستم‌های هوشمند و خودکار منجر شود. کاربردهای گسترده \lr{MARL} در حوزه‌های مختلف علمی و صنعتی نشان‌دهنده اهمیت و پتانسیل بالای این حوزه است.

