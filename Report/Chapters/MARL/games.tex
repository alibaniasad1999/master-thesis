\section
{
 نظریه بازی‌ها
}\label{sec:marl_games}
نظریه بازی‌ها شاخه‌ای از ریاضیات است که به مطالعه تصمیم‌گیری در موقعیت‌هایی می‌پردازد که نتیجه انتخاب‌های هر فرد به تصمیمات دیگران وابسته است. این نظریه چارچوبی برای تحلیل تعاملات میان بازیکنان ارائه می‌دهد و در حوزه‌های مختلفی مانند اقتصاد، علوم سیاسی، زیست‌شناسی و علوم کامپیوتر کاربرد دارد. 
در این بخش، دو مفهوم کلیدی نظریه‌ی بازی‌ها، یعنی تعادل نش و بازی‌های مجموع‌صفر، بررسی می‌شوند.