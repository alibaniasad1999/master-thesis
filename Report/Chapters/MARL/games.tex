\subsection{بازی‌های جمع صفر}

بازی‌های جمع صفر\LTRfootnote{Zero-Sum}
 یکی از انواع اصلی بازی‌های چندعاملی هستند که در آن سود یک بازیکن به طور مستقیم با ضرر بازیکنان دیگر مرتبط است. در این بازی‌ها، مجموع پاداش‌ها برای همه بازیکنان در هر حالت برابر با صفر است، به این معنی که هر افزایشی در پاداش یکی از بازیکنان منجر به کاهش معادل آن در بازیکنان دیگر می‌شود. این نوع بازی‌ها به خوبی می‌توانند رقابت‌های شدید و استراتژی‌های بهینه را مدل‌سازی کنند.

بازی‌های جمع صفر می‌توانند بر اساس سناریوهای مختلف به دسته‌های متنوعی تقسیم‌بندی شوند. دو دسته اصلی این بازی‌ها عبارتند از بازی‌های ثابت و بازی‌های تکراری.

\begin{itemize}
	\item \textbf{بازی ثابت (\lr{Static Game}):} بازی ثابت ساده‌ترین شکل برای مدل‌سازی تعاملات میان عوامل است. در بازی ثابت، هر عامل تنها یک تصمیم‌گیری واحد را انجام می‌دهد. از آنجایی که هر عامل تنها یک بار عمل می‌کند، تقلب و خیانت غیرمنتظره می‌تواند در این نوع بازی‌ها سودآور باشد. بنابراین، هر عامل نیاز دارد تا به دقت استراتژی‌های سایر عوامل را پیش‌بینی کند تا بتواند به طور هوشمندانه عمل کرده و بیشترین سود ممکن را کسب کند. بازی‌های ثابت معمولاً در سناریوهای رقابتی با تعاملات کوتاه مدت کاربرد دارند.
	
	\item \textbf{بازی تکراری (\lr{Repeated Game}):} بازی تکراری به وضعیتی اشاره دارد که در آن تمام عوامل می‌توانند بر اساس همان وضعیت برای چندین تکرار اقداماتی انجام دهند. سود کلی هر عامل مجموع سودهای تخفیف‌شده برای هر تکرار از بازی است. به دلیل اقدامات مکرر تمام عوامل، تقلب و خیانت در طول تعاملات می‌تواند منجر به مجازات یا انتقام از سوی سایر عوامل در تکرارهای آینده شود. بنابراین، بازی تکراری از رفتارهای مخرب عوامل جلوگیری می‌کند و به طور کلی سود کل برای تمام عوامل را افزایش می‌دهد. بازی‌های تکراری معمولاً در سناریوهای همکاری بلندمدت و تعاملات پویا کاربرد دارند.
\end{itemize}

این دسته‌بندی‌ها به محققان و توسعه‌دهندگان کمک می‌کنند تا بازی‌های چندعاملی را بر اساس ویژگی‌های مختلف آن‌ها شناسایی و تحلیل کنند. در بازی‌های ثابت، تمرکز بر پیش‌بینی دقیق استراتژی‌های دیگر عوامل و اتخاذ بهترین تصمیم در یک لحظه زمانی است. در مقابل، بازی‌های تکراری نیازمند توسعه استراتژی‌های پایدار و قابل اعتماد هستند که نه تنها در تکرار اول بلکه در تکرارهای بعدی نیز موثر باشند.

بازی‌های جمع صفر در یادگیری تقویتی چندعاملی به دلیل سادگی و قابلیت مدل‌سازی دقیق تعاملات رقابتی، به عنوان یک ابزار قدرتمند برای تحلیل و توسعه الگوریتم‌های \lr{MARL} مورد استفاده قرار می‌گیرند. این بازی‌ها امکان بررسی رفتارهای استراتژیک، بهینه‌سازی سیاست‌ها و تحلیل تعادل‌های نش (\lr{Nash Equilibrium}) را فراهم می‌کنند که در نهایت به بهبود عملکرد سیستم‌های چندعاملی منجر می‌شود.

\paragraph{مثال‌ها و کاربردها}
یکی از مثال‌های معروف بازی‌های جمع صفر، بازی شطرنج است که در آن هر حرکت یک بازیکن مستقیماً به نفع یا ضرر بازیکن دیگر است. سایر مثال‌ها شامل بازی‌های استراتژیک مانند \lr{Poker} و \lr{Go} می‌باشند که در آن‌ها تعاملات رقابتی میان بازیکنان به طور کامل با اصول بازی‌های جمع صفر مطابقت دارند.

در حوزه‌های عملی، بازی‌های جمع صفر می‌توانند برای مدل‌سازی رقابت‌های بازار، مذاکرات اقتصادی و حتی تعاملات میان ربات‌های خودران در محیط‌های رقابتی مورد استفاده قرار گیرند. این کاربردها به محققان امکان می‌دهند تا الگوریتم‌هایی طراحی کنند که قادر به بهینه‌سازی عملکرد در شرایط رقابتی و متغیر باشند.

\paragraph{چالش‌ها و فرصت‌ها}
یکی از چالش‌های اصلی در بازی‌های جمع صفر، پیش‌بینی دقیق رفتارهای رقبا و اتخاذ تصمیم‌های بهینه در مواجهه با استراتژی‌های متغیر آن‌ها است. همچنین، در بازی‌های تکراری، ایجاد تعادل‌های پایدار و جلوگیری از رفتارهای مخرب به عنوان یک چالش مهم مطرح است. با این حال، این چالش‌ها فرصت‌های قابل توجهی برای توسعه الگوریتم‌های پیشرفته و افزایش قابلیت‌های یادگیری تقویتی چندعاملی فراهم می‌کنند که می‌توانند در شرایط پیچیده‌تر و پویا نیز عملکرد مطلوبی داشته باشند.
