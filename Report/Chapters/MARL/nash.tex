\subsection{تعادل نش}
تعادل نش\LTRfootnote{Nash Equilibrium}
 یکی از بنیادی‌ترین مفاهیم در نظریه بازی‌ها است که توسط جان نش در سال 1950 معرفی شد. این مفهوم به مجموعه‌ای از ‌ها اشاره دارد که در آن هیچ بازیکنی نمی‌تواند با تغییر یک‌جانبه سیاست خود، سود بیشتری به دست آورد، به شرطی که سیاست‌های سایر بازیکنان ثابت بماند.


\begin{itemize}
	\item تعریف تعادل نش:
	فرض کنید یک بازی با \( n \) بازیکن داریم. هر بازیکن \( i \) دارای مجموعه سیاست‌های \( \Pi_i \) و تابع مطلوبیت \( u_i: \Pi_1 \times \Pi_2 \times \cdots \times \Pi_n \rightarrow \mathbb{R} \) است. یک مجموعه سیاست \( \pi^* = (\pi_1^*, \pi_2^*, \ldots, \pi_n^*) \) تعادل نش نامیده می‌شود اگر برای هر بازیکن \( i \) و هر سیاست \( \pi_i \in \Pi_i \)
	 در وضعیت \(s\)
	 داشته باشیم:
\begin{equation}
		u_i(\pi_i^*, \pi_{-i}^*, s) \geq u_i(\pi_i, \pi_{-i}^*, s)
\end{equation}
	
	در اینجا، \( \pi_{-i}^* \) نشان‌دهنده سیاست‌های همه بازیکنان به جز بازیکن \( i \) است. در ادامه پژوهش جهت استفاده از چارچوب نظریه بازی در یادگیری تقویتی تابع مطلوبیت به‌گونه‌ای تعریف شده است که برابر با تابع ارزش
	\(u_i(\pi_i, \pi_{-i}, s)
	= V_i^{\pi_i, \pi_{-i}}(s)\)
	 باشد.
	
	\item اهمیت تعادل نش:
	تعادل نش نقطه‌ای را در بازی مشخص می‌کند که هر بازیکن بهترین پاسخ را نسبت به انتخاب‌های دیگران ارائه داده است. این مفهوم به‌ویژه در بازی‌های غیرهمکارانه، به‌عنوان پیش‌بینی رفتار منطقی بازیکنان استفاده می‌شود و در زمینه‌هایی مانند یادگیری تقویتی چند عامله کاربرد گسترده‌ای دارد.
\end{itemize}



