\subsection{تعادل نش}
تعادل نش (\lr{Nash Equilibrium}) یکی از مفاهیم بنیادی در نظریه بازی‌ها است که به تحلیل تعاملات میان عوامل در محیط‌های رقابتی و همکاری می‌پردازد. این مفهوم به ما امکان می‌دهد تا نقاط تعادلی را شناسایی کنیم که در آن هیچ کدام از عوامل نمی‌توانند با تغییر یکجانبه استراتژی خود سود بیشتری کسب کنند، مشروط بر این که سایر عوامل استراتژی‌های خود را ثابت نگه دارند. در این بخش، به دو نوع اصلی تعادل نش، یعنی تعادل نش خالص و تعادل نش ترکیبی، پرداخته می‌شود.

\subsubsection{تعادل نش خالص (\lr{Pure Strategy Nash Equilibrium})}
تعادل نش خالص به حالتی از بازی گفته می‌شود که در آن هر عامل یک استراتژی مشخص و غیر تصادفی را انتخاب کرده است و هیچ کدام از عوامل نمی‌توانند با تغییر استراتژی خود بدون تغییر استراتژی‌های سایر عوامل سود بیشتری کسب کنند. به عبارت دیگر، در یک تعادل نش خالص، هیچ عاملی انگیزه‌ای برای انحراف از استراتژی انتخابی خود ندارد.

\paragraph{تعریف ریاضی}
فرض کنید یک بازی با \( n \) بازیکن داریم که هر بازیکن \( i \) دارای مجموعه‌ای از استراتژی‌ها \( S_i \) است. تابع سود هر بازیکن \( i \) به صورت \( u_i: S_1 \times S_2 \times \dots \times S_n \rightarrow \mathbb{R} \) تعریف می‌شود. یک پروفایل استراتژی \( (s_1^*, s_2^*, \dots, s_n^*) \) تعادل نش خالص است اگر برای هر بازیکن \( i \) و برای هر استراتژی \( s_i \in S_i \)، شرط زیر برقرار باشد:
\[
u_i(s_i^*, s_{-i}^*) \geq u_i(s_i, s_{-i}^*)
\]
که در آن \( s_{-i}^* \) نشان‌دهنده مجموعه استراتژی‌های سایر بازیکنان است.

\paragraph{مثال}
یک مثال کلاسیک از تعادل نش خالص، بازی \lr{Prisoner's Dilemma} است. در این بازی، هر دو بازیکن انتخاب می‌کنند که همکاری کنند یا خیانت کنند. اگر هر دو بازیکن همکاری کنند، هر کدام دو سال زندانی می‌شوند. اگر یکی خیانت کند و دیگری همکاری کند، خیانتکار بی‌نیازی و همکار یک سال زندانی می‌شود. اگر هر دو خیانت کنند، هر کدام سه سال زندانی می‌شوند. تعادل نش خالص در این بازی زمانی رخ می‌دهد که هر دو بازیکن انتخاب به خیانت کردن هستند، زیرا هیچ یک از بازیکنان نمی‌توانند با تغییر استراتژی خود سود بیشتری کسب کنند.

\subsubsection{تعادل نش ترکیبی (\lr{Mixed Strategy Nash Equilibrium})}
تعادل نش ترکیبی به حالتی از بازی گفته می‌شود که در آن هر عامل می‌تواند با احتمال معینی بین استراتژی‌های مختلف خود انتخاب کند. در این نوع تعادل نش، هیچ کدام از عوامل نمی‌توانند با تغییر احتمالات انتخاب استراتژی خود بدون تغییر احتمالات انتخاب استراتژی‌های سایر عوامل سود بیشتری کسب کنند.

\paragraph{تعریف ریاضی}
در بازی‌هایی که تعادل نش خالص وجود ندارد، ممکن است تعادل نش ترکیبی وجود داشته باشد. فرض کنید مجموعه استراتژی‌های بازیکن \( i \) شامل \( S_i = \{s_i^1, s_i^2, \dots, s_i^m\} \) است. یک تعادل نش ترکیبی شامل توزیع‌های احتمالاتی \( \sigma_i^* \) برای هر بازیکن \( i \) است که در آن هیچ بازیکن نمی‌تواند با تغییر توزیع احتمالات انتخاب استراتژی خود سود بیشتری کسب کند. به طور ریاضی، برای هر بازیکن \( i \) و هر استراتژی \( s_i^j \in S_i \)، شرط زیر برقرار است:
\[
\sum_{s_i \in S_i} \sigma_i^*(s_i) u_i(s_i, \sigma_{-i}^*) \geq \sum_{s_i \in S_i} \sigma_i(s_i) u_i(s_i, \sigma_{-i}^*)
\]
که در آن \( \sigma_{-i}^* \) توزیع‌های احتمالاتی سایر بازیکنان است.

\paragraph{مثال}
یک مثال از تعادل نش ترکیبی، بازی \lr{Rock-Paper-Scissors} است. در این بازی، هر بازیکن می‌تواند سنگ، کاغذ یا قیچی انتخاب کند. اگر هر دو بازیکن انتخاب یک گزینه را داشته باشند، بازی مساوی است. اگر یک بازیکن سنگ انتخاب کند و دیگری قیچی، سنگ برقی قیچی را شکست می‌دهد، و به همین ترتیب. در این بازی، تعادل نش ترکیبی زمانی رخ می‌دهد که هر بازیکن با احتمال برابر 1/3 سنگ، کاغذ و قیچی را انتخاب کند. در این حالت، هیچ کدام از بازیکنان نمی‌توانند با تغییر احتمالات انتخاب استراتژی خود سود بیشتری کسب کنند.

\paragraph{پیش‌بینی و استراتژی‌ها}
در تعادل نش ترکیبی، بازیکنان به گونه‌ای استراتژی‌های خود را انتخاب می‌کنند که بازیکنان دیگر نیز مجبور به انتخاب استراتژی‌های ترکیبی خود می‌شوند تا تعادل حفظ شود. این نوع تعادل نش به ویژه در بازی‌هایی که تعادل نش خالص وجود ندارد یا تعادل نش خالص ناامن است، اهمیت دارد.

\paragraph{چالش‌ها و فرصت‌ها}
یکی از چالش‌های اصلی در تعادل نش ترکیبی، محاسبه و پیش‌بینی توزیع‌های احتمالاتی مناسب برای هر بازیکن است. این امر به ویژه در بازی‌های با تعداد زیاد استراتژی‌ها و بازیکنان پیچیده‌تر است. با این حال، تعادل نش ترکیبی فرصت‌های زیادی را برای تحلیل و بهبود استراتژی‌های بازیکنان فراهم می‌کند، به ویژه در بازی‌هایی که نیاز به تعادل میان استراتژی‌های مختلف دارند.

\paragraph{کاربردها}
تعادل نش ترکیبی در بسیاری از حوزه‌ها کاربرد دارد، از جمله اقتصاد، سیاست، و هوش مصنوعی. در اقتصاد، این مفهوم می‌تواند برای تحلیل رقابت‌های بازار و تصمیم‌گیری‌های استراتژیک شرکت‌ها مورد استفاده قرار گیرد. در هوش مصنوعی، تعادل نش ترکیبی می‌تواند به طراحی الگوریتم‌های هوشمند برای بازی‌ها و سیستم‌های چندعاملی کمک کند که قادر به تعامل بهینه و تعادلی با سایر عوامل هستند.

\paragraph{نتیجه‌گیری}
تعادل نش، چه خالص و چه ترکیبی، ابزار قدرتمندی در تحلیل تعاملات میان عوامل در محیط‌های رقابتی و همکاری فراهم می‌کند. فهم و کاربرد این تعادل‌ها به محققان و توسعه‌دهندگان امکان می‌دهد تا الگوریتم‌های هوشمندتری طراحی کنند که قادر به اتخاذ تصمیمات بهینه در مواجهه با استراتژی‌های متغیر و پیچیده دیگر عوامل هستند. با ادامه تحقیقات در این زمینه، انتظار می‌رود که تعادل نش نقش مهم‌تری در بهبود عملکرد سیستم‌های چندعاملی ایفا کند.
