%\subsection{بازی مجموع صفر}
%
%بازی‌های مجموع صفر\LTRfootnote{Zero-Sum Games}
% دسته‌ای از بازی‌ها هستند که در آن‌ها تابع ارزش یک بازیکن دقیقاً برابر با ضرر بازیکن دیگر است. به عبارت دیگر، مجموع ارزشهای همه بازیکنان در هر مرحله صفر است.
%
%
%
%\begin{itemize}
%	\item تعریف بازی مجموع صفر:
%	در یک بازی دو نفره، اگر تابع ارزش بازیکن اول (\( V_1^{(\pi_1 ,\pi_2)}(s)
%	\)) و بازیکن دوم (\( V_2^{(\pi_1 ,\pi_2)}(s)
%	\)) به‌گونه‌ای باشد که برای هر مجموعه سیاست
%	\( (\pi_1, \pi_2) \) 
%به صورت زیر باشد را یک بازی مجموع صفر نامیده می‌شود.
%	\begin{equation}\label{eq:game_v}
%		V_1^{(\pi_1 ,\pi_2)}(s) + V_2^{(\pi_1 ,\pi_2)}(s) = 0 \to V_1^{(\pi_1 ,\pi_2)}(s) = -V_2^{(\pi_1 ,\pi_2)}(s)
%%		V_1^{(\pi_1 ,\pi_2)}(s) =- V_2^{(\pi_1, \pi_2)}(s)
%	\end{equation}
%	\item سیاست بهینه در بازی مجموع صفر:
%	در بازی‌های مجموع صفر، سیاست بهینه هر بازیکن، انتخابی است که  تابع ارزش خود را در برابر بهترین پاسخ حریف به حداکثر برساند. این سیاست اغلب به تعادل نش منجر می‌شود. سیاست بهینه دو بازیکن در بازی مجموع صفر با تابع  ارزش معادله
%	\eqref{eq:game_v}
%	به صورت زیر است.
%	
%	\begin{align}
%		V_1^*(s) = \max_{\pi_1} \min_{\pi_2} V_1^{(\pi_1 ,\pi_2)}(s) \\
%		V_2^*(s) = \max_{\pi_2} \min_{\pi_1} V_2^{(\pi_1 ,\pi_2)}(s)
%	\end{align}
%	
%	
%\end{itemize}


















\subsection{بازی مجموع صفر}

بازی‌های مجموع صفر\LTRfootnote{Zero-Sum Games}
دسته‌ای از بازی‌ها هستند که در آن‌ها تابع ارزش یک بازیکن دقیقاً برابر با ضرر بازیکن دیگر است؛ بنابراین، مجموع ارزش‌های همهٔ بازیکنان در هر مرحله صفر خواهد بود.

\begin{itemize}
	%------------------------------------
	\item \textbf{تعریف بازی مجموع صفر:}
	
	در یک بازی دو نفره، اگر تابع ارزشِ حالت (value) بازیکن اوّل 
	\(V_1^{(\pi_1 ,\pi_2)}(s)\)
	و بازیکن دوم 
	\(V_2^{(\pi_1 ,\pi_2)}(s)\)
	برای هر مجموعه سیاست 
	\((\pi_1,\pi_2)\)
	به‌گونه‌ای باشند که:
	\begin{equation}\label{eq:game_v}
		V_1^{(\pi_1 ,\pi_2)}(s) + V_2^{(\pi_1 ,\pi_2)}(s) = 0 
		\;\;\Longrightarrow\;\;
		V_1^{(\pi_1 ,\pi_2)}(s) = -\,V_2^{(\pi_1 ,\pi_2)}(s),
	\end{equation}
	آنگاه آن بازی را \emph{بازی مجموع صفر} می‌نامیم.
	
	به‌طور مشابه، اگر تابع ارزش–عمل برای دو بازیکن را با
	\(Q_1^{(\pi_1,\pi_2)}(s,a_1,a_2)\)
	و
	\(Q_2^{(\pi_1,\pi_2)}(s,a_1,a_2)\)
	نشان دهیم، باید برقرار باشد:
	\begin{equation}\label{eq:game_q}
		Q_1^{(\pi_1,\pi_2)}(s,a_1,a_2) + 
		Q_2^{(\pi_1,\pi_2)}(s,a_1,a_2) = 0
		\;\;\Longrightarrow\;\;
		Q_1^{(\pi_1,\pi_2)}(s,a_1,a_2) = -\,Q_2^{(\pi_1,\pi_2)}(s,a_1,a_2).
	\end{equation}
	%------------------------------------
	
	\item \textbf{سیاست بهینه در بازی مجموع صفر:}
	
	در این بازی‌ها، هر بازیکن سیاستی را برمی‌گزیند که تابع ارزش خود را
	در برابر بهترین پاسخِ حریف بیشینه کند؛ این انتخاب در نهایت به
	تعادل نش منجر می‌شود.
	
	به‌صورت تابع ارزشِ حالت:
	\begin{align}
		V_1^*(s) &= \max_{\pi_1}\,\min_{\pi_2} \;
		V_1^{(\pi_1 ,\pi_2)}(s), \\
		V_2^*(s) &= \max_{\pi_2}\,\min_{\pi_1} \;
		V_2^{(\pi_1 ,\pi_2)}(s).
	\end{align}
	
	و به‌صورت تابع ارزش–عمل:
	\begin{align}
		Q_1^*(s,a_1,a_2) &= \max_{\pi_1}\,\min_{\pi_2} \;
		Q_1^{(\pi_1 ,\pi_2)}(s,a_1,a_2), \\
		Q_2^*(s,a_1,a_2) &= \max_{\pi_2}\,\min_{\pi_1} \;
		Q_2^{(\pi_1 ,\pi_2)}(s,a_1,a_2).
	\end{align}
\end{itemize}


