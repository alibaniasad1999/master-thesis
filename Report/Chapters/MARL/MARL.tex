\chapter{یادگیری تقویتی چند عاملی}
کاربردهای پیچیده در یادگیری تقویتی نیازمند اضافه کردن چندین عامل\LTRfootnote{Multi-Agent}
 برای انجام همزمان وظایف مختلف هستند. با این حال، افزایش تعداد عامل‌ها چالش‌هایی در مدیریت تعاملات میان آن‌ها به همراه دارد. در این فصل، بر اساس مسئله بهینه‌سازی برای هر عامل، مفهوم تعادل\LTRfootnote{Equilibrium}
  معرفی شده تا رفتارهای توزیعی چندعاملی را تنظیم کنند. روابط همکاری و رقابت میان عامل‌ها را در سناریوهای مختلف تحلیل شده و آن‌ها با الگوریتم‌های معمول یادگیری تقویتی چندعاملی ترکیب شده است. بر اساس انواع تعاملات، یک چارچوب نظریه بازی برای مدل‌سازی عمومی در سناریوهای چندعاملی استفاده شده است. با تحلیل بهینه‌سازی و وضعیت تعادل برای هر بخش از چارچوب، سیاست بهینه یادگیری تقویتی چندعاملی برای هر عامل بررسی شد.
  
  \section{تعاریف و مفاهیم اساسی }
یادگیری تقویتی چندعاملی\LTRfootnote{Multi-Agent Reinforcement Learning (MARL)} به بررسی چگونگی یادگیری و تصمیم‌گیری چندین عامل مستقل در یک محیط مشترک پرداخته می‌شود. برای تحلیل دقیق و درک بهتر این حوزه، اجزای اصلی آن شامل عامل، سیاست و مطلوبیت\LTRfootnote{Utility} در نظر گرفته می‌شوند که در ادامه به صورت مختصر و منسجم تشریح می‌گردند.

\begin{itemize}
	\item عامل: یک موجودیت مستقل به عنوان عامل تعریف می‌شود که به صورت خودمختار با محیط تعامل کرده و بر اساس مشاهدات رفتار سایر عامل‌ها، سیاست‌هایش انتخاب می‌گردند تا سود حداکثر یا ضرر حداقل حاصل شود. در سناریوهای مورد بررسی، چندین عامل به صورت مستقل عمل می‌کنند؛ اما اگر تعداد عامل‌ها به یک کاهش یابد، \lr{MARL} به یادگیری تقویتی معمولی تبدیل می‌شود.
	
	\item سیاست: برای هر عامل در \lr{MARL}، سیاستی خاص در نظر گرفته می‌شود که به عنوان روشی برای انتخاب اقدامات بر اساس وضعیت محیط و رفتار سایر عامل‌ها تعریف می‌گردد. این سیاست‌ها با هدف به حداکثر رساندن سود و به حداقل رساندن هزینه طراحی شده و تحت تأثیر محیط و سیاست‌های دیگر عامل‌ها قرار می‌گیرند.
	
	\item مطلوبیت: مطلوبیت
	هر عامل بر اساس نیازها و وابستگی‌هایش به محیط و سایر عامل‌ها تعریف شده و به صورت سود منهای هزینه، با توجه به اهداف مختلف محاسبه می‌شود. در سناریوهای چندعاملی، از طریق یادگیری از محیط و تعامل با دیگران، مطلوبیت هر عامل بهینه می‌گردد.
\end{itemize}

در این چارچوب، برای هر عامل در \lr{MARL} تابع مطلوبیت خاصی در نظر گرفته شده و بر اساس مشاهدات و تجربیات حاصل از تعاملات، یادگیری سیاست به صورت مستقل انجام می‌شود تا ارزش مطلوبیت به حداکثر برسد، بدون اینکه مستقیماً به مطلوبیت سایر عامل‌ها توجه شود. این فرآیند ممکن است به رقابت یا همکاری میان عامل‌ها منجر گردد.
 با توجه به پیچیدگی تعاملات میان چندین عامل، تحلیل نظریه بازی‌ها به عنوان ابزاری مؤثر برای تصمیم‌گیری در این حوزه به کار گرفته می‌شود. بسته به سناریوهای مختلف، این بازی‌ها در دسته‌بندی‌های متفاوتی قرار داده شده که در بخش‌های بعدی بررسی خواهند شد.

  % \section{اهمیت یادگیری تقویتی چندعاملی}

یادگیری تقویتی چندعاملی به دلیل قابلیت‌های بالقوه‌اش در مدل‌سازی و حل مسائل پیچیده و پویا، اهمیت زیادی در حوزه‌های مختلف علمی و صنعتی دارد. در این بخش، به بررسی اهمیت \lr{MARL} در زمینه‌های مختلف پرداخته و نقش آن را در توسعه سیستم‌های هوشمند متعدد بررسی می‌کنیم.

\subsubsection{مدل‌سازی سیستم‌های پیچیده و پویا}
یکی از دلایل اصلی اهمیت \lr{MARL}، توانایی آن در مدل‌سازی سیستم‌های پیچیده و پویا است. در بسیاری از کاربردهای واقعی، سیستم‌ها شامل چندین عامل هستند که به صورت همزمان و مستقل به تعامل می‌پردازند. به عنوان مثال، در شبکه‌های ترافیکی، هر خودرو می‌تواند به عنوان یک عامل مستقل عمل کند که نیاز به هماهنگی و تعامل با سایر خودروها برای بهینه‌سازی جریان ترافیک دارد. \lr{MARL} با فراهم کردن چارچوبی برای تعامل و یادگیری میان این عوامل، امکان بهبود کارایی و کاهش ترافیک را فراهم می‌کند.

\subsubsection{کاربرد در رباتیک چندعاملی}
در حوزه رباتیک، سیستم‌های چندعاملی می‌توانند برای انجام وظایف پیچیده‌ای مانند جست‌وجو و نجات، حمل و نقل مواد، و عملیات هماهنگ در محیط‌های غیرقابل پیش‌بینی مورد استفاده قرار گیرند. به عنوان مثال، گروهی از ربات‌های پرنده (دُرون‌ها) می‌توانند با همکاری و تبادل اطلاعات، منطقه‌ای وسیع را برای شناسایی اهداف نظارت کنند یا به سرعت به تغییرات محیطی واکنش نشان دهند. \lr{MARL} در این زمینه بهبود هماهنگی میان ربات‌ها و افزایش کارایی عملیات‌های چندعاملی را ممکن می‌سازد.

\subsubsection{مدیریت منابع در شبکه‌های ارتباطی}
شبکه‌های ارتباطی مدرن نیازمند مدیریت بهینه منابع مانند پهنای باند، انرژی و ظرفیت ذخیره‌سازی هستند. در این راستا، \lr{MARL} می‌تواند به عنوان یک ابزار قدرتمند برای تخصیص بهینه منابع به عوامل مختلف شبکه عمل کند. به عنوان مثال، در شبکه‌های بی‌سیم، هر دستگاه کاربر می‌تواند به عنوان یک عامل مستقل عمل کرده و با یادگیری و تعامل با سایر دستگاه‌ها، نحوه بهینه‌سازی مصرف انرژی و پهنای باند را پیدا کند. این امر منجر به افزایش کارایی شبکه و کاهش هزینه‌های عملیاتی می‌شود.

\subsubsection{توسعه الگوریتم‌های پیشرفته‌تر و قابل اعتمادتر}
یکی دیگر از جنبه‌های مهم \lr{MARL}، فهم و تحلیل تعاملات میان عوامل مختلف است که می‌تواند به توسعه الگوریتم‌های پیشرفته‌تر و قابل اعتمادتر منجر شود. با مطالعه رفتارها و استراتژی‌های مختلف در محیط‌های چندعاملی، پژوهشگران قادر به طراحی الگوریتم‌هایی می‌شوند که نه تنها بهینه عمل می‌کنند بلکه مقاومت بالایی در برابر تغییرات محیطی و رفتارهای غیرمنتظره دارند. این الگوریتم‌ها می‌توانند در شرایط متنوع و پیچیده‌تر به خوبی عمل کنند و از خطاها و ناهنجاری‌های احتمالی جلوگیری نمایند.

\subsubsection{کاربرد در بازی‌های چندعاملی و شبیه‌سازی‌های اقتصادی}
بازی‌های چندعاملی و شبیه‌سازی‌های اقتصادی از دیگر حوزه‌هایی هستند که به شدت از \lr{MARL} بهره‌مند می‌شوند. در بازی‌های استراتژیک چند نفره، \lr{MARL} می‌تواند به بازیگران کمک کند تا استراتژی‌های بهینه‌ای برای رقابت و همکاری با یکدیگر توسعه دهند. همچنین، در شبیه‌سازی‌های اقتصادی، \lr{MARL} می‌تواند به مدل‌سازی و تحلیل رفتارهای بازار و تصمیم‌گیری‌های اقتصادی کمک کند، که این امر به پیش‌بینی دقیق‌تر روندهای اقتصادی و بهبود سیاست‌گذاری‌های مالی منجر می‌شود.

\subsubsection{افزایش قابلیت انعطاف‌پذیری و مقیاس‌پذیری سیستم‌ها}
سیستم‌های چندعاملی معمولاً نیازمند قابلیت انعطاف‌پذیری و مقیاس‌پذیری بالا هستند تا بتوانند با تغییرات محیطی و افزایش تعداد عوامل سازگار شوند. \lr{MARL} با استفاده از الگوریتم‌های توزیع‌شده و یادگیری محلی، امکان توسعه سیستم‌هایی با مقیاس بزرگ و پیچیدگی بالا را فراهم می‌کند. این امر به ویژه در کاربردهایی مانند اینترنت اشیاء\LTRfootnote{Internet of Things (IoT)}، هوش مصنوعی توزیع‌شده و سیستم‌های بزرگ‌مقیاس داده‌های بزرگ\LTRfootnote{Big Data}
 بسیار حائز اهمیت است.

%\subsubsection{نتیجه‌گیری}
%در نهایت، یادگیری تقویتی چندعاملی به عنوان یک ابزار قدرتمند در حل مسائل پیچیده و پویا مطرح است که قابلیت‌های متنوعی در حوزه‌های مختلف علمی و صنعتی ارائه می‌دهد. از مدل‌سازی سیستم‌های پیچیده و رباتیک چندعاملی گرفته تا مدیریت منابع در شبکه‌های ارتباطی و توسعه الگوریتم‌های پیشرفته‌تر، \lr{MARL} نقش کلیدی در پیشرفت تکنولوژی‌های هوشمند و خودکار ایفا می‌کند. با ادامه تحقیقات و بهبود الگوریتم‌های موجود، انتظار می‌رود که کاربردهای \lr{MARL} همچنان گسترش یافته و تاثیرات مثبتی در حوزه‌های مختلف داشته باشد.


    \subsection{بازی‌های جمع صفر}

بازی‌های جمع صفر\LTRfootnote{Zero-Sum}
 یکی از انواع اصلی بازی‌های چندعاملی هستند که در آن سود یک بازیکن به طور مستقیم با ضرر بازیکنان دیگر مرتبط است. در این بازی‌ها، مجموع پاداش‌ها برای همه بازیکنان در هر حالت برابر با صفر است، به این معنی که هر افزایشی در پاداش یکی از بازیکنان منجر به کاهش معادل آن در بازیکنان دیگر می‌شود. این نوع بازی‌ها به خوبی می‌توانند رقابت‌های شدید و استراتژی‌های بهینه را مدل‌سازی کنند.

بازی‌های جمع صفر می‌توانند بر اساس سناریوهای مختلف به دسته‌های متنوعی تقسیم‌بندی شوند. دو دسته اصلی این بازی‌ها عبارتند از بازی‌های ثابت و بازی‌های تکراری.

\begin{itemize}
	\item \textbf{بازی ثابت (\lr{Static Game}):} بازی ثابت ساده‌ترین شکل برای مدل‌سازی تعاملات میان عوامل است. در بازی ثابت، هر عامل تنها یک تصمیم‌گیری واحد را انجام می‌دهد. از آنجایی که هر عامل تنها یک بار عمل می‌کند، تقلب و خیانت غیرمنتظره می‌تواند در این نوع بازی‌ها سودآور باشد. بنابراین، هر عامل نیاز دارد تا به دقت استراتژی‌های سایر عوامل را پیش‌بینی کند تا بتواند به طور هوشمندانه عمل کرده و بیشترین سود ممکن را کسب کند. بازی‌های ثابت معمولاً در سناریوهای رقابتی با تعاملات کوتاه مدت کاربرد دارند.
	
	\item \textbf{بازی تکراری (\lr{Repeated Game}):} بازی تکراری به وضعیتی اشاره دارد که در آن تمام عوامل می‌توانند بر اساس همان وضعیت برای چندین تکرار اقداماتی انجام دهند. سود کلی هر عامل مجموع سودهای تخفیف‌شده برای هر تکرار از بازی است. به دلیل اقدامات مکرر تمام عوامل، تقلب و خیانت در طول تعاملات می‌تواند منجر به مجازات یا انتقام از سوی سایر عوامل در تکرارهای آینده شود. بنابراین، بازی تکراری از رفتارهای مخرب عوامل جلوگیری می‌کند و به طور کلی سود کل برای تمام عوامل را افزایش می‌دهد. بازی‌های تکراری معمولاً در سناریوهای همکاری بلندمدت و تعاملات پویا کاربرد دارند.
\end{itemize}

این دسته‌بندی‌ها به محققان و توسعه‌دهندگان کمک می‌کنند تا بازی‌های چندعاملی را بر اساس ویژگی‌های مختلف آن‌ها شناسایی و تحلیل کنند. در بازی‌های ثابت، تمرکز بر پیش‌بینی دقیق استراتژی‌های دیگر عوامل و اتخاذ بهترین تصمیم در یک لحظه زمانی است. در مقابل، بازی‌های تکراری نیازمند توسعه استراتژی‌های پایدار و قابل اعتماد هستند که نه تنها در تکرار اول بلکه در تکرارهای بعدی نیز موثر باشند.

بازی‌های جمع صفر در یادگیری تقویتی چندعاملی به دلیل سادگی و قابلیت مدل‌سازی دقیق تعاملات رقابتی، به عنوان یک ابزار قدرتمند برای تحلیل و توسعه الگوریتم‌های \lr{MARL} مورد استفاده قرار می‌گیرند. این بازی‌ها امکان بررسی رفتارهای استراتژیک، بهینه‌سازی سیاست‌ها و تحلیل تعادل‌های نش (\lr{Nash Equilibrium}) را فراهم می‌کنند که در نهایت به بهبود عملکرد سیستم‌های چندعاملی منجر می‌شود.

\paragraph{مثال‌ها و کاربردها}
یکی از مثال‌های معروف بازی‌های جمع صفر، بازی شطرنج است که در آن هر حرکت یک بازیکن مستقیماً به نفع یا ضرر بازیکن دیگر است. سایر مثال‌ها شامل بازی‌های استراتژیک مانند \lr{Poker} و \lr{Go} می‌باشند که در آن‌ها تعاملات رقابتی میان بازیکنان به طور کامل با اصول بازی‌های جمع صفر مطابقت دارند.

در حوزه‌های عملی، بازی‌های جمع صفر می‌توانند برای مدل‌سازی رقابت‌های بازار، مذاکرات اقتصادی و حتی تعاملات میان ربات‌های خودران در محیط‌های رقابتی مورد استفاده قرار گیرند. این کاربردها به محققان امکان می‌دهند تا الگوریتم‌هایی طراحی کنند که قادر به بهینه‌سازی عملکرد در شرایط رقابتی و متغیر باشند.

\paragraph{چالش‌ها و فرصت‌ها}
یکی از چالش‌های اصلی در بازی‌های جمع صفر، پیش‌بینی دقیق رفتارهای رقبا و اتخاذ تصمیم‌های بهینه در مواجهه با استراتژی‌های متغیر آن‌ها است. همچنین، در بازی‌های تکراری، ایجاد تعادل‌های پایدار و جلوگیری از رفتارهای مخرب به عنوان یک چالش مهم مطرح است. با این حال، این چالش‌ها فرصت‌های قابل توجهی برای توسعه الگوریتم‌های پیشرفته و افزایش قابلیت‌های یادگیری تقویتی چندعاملی فراهم می‌کنند که می‌توانند در شرایط پیچیده‌تر و پویا نیز عملکرد مطلوبی داشته باشند.

    \section{تعادل نش}

    \subsection{ایمنی و مقاومت در یادگیری تقویتی چندعاملی}

در استفاده از یادگیری تقویتی چندعاملی (\lr{Multi-Agent Reinforcement Learning - \lr{MARL}})، مسائل مربوط به ایمنی و مقاومت در برابر اختلالات یکی از چالش‌های اساسی مطرح می‌گردد. به منظور اطمینان از عملکرد قابل اعتماد و ایمن الگوریتم‌های یادگیری تقویتی چندعاملی، نیازمند توسعه روش‌هایی هستیم که بتوانند در مواجهه با رفتارهای غیرمنتظره یا مخرب سایر عوامل، پایداری و ایمنی سیستم را حفظ نمایند. در این بخش، به بررسی مفاهیم ایمنی و مقاومت در \lr{MARL} پرداخته شده و چگونگی افزایش مقاومت الگوریتم‌ها از طریق در نظر گرفتن عوامل به عنوان اختلالات مورد بحث قرار گرفته است.

ایمنی در \lr{MARL} به معنای تضمین این است که تعاملات میان عوامل منجر به نتایج نامطلوب یا خطرناک نشوند. برای دستیابی به ایمنی، روش‌هایی نظیر محدود کردن فضای عملیاتی، اعمال قیود بر سیاست‌های یادگیری و استفاده از الگوریتم‌های مقاوم در برابر خطا به کار گرفته شده‌اند. یکی از رویکردهای موثر در افزایش مقاومت سیستم، فرض کردن یکی از عوامل به عنوان اختلال (\lr{Disturbance}) در محیط است. با این فرض، الگوریتم‌ها قادر خواهند بود تا به گونه‌ای طراحی شوند که در حضور اختلالات احتمالی، عملکرد سیستم همچنان قابل اعتماد باقی بماند.

\subsubsection{فرض کردن اختلال به عنوان عامل}

در محیط‌های چندعاملی، برخی از عوامل ممکن است رفتارهای مخرب یا غیرمنتظره‌ای را از خود نشان دهند که می‌تواند به عملکرد کلی سیستم آسیب برساند. برای مقابله با این مسئله، فرض می‌شود که یک یا چند عامل به عنوان اختلالات در نظر گرفته شوند. این اختلالات می‌توانند به صورت عمدی یا غیرعمدی ایجاد شوند و هدف آن‌ها کاهش کارایی سیستم است. با فرض کردن این اختلالات، الگوریتم‌های \lr{MARL} قادر خواهند بود تا سیاست‌هایی را یاد بگیرند که در مواجهه با این اختلالات نیز عملکرد بهینه و ایمنی را حفظ کنند.

\subsubsection{تعریف مقاومت و ایمنی در \lr{MARL}}

مقاومت در \lr{MARL} به معنای توانایی الگوریتم در حفظ عملکرد مطلوب در حضور اختلالات و تغییرات محیطی است. این مقاومت می‌تواند از طریق طراحی سیاست‌های بهینه که به گونه‌ای تنظیم شده‌اند که تأثیر اختلالات را به حداقل برسانند، به دست آید. به علاوه، ایمنی می‌تواند از طریق تضمین عدم وقوع رفتارهای خطرناک و حفظ تعادل سیستم در مواجهه با رفتارهای مخرب حاصل شود.

\paragraph{تعریف ریاضی مقاومت}

فرض کنید یک محیط چندعاملی با مجموعه‌ای از عوامل \( \mathcal{A} = \{A_1, A_2, \dots, A_n\} \) وجود دارد که در آن یک عامل \( A_d \) به عنوان اختلال تعریف شده است. هدف این است که الگوریتم \lr{MARL} به گونه‌ای طراحی شود که سیاست‌های یادگرفته شده \( \pi = \{\pi_1, \pi_2, \dots, \pi_n\} \) بتوانند عملکرد بهینه را حتی در حضور \( A_d \) حفظ کنند. به طور ریاضی، مقاومت به صورت زیر تعریف می‌شود:
\[
\forall A_i \in \mathcal{A}, \quad \text{اگر } A_d \text{ رفتار مخرب نشان دهد، } \pi_i \text{ باید همچنان به حداکثر رساندن پاداش خود ادامه دهد.}
\]

\paragraph{تعریف ریاضی ایمنی}

ایمنی در \lr{MARL} به معنای اطمینان از این است که سیستم در هیچ حالت خطرناکی وارد نمی‌شود. به طور ریاضی، ایمنی می‌تواند به صورت مجموعه‌ای از قیود تعریف شود که سیاست‌های یادگرفته شده باید آن‌ها را رعایت کنند:
\[
\forall A_i \in \mathcal{A}, \quad \text{ایمنی: } u_i(s_i, s_{-i}) \geq \theta_i \quad \text{برای همه } s_i \in S_i, \ s_{-i} \in S_{-i}
\]
که در آن \( \theta_i \) آستانه‌ای است که برای هر عامل \( A_i \) تعیین شده و نشان‌دهنده حداقل پاداش قابل قبول است.

\subsubsection{روش‌های افزایش مقاومت و ایمنی}

برای افزایش مقاومت و ایمنی در \lr{MARL}، روش‌های متعددی مورد استفاده قرار گرفته‌اند که در زیر به برخی از آن‌ها پرداخته می‌شود:

\begin{itemize}
	\item \textbf{الگوریتم‌های مقاوم در برابر اختلالات:} این الگوریتم‌ها به گونه‌ای طراحی شده‌اند که بتوانند به سرعت با تغییرات محیطی و حضور اختلالات سازگار شوند. به عنوان مثال، الگوریتم‌های مبتنی بر یادگیری تطبیقی که قادر به تغییر سیاست‌های خود در پاسخ به تغییرات محیط هستند.
	
	\item \textbf{فریم‌ورک‌های ایمنی:} چارچوب‌هایی برای تضمین ایمنی در تعاملات چندعاملی طراحی شده‌اند که شامل محدود کردن فضای عملیاتی و اعمال قیود بر سیاست‌های یادگیری است. این فریم‌ورک‌ها معمولاً شامل روش‌هایی برای نظارت و تنظیم رفتار عامل‌ها به منظور جلوگیری از وقوع رفتارهای خطرناک هستند.
	
	\item \textbf{آموزش در حضور اختلالات:} با آموزش الگوریتم‌ها در محیط‌هایی که شامل اختلالات هستند، می‌توان مقاومت الگوریتم‌ها را افزایش داد. این روش به الگوریتم اجازه می‌دهد تا در مواجهه با اختلالات غیرمنتظره، سیاست‌های مقاومتی یاد بگیرد.
	
	\item \textbf{استفاده از اصول نظریه بازی‌ها:} با بهره‌گیری از تعادل‌های نظریه بازی‌ها مانند تعادل نش (\lr{Nash Equilibrium}), می‌توان سیاست‌هایی طراحی کرد که در مواجهه با استراتژی‌های متغیر سایر عوامل، پایداری و ایمنی سیستم حفظ شود.
\end{itemize}

\subsubsection{نمونه‌های کاربردی}

برای نشان دادن کاربردهای عملی \lr{MARL} در افزایش ایمنی و مقاومت، به چند مثال اشاره می‌شود:

\paragraph{سامانه‌های خودران:} در خودروهای خودران چندعاملی، ایمنی یکی از اولویت‌های اصلی است. با استفاده از \lr{MARL} و فرض کردن سایر خودروها به عنوان عوامل یا اختلالات، می‌توان الگوریتم‌هایی توسعه داد که در مواجهه با رفتارهای غیرمنتظره سایر خودروها، ایمن باقی بمانند.

\paragraph{مدیریت انرژی در شبکه‌های هوشمند:} در شبکه‌های انرژی هوشمند، \lr{MARL} می‌تواند برای مدیریت بهینه انرژی در حضور اختلالات مانند خرابی‌ها یا حملات سایبری استفاده شود. الگوریتم‌های مقاوم می‌توانند با تغییرات ناگهانی در تقاضا یا عرضه انرژی سازگار شده و پایداری شبکه را حفظ کنند.

\paragraph{ربات‌های همکاری‌کننده:} در سیستم‌های رباتیک همکاری‌کننده، اطمینان از ایمنی تعاملات میان ربات‌ها حیاتی است. \lr{MARL} می‌تواند برای طراحی سیاست‌های ایمن که در مواجهه با رفتارهای مخرب یا اختلالات داخلی، سیستم را پایدار نگه دارند، به کار رود.

\paragraph{محیط‌های صنعتی:} در محیط‌های صنعتی که شامل چندین عامل نظیر ربات‌ها و ماشین‌آلات است، ایمنی و مقاومت سیستم‌ها از اهمیت بالایی برخوردار است. با استفاده از \lr{MARL}، می‌توان الگوریتم‌هایی توسعه داد که در مواجهه با اختلالات مانند خرابی تجهیزات یا خطاهای انسانی، عملکرد سیستم را حفظ کنند.

\subsubsection{چالش‌ها و فرصت‌ها}

با وجود مزایای متعدد \lr{MARL} در افزایش ایمنی و مقاومت سیستم‌ها، چالش‌هایی نیز در این زمینه وجود دارد:

\begin{itemize}
	\item \textbf{پیچیدگی مدل‌سازی اختلالات:} مدل‌سازی دقیق اختلالات و رفتارهای مخرب می‌تواند پیچیده و زمان‌بر باشد، به ویژه در محیط‌های پیچیده و پویا.
	
	\item \textbf{هماهنگی میان عوامل:} هماهنگی موثر میان عوامل در حضور اختلالات نیازمند الگوریتم‌های پیچیده و کارآمد است که بتوانند به سرعت و بهینه با تغییرات محیطی سازگار شوند.
	
	\item \textbf{تعادل میان کارایی و ایمنی:} حفظ تعادل میان بهینه‌سازی کارایی سیستم و تضمین ایمنی در مواجهه با اختلالات یک چالش بزرگ است که نیازمند طراحی دقیق الگوریتم‌ها است.
	
	\item \textbf{نیاز به داده‌های بزرگ:} برای آموزش الگوریتم‌های مقاوم و ایمن، نیاز به مجموعه‌های داده بزرگ و متنوعی است که شامل انواع اختلالات و رفتارهای مخرب باشند.
\end{itemize}

با این حال، فرصت‌های بزرگی نیز در این زمینه وجود دارد. توسعه الگوریتم‌های پیشرفته \lr{MARL} که بتوانند به طور موثر با اختلالات مواجه شوند، می‌تواند منجر به سیستم‌های هوشمند و خودکار با کارایی و ایمنی بالاتر شود. همچنین، پژوهش‌های ادامه‌دار در زمینه نظریه بازی‌ها و روش‌های مقاوم‌سازی الگوریتم‌های یادگیری تقویتی، پتانسیل بالایی برای بهبود \lr{MARL} و کاربردهای آن فراهم می‌آورد.

\subsubsection{نتیجه‌گیری}

در نهایت، ایمنی و مقاومت در یادگیری تقویتی چندعاملی به عنوان عوامل کلیدی در توسعه سیستم‌های هوشمند و خودکار مطرح می‌گردند. با فرض کردن اختلالات و توسعه الگوریتم‌های مقاوم، می‌توان عملکرد سیستم‌های \lr{MARL} را در مواجهه با چالش‌های مختلف بهبود بخشید. با توجه به اهمیت و کاربردهای گسترده \lr{MARL} در حوزه‌های علمی و صنعتی، تحقیقات در زمینه افزایش ایمنی و مقاومت این الگوریتم‌ها همچنان ادامه خواهد یافت تا به سیستم‌هایی با قابلیت‌های بیشتر و عملکردی پایدارتر دست یابند.

    \subsection{الگوریتم‌های یادگیری تقویتی چندعاملی}

    