\chapter{یادگیری تقویتی چندعاملی}
کاربردهای پیچیده در یادگیری تقویتی نیازمند اضافه کردن چندین عامل\LTRfootnote{Multi-Agent} برای انجام همزمان وظایف مختلف هستند.
با این حال، افزایش تعداد عامل‌ها چالش‌هایی در مدیریت تعاملات میان آن‌ها به همراه دارد.
در این فصل، بر اساس مسئله بهینه‌سازی برای هر عامل، مفهوم تعادل\LTRfootnote{Equilibrium} معرفی شده تا رفتارهای توزیعی چندعاملی را تنظیم کند.
رابطه رقابت میان عامل‌ها در سناریوهای مختلف تحلیل شده و آن‌ها با الگوریتم‌های معمول یادگیری تقویتی چندعاملی ترکیب شده‌اند. بر اساس انواع تعاملات، یک چارچوب نظریه بازی برای مدل‌سازی عمومی در سناریوهای چندعاملی استفاده شده است. با تحلیل بهینه‌سازی و وضعیت تعادل برای هر بخش از چارچوب، سیاست بهینه یادگیری تقویتی چندعاملی برای هر عامل بررسی شده است.




  \section{تعاریف و مفاهیم اساسی }
یادگیری تقویتی چندعاملی\LTRfootnote{Multi-Agent Reinforcement Learning (MARL)} به بررسی چگونگی یادگیری و تصمیم‌گیری چندین عامل مستقل در یک محیط مشترک پرداخته می‌شود. برای تحلیل دقیق و درک بهتر این حوزه، اجزای اصلی آن شامل عامل، سیاست و مطلوبیت\LTRfootnote{Utility} در نظر گرفته می‌شوند که در ادامه به صورت مختصر و منسجم تشریح می‌گردند.

\begin{itemize}
	\item عامل: یک موجودیت مستقل به عنوان عامل تعریف می‌شود که به صورت خودمختار با محیط تعامل کرده و بر اساس مشاهدات رفتار سایر عامل‌ها، سیاست‌هایش انتخاب می‌گردند تا سود حداکثر یا ضرر حداقل حاصل شود. در سناریوهای مورد بررسی، چندین عامل به صورت مستقل عمل می‌کنند؛ اما اگر تعداد عامل‌ها به یک کاهش یابد، \lr{MARL} به یادگیری تقویتی معمولی تبدیل می‌شود.
	
	\item سیاست: برای هر عامل در \lr{MARL}، سیاستی خاص در نظر گرفته می‌شود که به عنوان روشی برای انتخاب اقدامات بر اساس وضعیت محیط و رفتار سایر عامل‌ها تعریف می‌گردد. این سیاست‌ها با هدف به حداکثر رساندن سود و به حداقل رساندن هزینه طراحی شده و تحت تأثیر محیط و سیاست‌های دیگر عامل‌ها قرار می‌گیرند.
	
	\item مطلوبیت: مطلوبیت
	هر عامل بر اساس نیازها و وابستگی‌هایش به محیط و سایر عامل‌ها تعریف شده و به صورت سود منهای هزینه، با توجه به اهداف مختلف محاسبه می‌شود. در سناریوهای چندعاملی، از طریق یادگیری از محیط و تعامل با دیگران، مطلوبیت هر عامل بهینه می‌گردد.
\end{itemize}

در این چارچوب، برای هر عامل در \lr{MARL} تابع مطلوبیت خاصی در نظر گرفته شده و بر اساس مشاهدات و تجربیات حاصل از تعاملات، یادگیری سیاست به صورت مستقل انجام می‌شود تا ارزش مطلوبیت به حداکثر برسد، بدون اینکه مستقیماً به مطلوبیت سایر عامل‌ها توجه شود. این فرآیند ممکن است به رقابت یا همکاری میان عامل‌ها منجر گردد.
 با توجه به پیچیدگی تعاملات میان چندین عامل، تحلیل نظریه بازی‌ها به عنوان ابزاری مؤثر برای تصمیم‌گیری در این حوزه به کار گرفته می‌شود. بسته به سناریوهای مختلف، این بازی‌ها در دسته‌بندی‌های متفاوتی قرار داده شده که در بخش‌های بعدی بررسی خواهند شد.

  % \section{اهمیت یادگیری تقویتی چندعاملی}

یادگیری تقویتی چندعاملی به دلیل قابلیت‌های بالقوه‌اش در مدل‌سازی و حل مسائل پیچیده و پویا، اهمیت زیادی در حوزه‌های مختلف علمی و صنعتی دارد. در این بخش، به بررسی اهمیت \lr{MARL} در زمینه‌های مختلف پرداخته و نقش آن را در توسعه سیستم‌های هوشمند متعدد بررسی می‌کنیم.

\subsubsection{مدل‌سازی سیستم‌های پیچیده و پویا}
یکی از دلایل اصلی اهمیت \lr{MARL}، توانایی آن در مدل‌سازی سیستم‌های پیچیده و پویا است. در بسیاری از کاربردهای واقعی، سیستم‌ها شامل چندین عامل هستند که به صورت همزمان و مستقل به تعامل می‌پردازند. به عنوان مثال، در شبکه‌های ترافیکی، هر خودرو می‌تواند به عنوان یک عامل مستقل عمل کند که نیاز به هماهنگی و تعامل با سایر خودروها برای بهینه‌سازی جریان ترافیک دارد. \lr{MARL} با فراهم کردن چارچوبی برای تعامل و یادگیری میان این عوامل، امکان بهبود کارایی و کاهش ترافیک را فراهم می‌کند.

\subsubsection{کاربرد در رباتیک چندعاملی}
در حوزه رباتیک، سیستم‌های چندعاملی می‌توانند برای انجام وظایف پیچیده‌ای مانند جست‌وجو و نجات، حمل و نقل مواد، و عملیات هماهنگ در محیط‌های غیرقابل پیش‌بینی مورد استفاده قرار گیرند. به عنوان مثال، گروهی از ربات‌های پرنده (دُرون‌ها) می‌توانند با همکاری و تبادل اطلاعات، منطقه‌ای وسیع را برای شناسایی اهداف نظارت کنند یا به سرعت به تغییرات محیطی واکنش نشان دهند. \lr{MARL} در این زمینه بهبود هماهنگی میان ربات‌ها و افزایش کارایی عملیات‌های چندعاملی را ممکن می‌سازد.

\subsubsection{مدیریت منابع در شبکه‌های ارتباطی}
شبکه‌های ارتباطی مدرن نیازمند مدیریت بهینه منابع مانند پهنای باند، انرژی و ظرفیت ذخیره‌سازی هستند. در این راستا، \lr{MARL} می‌تواند به عنوان یک ابزار قدرتمند برای تخصیص بهینه منابع به عوامل مختلف شبکه عمل کند. به عنوان مثال، در شبکه‌های بی‌سیم، هر دستگاه کاربر می‌تواند به عنوان یک عامل مستقل عمل کرده و با یادگیری و تعامل با سایر دستگاه‌ها، نحوه بهینه‌سازی مصرف انرژی و پهنای باند را پیدا کند. این امر منجر به افزایش کارایی شبکه و کاهش هزینه‌های عملیاتی می‌شود.

\subsubsection{توسعه الگوریتم‌های پیشرفته‌تر و قابل اعتمادتر}
یکی دیگر از جنبه‌های مهم \lr{MARL}، فهم و تحلیل تعاملات میان عوامل مختلف است که می‌تواند به توسعه الگوریتم‌های پیشرفته‌تر و قابل اعتمادتر منجر شود. با مطالعه رفتارها و استراتژی‌های مختلف در محیط‌های چندعاملی، پژوهشگران قادر به طراحی الگوریتم‌هایی می‌شوند که نه تنها بهینه عمل می‌کنند بلکه مقاومت بالایی در برابر تغییرات محیطی و رفتارهای غیرمنتظره دارند. این الگوریتم‌ها می‌توانند در شرایط متنوع و پیچیده‌تر به خوبی عمل کنند و از خطاها و ناهنجاری‌های احتمالی جلوگیری نمایند.

\subsubsection{کاربرد در بازی‌های چندعاملی و شبیه‌سازی‌های اقتصادی}
بازی‌های چندعاملی و شبیه‌سازی‌های اقتصادی از دیگر حوزه‌هایی هستند که به شدت از \lr{MARL} بهره‌مند می‌شوند. در بازی‌های استراتژیک چند نفره، \lr{MARL} می‌تواند به بازیگران کمک کند تا استراتژی‌های بهینه‌ای برای رقابت و همکاری با یکدیگر توسعه دهند. همچنین، در شبیه‌سازی‌های اقتصادی، \lr{MARL} می‌تواند به مدل‌سازی و تحلیل رفتارهای بازار و تصمیم‌گیری‌های اقتصادی کمک کند، که این امر به پیش‌بینی دقیق‌تر روندهای اقتصادی و بهبود سیاست‌گذاری‌های مالی منجر می‌شود.

\subsubsection{افزایش قابلیت انعطاف‌پذیری و مقیاس‌پذیری سیستم‌ها}
سیستم‌های چندعاملی معمولاً نیازمند قابلیت انعطاف‌پذیری و مقیاس‌پذیری بالا هستند تا بتوانند با تغییرات محیطی و افزایش تعداد عوامل سازگار شوند. \lr{MARL} با استفاده از الگوریتم‌های توزیع‌شده و یادگیری محلی، امکان توسعه سیستم‌هایی با مقیاس بزرگ و پیچیدگی بالا را فراهم می‌کند. این امر به ویژه در کاربردهایی مانند اینترنت اشیاء\LTRfootnote{Internet of Things (IoT)}، هوش مصنوعی توزیع‌شده و سیستم‌های بزرگ‌مقیاس داده‌های بزرگ\LTRfootnote{Big Data}
 بسیار حائز اهمیت است.

%\subsubsection{نتیجه‌گیری}
%در نهایت، یادگیری تقویتی چندعاملی به عنوان یک ابزار قدرتمند در حل مسائل پیچیده و پویا مطرح است که قابلیت‌های متنوعی در حوزه‌های مختلف علمی و صنعتی ارائه می‌دهد. از مدل‌سازی سیستم‌های پیچیده و رباتیک چندعاملی گرفته تا مدیریت منابع در شبکه‌های ارتباطی و توسعه الگوریتم‌های پیشرفته‌تر، \lr{MARL} نقش کلیدی در پیشرفت تکنولوژی‌های هوشمند و خودکار ایفا می‌کند. با ادامه تحقیقات و بهبود الگوریتم‌های موجود، انتظار می‌رود که کاربردهای \lr{MARL} همچنان گسترش یافته و تاثیرات مثبتی در حوزه‌های مختلف داشته باشد.


    \subsection{بازی‌های جمع صفر}

بازی‌های جمع صفر\LTRfootnote{Zero-Sum}
 یکی از انواع اصلی بازی‌های چندعاملی هستند که در آن سود یک بازیکن به طور مستقیم با ضرر بازیکنان دیگر مرتبط است. در این بازی‌ها، مجموع پاداش‌ها برای همه بازیکنان در هر حالت برابر با صفر است، به این معنی که هر افزایشی در پاداش یکی از بازیکنان منجر به کاهش معادل آن در بازیکنان دیگر می‌شود. این نوع بازی‌ها به خوبی می‌توانند رقابت‌های شدید و استراتژی‌های بهینه را مدل‌سازی کنند.

بازی‌های جمع صفر می‌توانند بر اساس سناریوهای مختلف به دسته‌های متنوعی تقسیم‌بندی شوند. دو دسته اصلی این بازی‌ها عبارتند از بازی‌های ثابت و بازی‌های تکراری.

\begin{itemize}
	\item \textbf{بازی ثابت (\lr{Static Game}):} بازی ثابت ساده‌ترین شکل برای مدل‌سازی تعاملات میان عوامل است. در بازی ثابت، هر عامل تنها یک تصمیم‌گیری واحد را انجام می‌دهد. از آنجایی که هر عامل تنها یک بار عمل می‌کند، تقلب و خیانت غیرمنتظره می‌تواند در این نوع بازی‌ها سودآور باشد. بنابراین، هر عامل نیاز دارد تا به دقت استراتژی‌های سایر عوامل را پیش‌بینی کند تا بتواند به طور هوشمندانه عمل کرده و بیشترین سود ممکن را کسب کند. بازی‌های ثابت معمولاً در سناریوهای رقابتی با تعاملات کوتاه مدت کاربرد دارند.
	
	\item \textbf{بازی تکراری (\lr{Repeated Game}):} بازی تکراری به وضعیتی اشاره دارد که در آن تمام عوامل می‌توانند بر اساس همان وضعیت برای چندین تکرار اقداماتی انجام دهند. سود کلی هر عامل مجموع سودهای تخفیف‌شده برای هر تکرار از بازی است. به دلیل اقدامات مکرر تمام عوامل، تقلب و خیانت در طول تعاملات می‌تواند منجر به مجازات یا انتقام از سوی سایر عوامل در تکرارهای آینده شود. بنابراین، بازی تکراری از رفتارهای مخرب عوامل جلوگیری می‌کند و به طور کلی سود کل برای تمام عوامل را افزایش می‌دهد. بازی‌های تکراری معمولاً در سناریوهای همکاری بلندمدت و تعاملات پویا کاربرد دارند.
\end{itemize}

این دسته‌بندی‌ها به محققان و توسعه‌دهندگان کمک می‌کنند تا بازی‌های چندعاملی را بر اساس ویژگی‌های مختلف آن‌ها شناسایی و تحلیل کنند. در بازی‌های ثابت، تمرکز بر پیش‌بینی دقیق استراتژی‌های دیگر عوامل و اتخاذ بهترین تصمیم در یک لحظه زمانی است. در مقابل، بازی‌های تکراری نیازمند توسعه استراتژی‌های پایدار و قابل اعتماد هستند که نه تنها در تکرار اول بلکه در تکرارهای بعدی نیز موثر باشند.

بازی‌های جمع صفر در یادگیری تقویتی چندعاملی به دلیل سادگی و قابلیت مدل‌سازی دقیق تعاملات رقابتی، به عنوان یک ابزار قدرتمند برای تحلیل و توسعه الگوریتم‌های \lr{MARL} مورد استفاده قرار می‌گیرند. این بازی‌ها امکان بررسی رفتارهای استراتژیک، بهینه‌سازی سیاست‌ها و تحلیل تعادل‌های نش (\lr{Nash Equilibrium}) را فراهم می‌کنند که در نهایت به بهبود عملکرد سیستم‌های چندعاملی منجر می‌شود.

\paragraph{مثال‌ها و کاربردها}
یکی از مثال‌های معروف بازی‌های جمع صفر، بازی شطرنج است که در آن هر حرکت یک بازیکن مستقیماً به نفع یا ضرر بازیکن دیگر است. سایر مثال‌ها شامل بازی‌های استراتژیک مانند \lr{Poker} و \lr{Go} می‌باشند که در آن‌ها تعاملات رقابتی میان بازیکنان به طور کامل با اصول بازی‌های جمع صفر مطابقت دارند.

در حوزه‌های عملی، بازی‌های جمع صفر می‌توانند برای مدل‌سازی رقابت‌های بازار، مذاکرات اقتصادی و حتی تعاملات میان ربات‌های خودران در محیط‌های رقابتی مورد استفاده قرار گیرند. این کاربردها به محققان امکان می‌دهند تا الگوریتم‌هایی طراحی کنند که قادر به بهینه‌سازی عملکرد در شرایط رقابتی و متغیر باشند.

\paragraph{چالش‌ها و فرصت‌ها}
یکی از چالش‌های اصلی در بازی‌های جمع صفر، پیش‌بینی دقیق رفتارهای رقبا و اتخاذ تصمیم‌های بهینه در مواجهه با استراتژی‌های متغیر آن‌ها است. همچنین، در بازی‌های تکراری، ایجاد تعادل‌های پایدار و جلوگیری از رفتارهای مخرب به عنوان یک چالش مهم مطرح است. با این حال، این چالش‌ها فرصت‌های قابل توجهی برای توسعه الگوریتم‌های پیشرفته و افزایش قابلیت‌های یادگیری تقویتی چندعاملی فراهم می‌کنند که می‌توانند در شرایط پیچیده‌تر و پویا نیز عملکرد مطلوبی داشته باشند.

    \section{تعادل نش}

    %\subsection{بازی مجموع صفر}
%
%بازی‌های مجموع صفر\LTRfootnote{Zero-Sum Games}
% دسته‌ای از بازی‌ها هستند که در آن‌ها تابع ارزش یک بازیکن دقیقاً برابر با ضرر بازیکن دیگر است. به عبارت دیگر، مجموع ارزشهای همه بازیکنان در هر مرحله صفر است.
%
%
%
%\begin{itemize}
%	\item تعریف بازی مجموع صفر:
%	در یک بازی دو نفره، اگر تابع ارزش بازیکن اول (\( V_1^{(\pi_1 ,\pi_2)}(s)
%	\)) و بازیکن دوم (\( V_2^{(\pi_1 ,\pi_2)}(s)
%	\)) به‌گونه‌ای باشد که برای هر مجموعه سیاست
%	\( (\pi_1, \pi_2) \) 
%به صورت زیر باشد را یک بازی مجموع صفر نامیده می‌شود.
%	\begin{equation}\label{eq:game_v}
%		V_1^{(\pi_1 ,\pi_2)}(s) + V_2^{(\pi_1 ,\pi_2)}(s) = 0 \to V_1^{(\pi_1 ,\pi_2)}(s) = -V_2^{(\pi_1 ,\pi_2)}(s)
%%		V_1^{(\pi_1 ,\pi_2)}(s) =- V_2^{(\pi_1, \pi_2)}(s)
%	\end{equation}
%	\item سیاست بهینه در بازی مجموع صفر:
%	در بازی‌های مجموع صفر، سیاست بهینه هر بازیکن، انتخابی است که  تابع ارزش خود را در برابر بهترین پاسخ حریف به حداکثر برساند. این سیاست اغلب به تعادل نش منجر می‌شود. سیاست بهینه دو بازیکن در بازی مجموع صفر با تابع  ارزش معادله
%	\eqref{eq:game_v}
%	به صورت زیر است.
%	
%	\begin{align}
%		V_1^*(s) = \max_{\pi_1} \min_{\pi_2} V_1^{(\pi_1 ,\pi_2)}(s) \\
%		V_2^*(s) = \max_{\pi_2} \min_{\pi_1} V_2^{(\pi_1 ,\pi_2)}(s)
%	\end{align}
%	
%	
%\end{itemize}


















\subsection{بازی مجموع صفر}

بازی‌های مجموع صفر\LTRfootnote{Zero-Sum Games}
دسته‌ای از بازی‌ها هستند که در آن‌ها تابع ارزش یک بازیکن دقیقاً برابر با ضرر بازیکن دیگر است؛ بنابراین، مجموع ارزش‌های همهٔ بازیکنان در هر مرحله صفر خواهد بود.

\begin{itemize}
	%------------------------------------
	\item \textbf{تعریف بازی مجموع صفر:}
	
	در یک بازی دو نفره، اگر تابع ارزشِ حالت (value) بازیکن اوّل 
	\(V_1^{(\pi_1 ,\pi_2)}(s)\)
	و بازیکن دوم 
	\(V_2^{(\pi_1 ,\pi_2)}(s)\)
	برای هر مجموعه سیاست 
	\((\pi_1,\pi_2)\)
	به‌گونه‌ای باشند که:
	\begin{equation}\label{eq:game_v}
		V_1^{(\pi_1 ,\pi_2)}(s) + V_2^{(\pi_1 ,\pi_2)}(s) = 0 
		\;\;\Longrightarrow\;\;
		V_1^{(\pi_1 ,\pi_2)}(s) = -\,V_2^{(\pi_1 ,\pi_2)}(s),
	\end{equation}
	آنگاه آن بازی را \emph{بازی مجموع صفر} می‌نامیم.
	
	به‌طور مشابه، اگر تابع ارزش–عمل برای دو بازیکن را با
	\(Q_1^{(\pi_1,\pi_2)}(s,a_1,a_2)\)
	و
	\(Q_2^{(\pi_1,\pi_2)}(s,a_1,a_2)\)
	نشان دهیم، باید برقرار باشد:
	\begin{equation}\label{eq:game_q}
		Q_1^{(\pi_1,\pi_2)}(s,a_1,a_2) + 
		Q_2^{(\pi_1,\pi_2)}(s,a_1,a_2) = 0
		\;\;\Longrightarrow\;\;
		Q_1^{(\pi_1,\pi_2)}(s,a_1,a_2) = -\,Q_2^{(\pi_1,\pi_2)}(s,a_1,a_2).
	\end{equation}
	%------------------------------------
	
	\item \textbf{سیاست بهینه در بازی مجموع صفر:}
	
	در این بازی‌ها، هر بازیکن سیاستی را برمی‌گزیند که تابع ارزش خود را
	در برابر بهترین پاسخِ حریف بیشینه کند؛ این انتخاب در نهایت به
	تعادل نش منجر می‌شود.
	
	به‌صورت تابع ارزشِ حالت:
	\begin{align}
		V_1^*(s) &= \max_{\pi_1}\,\min_{\pi_2} \;
		V_1^{(\pi_1 ,\pi_2)}(s), \\
		V_2^*(s) &= \max_{\pi_2}\,\min_{\pi_1} \;
		V_2^{(\pi_1 ,\pi_2)}(s).
	\end{align}
	
	و به‌صورت تابع ارزش–عمل:
	\begin{align}
		Q_1^*(s,a_1,a_2) &= \max_{\pi_1}\,\min_{\pi_2} \;
		Q_1^{(\pi_1 ,\pi_2)}(s,a_1,a_2), \\
		Q_2^*(s,a_1,a_2) &= \max_{\pi_2}\,\min_{\pi_1} \;
		Q_2^{(\pi_1 ,\pi_2)}(s,a_1,a_2).
	\end{align}
\end{itemize}



    \section{ گرادیان سیاست عمیق قطعی 
در بازی‌های دو­عاملیِ مجموع‌­صفر
}
% insert_edit_into_file("/Users/Ali/Documents/BAI/Master/master-thesis/Report/Chapters/MARL/MADDPG.tex", r"\section{ گرادیان سیاست عمیق قطعی 
% در بازی‌های دو­عاملیِ مجموع‌­صفر
% }", """
% \section{ گرادیان سیاست عمیق قطعی 
% در بازی‌های دو­عاملیِ مجموع‌­صفر
% }\label{sec:MADDPG}

گرادیان سیاست عمیق قطعی چند­عاملی\LTRfootnote{Multi-Agent Deep Deterministic Policy Gradient (MADDPG)}
توسعه‌ای از الگوریتم \lr{DDPG} برای محیط‌های چند­عاملی است. در این بخش، به بررسی این الگوریتم در چارچوب بازی‌های دو­عاملیِ مجموع­‌صفر می‌پردازیم که در آن مجموع پاداش‌های دو عامل همواره صفر است (آنچه یک عامل به دست می‌آورد، عامل دیگر از دست می‌دهد).

\subsection{چالش‌های یادگیری تقویتی در محیط‌های چند­عاملی}

در محیط‌های چند­عاملی، سیاست هر عامل مدام در حال تغییر است، که باعث می‌شود محیط از دید هر عامل غیرایستا\LTRfootnote{Non-stationary} شود. این مسئله چالش بزرگی برای الگوریتم‌های یادگیری تقویتی تک‌عاملی مانند \lr{DDPG} ایجاد می‌کند، زیرا فرض ایستایی محیط را نقض می‌کند.

\lr{MADDPG} با استفاده از رویکرد آموزش متمرکز، اجرای غیرمتمرکز\LTRfootnote{Centralized Training, Decentralized Execution} این مشکل را حل می‌کند. در این رویکرد، هر عامل در زمان آموزش به اطلاعات کامل محیط دسترسی دارد، اما در زمان اجرا تنها از مشاهدات محلی خود استفاده می‌کند.

\subsection{معماری \lr{MADDPG} در بازی‌های مجموع­‌صفر}

در یک بازی دو­عاملیِ مجموع­‌صفر، دو عامل با نمادهای 1 و 2 نشان داده می‌شوند. هر عامل دارای شبکه‌های منحصر به فرد خود است:

\begin{itemize}
    \item \textbf{شبکه‌های بازیگر:} $\mu_{\theta_1}(o_1)$ و $\mu_{\theta_2}(o_2)$ که مشاهدات محلی $o_1$ و $o_2$ را به اعمال $a_1$ و $a_2$ نگاشت می‌کنند.
    \item \textbf{شبکه‌های منتقد:} $Q_{\phi_1}(o_1, a_1, a_2)$ و $Q_{\phi_2}(o_2, a_2, a_1)$ که ارزش حالت-عمل را با توجه به مشاهدات و اعمال تمام عامل‌ها تخمین می‌زنند.
    \item \textbf{شبکه‌های هدف:} مشابه \lr{DDPG}، برای پایدار کردن آموزش از شبکه‌های هدف استفاده می‌شود.
\end{itemize}

در بازی‌های مجموع­‌صفر، پاداش‌ها رابطه $r_1 + r_2 = 0$ دارند که در آن $r_1$ و $r_2$ پاداش‌های دریافتی عامل‌ها هستند. در نتیجه، $r_2 = -r_1$ است که نمایانگر تضاد کامل منافع بین عامل‌هاست.

\subsection{آموزش \lr{MADDPG} در بازی‌های مجموع­‌صفر}

فرایند آموزش \lr{MADDPG} برای بازی‌های مجموع­‌صفر به شرح زیر است:

\subsubsection{یادگیری تابع \lr{Q}}

برای هر عامل $i \in \{1, 2\}$، تابع \lr{Q} با کمینه کردن خطای میانگین مربعات بلمن به‌روزرسانی می‌شود:

\begin{equation}
    L(\phi_i, \mathcal{D}) = \underset{(\boldsymbol{o}, \boldsymbol{a}, r_i, \boldsymbol{o}', d) \sim \mathcal{D}}{\mathrm{E}}\left[ 
    \Bigg( Q_{\phi_i}(o_i, a_1, a_2) - y_i \Bigg)^2
    \right]
\end{equation}

که در آن $\boldsymbol{o} = (o_1, o_2)$ بردار مشاهدات، $\boldsymbol{a} = (a_1, a_2)$ بردار اعمال، و $y_i$ هدف برای عامل $i$ است:

\begin{equation}
    y_i = r_i + \gamma (1 - d) Q_{\phi_{i,\text{targ}}}(o_i', \mu_{\theta_{1,\text{targ}}}(o_1'), \mu_{\theta_{2,\text{targ}}}(o_2'))
\end{equation}

توجه کنید که منتقد هر عامل به اعمال همه عامل‌ها دسترسی دارد، اما در بازی‌های مجموع­‌صفر، عامل شماره 2 جهت مخالف هدف عامل 1 را دنبال می‌کند.

\subsubsection{یادگیری سیاست}

سیاست هر عامل با بیشینه کردن تابع \lr{Q} مربوط به آن عامل به‌روزرسانی می‌شود:

\begin{equation}
    \max_{\theta_i} \underset{\boldsymbol{o} \sim \mathcal{D}}{\mathrm{E}}\left[ Q_{\phi_i}(o_i, \mu_{\theta_i}(o_i), \mu_{\theta_{-i}}(o_{-i})) \right]
\end{equation}

که در آن $-i$ نشان‌دهنده عامل مقابل است. با توجه به ماهیت بازی مجموع­‌صفر، هر عامل تلاش می‌کند تا مطلوبیت خود را افزایش دهد، در حالی که مطلوبیت عامل دیگر به طور همزمان کاهش می‌یابد.

\subsubsection{شبکه‌های هدف و بافر تجربه}

مشابه \lr{DDPG}، برای پایدار کردن آموزش، شبکه‌های هدف با میانگین‌گیری پولیاک به‌روزرسانی می‌شوند:

\begin{align*}
    \phi_{i,\text{targ}} &\leftarrow \rho \phi_{i,\text{targ}} + (1 - \rho) \phi_i \\
    \theta_{i,\text{targ}} &\leftarrow \rho \theta_{i,\text{targ}} + (1 - \rho) \theta_i
\end{align*}

همچنین، از یک بافر تکرار بازی مشترک برای ذخیره تجربیات استفاده می‌شود که شامل وضعیت‌ها، اعمال و پاداش‌های همه عامل‌هاست.

\subsection{اکتشاف در \lr{MADDPG}}

اکتشاف در \lr{MADDPG} مشابه \lr{DDPG} است، اما برای هر عامل به طور جداگانه اعمال می‌شود. در طی آموزش، به اعمال هر عامل نویز اضافه می‌شود:

\begin{equation}
    a_i = \text{clip}(\mu_{\theta_i}(o_i) + \epsilon_i, a_{\text{Low}}, a_{\text{High}})
\end{equation}

که در آن $\epsilon_i$ نویز اضافه شده به عامل $i$ است.

\subsection{شبه‌کد \lr{MADDPG} برای بازی‌های دو­عاملیِ مجموع­‌صفر}

\begin{algorithm}[H]
    \caption{گرادیان سیاست عمیق قطعی چند­عاملی برای بازی‌های مجموع­‌صفر}\label{alg:MADDPG}
    \begin{algorithmic}[1]
        \ورودی پارامترهای اولیه سیاست عامل‌ها $(\theta_1, \theta_2)$، پارامترهای تابع \lr{Q} $(\phi_1, \phi_2)$، بافر تکرار بازی خالی $(\mathcal{D})$
        \State پارامترهای هدف را برابر با پارامترهای اصلی قرار دهید: $\theta_{i,\text{targ}} \leftarrow \theta_i$, $\phi_{i,\text{targ}} \leftarrow \phi_i$ برای $i \in \{1, 2\}$
        
        \While{همگرایی رخ دهد}
            \State \parbox[t]{\dimexpr\linewidth-\algorithmicindent}{
            مشاهدات $(o_1, o_2)$ را دریافت کنید
            \strut}
            \State \parbox[t]{\dimexpr\linewidth-\algorithmicindent}{
            برای هر عامل $i$، عمل $a_i = \text{clip}(\mu_{\theta_i}(o_i) + \epsilon_i, a_{\text{Low}}, a_{\text{High}})$ را انتخاب کنید، به‌طوری که $\epsilon_i \sim \mathcal{N}$ است
            \strut}
            \State اعمال $(a_1, a_2)$ را در محیط اجرا کنید
            \State \parbox[t]{\dimexpr\linewidth-\algorithmicindent}{
            مشاهدات بعدی $(o_1', o_2')$، پاداش‌ها $(r_1, r_2=-r_1)$ و سیگنال پایان $d$ را دریافت کنید
            \strut}
            \State تجربه $(o_1, o_2, a_1, a_2, r_1, r_2, o_1', o_2', d)$ را در بافر $\mathcal{D}$ ذخیره کنید
            \State اگر $d=1$ است، وضعیت محیط را بازنشانی کنید
            
            \If{زمان به‌روزرسانی فرا رسیده است}
                \For{هر تعداد به‌روزرسانی}
                    \State یک دسته تصادفی از تجربیات، $B = \{(\boldsymbol{o}, \boldsymbol{a}, r_1, r_2, \boldsymbol{o}', d)\}$، از $\mathcal{D}$ نمونه‌گیری کنید
                    \State \parbox[t]{\dimexpr\linewidth-\algorithmicindent}{
                    اهداف را محاسبه کنید:
                    \begin{align*}
                        y_1 &= r_1 + \gamma (1-d) Q_{\phi_{1,\text{targ}}}(o_1', \mu_{\theta_{1,\text{targ}}}(o_1'), \mu_{\theta_{2,\text{targ}}}(o_2')) \\
                        y_2 &= r_2 + \gamma (1-d) Q_{\phi_{2,\text{targ}}}(o_2', \mu_{\theta_{2,\text{targ}}}(o_2'), \mu_{\theta_{1,\text{targ}}}(o_1'))
                    \end{align*}
                    \strut}
                    \State \parbox[t]{\dimexpr\linewidth-\algorithmicindent}{
                    توابع \lr{Q} را با نزول گرادیان به‌روزرسانی کنید:
                    \begin{align*}
                        \nabla_{\phi_1} \frac{1}{|B|}\sum_{(\boldsymbol{o}, \boldsymbol{a}, r_1, r_2, \boldsymbol{o}', d) \in B} \left( Q_{\phi_1}(o_1, a_1, a_2) - y_1 \right)^2 \\
                        \nabla_{\phi_2} \frac{1}{|B|}\sum_{(\boldsymbol{o}, \boldsymbol{a}, r_1, r_2, \boldsymbol{o}', d) \in B} \left( Q_{\phi_2}(o_2, a_2, a_1) - y_2 \right)^2
                    \end{align*}
                    \strut}
                    
                    \State \parbox[t]{\dimexpr\linewidth-\algorithmicindent}{
                    سیاست‌ها را با صعود گرادیان به‌روزرسانی کنید:
                    \begin{align*}
                        \nabla_{\theta_1} \frac{1}{|B|}\sum_{\boldsymbol{o} \in B}Q_{\phi_1}(o_1, \mu_{\theta_1}(o_1), a_2) \\
                        \nabla_{\theta_2} \frac{1}{|B|}\sum_{\boldsymbol{o} \in B}Q_{\phi_2}(o_2, \mu_{\theta_2}(o_2), a_1)
                    \end{align*}
                    \strut}
                    
                    \State \parbox[t]{\dimexpr\linewidth-\algorithmicindent}{
                    شبکه‌های هدف را به‌روزرسانی کنید:
                    \begin{align*}
                        \phi_{1,\text{targ}} &\leftarrow \rho \phi_{1,\text{targ}} + (1-\rho) \phi_1 \\
                        \phi_{2,\text{targ}} &\leftarrow \rho \phi_{2,\text{targ}} + (1-\rho) \phi_2 \\
                        \theta_{1,\text{targ}} &\leftarrow \rho \theta_{1,\text{targ}} + (1-\rho) \theta_1 \\
                        \theta_{2,\text{targ}} &\leftarrow \rho \theta_{2,\text{targ}} + (1-\rho) \theta_2
                    \end{align*}
                    \strut}
                \EndFor
            \EndIf
        \EndWhile
    \end{algorithmic}
\end{algorithm}

\subsection{مزایای \lr{MADDPG} در بازی‌های مجموع­‌صفر}

\lr{MADDPG} چندین مزیت برای یادگیری در بازی‌های دو­عاملیِ مجموع­‌صفر ارائه می‌دهد:

\begin{itemize}
    \item \textbf{مقابله با غیرایستایی:} با استفاده از منتقدهایی که به اطلاعات کامل دسترسی دارند، مشکل غیرایستایی محیط از دید هر عامل حل می‌شود.
    \item \textbf{همگرایی بهتر:} در بازی‌های مجموع­‌صفر، \lr{MADDPG} معمولاً همگرایی بهتری نسبت به آموزش مستقل عامل‌ها با \lr{DDPG} نشان می‌دهد.
    \item \textbf{یادگیری استراتژی‌های متقابل:} عامل‌ها می‌توانند استراتژی‌های متقابل پیچیده را یاد بگیرند که در آموزش مستقل امکان‌پذیر نیست.
\end{itemize}

در بازی‌های دو­عاملیِ مجموع­‌صفر، این رویکرد به رقابت کامل بین عامل‌ها منجر می‌شود، که هر یک تلاش می‌کند بهترین استراتژی را در برابر استراتژی رقیب پیدا کند.
    \section{عامل گرادیان سیاست عمیق قطعی تاخیری دوگانه چند­عاملی}\label{sec:MATD3}

عامل گرادیان سیاست عمیق قطعی تاخیری دوگانه چند­عاملی\LTRfootnote{Multi-Agent Twin Delayed Deep Deterministic Policy Gradient (MA-TD3)}
توسعه‌ای از الگوریتم \lr{TD3} برای محیط‌های چند­عاملی است. در این بخش، به بررسی این الگوریتم در چارچوب بازی‌های چندعاملیِ مجموع­‌صفر می‌پردازیم که در آن ترکیب ویژگی‌های \lr{TD3} با رویکرد چند­عاملی \lr{MA-DDPG} به پایداری و کارایی بیشتر در یادگیری منجر می‌شود.

\subsection{چالش‌های یادگیری تقویتی در محیط‌های چند­عاملی و راه‌حل \lr{MA-TD3}}

در محیط‌های چند­عاملی، عامل‌ها همزمان سیاست‌های خود را تغییر می‌دهند که باعث غیرایستایی محیط از دید هر عامل می‌شود. علاوه بر این، بیش‌برآورد تابع \lr{Q} که در \lr{DDPG} دیده می‌شود، در محیط‌های چند­عاملی می‌تواند تشدید شود.

\lr{MA-TD3} هر دو چالش را با ترکیب رویکردهای زیر حل می‌کند:
\begin{itemize}
    \item \textbf{آموزش متمرکز، اجرای غیرمتمرکز:} مشابه \lr{MA-DDPG}، از منتقدهایی استفاده می‌کند که به اطلاعات کامل دسترسی دارند.
    \item \textbf{منتقدهای دوگانه:} برای هر عامل، از دو شبکه منتقد استفاده می‌کند تا بیش‌برآورد تابع \lr{Q} را کاهش دهد.
    \item \textbf{به‌روزرسانی‌های تاخیری سیاست:} سیاست‌ها را با تواتر کمتری نسبت به منتقدها به‌روزرسانی می‌کند.
\end{itemize}

\subsection{معماری \lr{MA-TD3} در بازی‌های مجموع­‌صفر}

در یک بازی چندعاملیِ مجموع­‌صفر، هر عامل دارای شبکه‌های زیر است:

\begin{itemize}
    \item \textbf{شبکه بازیگر:} $\mu_{\theta_i}(o_i)$ که مشاهدات محلی $o_i$ را به اعمال $a_i$ نگاشت می‌کند.
    \item \textbf{شبکه‌های منتقد دوگانه:} $Q_{\phi_{i,1}}(o_i, a_1, a_2)$ و $Q_{\phi_{i,2}}(o_i, a_1, a_2)$ که ارزش حالت-عمل را تخمین می‌زنند.
    \item \textbf{شبکه‌های هدف:} برای پایدارسازی آموزش، از نسخه‌های هدف بازیگر و منتقدها استفاده می‌شود.
\end{itemize}

%در بازی‌های مجموع­‌صفر، پاداش‌ها رابطه $r_1 + r_2 = 0$ دارند، بنابراین $r_2 = -r_1$ است.

\subsection{آموزش \lr{MA-TD3}}

فرایند آموزش \lr{MA-TD3} به شرح زیر است:

\subsubsection{یادگیری تابع \lr{Q}}

برای هر عامل $i \in \{1, 2\}$ و هر منتقد $j \in \{1, 2\}$، تابع \lr{Q} با کمینه کردن خطای میانگین مربعات بلمن به‌روزرسانی می‌شود:

\begin{equation}
    L(\phi_{i,j}, \mathcal{D}) = \underset{(\boldsymbol{o}, \boldsymbol{a}, r_i, \boldsymbol{o}', d) \sim \mathcal{D}}{\mathrm{E}}\left[ 
    \Bigg( Q_{\phi_{i,j}}(o_i, a_1, a_2) - y_i \Bigg)^2
    \right]
\end{equation}

که در آن $y_i$ هدف برای عامل $i$ است:

\begin{equation}
    y_i = r_i + \gamma (1 - d) \min_{j=1,2} Q_{\phi_{i,j,\text{targ}}}(o_i', \mu_{\theta_{1,\text{targ}}}(o_1'), \mu_{\theta_{2,\text{targ}}}(o_2'))
\end{equation}

استفاده از عملگر حداقل روی دو منتقد، بیش‌برآورد را کاهش می‌دهد که منجر به تخمین‌های محتاطانه‌تر و پایدارتر می‌شود.

\subsubsection{یادگیری سیاست با تاخیر}

سیاست هر عامل با تاخیر (معمولاً پس از هر دو به‌روزرسانی منتقدها) و با بیشینه کردن تابع \lr{Q} اول به‌روزرسانی می‌شود:

\begin{equation}
    \max_{\theta_i} \underset{\boldsymbol{o} \sim \mathcal{D}}{\mathrm{E}}\left[ 
    Q_{\phi_{i,1}}\big(o_i, \mu_{\theta_i}(o_i), \mu_{\theta_{-i}}(o_{-i})\big) 
    \right]
\end{equation}

% در عمل، بسته به پیاده‌سازی CTDE، می‌توان از عملِ عامل مقابل در بافر ($a_{-i}$) یا از سیاست فعلیِ او استفاده کرد.

به‌روزرسانی تاخیری سیاست اجازه می‌دهد تا منتقدها قبل از تغییر سیاست به مقادیر دقیق‌تری همگرا شوند.

\subsubsection{شبکه‌های هدف}

مشابه \lr{TD3}، شبکه‌های هدف با میانگین‌گیری پولیاک به‌روزرسانی می‌شوند.

%\begin{align*}
%    \phi_{i,j,\text{targ}} &\leftarrow \rho \phi_{i,j,\text{targ}} + (1 - \rho) \phi_{i,j} \quad \text{برای } j=1,2 \\
%    \theta_{i,\text{targ}} &\leftarrow \rho \theta_{i,\text{targ}} + (1 - \rho) \theta_i
%\end{align*}

\subsection{اکتشاف در \lr{MA-TD3}}

اکتشاف در \lr{MA-TD3} با افزودن نویز به اعمال هر عامل انجام می‌شود:

\begin{equation}
    a_i = \text{clip}(\mu_{\theta_i}(o_i) + \epsilon_i, a_{\text{Low}}, a_{\text{High}})
\end{equation}

که در آن $\epsilon_i \sim \mathcal{N}(0, \sigma_i)$ است و مقدار $\sigma_i$ به مرور زمان کاهش می‌یابد.

\subsection{شبه‌کد \lr{MA-TD3} برای بازی‌های چندعاملیِ مجموع­‌صفر}

در این بخش، شبه‌کد الگوریتم \lr{MA-TD3} پیاده‌سازی‌شده آورده شده‌است. در این پژوهش الگوریتم~\رجوع{alg:MA-TD3} در محیط پایتون با استفاده از کتابخانه \lr{PyTorch} \cite{paszke2017automatic} پیاده‌سازی شده‌است.

\begin{algorithm}[H]
    \caption{عامل گرادیان سیاست عمیق قطعی تاخیری دوگانه چند­عاملی}\label{alg:MA-TD3}
    \begin{algorithmic}[1]
        \ورودی پارامترهای اولیه سیاست عامل‌ها $(\theta_1, \theta_2)$، پارامترهای توابع \lr{Q} $(\phi_{1,1}, \phi_{1,2}, \phi_{2,1}, \phi_{2,2})$، بافر تکرار بازی خالی $(\mathcal{D})$
        \State پارامترهای هدف را برابر با پارامترهای اصلی قرار دهید: 
        \Statex \hspace{\algorithmicindent}
        $\theta_{i,\text{targ}} \leftarrow \theta_i$, $\phi_{i,j,\text{targ}} \leftarrow \phi_{i,j}$ برای $i \in \{1, 2\}$ و $j \in \{1, 2\}$
        
        \While{همگرایی رخ دهد}
            \State \parbox[t]{\dimexpr\linewidth-\algorithmicindent}{
            مشاهدات $(o_1, o_2)$ را دریافت کنید
            \strut}
            \State \parbox[t]{\dimexpr\linewidth-\algorithmicindent}{
            برای هر عامل $i$، عمل $a_i = \text{clip}(\mu_{\theta_i}(o_i) + \epsilon_i, a_{\text{Low}}, a_{\text{High}})$ را انتخاب کنید، به‌طوری که $\epsilon_i \sim \mathcal{N}(0, \sigma_i)$ است
            \strut}
            \State اعمال $(a_1, a_2)$ را در محیط اجرا کنید
            \State \parbox[t]{\dimexpr\linewidth-\algorithmicindent}{
            مشاهدات بعدی $(o_1', o_2')$، پاداش‌ها $(r_1, r_2=-r_1)$ و سیگنال پایان $d$ را دریافت کنید
            \strut}
            \State تجربه $(o_1, o_2, a_1, a_2, r_1, r_2, o_1', o_2', d)$ را در بافر $\mathcal{D}$ ذخیره کنید
            \State اگر $d=1$ است، وضعیت محیط را بازنشانی کنید
            
            \If{زمان به‌روزرسانی فرا رسیده است}
                \For{$j$ در هر تعداد به‌روزرسانی}
                    \State % \parbox[t]{\dimexpr\linewidth-\algorithmicindent}{
                    یک دسته تصادفی از تجربیات، $B = \{(\boldsymbol{o}, \boldsymbol{a}, r_1, r_2, \boldsymbol{o}', d)\}$، از $\mathcal{D}$ نمونه‌گیری کنید.
%                    \strut}
                    \State اهداف را محاسبه کنید:
                     \vspace{-15pt}
                    \begin{align*}
                        y_1 &= r_1 + \gamma (1-d) \min_{k=1,2} Q_{\phi_{1,k,\text{targ}}}(o_1', \mu_{\theta_{1,\text{targ}}}(o_1'), \mu_{\theta_{2,\text{targ}}}(o_2')) \\
                        y_2 &= r_2 + \gamma (1-d) \min_{k=1,2} Q_{\phi_{2,k,\text{targ}}}(o_2', \mu_{\theta_{2,\text{targ}}}(o_2'), \mu_{\theta_{1,\text{targ}}}(o_1'))
                    \end{align*}
                    \vspace{-35pt}
                    
                    \State توابع \lr{Q} را با نزول گرادیان به‌روزرسانی کنید:
                    \vspace{-15pt}
                    \begin{align*}
                        \nabla_{\phi_{1,k}} \frac{1}{|B|}\sum_{B} \left( Q_{\phi_{1,k}}(o_1, a_1, a_2) - y_1 \right)^2 \quad \text{برای } k=1,2 \\
                        \nabla_{\phi_{2,k}} \frac{1}{|B|}\sum_{B} \left( Q_{\phi_{2,k}}(o_2, a_2, a_1) - y_2 \right)^2 \quad \text{برای } k=1,2
                    \end{align*}
                    \vspace{-30pt}
                    
                    \If{باقیمانده $j$ بر تاخیر سیاست برابر $0$ باشد}
                        \State سیاست‌ها را با صعود گرادیان به‌روزرسانی کنید:
                        \vspace{-15pt}
                        \begin{align*}
                            \nabla_{\theta_1} \frac{1}{|B|}\sum_{\boldsymbol{o} \in B}Q_{\phi_{1,1}}(o_1, \mu_{\theta_1}(o_1), a_2) \\
                            \nabla_{\theta_2} \frac{1}{|B|}\sum_{\boldsymbol{o} \in B}Q_{\phi_{2,1}}(o_2, \mu_{\theta_2}(o_2), a_1)
                        \end{align*}
                        \vspace{-35pt}
                        
                        \State شبکه‌های هدف را به‌روزرسانی کنید:
                        \vspace{-15pt}
                        \begin{align*}
                            \phi_{i,k,\text{targ}} &\leftarrow \rho \phi_{i,k,\text{targ}} + (1-\rho) \phi_{i,k} \quad \text{برای } i,k \in \{1,2\} \\
                            \theta_{i,\text{targ}} &\leftarrow \rho \theta_{i,\text{targ}} + (1-\rho) \theta_i \quad \text{برای } i \in \{1,2\}
                        \end{align*}
                    \EndIf
                \EndFor
            \EndIf
        \EndWhile
        \vspace{-15pt}
    \end{algorithmic}
\end{algorithm}

\subsection{مزایای \lr{MA-TD3} در بازی‌های مجموع­‌صفر}

\lr{MA-TD3} مزایای زیر را نسبت به \lr{MA-DDPG} در بازی‌های چندعاملیِ مجموع­‌صفر ارائه می‌دهد:

\begin{itemize}
    \item \textbf{پایداری بیشتر:} با استفاده از منتقدهای دوگانه، بیش‌برآورد تابع \lr{Q} که در محیط‌های غیرایستای چند­عاملی شدیدتر است، کاهش می‌یابد.
    \item \textbf{یادگیری کارآمدتر:} به‌روزرسانی‌های تاخیری سیاست اجازه می‌دهد منتقدها به تخمین‌های دقیق‌تری دست یابند، که منجر به بهبود کیفیت یادگیری سیاست می‌شود.
    \item \textbf{مقاومت در برابر نویز:} ترکیب منتقدهای دوگانه با رویکرد آموزش متمرکز، مقاومت الگوریتم در برابر نویز و تغییرات محیط را افزایش می‌دهد.
    \item \textbf{همگرایی بهتر:} بهبودهای \lr{TD3} در کنار رویکرد چند­عاملی، به همگرایی سریع‌تر و پایدارتر در بازی‌های رقابتی منجر می‌شود.
\end{itemize}

در مجموع، \lr{MA-TD3} ترکیبی از بهترین ویژگی‌های \lr{TD3} و \lr{MA-DDPG} را ارائه می‌دهد که آن را به گزینه‌ای مناسب برای یادگیری سیاست‌های پیچیده در بازی‌های چندعاملیِ مجموع­‌صفر تبدیل می‌کند.
    \section{عامل عملگر نقاد نرم دو­عاملی}\label{sec:MASAC}

عامل عملگر نقاد نرم دو­عاملی\LTRfootnote{Multi-Agent Soft Actor-Critic (MASAC)}
توسعه‌ای از الگوریتم \lr{SAC} برای محیط‌های چند­عاملی است. در این بخش، به بررسی این الگوریتم در چارچوب بازی‌های دو­عاملیِ مجموع­‌صفر می‌پردازیم که در آن ترکیب ویژگی‌های \lr{SAC} با رویکرد چند­عاملی به پایداری و کارایی بیشتر در یادگیری منجر می‌شود.

\subsection{چالش‌های یادگیری تقویتی در محیط‌های چند­عاملی و راه‌حل \lr{MASAC}}

در محیط‌های چند­عاملی، عامل‌ها همزمان سیاست‌های خود را تغییر می‌دهند که باعث غیرایستایی محیط از دید هر عامل می‌شود. علاوه بر این، چالش‌های مربوط به تعادل اکتشاف-بهره‌برداری در محیط‌های چند­عاملی پیچیده‌تر است.

\lr{MASAC} این چالش‌ها را با ترکیب رویکردهای زیر حل می‌کند:
\begin{itemize}
    \item \textbf{آموزش متمرکز، اجرای غیرمتمرکز:} مشابه \lr{MADDPG}، از منتقدهایی استفاده می‌کند که به اطلاعات کامل دسترسی دارند.
    \item \textbf{سیاست‌های تصادفی:} برخلاف \lr{MADDPG} و \lr{MATD3} که سیاست‌های قطعی دارند، \lr{MASAC} از سیاست‌های تصادفی استفاده می‌کند.
    \item \textbf{تنظیم آنتروپی:} با استفاده از تنظیم آنتروپی، اکتشاف و همگرایی به سیاست‌های بهتر را بهبود می‌بخشد.
    \item \textbf{منتقدهای دوگانه:} برای هر عامل، از دو شبکه منتقد استفاده می‌کند تا بیش‌برآورد تابع \lr{Q} را کاهش دهد.
\end{itemize}

\subsection{معماری \lr{MASAC} در بازی‌های مجموع­‌صفر}

در یک بازی دو­عاملیِ مجموع­‌صفر، هر عامل دارای شبکه‌های زیر است:

\begin{itemize}
    \item \textbf{شبکه بازیگر:} $\pi_{\theta_i}(a_i|o_i)$ که توزیع احتمال اعمال را با توجه به مشاهدات محلی تعیین می‌کند.
    \item \textbf{شبکه‌های منتقد دوگانه:} $Q_{\phi_{i,1}}(o_i, a_1, a_2)$ و $Q_{\phi_{i,2}}(o_i, a_1, a_2)$ که ارزش حالت-عمل را تخمین می‌زنند.
    \item \textbf{شبکه‌های هدف:} برای پایدارسازی آموزش، از نسخه‌های هدف منتقدها استفاده می‌شود.
\end{itemize}

در بازی‌های مجموع­‌صفر، پاداش‌ها رابطه $r_1 + r_2 = 0$ دارند، بنابراین $r_2 = -r_1$ است.

\subsection{آموزش \lr{MASAC}}

فرایند آموزش \lr{MASAC} به شرح زیر است:

\subsubsection{یادگیری تابع \lr{Q}}

برای هر عامل $i \in \{1, 2\}$ و هر منتقد $j \in \{1, 2\}$، تابع \lr{Q} با کمینه کردن خطای میانگین مربعات بلمن به‌روزرسانی می‌شود:

\begin{equation}
    L(\phi_{i,j}, \mathcal{D}) = \underset{(\boldsymbol{o}, \boldsymbol{a}, r_i, \boldsymbol{o}', d) \sim \mathcal{D}}{\mathrm{E}}\left[ 
    \Bigg( Q_{\phi_{i,j}}(o_i, a_1, a_2) - y_i \Bigg)^2
    \right]
\end{equation}

که در آن $y_i$ هدف برای عامل $i$ است:

\begin{equation}
    y_i = r_i + \gamma (1 - d) \Big( \min_{j=1,2} Q_{\phi_{i,j,\text{targ}}}(o_i', \tilde{a}_1', \tilde{a}_2') - \alpha_i \log \pi_{\theta_i}(\tilde{a}_i'|o_i') \Big)
\end{equation}

که در آن $\tilde{a}_i' \sim \pi_{\theta_i}(\cdot|o_i')$ است. استفاده از عملگر حداقل روی دو منتقد، بیش‌برآورد را کاهش می‌دهد که منجر به تخمین‌های محتاطانه‌تر و پایدارتر می‌شود.

\subsubsection{یادگیری سیاست}

سیاست هر عامل با بیشینه کردن ترکیبی از تابع \lr{Q} و آنتروپی به‌روزرسانی می‌شود:

\begin{equation}
    \max_{\theta_i} \underset{\boldsymbol{o} \sim \mathcal{D}}{\mathrm{E}}\left[ \min_{j=1,2}Q_{\phi_{i,j}}(o_i, \tilde{a}_i, a_{-i}) - \alpha_i \log \pi_{\theta_i}(\tilde{a}_i|o_i) \right]
\end{equation}

که در آن $\tilde{a}_i \sim \pi_{\theta_i}(\cdot|o_i)$ است و از ترفند پارامترسازی مجدد برای استخراج گرادیان استفاده می‌شود:

\begin{equation}
    \tilde{a}_{i,\theta_i}(o_i, \xi_i) = \tanh\left( \mu_{\theta_i}(o_i) + \sigma_{\theta_i}(o_i) \odot \xi_i \right), \;\;\;\;\; \xi_i \sim \mathcal{N}
\end{equation}

\subsubsection{شبکه‌های هدف}

مشابه \lr{SAC}، شبکه‌های هدف منتقد با میانگین‌گیری پولیاک به‌روزرسانی می‌شوند:

\begin{equation}
    \phi_{i,j,\text{targ}} \leftarrow \rho \phi_{i,j,\text{targ}} + (1 - \rho) \phi_{i,j} \quad \text{برای } j=1,2
\end{equation}

\subsubsection{تنظیم ضریب آنتروپی}

یکی از مزایای \lr{MASAC}، توانایی تنظیم خودکار ضریب آنتروپی $\alpha_i$ برای هر عامل است که می‌تواند با استفاده از یک تابع هزینه مجزا بهینه شود:

\begin{equation}
    \min_{\alpha_i} \underset{\boldsymbol{o} \sim \mathcal{D}, \tilde{a}_i \sim \pi_{\theta_i}}{\mathrm{E}}\left[ -\alpha_i \Big(\log \pi_{\theta_i}(\tilde{a}_i|o_i) + H_{\text{target}} \Big) \right]
\end{equation}

که در آن $H_{\text{target}}$ آنتروپی هدف است که به عنوان یک ابرپارامتر تعیین می‌شود.

\subsection{اکتشاف در \lr{MASAC}}

اکتشاف در \lr{MASAC} به صورت ذاتی از طریق سیاست‌های تصادفی و تنظیم آنتروپی انجام می‌شود. برخلاف \lr{MADDPG} و \lr{MATD3} که به افزودن نویز به اعمال نیاز دارند، \lr{MASAC} اعمال را مستقیماً از توزیع احتمال سیاست نمونه‌گیری می‌کند:

\begin{equation}
    a_i \sim \pi_{\theta_i}(\cdot|o_i)
\end{equation}

این رویکرد امکان اکتشاف ساختاریافته‌تر و کارآمدتر را فراهم می‌کند که در محیط‌های چند­عاملی پیچیده مفید است.

\subsection{شبه‌کد \lr{MASAC} برای بازی‌های دو­عاملیِ مجموع­‌صفر}

در این بخش، شبه‌کد الگوریتم \lr{MASAC} پیاده‌سازی‌شده آورده شده‌است. در این پژوهش الگوریتم~\رجوع{alg:MASAC} در محیط پایتون با استفاده از کتابخانه \lr{PyTorch} \cite{paszke2017automatic} پیاده‌سازی شده‌است.

\begin{algorithm}[H]
    \caption{عامل عملگر نقاد نرم دو­عاملی}\label{alg:MASAC}
    \begin{algorithmic}[1]
        \ورودی پارامترهای اولیه سیاست عامل‌ها $(\theta_1, \theta_2)$، پارامترهای توابع \lr{Q} $(\phi_{1,1}, \phi_{1,2}, \phi_{2,1}, \phi_{2,2})$، ضرایب آنتروپی $(\alpha_1, \alpha_2)$، بافر تکرار بازی خالی $(\mathcal{D})$
        \State پارامترهای هدف را برابر با پارامترهای اصلی قرار دهید: 
        \Statex \hspace{\algorithmicindent}
        $\phi_{i,j,\text{targ}} \leftarrow \phi_{i,j}$ برای $i \in \{1, 2\}$ و $j \in \{1, 2\}$
        
        \While{همگرایی رخ دهد}
            \State \parbox[t]{\dimexpr\linewidth-\algorithmicindent}{
            مشاهدات $(o_1, o_2)$ را دریافت کنید
            \strut}
            \State \parbox[t]{\dimexpr\linewidth-\algorithmicindent}{
            برای هر عامل $i$، عمل $a_i \sim \pi_{\theta_i}(\cdot|o_i)$ را انتخاب کنید
            \strut}
            \State اعمال $(a_1, a_2)$ را در محیط اجرا کنید
            \State \parbox[t]{\dimexpr\linewidth-\algorithmicindent}{
            مشاهدات بعدی $(o_1', o_2')$، پاداش‌ها $(r_1, r_2=-r_1)$ و سیگنال پایان $d$ را دریافت کنید
            \strut}
            \State تجربه $(o_1, o_2, a_1, a_2, r_1, r_2, o_1', o_2', d)$ را در بافر $\mathcal{D}$ ذخیره کنید
            \State اگر $d=1$ است، وضعیت محیط را بازنشانی کنید
            
            \If{زمان به‌روزرسانی فرا رسیده است}
                \For{هر تعداد به‌روزرسانی}
                    \State \parbox[t]{\dimexpr\linewidth-\algorithmicindent}{
                    یک دسته تصادفی از تجربیات، $B = \{(\boldsymbol{o}, \boldsymbol{a}, r_1, r_2, \boldsymbol{o}', d)\}$، از $\mathcal{D}$ نمونه‌گیری کنید
                    \strut}
                    \State اهداف را محاسبه کنید:
                     \vspace{-15pt}
                    \begin{align*}
                        y_1 &= r_1 + \gamma (1-d) \Big(\min_{j=1,2} Q_{\phi_{1,j,\text{targ}}}(o_1', \tilde{a}_1', \tilde{a}_2') - \alpha_1 \log \pi_{\theta_1}(\tilde{a}_1'|o_1') \Big) \\
                        y_2 &= r_2 + \gamma (1-d) \Big(\min_{j=1,2} Q_{\phi_{2,j,\text{targ}}}(o_2', \tilde{a}_2', \tilde{a}_1') - \alpha_2 \log \pi_{\theta_2}(\tilde{a}_2'|o_2') \Big)
                    \end{align*}
                    \vspace{-35pt}
                    
                    \State توابع \lr{Q} را با نزول گرادیان به‌روزرسانی کنید:
                    \vspace{-15pt}
                    \begin{align*}
                        \nabla_{\phi_{1,j}} \frac{1}{|B|}\sum_{B} \left( Q_{\phi_{1,j}}(o_1, a_1, a_2) - y_1 \right)^2 \quad \text{برای } j=1,2 \\
                        \nabla_{\phi_{2,j}} \frac{1}{|B|}\sum_{B} \left( Q_{\phi_{2,j}}(o_2, a_2, a_1) - y_2 \right)^2 \quad \text{برای } j=1,2
                    \end{align*}
                    \vspace{-30pt}
                    
                    \State سیاست‌ها را با صعود گرادیان به‌روزرسانی کنید:
                    \vspace{-15pt}
                    \begin{align*}
                        \nabla_{\theta_1} \frac{1}{|B|}\sum_{\boldsymbol{o} \in B}\Big[\min_{j=1,2} Q_{\phi_{1,j}}(o_1, \tilde{a}_{1,\theta_1}(o_1, \xi_1), a_2) - \alpha_1 \log \pi_{\theta_1}(\tilde{a}_{1,\theta_1}(o_1, \xi_1)|o_1) \Big] \\
                        \nabla_{\theta_2} \frac{1}{|B|}\sum_{\boldsymbol{o} \in B}\Big[\min_{j=1,2} Q_{\phi_{2,j}}(o_2, \tilde{a}_{2,\theta_2}(o_2, \xi_2), a_1) - \alpha_2 \log \pi_{\theta_2}(\tilde{a}_{2,\theta_2}(o_2, \xi_2)|o_2) \Big]
                    \end{align*}
                    \vspace{-35pt}
                    
                    \State ضرایب آنتروپی را با نزول گرادیان به‌روزرسانی کنید (اختیاری):
                    \vspace{-15pt}
                    \begin{align*}
                        \nabla_{\alpha_1} \frac{1}{|B|}\sum_{\boldsymbol{o} \in B} -\alpha_1 \Big(\log \pi_{\theta_1}(\tilde{a}_{1,\theta_1}(o_1, \xi_1)|o_1) + H_{\text{target}} \Big) \\
                        \nabla_{\alpha_2} \frac{1}{|B|}\sum_{\boldsymbol{o} \in B} -\alpha_2 \Big(\log \pi_{\theta_2}(\tilde{a}_{2,\theta_2}(o_2, \xi_2)|o_2) + H_{\text{target}} \Big)
                    \end{align*}
                    \vspace{-35pt}
                    
                    \State شبکه‌های هدف را به‌روزرسانی کنید:
                    \vspace{-15pt}
                    \begin{align*}
                        \phi_{i,j,\text{targ}} &\leftarrow \rho \phi_{i,j,\text{targ}} + (1-\rho) \phi_{i,j} \quad \text{برای } i,j \in \{1,2\}
                    \end{align*}
                \EndFor
            \EndIf
        \EndWhile
        \vspace{-15pt}
    \end{algorithmic}
\end{algorithm}

\subsection{مزایای \lr{MASAC} در بازی‌های مجموع­‌صفر}

\lr{MASAC} مزایای زیر را نسبت به سایر الگوریتم‌های چند­عاملی در بازی‌های دو­عاملیِ مجموع­‌صفر ارائه می‌دهد:

\begin{itemize}
    \item \textbf{اکتشاف بهتر:} استفاده از سیاست‌های تصادفی و تنظیم آنتروپی، اکتشاف فضای حالت-عمل را بهبود می‌بخشد که برای یافتن راه‌حل‌های بهینه در بازی‌های دو­عاملی ضروری است.
    \item \textbf{ثبات بیشتر:} ترکیب منتقدهای دوگانه با تنظیم آنتروپی، یادگیری را پایدارتر می‌کند و از همگرایی زودهنگام به سیاست‌های ضعیف جلوگیری می‌کند.
    \item \textbf{سازگاری با محیط‌های پیچیده:} توانایی تنظیم خودکار تعادل بین اکتشاف و بهره‌برداری، \lr{MASAC} را برای محیط‌های چند­عاملی پیچیده مناسب می‌سازد.
    \item \textbf{عملکرد بهتر در مسائل با چندین بهینه محلی:} سیاست‌های تصادفی می‌توانند از دام‌های بهینه محلی فرار کنند و به راه‌حل‌های بهتر برسند.
\end{itemize}

در مجموع، \lr{MASAC} ترکیبی از ویژگی‌های مثبت \lr{SAC} و رویکردهای چند­عاملی را ارائه می‌دهد که آن را به گزینه‌ای قدرتمند برای یادگیری سیاست‌های پیچیده در بازی‌های دو­عاملیِ مجموع­‌صفر تبدیل می‌کند، به‌ویژه در محیط‌هایی که اکتشاف کارآمد و سیاست‌های تصادفی اهمیت دارند.

     \section{عامل بهینه‌سازی سیاست مجاور چند‌عاملی}\label{sec:MAPPO}

عامل بهینه‌سازی سیاست مجاور دو­عاملی\LTRfootnote{Multi-Agent Proximal Policy Optimization (MA-PPO)}
توسعه‌ای از الگوریتم \lr{PPO} برای محیط‌های چند­عاملی است. در این بخش، به بررسی این الگوریتم در چارچوب بازی‌های چندعاملیِ مجموع­‌صفر می‌پردازیم که در آن ترکیب ویژگی‌های \lr{PPO} با رویکرد چند­عاملی به پایداری و کارایی بیشتر در یادگیری منجر می‌شود.

\subsection{چالش‌های یادگیری تقویتی در محیط‌های چند­عاملی و راه‌حل \lr{MA-PPO}}

در محیط‌های چند­عاملی، عامل‌ها همزمان سیاست‌های خود را تغییر می‌دهند که باعث غیرایستایی محیط از دید هر عامل می‌شود. این چالش با پیچیدگی‌های ذاتی الگوریتم‌های مبتنی بر گرادیان سیاست مانند \lr{PPO} ترکیب می‌شود.

\lr{MA-PPO} این چالش‌ها را با ترکیب رویکردهای زیر حل می‌کند:
\begin{itemize}
    \item \textbf{آموزش متمرکز، اجرای غیرمتمرکز:} مشابه سایر الگوریتم‌های چندعاملی، از منتقدهایی استفاده می‌کند که به اطلاعات کامل دسترسی دارند، اما بازیگران تنها به مشاهدات محلی خود دسترسی دارند.
    \item \textbf{به‌روزرسانی کلیپ‌شده:} استفاده از مکانیسم کلیپ شده \lr{PPO} برای محدود کردن به‌روزرسانی‌های سیاست، که به پایداری بیشتر در یادگیری چند‌عاملی کمک می‌کند.
    \item \textbf{بافر تجربه مشترک:} استفاده از یک بافر تجربه مشترک که تعاملات بین عامل‌ها را ثبت می‌کند.
\end{itemize}

\subsection{معماری \lr{MA-PPO} در بازی‌های مجموع­‌صفر}

در یک بازی چندعاملیِ مجموع­‌صفر، هر عامل دارای شبکه‌های زیر است:

\begin{itemize}
    \item \textbf{شبکه بازیگر:} $\pi_{\theta_i}(a_i|o_i)$ که توزیع احتمال اعمال را با توجه به مشاهدات محلی تعیین می‌کند.
    \item \textbf{شبکه منتقد:} $V_{\phi_i}(\boldsymbol{o})$ که ارزش حالتِ متمرکز را (با دسترسی به مشاهدات همهٔ عامل‌ها) تخمین می‌زند و برای محاسبهٔ تابع مزیت استفاده می‌شود.
\end{itemize}

%در بازی‌های مجموع­‌صفر، پاداش‌ها رابطه $r_1 + r_2 = 0$ دارند، بنابراین $r_2 = -r_1$ است.

\subsection{آموزش \lr{MA-PPO}}

فرایند آموزش \lr{MA-PPO} به شرح زیر است:

\subsubsection{جمع‌آوری تجربیات}

در هر تکرار، عامل‌ها با استفاده از سیاست‌های فعلی خود در محیط تعامل می‌کنند و مجموعه‌ای از مسیرها را جمع‌آوری می‌کنند:

\begin{equation}
    \mathcal{D}_k = \{(o_1^t, o_2^t, a_1^t, a_2^t, r_1^t, r_2^t, o_1^{t+1}, o_2^{t+1})\}
\end{equation}

\subsubsection{محاسبه مزیت}

برای هر عامل $i \in \{1, 2\}$، تابع مزیت با استفاده از تابع ارزش فعلی محاسبه می‌شود. روش‌های مختلفی برای محاسبه مزیت وجود دارد؛ یک روش متداول استفاده از تخمین‌زننده مزیت تعمیم‌یافته (\lr{GAE}) است:

\begin{equation}
    \hat{A}_i^t = \sum_{l=0}^{\infty} (\gamma\lambda)^l \delta_{i,t+l}
\end{equation}

که در آن $\delta_{i,t} = r_i^t + \gamma V_{\phi_i}(\boldsymbol{o}^{t+1}) - V_{\phi_i}(\boldsymbol{o}^t)$ است.

\subsubsection{به‌روزرسانی سیاست}

سیاست هر عامل با بیشینه کردن تابع هدف \lr{PPO-Clip} به‌روزرسانی می‌شود:

\begin{equation}
    \max_{\theta_i} \underset{(o_i,a_i) \sim \mathcal{D}_k}{\mathrm{E}}\left[ \min\left( \frac{\pi_{\theta_i}(a_i|o_i)}{\pi_{\theta_{i,k}}(a_i|o_i)} \hat{A}_i, \;\; \text{clip}\left(\frac{\pi_{\theta_i}(a_i|o_i)}{\pi_{\theta_{i,k}}(a_i|o_i)}, 1 - \epsilon, 1+\epsilon \right) \hat{A}_i \right) \right]
\end{equation}

یا با استفاده از همان فرمول‌بندی ساده‌تر:

\begin{equation}
    \max_{\theta_i} \underset{(o_i,a_i) \sim \mathcal{D}_k}{\mathrm{E}}\left[ \min\left( \frac{\pi_{\theta_i}(a_i|o_i)}{\pi_{\theta_{i,k}}(a_i|o_i)} \hat{A}_i, \;\; g(\epsilon, \hat{A}_i) \right) \right]
\end{equation}

که تابع $g$ به صورت زیر تعریف شده‌است:

\begin{align}
    g(\epsilon, A) = \left\{
    \begin{array}{ll}
        (1 + \epsilon) A & A \geq 0 \\
        (1 - \epsilon) A & A < 0
    \end{array}
    \right.
\end{align}

\subsubsection{به‌روزرسانی منتقد}

تابع ارزش هر عامل با کمینه کردن خطای میانگین مربعات به‌روزرسانی می‌شود:

\begin{equation}
    \min_{\phi_i} \underset{(o_i,\hat{R}_i) \sim \mathcal{D}_k}{\mathrm{E}}\left[ \left( V_{\phi_i}(o_i) - \hat{R}_i \right)^2 \right]
\end{equation}

که در آن $\hat{R}_i$ بازده تنزیل‌شده برای عامل $i$ است.

\subsection{اکتشاف در \lr{MA-PPO}}

اکتشاف در \lr{MA-PPO} به صورت ذاتی از طریق سیاست‌های تصادفی انجام می‌شود. برخلاف الگوریتم‌های مبتنی بر \lr{DDPG} که به افزودن نویز به اعمال نیاز دارند، \lr{MA-PPO} از توزیع احتمال سیاست برای اکتشاف استفاده می‌کند:

\begin{equation}
    a_i \sim \pi_{\theta_i}(\cdot|o_i)
\end{equation}

این رویکرد اکتشاف سیاست‌محور، در ترکیب با مکانیسم کلیپ \lr{PPO} که از به‌روزرسانی‌های بزرگ سیاست جلوگیری می‌کند، به ثبات بیشتر در یادگیری چند‌عاملی کمک می‌کند.

\subsection{شبه‌کد \lr{MA-PPO} برای بازی‌های چندعاملیِ مجموع­‌صفر}

در این بخش، شبه‌کد الگوریتم \lr{MA-PPO} پیاده‌سازی‌شده آورده شده‌است. در این پژوهش الگوریتم~\رجوع{alg:MA-PPO} در محیط پایتون با استفاده از کتابخانه \lr{PyTorch} \cite{paszke2017automatic} پیاده‌سازی شده‌است.

\begin{algorithm}[H]
    \caption{عامل بهینه‌سازی سیاست مجاور چند­عاملی}\label{alg:MA-PPO}
    \begin{algorithmic}[1]
        \ورودی پارامترهای اولیه سیاست عامل‌ها $(\theta_1, \theta_2)$، پارامترهای تابع ارزش $(\phi_1, \phi_2)$
        
        \For{$k = 0, 1, 2, ...$}
            \State \parbox[t]{\dimexpr\linewidth-\algorithmicindent}{
            مجموعه‌ای از مسیرها به نام $\mathcal{D}_k = \{(o_1^t, o_2^t, a_1^t, a_2^t, r_1^t, r_2^t, o_1^{t+1}, o_2^{t+1})\}$ با اجرای سیاست‌های $\pi_{\theta_1}$ و $\pi_{\theta_2}$ در محیط جمع‌آوری شود.
            \strut}
            
            \State \parbox[t]{\dimexpr\linewidth-\algorithmicindent}{
            برای هر عامل $i$، پاداش‌های باقی‌مانده $\hat{R}_i^t$ را محاسبه کنید.
            \strut}
            
            \State \parbox[t]{\dimexpr\linewidth-\algorithmicindent}{
            برای هر عامل $i$، برآوردهای مزیت $\hat{A}_i^t$ را با استفاده از تابع ارزش فعلی $V_{\phi_i}$ محاسبه کنید.
            \strut}
            
            \State \parbox[t]{\dimexpr\linewidth-\algorithmicindent}{
            برای هر عامل $i$، سیاست را با به حداکثر رساندن تابع هدف \lr{PPO-Clip} به‌روزرسانی کنید:
%            \vspace{-15pt}
            \begin{align*}
                \theta_{i,k+1} = \arg \max_{\theta_i} \frac{1}{|\mathcal{D}_k|} \sum_{(o_i,a_i) \in \mathcal{D}_k} \min\left( \frac{\pi_{\theta_i}(a_i|o_i)}{\pi_{\theta_{i,k}}(a_i|o_i)} \hat{A}_i, \;\; g(\epsilon, \hat{A}_i) \right)
            \end{align*}
            \strut}
%            \vspace{-30pt}
            
            \State \parbox[t]{\dimexpr\linewidth-\algorithmicindent}{
            برای هر عامل $i$، تابع ارزش را با رگرسیون بر روی میانگین مربعات خطا به‌روزرسانی کنید:
%            \vspace{-15pt}
            \begin{align*}
                \phi_{i,k+1} = \arg \min_{\phi_i} \frac{1}{|\mathcal{D}_k|} \sum_{(o_i) \in \mathcal{D}_k} \left( V_{\phi_i}(o_i) - \hat{R}_i \right)^2
            \end{align*}
            \strut}
        \EndFor
        \vspace{-15pt}
    \end{algorithmic}
\end{algorithm}

\subsection{مزایای \lr{MA-PPO} در بازی‌های مجموع­‌صفر}

\lr{MA-PPO} مزایای زیر را نسبت به سایر الگوریتم‌های چند­عاملی در بازی‌های چندعاملیِ مجموع­‌صفر ارائه می‌دهد:

\begin{itemize}
    \item \textbf{پایداری یادگیری:} مکانیسم کلیپ \lr{PPO} از به‌روزرسانی‌های بزرگ سیاست جلوگیری می‌کند که به پایداری بیشتر در محیط‌های غیرایستای چند­عاملی منجر می‌شود.
    \item \textbf{کارایی نمونه:} به عنوان یک روش درون‌سیاست، \lr{MA-PPO} معمولاً کارایی نمونهٔ کمتری نسبت به روش‌های برون‌سیاست مانند \lr{MA-TD3} و \lr{MA-SAC} دارد، اما پایداری بهتری در به‌روزرسانی‌ها ارائه می‌کند.
    \item \textbf{اکتشاف سیاست‌محور:} اکتشاف ذاتی از طریق سیاست‌های تصادفی به جای افزودن نویز به اعمال، به اکتشاف کارآمدتر فضای حالت-عمل کمک می‌کند.
    \item \textbf{مقیاس‌پذیری:} \lr{MA-PPO} به‌راحتی به سیستم‌های با تعداد بیشتری از عامل‌ها قابل گسترش است، اگرچه در این پژوهش بر بازی‌های دو­عاملی تمرکز شده‌است.
\end{itemize}

در مجموع، \lr{MA-PPO} ترکیبی از سادگی و کارایی \lr{PPO} با رویکردهای چند­عاملی را ارائه می‌دهد که آن را به گزینه‌ای قدرتمند برای یادگیری در بازی‌های چندعاملیِ مجموع­‌صفر تبدیل می‌کند.

%    \subsection{ایمنی و مقاومت در یادگیری تقویتی چندعاملی}

در استفاده از یادگیری تقویتی چندعاملی (\lr{Multi-Agent Reinforcement Learning - \lr{MARL}})، مسائل مربوط به ایمنی و مقاومت در برابر اختلالات یکی از چالش‌های اساسی مطرح می‌گردد. به منظور اطمینان از عملکرد قابل اعتماد و ایمن الگوریتم‌های یادگیری تقویتی چندعاملی، نیازمند توسعه روش‌هایی هستیم که بتوانند در مواجهه با رفتارهای غیرمنتظره یا مخرب سایر عوامل، پایداری و ایمنی سیستم را حفظ نمایند. در این بخش، به بررسی مفاهیم ایمنی و مقاومت در \lr{MARL} پرداخته شده و چگونگی افزایش مقاومت الگوریتم‌ها از طریق در نظر گرفتن عوامل به عنوان اختلالات مورد بحث قرار گرفته است.

ایمنی در \lr{MARL} به معنای تضمین این است که تعاملات میان عوامل منجر به نتایج نامطلوب یا خطرناک نشوند. برای دستیابی به ایمنی، روش‌هایی نظیر محدود کردن فضای عملیاتی، اعمال قیود بر سیاست‌های یادگیری و استفاده از الگوریتم‌های مقاوم در برابر خطا به کار گرفته شده‌اند. یکی از رویکردهای موثر در افزایش مقاومت سیستم، فرض کردن یکی از عوامل به عنوان اختلال (\lr{Disturbance}) در محیط است. با این فرض، الگوریتم‌ها قادر خواهند بود تا به گونه‌ای طراحی شوند که در حضور اختلالات احتمالی، عملکرد سیستم همچنان قابل اعتماد باقی بماند.

\subsubsection{فرض کردن اختلال به عنوان عامل}

در محیط‌های چندعاملی، برخی از عوامل ممکن است رفتارهای مخرب یا غیرمنتظره‌ای را از خود نشان دهند که می‌تواند به عملکرد کلی سیستم آسیب برساند. برای مقابله با این مسئله، فرض می‌شود که یک یا چند عامل به عنوان اختلالات در نظر گرفته شوند. این اختلالات می‌توانند به صورت عمدی یا غیرعمدی ایجاد شوند و هدف آن‌ها کاهش کارایی سیستم است. با فرض کردن این اختلالات، الگوریتم‌های \lr{MARL} قادر خواهند بود تا سیاست‌هایی را یاد بگیرند که در مواجهه با این اختلالات نیز عملکرد بهینه و ایمنی را حفظ کنند.

\subsubsection{تعریف مقاومت و ایمنی در \lr{MARL}}

مقاومت در \lr{MARL} به معنای توانایی الگوریتم در حفظ عملکرد مطلوب در حضور اختلالات و تغییرات محیطی است. این مقاومت می‌تواند از طریق طراحی سیاست‌های بهینه که به گونه‌ای تنظیم شده‌اند که تأثیر اختلالات را به حداقل برسانند، به دست آید. به علاوه، ایمنی می‌تواند از طریق تضمین عدم وقوع رفتارهای خطرناک و حفظ تعادل سیستم در مواجهه با رفتارهای مخرب حاصل شود.

\paragraph{تعریف ریاضی مقاومت}

فرض کنید یک محیط چندعاملی با مجموعه‌ای از عوامل \( \mathcal{A} = \{A_1, A_2, \dots, A_n\} \) وجود دارد که در آن یک عامل \( A_d \) به عنوان اختلال تعریف شده است. هدف این است که الگوریتم \lr{MARL} به گونه‌ای طراحی شود که سیاست‌های یادگرفته شده \( \pi = \{\pi_1, \pi_2, \dots, \pi_n\} \) بتوانند عملکرد بهینه را حتی در حضور \( A_d \) حفظ کنند. به طور ریاضی، مقاومت به صورت زیر تعریف می‌شود:
\[
\forall A_i \in \mathcal{A}, \quad \text{اگر } A_d \text{ رفتار مخرب نشان دهد، } \pi_i \text{ باید همچنان به حداکثر رساندن پاداش خود ادامه دهد.}
\]

\paragraph{تعریف ریاضی ایمنی}

ایمنی در \lr{MARL} به معنای اطمینان از این است که سیستم در هیچ حالت خطرناکی وارد نمی‌شود. به طور ریاضی، ایمنی می‌تواند به صورت مجموعه‌ای از قیود تعریف شود که سیاست‌های یادگرفته شده باید آن‌ها را رعایت کنند:
\[
\forall A_i \in \mathcal{A}, \quad \text{ایمنی: } u_i(s_i, s_{-i}) \geq \theta_i \quad \text{برای همه } s_i \in S_i, \ s_{-i} \in S_{-i}
\]
که در آن \( \theta_i \) آستانه‌ای است که برای هر عامل \( A_i \) تعیین شده و نشان‌دهنده حداقل پاداش قابل قبول است.

\subsubsection{روش‌های افزایش مقاومت و ایمنی}

برای افزایش مقاومت و ایمنی در \lr{MARL}، روش‌های متعددی مورد استفاده قرار گرفته‌اند که در زیر به برخی از آن‌ها پرداخته می‌شود:

\begin{itemize}
	\item \textbf{الگوریتم‌های مقاوم در برابر اختلالات:} این الگوریتم‌ها به گونه‌ای طراحی شده‌اند که بتوانند به سرعت با تغییرات محیطی و حضور اختلالات سازگار شوند. به عنوان مثال، الگوریتم‌های مبتنی بر یادگیری تطبیقی که قادر به تغییر سیاست‌های خود در پاسخ به تغییرات محیط هستند.
	
	\item \textbf{فریم‌ورک‌های ایمنی:} چارچوب‌هایی برای تضمین ایمنی در تعاملات چندعاملی طراحی شده‌اند که شامل محدود کردن فضای عملیاتی و اعمال قیود بر سیاست‌های یادگیری است. این فریم‌ورک‌ها معمولاً شامل روش‌هایی برای نظارت و تنظیم رفتار عامل‌ها به منظور جلوگیری از وقوع رفتارهای خطرناک هستند.
	
	\item \textbf{آموزش در حضور اختلالات:} با آموزش الگوریتم‌ها در محیط‌هایی که شامل اختلالات هستند، می‌توان مقاومت الگوریتم‌ها را افزایش داد. این روش به الگوریتم اجازه می‌دهد تا در مواجهه با اختلالات غیرمنتظره، سیاست‌های مقاومتی یاد بگیرد.
	
	\item \textbf{استفاده از اصول نظریه بازی‌ها:} با بهره‌گیری از تعادل‌های نظریه بازی‌ها مانند تعادل نش (\lr{Nash Equilibrium}), می‌توان سیاست‌هایی طراحی کرد که در مواجهه با استراتژی‌های متغیر سایر عوامل، پایداری و ایمنی سیستم حفظ شود.
\end{itemize}

\subsubsection{نمونه‌های کاربردی}

برای نشان دادن کاربردهای عملی \lr{MARL} در افزایش ایمنی و مقاومت، به چند مثال اشاره می‌شود:

\paragraph{سامانه‌های خودران:} در خودروهای خودران چندعاملی، ایمنی یکی از اولویت‌های اصلی است. با استفاده از \lr{MARL} و فرض کردن سایر خودروها به عنوان عوامل یا اختلالات، می‌توان الگوریتم‌هایی توسعه داد که در مواجهه با رفتارهای غیرمنتظره سایر خودروها، ایمن باقی بمانند.

\paragraph{مدیریت انرژی در شبکه‌های هوشمند:} در شبکه‌های انرژی هوشمند، \lr{MARL} می‌تواند برای مدیریت بهینه انرژی در حضور اختلالات مانند خرابی‌ها یا حملات سایبری استفاده شود. الگوریتم‌های مقاوم می‌توانند با تغییرات ناگهانی در تقاضا یا عرضه انرژی سازگار شده و پایداری شبکه را حفظ کنند.

\paragraph{ربات‌های همکاری‌کننده:} در سیستم‌های رباتیک همکاری‌کننده، اطمینان از ایمنی تعاملات میان ربات‌ها حیاتی است. \lr{MARL} می‌تواند برای طراحی سیاست‌های ایمن که در مواجهه با رفتارهای مخرب یا اختلالات داخلی، سیستم را پایدار نگه دارند، به کار رود.

\paragraph{محیط‌های صنعتی:} در محیط‌های صنعتی که شامل چندین عامل نظیر ربات‌ها و ماشین‌آلات است، ایمنی و مقاومت سیستم‌ها از اهمیت بالایی برخوردار است. با استفاده از \lr{MARL}، می‌توان الگوریتم‌هایی توسعه داد که در مواجهه با اختلالات مانند خرابی تجهیزات یا خطاهای انسانی، عملکرد سیستم را حفظ کنند.

\subsubsection{چالش‌ها و فرصت‌ها}

با وجود مزایای متعدد \lr{MARL} در افزایش ایمنی و مقاومت سیستم‌ها، چالش‌هایی نیز در این زمینه وجود دارد:

\begin{itemize}
	\item \textbf{پیچیدگی مدل‌سازی اختلالات:} مدل‌سازی دقیق اختلالات و رفتارهای مخرب می‌تواند پیچیده و زمان‌بر باشد، به ویژه در محیط‌های پیچیده و پویا.
	
	\item \textbf{هماهنگی میان عوامل:} هماهنگی موثر میان عوامل در حضور اختلالات نیازمند الگوریتم‌های پیچیده و کارآمد است که بتوانند به سرعت و بهینه با تغییرات محیطی سازگار شوند.
	
	\item \textbf{تعادل میان کارایی و ایمنی:} حفظ تعادل میان بهینه‌سازی کارایی سیستم و تضمین ایمنی در مواجهه با اختلالات یک چالش بزرگ است که نیازمند طراحی دقیق الگوریتم‌ها است.
	
	\item \textbf{نیاز به داده‌های بزرگ:} برای آموزش الگوریتم‌های مقاوم و ایمن، نیاز به مجموعه‌های داده بزرگ و متنوعی است که شامل انواع اختلالات و رفتارهای مخرب باشند.
\end{itemize}

با این حال، فرصت‌های بزرگی نیز در این زمینه وجود دارد. توسعه الگوریتم‌های پیشرفته \lr{MARL} که بتوانند به طور موثر با اختلالات مواجه شوند، می‌تواند منجر به سیستم‌های هوشمند و خودکار با کارایی و ایمنی بالاتر شود. همچنین، پژوهش‌های ادامه‌دار در زمینه نظریه بازی‌ها و روش‌های مقاوم‌سازی الگوریتم‌های یادگیری تقویتی، پتانسیل بالایی برای بهبود \lr{MARL} و کاربردهای آن فراهم می‌آورد.

\subsubsection{نتیجه‌گیری}

در نهایت، ایمنی و مقاومت در یادگیری تقویتی چندعاملی به عنوان عوامل کلیدی در توسعه سیستم‌های هوشمند و خودکار مطرح می‌گردند. با فرض کردن اختلالات و توسعه الگوریتم‌های مقاوم، می‌توان عملکرد سیستم‌های \lr{MARL} را در مواجهه با چالش‌های مختلف بهبود بخشید. با توجه به اهمیت و کاربردهای گسترده \lr{MARL} در حوزه‌های علمی و صنعتی، تحقیقات در زمینه افزایش ایمنی و مقاومت این الگوریتم‌ها همچنان ادامه خواهد یافت تا به سیستم‌هایی با قابلیت‌های بیشتر و عملکردی پایدارتر دست یابند.

%    \subsection{الگوریتم‌های یادگیری تقویتی چندعاملی}

%       \section{تنظیمات آزمایشی}
%    \section{نتایج عملکرد الگوریتم‌ها}
%    \section{تحلیل پایداری و همگرایی}
%    \section{مقایسه روش‌های تک‌عاملی و چندعاملی}
%    \section{ارزیابی مقاومت الگوریتم‌ها در برابر اختلالات}
%    \section{تحلیل آماری نتایج}
%    \section{بحث و تفسیر نتایج}
%    \section{مقایسه با معیارهای مرجع}
%    \section{جمع‌بندی و پیشنهادات آتی} 