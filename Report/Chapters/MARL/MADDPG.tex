\section{ گرادیان سیاست عمیق قطعی 
در بازی‌های دو­عاملیِ مجموع‌­صفر
}
% insert_edit_into_file("/Users/Ali/Documents/BAI/Master/master-thesis/Report/Chapters/MARL/MA-DDPG.tex", r"\section{ گرادیان سیاست عمیق قطعی 
% در بازی‌های دو­عاملیِ مجموع‌­صفر
% }", """
% \section{ گرادیان سیاست عمیق قطعی 
% در بازی‌های دو­عاملیِ مجموع‌­صفر
% }\label{sec:MA-DDPG}

گرادیان سیاست عمیق قطعی چند­عاملی\LTRfootnote{Multi-Agent Deep Deterministic Policy Gradient (MA-DDPG)}
توسعه‌ای از الگوریتم \lr{DDPG} برای محیط‌های چند­عاملی است. در این بخش، به بررسی این الگوریتم در چارچوب بازی‌های دو­عاملیِ مجموع­‌صفر می‌پردازیم که در آن مجموع پاداش‌های دو عامل همواره صفر است (آنچه یک عامل به دست می‌آورد، عامل دیگر از دست می‌دهد).

\subsection{چالش‌های یادگیری تقویتی در محیط‌های چند­عاملی}

در محیط‌های چند­عاملی، سیاست هر عامل مدام در حال تغییر است، که باعث می‌شود محیط از دید هر عامل غیرایستا\LTRfootnote{Non-stationary} شود. این مسئله چالش بزرگی برای الگوریتم‌های یادگیری تقویتی تک‌عاملی مانند \lr{DDPG} ایجاد می‌کند، زیرا فرض ایستایی محیط را نقض می‌کند.

\lr{MA-DDPG} با استفاده از رویکرد آموزش متمرکز، اجرای غیرمتمرکز\LTRfootnote{Centralized Training, Decentralized Execution} این مشکل را حل می‌کند. در این رویکرد، هر عامل در زمان آموزش به اطلاعات کامل محیط دسترسی دارد، اما در زمان اجرا تنها از مشاهدات محلی خود استفاده می‌کند.

\subsection{معماری \lr{MA-DDPG} در بازی‌های مجموع­‌صفر}

در یک بازی دو­عاملیِ مجموع­‌صفر، دو عامل با نمادهای 1 و 2 نشان داده می‌شوند. هر عامل دارای شبکه‌های منحصر به فرد خود است:

\begin{itemize}
    \item \textbf{شبکه‌های بازیگر:} $\mu_{\theta_1}(o_1)$ و $\mu_{\theta_2}(o_2)$ که مشاهدات محلی $o_1$ و $o_2$ را به اعمال $a_1$ و $a_2$ نگاشت می‌کنند.
    \item \textbf{شبکه‌های منتقد:} $Q_{\phi_1}(o_1, a_1, a_2)$ و $Q_{\phi_2}(o_2, a_2, a_1)$ که ارزش حالت-عمل را با توجه به مشاهدات و اعمال تمام عامل‌ها تخمین می‌زنند.
    \item \textbf{شبکه‌های هدف:} مشابه \lr{DDPG}، برای پایدار کردن آموزش از شبکه‌های هدف استفاده می‌شود.
\end{itemize}

در بازی‌های مجموع­‌صفر، پاداش‌ها رابطه $r_1 + r_2 = 0$ دارند که در آن $r_1$ و $r_2$ پاداش‌های دریافتی عامل‌ها هستند. در نتیجه، $r_2 = -r_1$ است که نمایانگر تضاد کامل منافع بین عامل‌هاست.

\subsection{آموزش \lr{MA-DDPG} در بازی‌های مجموع­‌صفر}

فرایند آموزش \lr{MA-DDPG} برای بازی‌های مجموع­‌صفر به شرح زیر است:

\subsubsection{یادگیری تابع \lr{Q}}

برای هر عامل $i \in \{1, 2\}$، تابع \lr{Q} با کمینه کردن خطای میانگین مربعات بلمن به‌روزرسانی می‌شود:

\begin{equation}
    L(\phi_i, \mathcal{D}) = \underset{(\boldsymbol{o}, \boldsymbol{a}, r_i, \boldsymbol{o}', d) \sim \mathcal{D}}{\mathrm{E}}\left[ 
    \Bigg( Q_{\phi_i}(o_i, a_1, a_2) - y_i \Bigg)^2
    \right]
\end{equation}

که در آن $\boldsymbol{o} = (o_1, o_2)$ بردار مشاهدات، $\boldsymbol{a} = (a_1, a_2)$ بردار اعمال، و $y_i$ هدف برای عامل $i$ است:

\begin{equation}
    y_i = r_i + \gamma (1 - d) Q_{\phi_{i,\text{targ}}}(o_i', \mu_{\theta_{1,\text{targ}}}(o_1'), \mu_{\theta_{2,\text{targ}}}(o_2'))
\end{equation}

توجه کنید که منتقد هر عامل به اعمال همه عامل‌ها دسترسی دارد، اما در بازی‌های مجموع­‌صفر، عامل شماره 2 جهت مخالف هدف عامل 1 را دنبال می‌کند.

\subsubsection{یادگیری سیاست}

سیاست هر عامل با بیشینه کردن تابع \lr{Q} مربوط به آن عامل به‌روزرسانی می‌شود:

\begin{equation}
    \max_{\theta_i} \underset{\boldsymbol{o} \sim \mathcal{D}}{\mathrm{E}}\left[ Q_{\phi_i}(o_i, \mu_{\theta_i}(o_i), \mu_{\theta_{-i}}(o_{-i})) \right]
\end{equation}

که در آن $-i$ نشان‌دهنده عامل مقابل است. با توجه به ماهیت بازی مجموع­‌صفر، هر عامل تلاش می‌کند تا مطلوبیت خود را افزایش دهد، در حالی که مطلوبیت عامل دیگر به طور همزمان کاهش می‌یابد.

\subsubsection{شبکه‌های هدف و بافر تجربه}

مشابه \lr{DDPG}، برای پایدار کردن آموزش، شبکه‌های هدف با میانگین‌گیری پولیاک به‌روزرسانی می‌شوند:

\begin{align*}
    \phi_{i,\text{targ}} &\leftarrow \rho \phi_{i,\text{targ}} + (1 - \rho) \phi_i \\
    \theta_{i,\text{targ}} &\leftarrow \rho \theta_{i,\text{targ}} + (1 - \rho) \theta_i
\end{align*}

همچنین، از یک بافر تکرار بازی مشترک برای ذخیره تجربیات استفاده می‌شود که شامل وضعیت‌ها، اعمال و پاداش‌های همه عامل‌هاست.

\subsection{اکتشاف در \lr{MA-DDPG}}

اکتشاف در \lr{MA-DDPG} مشابه \lr{DDPG} است، اما برای هر عامل به طور جداگانه اعمال می‌شود. در طی آموزش، به اعمال هر عامل نویز اضافه می‌شود:

\begin{equation}
    a_i = \text{clip}(\mu_{\theta_i}(o_i) + \epsilon_i, a_{\text{Low}}, a_{\text{High}})
\end{equation}

که در آن $\epsilon_i$ نویز اضافه شده به عامل $i$ است.

\subsection{شبه‌کد \lr{MA-DDPG} برای بازی‌های دو­عاملیِ مجموع­‌صفر}
       در این بخش، شبه‌کد الگوریتم
\lr{MA-DDPG}
پیاده‌سازی‌شده آورده شده‌است. در این پژوهش الگوریتم~\رجوع{alg:MA-DDPG} در محیط پایتون با استفاده از کتابخانه
\lr{PyTorch} \cite{paszke2017automatic}
پیاده‌سازی شده‌است.

\begin{algorithm}[H]
    \caption{گرادیان سیاست عمیق قطعی چند­عاملی برای بازی‌های مجموع­‌صفر}\label{alg:MA-DDPG}
    \begin{algorithmic}[1]
        \ورودی پارامترهای اولیه سیاست عامل‌ها $(\theta_1, \theta_2)$، پارامترهای تابع \lr{Q} $(\phi_1, \phi_2)$، بافر تکرار بازی خالی $(\mathcal{D})$
        \State پارامترهای هدف را برابر با پارامترهای اصلی قرار دهید: $\theta_{i,\text{targ}} \leftarrow \theta_i$, $\phi_{i,\text{targ}} \leftarrow \phi_i$ برای $i \in \{1, 2\}$
        
        \While{همگرایی رخ دهد}
            \State \parbox[t]{\dimexpr\linewidth-\algorithmicindent}{
            مشاهدات $(o_1, o_2)$ را دریافت کنید
            \strut}
            \State \parbox[t]{\dimexpr\linewidth-\algorithmicindent}{
            برای هر عامل $i$، عمل $a_i = \text{clip}(\mu_{\theta_i}(o_i) + \epsilon_i, a_{\text{Low}}, a_{\text{High}})$ را انتخاب کنید، به‌طوری که $\epsilon_i \sim \mathcal{N}$ است
            \strut}
            \State اعمال $(a_1, a_2)$ را در محیط اجرا کنید
            \State \parbox[t]{\dimexpr\linewidth-\algorithmicindent}{
            مشاهدات بعدی $(o_1', o_2')$، پاداش‌ها $(r_1, r_2=-r_1)$ و سیگنال پایان $d$ را دریافت کنید
            \strut}
            \State تجربه $(o_1, o_2, a_1, a_2, r_1, r_2, o_1', o_2', d)$ را در بافر $\mathcal{D}$ ذخیره کنید
            \State اگر $d=1$ است، وضعیت محیط را بازنشانی کنید
            
            \If{زمان به‌روزرسانی فرا رسیده است}
                \For{هر تعداد به‌روزرسانی}
                    \State یک دسته تصادفی از تجربیات، $B = \{(\boldsymbol{o}, \boldsymbol{a}, r_1, r_2, \boldsymbol{o}', d)\}$، از $\mathcal{D}$ نمونه‌گیری کنید
%                    \State \parbox[t]{\dimexpr\linewidth-\algorithmicindent}{
                    اهداف
                    \Statex \hspace{\algorithmicindent} \hspace{\algorithmicindent} ~
                     را محاسبه کنید:
                     \vspace{-15pt}
                    \begin{align*}
                        y_1 &= r_1 + \gamma (1-d) Q_{\phi_{1,\text{targ}}}(o_1', \mu_{\theta_{1,\text{targ}}}(o_1'), \mu_{\theta_{2,\text{targ}}}(o_2')) \\
                        y_2 &= r_2 + \gamma (1-d) Q_{\phi_{2,\text{targ}}}(o_2', \mu_{\theta_{2,\text{targ}}}(o_2'), \mu_{\theta_{1,\text{targ}}}(o_1'))
                    \end{align*}
                    \vspace{-35pt}
%                    \strut}
                    \State% \parbox[t]{\dimexpr\linewidth-\algorithmicindent}{
                    توابع \lr{Q} را با نزول گرادیان به‌روزرسانی کنید:
                    \vspace{-15pt}
                    \begin{align*}
                        \nabla_{\phi_1} \frac{1}{|B|}\sum_{(\boldsymbol{o}, \boldsymbol{a}, r_1, r_2, \boldsymbol{o}', d) \in B} \left( Q_{\phi_1}(o_1, a_1, a_2) - y_1 \right)^2 \\
                        \nabla_{\phi_2} \frac{1}{|B|}\sum_{(\boldsymbol{o}, \boldsymbol{a}, r_1, r_2, \boldsymbol{o}', d) \in B} \left( Q_{\phi_2}(o_2, a_2, a_1) - y_2 \right)^2
                    \end{align*}
%                    \strut}
                    \vspace{-30pt}
                    \State %\parbox[t]{\dimexpr\linewidth-\algorithmicindent}{
                    سیاست‌ها را با صعود گرادیان به‌روزرسانی کنید:
                    \vspace{-15pt}
                    \begin{align*}
                        \nabla_{\theta_1} \frac{1}{|B|}\sum_{\boldsymbol{o} \in B}Q_{\phi_1}(o_1, \mu_{\theta_1}(o_1), a_2) \\
                        \nabla_{\theta_2} \frac{1}{|B|}\sum_{\boldsymbol{o} \in B}Q_{\phi_2}(o_2, \mu_{\theta_2}(o_2), a_1)
                    \end{align*}
%                    \strut}
                    \vspace{-35pt}
                    \State %\parbox[t]{\dimexpr\linewidth-\algorithmicindent}{
                    شبکه‌های هدف را به‌روزرسانی کنید:
                    \vspace{-15pt}
                    \begin{align*}
                        \phi_{1,\text{targ}} &\leftarrow \rho \phi_{1,\text{targ}} + (1-\rho) \phi_1 \\
                        \phi_{2,\text{targ}} &\leftarrow \rho \phi_{2,\text{targ}} + (1-\rho) \phi_2 \\
                        \theta_{1,\text{targ}} &\leftarrow \rho \theta_{1,\text{targ}} + (1-\rho) \theta_1 \\
                        \theta_{2,\text{targ}} &\leftarrow \rho \theta_{2,\text{targ}} + (1-\rho) \theta_2
                    \end{align*}
%                    \strut}
                \EndFor
            \EndIf
        \EndWhile
        \vspace{-15pt}
    \end{algorithmic}
\end{algorithm}

\subsection{مزایای \lr{MA-DDPG} در بازی‌های مجموع­‌صفر}

\lr{MA-DDPG} چندین مزیت برای یادگیری در بازی‌های دو­عاملیِ مجموع­‌صفر ارائه می‌دهد:

\begin{itemize}
    \item \textbf{مقابله با غیرایستایی:} با استفاده از منتقدهایی که به اطلاعات کامل دسترسی دارند، مشکل غیرایستایی محیط از دید هر عامل حل می‌شود.
    \item \textbf{همگرایی بهتر:} در بازی‌های مجموع­‌صفر، \lr{MA-DDPG} معمولاً همگرایی بهتری نسبت به آموزش مستقل عامل‌ها با \lr{DDPG} نشان می‌دهد.
    \item \textbf{یادگیری استراتژی‌های متقابل:} عامل‌ها می‌توانند استراتژی‌های متقابل پیچیده را یاد بگیرند که در آموزش مستقل امکان‌پذیر نیست.
\end{itemize}

در بازی‌های دو­عاملیِ مجموع­‌صفر، این رویکرد به رقابت کامل بین عامل‌ها منجر می‌شود، که هر یک تلاش می‌کند بهترین استراتژی را در برابر استراتژی رقیب پیدا کند.