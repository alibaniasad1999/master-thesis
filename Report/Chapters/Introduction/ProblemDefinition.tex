\section{تعریف مسئله}\label{sec:problem_statement}
\noindent  در سال‌های اخیر، پیشرفت‌های فناوری در زمینه‌های مختلف، از جمله کنترل پرواز، پردازش سیگنال و هوش مصنوعی، به افزایش کاربردهای ماهواره با پیشران کم در منظومه زمین‐ماه کمک کرده است. ماهواره با پیشران کم می‌تواند برای تعقیب ماهواره‌ها، انتقال مداری و استقرار ماهواره‌ها استفاده شود.
روش‌های هدایت بهینه قدیمی جهت کنترل ماهواره‌ها اغلب نیازمند فرضیات ساده کننده، منابع محاسباتی فراوان و شرایط اولیه مناسب هستند. الگوریتم‌های مبتنی بر یادگیری تقویتی این توانایی را دارند که بدون مشکلات اشاره‌شده هدایت ماهواره را انجام دهند. به همین دلیل، این الگوریتم‌ها می‌توانند امکان محاسبات درونی 
(\lr{On-board Computing})  
را فراهم می‌کنند.



هدف، طراحی سیاست کنترلی برای یک فضاپیما به جرم~$m$ است که در میدان جاذبهٔ سیستم زمین–ماه به‌صورت دوبُعدی مدل می‌شود. 

\begin{itemize}
	\item \textbf{پویایی‌ها:} معادلات حرکت در چارچوب مرجع چرخان به‌صورت مجموعهٔ غیرخطی \mbox{$\dot{\mathbf{x}} = f(\mathbf{x}) + g(\mathbf{x})\,\mathbf{u}$} نوشته می‌شود که $\mathbf{x}\!=\![x,y,\dot x,\dot y]^\top$ بردار حالت و $\mathbf{u}$ بردار تراست با کران $\|\mathbf{u}\|\le u_{\max}$ است.
	\item \textbf{عدم‌قطعیت‌ها:} ضرایب تراست دچار نوفهٔ گاوسی با واریانس \mbox{$\sigma_u^2$} و خوانش‌های حسگر موقعیت و سرعت دارای خطای \mbox{$\sigma_s^2$} هستند؛ همچنین پارامتر جرمی نسبی $\mu$ با خطای مدل \mbox{$\delta\mu$} لحاظ می‌شود.
	\item \textbf{قیود مأموریت:} فضای مجاز حرکت حول نقطهٔ لاگرانژی~$L_1$ شعاعی معادل 500~km دارد و میزان مصرف پیشران \mbox{$J = \int_0^{t_f}\|\mathbf{u}(t)\|\,\mathrm dt$} باید حداقل شود.
	\item \textbf{صورت بازی دیفرانسیلی:} فضاپیما و «طبیعت» (اغتشاشات) به‌ترتیب به‌عنوان عامل کنترلی و حریفِ مزاحم مدل می‌شوند. مسئله به‌عنوان یک بازی \textsc{Minimax} در افق زمان محدود \mbox{$t_f$} فرمول‌بندی می‌گردد.
\end{itemize}

صورت کامل مسأله را می‌توان با یافتن سیاست \mbox{$\pi^\star : \mathcal{X}\to \mathcal{U}$} تعریف کرد که معیار هزینهٔ تجمعی رشدمی‌یافته 
\[ J(\pi,\omega) = \mathbb{E}\_{\pi,\omega}\Big[\,\alpha_1\,J\_{\text{fuel}} + \alpha_2\,J\_{\text{deviation}}\Big] \]
را در بدترین سناریوی اغتشاش~$\omega$ کمینه سازد. ضرایب $\alpha_{1,2}$ اهمیت نسبی سوخت و انحراف مسیر را نشان می‌دهد.
