\section{تعریف مسئله}\label{sec:problem_statement}
\noindent
در سال‌های اخیر، پیشرفت‌های فناوری در کنترل پرواز، پردازش و هوش مصنوعی به گسترشِ کاربردِ فضاپیماهای پیشران‌کم در منظومه‌ی زمین–ماه انجامیده است؛ از تعقیب و انتقال مداری تا استقرار و نگه‌داری. روش‌های هدایتِ بهینه‌ی کلاسیک، هرچند قدرتمند، عموماً به ساده‌سازی‌های بسیار، منابعِ محاسباتیِ زیاد و شرایط اولیه‌ی مناسب متکی بوده‌اند؛ در مقابل، بخشی از این محدودیت‌ها با الگوریتم‌های مبتنی بر یادگیری تقویتی و تکیه بر تعامل و امکان محاسباتِ {درون‌برد}\LTRfootnote{On-board Computing} برطرف می‌شود.

هدف، طراحیِ سیاستِ کنترلی برای فضاپیمایی با جرم~$m$ در میدانِ گرانشِ سامانه‌ی زمین–ماه (مدل دوبعدی در چارچوبِ چرخان) است. ویژگی‌ها به‌اختصار:

\begin{itemize}
  \item \textbf{پویایی‌ها:} معادلاتِ حرکت در چارچوب مرجعِ چرخان به‌صورت
  {$\dot{\boldsymbol{x}} = f(\boldsymbol{x}) + g(\boldsymbol{x})\,\boldsymbol{a}$}
  با $\boldsymbol{x}=[x,\,y,\,\dot x,\,\dot y]^\top$ و کنترلِ پیوسته‌ی $\boldsymbol{a}\in\mathcal{A}$ تعریف می‌شود، به‌طوری‌که کران
   $|\boldsymbol{a}|\le a_{\max}$
   برقرار است.
  \item \textbf{عدم‌قطعیت‌ها:} شرایطِ اولیه‌ی تصادفی، اغتشاش‌های عملگر، عدم‌تطابقِ مدل (در پارامترهای جرم)، مشاهده‌ی ناقص، نویز حسگر و تأخیر زمانی، که بر پایداری و کارایی اثرگذارند.
  \item \textbf{صورت‌بندیِ بازیِ دیفرانسیلی (جمع‌صفر):} فضاپیما و طبیعت (اغتشاشات) به‌ترتیب به‌عنوان عاملِ کنترل و {حریفِ مزاحم} مدل می‌شوند؛ با افق زمانیِ محدود {$t_f$}، هدف، دستیابی به سیاستی مقاوم در برابر بدترین سناریو است.
\end{itemize}

صورتِ فشرده‌ی بهینه‌سازی به‌صورت کمینه–بیشینه است:
\begin{equation}
  \min_{\pi}\ \max_{\omega}\ \mathbb{E}_{p,\pi,\omega}\!\left[\ \sum_{t=0}^{T} r\big(\boldsymbol{s}_t,\boldsymbol{a}_t,\boldsymbol{\delta}_t\big)\ \right],
\end{equation}
که در آن، پاداشِ $r$ به‌عنوان تابعی از {مصرفِ سوخت}، {انحراف از مسیر یا مدار نامی} و {قیودِ مسئله} تعریف می‌شود. خروجیِ مورد انتظار، سیاستی سبک و غیرمتمرکز برای اجرای {درون‌برد} مدنظر است.
