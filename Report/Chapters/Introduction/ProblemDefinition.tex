\section{تعریف مسئله}\label{sec:problem_statement}
\noindent  در سال‌های اخیر، پیشرفت‌های فناوری در زمینه‌های مختلف، از جمله کنترل پرواز، پردازش سیگنال و هوش مصنوعی، به افزایش کاربردهای ماهواره با پیشران کم در منظومه زمین‐ماه کمک کرده است. ماهواره با پیشران کم می‌تواند برای تعقیب ماهواره‌ها، انتقال مداری و استقرار ماهواره‌ها استفاده شود.
روش‌های هدایت بهینه قدیمی جهت کنترل ماهواره‌ها اغلب نیازمند فرضیات ساده کننده، منابع محاسباتی فراوان و شرایط اولیه مناسب هستند. الگوریتم‌های مبتنی بر یادگیری تقویتی این توانایی را دارند که بدون مشکلات اشاره‌شده هدایت ماهواره را انجام دهند. به‌دلیل ساختار شبکه‌ای، این الگوریتم‌ها می‌توانند امکان محاسبات درونی\LTRfootnote{On-board Computing}  
را فراهم کنند.



هدف از این پژوهش، طراحی سیاست کنترلی برای یک فضاپیما به جرم~$m$ است که در میدان جاذبه‌ی سیستم زمین–ماه به‌صورت دوبُعدی مدل شده‌است. ویژگی‌های این سامانه در ادامه آورده شده‌است.

\begin{itemize}
	\item \textbf{پویایی‌ها:} معادلات حرکت در چارچوب مرجع چرخان به‌صورت مجموعهٔ غیرخطی {$\dot{\boldsymbol{x}} = f(\boldsymbol{x}) + g(\boldsymbol{x})\,\boldsymbol{u}$} نوشته می‌شود که $\boldsymbol{x}\!=\![x,y,\dot x,\dot y]^\top$ بردار حالت و $\boldsymbol{u}$ بردار تراست با کران $\boldsymbol{u}\le u_{\max}$ است.
	\item \textbf{عدم‌قطعیت‌ها:} 
عوامل عدم قطعیت شامل شرایط اولیه تصادفی، اغتشاش در عملگرها، عدم تطابق مدل، مشاهده ناقص، نویز حسگر و تأخیر زمانی هستند که همگی بر عملکرد و پایداری سیستم تأثیر می‌گذارند.
	
%	\item \textbf{قیود مأموریت:} فضای مجاز حرکت حول نقطهٔ لاگرانژی~$L_1$ شعاعی معادل 500~km دارد و میزان مصرف پیشران {$J = \int_0^{t_f}\|\mathbf{u}(t)\|\,\mathrm dt$} باید حداقل شود.
	\item \textbf{صورت بازی دیفرانسیلی:} فضاپیما و طبیعت (اغتشاشات) به‌ترتیب به‌عنوان عامل کنترلی و حریفِ مزاحم مدل شده‌است. مسئله به‌عنوان یک بازی مجموع‌صفر \LTRfootnote{Zero-Sum} 
	در افق زمان محدود {$t_f$} فرمول‌بندی شده‌اند.
\end{itemize}

صورت کامل مسأله را می‌توان با یافتن سیاست {$\pi^\star : \mathcal{X}\to \mathcal{U}$} تعریف کرد که معیار بهینه‌سازی هزینه‌ی تجمعی است.

%رشدمی‌یافته
%\[ J(\pi,\omega) = \mathbb{E}\_{\pi,\omega}\Big[\,\alpha_1\,J\_{\text{fuel}} + \alpha_2\,J\_{\text{deviation}}\Big] \]
%را در بدترین سناریوی اغتشاش~$\omega$ کمینه سازد. ضرایب $\alpha_{1,2}$ اهمیت نسبی سوخت و انحراف مسیر را نشان می‌دهد.
