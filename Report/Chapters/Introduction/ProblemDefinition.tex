\section{تعریف مسئله}
\noindent  در سال‌های اخیر، پیشرفت‌های فناوری در زمینه‌های مختلف، از جمله کنترل پرواز، پردازش سیگنال و هوش مصنوعی، به افزایش کاربردهای ماهواره با پیشران کم در منظومه زمین‐ماه کمک کرده است. ماهواره با پیشران کم می‌تواند برای تعقیب ماهواره‌ها، انتقال مداری و استقرار ماهواره‌ها استفاده شود.
روش‌های هدایت بهینه قدیمی جهت کنترل ماهواره‌ها اغلب نیازمند فرضیات ساده کننده، منابع محاسباتی فراوان و شرایط اولیه مناسب هستند. الگوریتم‌های مبتنی بر یادگیری تقویتی این توانایی را دارند که بدون مشکلات اشاره‌شده هدایت ماهواره را انجام دهند. به همین دلیل، این الگوریتم‌ها می‌توانند امکان محاسبات درونی 
(\lr{On-board Computing})  
را فراهم می‌کنند.

