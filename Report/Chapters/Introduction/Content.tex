\section{ساختار گزارش}\label{sec:chapter_content}
در \textbf{فصل دوم} مروری انتقادی بر کارهای مرتبط در هدایت پیشران‌کم و یادگیری تقویتی (تک‌عاملی و چندعاملی) ارائه می‌شود. \textbf{فصل سوم} به مبانی یادگیری تقویتی اختصاص داده شده و الگوریتم‌های \lr{DDPG}، \lr{TD3}، \lr{SAC} و \lr{PPO} مرور می‌شوند. در \textbf{فصل چهارم} چارچوب یادگیری تقویتی چندعاملی و رویکرد \lr{CTDE} تشریح می‌شود و پیوند آن با بازی‌های جمع‌صفر و تعادل نش بیان می‌گردد. در \textbf{فصل پنجم} مدل‌سازی محیط آزمایش بر پایه‌ی \lr{CRTBP} ارائه می‌شود. در \textbf{فصل ششم} طراحی عامل‌ها، فضای حالت/عمل، تابع پاداش و جزئیات آموزش توضیح داده می‌شود. در \textbf{فصل هفتم} چارچوب «سخت‌افزار در حلقه» و ارزیابی زمان‌واقعی گزارش می‌شود. سرانجام، در \textbf{فصل هشتم} نتایج، مقایسه با معیارهای مرجع و مسیرهای آینده‌ی پژوهش جمع‌بندی می‌شود.
