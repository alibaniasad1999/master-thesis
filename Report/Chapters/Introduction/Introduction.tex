\chapter{مقدمه}
\noindent 
در سال‌های آغازین عصر فضا، فرآیند هدایت فضاپیماها عمدتاً بر مبانی دینامیک کلاسیک و کنترل خطی استوار بوده است. با این حال، پیچیدگی روزافزون مأموریت‌های کنونی مانند سفرهای میان‌سیاره‌ای با پیشران‌کم و شبکه‌های انبوه ماهواره‌ای در مدار زمین، موجب دوچندان شدن ضرورت بهره‌گیری از روش‌های هوشمند و تطبیق‌پذیر شده‌است.
%در این فصل، 
%به بررسی 
%با ترسیم نقشه راه پژوهش حاضر، ابتدا اهمیت هدایت بهینه در میدان‌های جاذبه غیرخطی روشن شده است. سپس، شکاف موجود میان کنترل کلاسیک و فناوری نوپای یادگیری تقویتی تبیین گردیده است. در ادامه، نشان داده خواهد شد که چگونه مرزهای کارایی و پایداری مأموریت‌های فضایی می‌تواند از طریق ترکیب رهیافت‌های یادگیری تقویتی تک‌عاملی و چندعاملی با مدل سه‌جسمی محدود و دایروی جابه‌جا شود.
%\section{انگیزه پژوهش}
%\section{انگیزه پژوهش}\label{sec:motivation}
%
%در دههٔ اخیر، گرایش روزافزون مأموریت‌های فضایی به استفاده از سامانه‌های سبک‌وزن و کم‌هزینه—از مدارگردهای زیردولتی گرفته تا صورت‌های متنوع ماهواره‌های مکعبی (CubeSat)—تقاضا برای روش‌های هدایت و کنترل خودکار و تضمین‌کنندهٔ امنیت مأموریت را به شکلی چشمگیر افزایش داده است. در این مأموریت‌ها، منابع پیشرانی محدود و محدودیت‌های توان پردازشی، هر دو، محیط طراحی را به سمت راهکارهایی سوق می‌دهند که ضمن بهینگی مصرف سوخت، نسبت به عدم‌قطعیت‌های «محیط سه‌جسمی محدود»ootnote{Circular Restricted Three‑Body Problem یا CRTBP} و آشفتگی‌های ناشی از خطاهای مدل و اختلالات بیرونی (مانند فشار تابشی خورشید و نامحوری جرم سیاره) مقاوم باشند.  
%
%از سوی دیگر، قرارگیری در نقطه‌های لاگرانژی \lr{\(L_1\)} و \lr{\(L_2\)} سامانهٔ خورشید–زمین، فرصت‌های اکتشافی بی‌سابقه‌ای—از رصد پس‌زمینهٔ کیهانی گرفته تا هواشناسی فضایی—فراهم کرده است، اما پایداری ذاتی پایین این نواحی، نیازمند راهبردهای ایمن نگه‌دارنده (station‑keeping) با دورهٔ کنترل کوتاه و واکنش سریع است. روش‌های کلاسیک همچون \lr{LQR} و هدایت بهینهٔ \lr{(OC)}, اگرچه در بسیاری از کاربردهای زمینی کارآمد بوده‌اند، در حضور سامانه‌های غیرخطی با ابعاد بالا و دینامیک آشوب‌گونهٔ CRTBP مستعد فروپاشی عملکرد (performance degradation) هستند؛ به ویژه زمانی که مدل دقیق جرم‌های آسمانی در دسترس نباشد یا حافظهٔ محاسباتی فضاپیما محدود شود.
%
%پیشرفت‌های اخیر در \lr{Machine Learning} و به طور خاص \lr{Reinforcement Learning (RL)}، چشم‌اندازی نو را پیش روی پژوهشگران سامانه‌های کنترلی قرار داده است؛ چراکه RL با یادگیری گام‌به‌گام از تعامل با محیط می‌تواند سیاست‌هایی را بیابد که نیازی به مدل صریح از دینامیک نداشته باشند. با این حال، اکثر کاربردهای موجود RL در حوزهٔ فضایی به صورت «تک‌عاملی» (Single‑Agent) و عمدتاً در چارچوب مسئله‌های \lr{Markov Decision Process (MDP)} صورت پذیرفته‌اند، حال آنکه تولد سامانه‌های مشارکتی و سناریوهای چندفازی (مانند مانورهای گله‌ای ماهواره‌ها یا مشارکت زمینی–مداری) مستلزم فریم‌ورک‌هایی چندعاملی است که بتواند برهم‌کنش‌های رقابتی–همکاری و همچنین نامعلومی‌های مشترک را مدیریت کند. الگوریتم‌های \lr{Multi‑Agent RL (MARL)}، به ویژه هنگامی که با تئوری بازی‌های دیفرانسیلی ترکیب شوند، قابلیتی فراهم می‌کنند تا عامل‌ها در برابر بدترین حالت اختلالات احتمالی (adversarial disturbances) آموزش ببینند و از آن طریق، ضریب اطمینان مأموریت ارتقا یابد.
%
%با وجود این پتانسیل، خلئی محسوس در ادبیات وجود دارد:  
%\begin{itemize}
%	\item \textbf{عدم تصریح چارچوبی جامع برای RL مقاوم در CRTBP} \cite{Naasz2022,Gabriel2023}؛ اکثر پژوهش‌ها یا به پایداری نقاط تعادل ساده اکتفا کرده یا تقارن کامل سامانه را فرض می‌کنند.
%	\item \textbf{نبود مقایسهٔ منصفانه بین الگوریتم‌های گرادیان‑سیاست (مانند TD3 و SAC) و روش‌های مبتنی بر توابع ارزش توزیعی} در سناریوهای فضایی گسسته–پیوسته.
%	\item \textbf{فقدان گزارش کمی از مزیت سوختی RL نسبت به کنترلرهای بهینهٔ کلاسیک}؛ آماری چون $\Delta m/m <1\%$ غالباً کیفی گزارش می‌شوند.
%\end{itemize}
%
%بر همین اساس، هدف از این پژوهش عبارت است از:  
%\begin{enumerate}
%	\item توسعهٔ یک چارچوب \textit{مقاوم} \lr{MARL}—بر پایهٔ بازی دیفرانسیلی صفرمجموع—برای مسئلهٔ هدایت کم‌پیشران در CRTBP، به نحوی که در برابر خطاهای مدل جرم اولیه و فشار تابشی تا $\pm20\,\%$ پایداری را حفظ کند.
%	\item طراحی یک تابع پاداش ترکیبی \(R=\alpha J_\text{fuel}+\beta J_\text{deviation}+\gamma J_\text{slack}\) که به طور هم‌زمان، مصرف سوخت و انحراف از مدار نامی را کمینه و محدودیت‌های پیشران را رعایت کند.
%	\item ارائهٔ مقایسهٔ جامع بین کنترلر پیشنهادی و رویکرد \lr{LQR/LQG}، از حیث $\Delta v$، تعداد اشتعال و خطای پسا‑ماموریت.
%\end{enumerate}
%
%تحقق این اهداف نه تنها می‌تواند سهم قابل‌توجهی در کاهش هزینه‌های مأموریت و افزایش عمر عملیاتی فضاپیماها داشته باشد، بلکه دانش طراحی کنترلی برای محیط‌های غیرخطی آشوبی را یک گام به افق «استقلال کامل تصمیم‌گیری در فضا» نزدیک‌تر می‌کند. به طور خاص، چارچوب ارائه شده قابلیت آن را دارد که در آینده به سادگی به سامانه‌های سه‌بعدی \lr{CR3BP} یا حتی دینامیک‌های چندجرمی تعمیم یابد، بدون آنکه نیاز به بازنویسی ساختار اصلی شبکهٔ عصبی باشد.
%
\section{انگیزه پژوهش}\label{sec:motivation}


%در دو دههٔ اخیر، چشم‌انداز مأموریت‌های فضایی به‌‌واسطهٔ کوچک‌سازی سامانه‌ها، پیشرفت الکترونیک ارزان‌قیمت و افزایش دسترس‌پذیری پرتابْ‌بارها، تغییر محسوسی را تجربه کرده است. از پروژه‌‌های علمی بین‌سیاره‌ای با محدودیت‌های منابع گرفته تا منظومه‌های پرشمارهٔ ماهواره‌ای در مدار زمین، همگی با چالش مشترکِ «هدایت بهینه در حضور عدم‌قطعیت» روبه‌رو هستند. در مسیرهای ترا-لونا و به‌طور خاص ناحیه‌های ناپایدار لاگرانژی در مدل «مسئلهٔ سه‌جسمی کروی محدود و دایروی» (CRTBP)، سامانهٔ کنترل باید قادر باشد با پیشران کم‌تراست (\emph{low‐thrust}) و underactuated، پایداری ایستا و کارایی مصرف سوخت را توأمان تضمین کند.  
%
%در عین حال، ظهور الگوریتم‌های یادگیری تقویتی عمیق (DRL) دریچهٔ تازه‌ای به‌سوی طراحی کنترلگرهای تطبیقی باز کرده است؛ ولی اغلب رویکردهای متداول بر سناریوهای تک‌عاملی و مدل‌های دقیق تکیه دارند. نبودِ استراتژی مقاوم در برابر اغتشاشات مدل و متغیرهای محیطی—از جمله خطای تراستِ موتورهای یونی و تأخیر حسگر—موجب می‌شود عملکرد در شرایط واقعی فاصلهٔ چشمگیری از نتایج شبیه‌سازی ایده‌آل پیدا کند. این پژوهش می‌کوشد شکاف مذکور را با استناد به چارچوب «یادگیری تقویتی چندعاملی مقاوم» پُر کرده و امکان هدایت کم‌سوخت در CRTBP را با وثوق عملیاتی ارتقا دهد.
%
%
%
%
%
%در دو دهه اخیر، چشم‌انداز مأموریت‌های فضایی به‌واسطهٔ کوچک‌سازی سامانه‌ها، پیشرفت الکترونیک ارزان‌قیمت، و افزایش دسترس‌پذیری پرتاب‌بارها، دگرگونی چشمگیری را تجربه کرده است. از پروژه‌های علمی بین‌سیاره‌ای با محدودیت‌های منابع گرفته تا منظومه‌های پرشمار ماهواره‌ای در مدار زمین، همگی با چالش مشترک «هدایت بهینه در حضور عدم‌قطعیت» روبه‌رو هستند. به‌ویژه در مسیرهای فرا-ماهی (ترا-لونا) و ناحیه‌های ناپایدار لاگرانژی در مدل «مسئلهٔ سه‌جسمی کروی محدود و دایروی» (CRTBP)، سامانه کنترل باید بتواند با پیشران کم‌تراست (low-thrust) و underactuated، پایداری ایستا و کارایی مصرف سوخت را توأمان تضمین کند.
%
%در همین حال، ظهور الگوریتم‌های یادگیری تقویتی عمیق (DRL)، افق‌های جدیدی را در طراحی کنترلگرهای تطبیقی گشوده است. با این وجود، اغلب رویکردهای متداول بر سناریوهای تک‌عاملی و مدل‌های دقیق متکی هستند. نبود یک استراتژی مقاوم در برابر اغتشاشات مدل و متغیرهای محیطی—مانند خطای تراست موتورهای یونی و تأخیر حسگر—باعث می‌شود که عملکرد در شرایط واقعی، فاصله چشمگیری از نتایج شبیه‌سازی ایده‌آل داشته باشد. این پژوهش می‌کوشد شکاف مذکور را با بهره‌گیری از چارچوب «یادگیری تقویتی چندعاملی مقاوم» پُر کرده و امکان هدایت کم‌سوخت در CRTBP را با اطمینان عملیاتی ارتقا دهد.

% "Translunar" refers to anything situated or traveling beyond the Moon or its orbit around Earth. It's commonly used in the context of spacecraft missions, particularly the translunar injection (TLI), which is a maneuver that puts a spacecraft on a trajectory towards the Moon. 
در دو دهه‌ی اخیر، ماموریت های فضایی به دلیل کوچک‌سازی سامانه ها، توسعه‌ی وسایل الکترونیک مقرون به صرفه و افزایش ظرفیت های پرتابی، تحولات بنیادینی را تجربه کرده است. از پروژه های علمی بین سیاره‌ایی گرفته تا منظومه های انبوه ماهواره‌ایی در مدارهای پایین زمین، همگی با چالش فراگیر هدایت بهینه در حضور عدم قطعیت های ذاتی مواجه‌اند. در مسیرهای فرا-قمری\LTRfootnote{Trans-lunar}
 و به طور خاص در ناحیه های ناپایدار نقاط لاگرانژ در چارچوب مسئله‌ی سه جسمی کروی محدود و دایروی\LTRfootnote{Circular Restricted Three-Body Problem (CRTBP)}، طراحی سامانه‌ی کنترل مستلزم توانایی تضمین همزمان پایداری ایستا و بهره‌وری سوخت با نیروی پیشران‌کم\LTRfootnote{Low-thrust}
 است.

هم‌راستا با این تحولات، ظهور و گسترش الگوریتم‌های یادگیری تقویتی عمیق\LTRfootnote{Deep Reinforcement Learning (DRL)}، امکانات نوینی را برای طراحی کنترل‌کننده‌های تطبیقی فراهم آورده است. با این حال، غالب رویکردهای رایج بر سناریوهای تک‌عاملی و اتکا به مدل‌های دینامیکی دقیق استوارند. غیاب یک استراتژی مقاوم در مواجهه با اغتشاشات مدل و تغییرات محیطی از جمله خطای تراست در پیشران‌ و تأخیر در سیگنال‌های حسگر منجر به فاصله‌ی چشمگیر عملکرد واقعی از پیش‌بینی‌های حاصل از شبیه‌سازی‌های ایده‌آل می‌گردد. این پژوهش بر آن است تا گسست اشاره‌شده را با بهره‌گیری از چارچوب یادگیری تقویتی چندعاملی مقاوم مرتفع سازد و بدین وسیله، اطمینان هدایت پیشران‌کم در \lr{CRTBP} را افزایش دهد.
\section{تعریف مسئله}\label{sec:problem_statement}
\noindent  در سال‌های اخیر، پیشرفت‌های فناوری در زمینه‌های مختلف، از جمله کنترل پرواز، پردازش سیگنال و هوش مصنوعی، به افزایش کاربردهای ماهواره با پیشران کم در منظومه زمین‐ماه کمک کرده است. ماهواره با پیشران کم می‌تواند برای تعقیب ماهواره‌ها، انتقال مداری و استقرار ماهواره‌ها استفاده شود.
روش‌های هدایت بهینه قدیمی جهت کنترل ماهواره‌ها اغلب نیازمند فرضیات ساده کننده، منابع محاسباتی فراوان و شرایط اولیه مناسب هستند. الگوریتم‌های مبتنی بر یادگیری تقویتی این توانایی را دارند که بدون مشکلات اشاره‌شده هدایت ماهواره را انجام دهند. به‌دلیل ساختار شبکه‌ای، این الگوریتم‌ها می‌توانند امکان محاسبات درونی\LTRfootnote{On-board Computing}  
را فراهم کنند.



هدف از این پژوهش، طراحی سیاست کنترلی برای یک فضاپیما به جرم~$m$ است که در میدان جاذبه‌ی سیستم زمین–ماه به‌صورت دوبُعدی مدل شده‌است. ویژگی‌های این سامانه در ادامه آورده شده‌است.

\begin{itemize}
	\item \textbf{پویایی‌ها:} معادلات حرکت در چارچوب مرجع چرخان به‌صورت مجموعهٔ غیرخطی {$\dot{\boldsymbol{x}} = f(\boldsymbol{x}) + g(\boldsymbol{x})\,\boldsymbol{u}$} نوشته می‌شود که $\boldsymbol{x}\!=\![x,y,\dot x,\dot y]^\top$ بردار حالت و $\boldsymbol{u}$ بردار تراست با کران $\boldsymbol{u}\le u_{\max}$ است.
	\item \textbf{عدم‌قطعیت‌ها:} 
عوامل عدم قطعیت شامل شرایط اولیه تصادفی، اغتشاش در عملگرها، عدم تطابق مدل، مشاهده ناقص، نویز حسگر و تأخیر زمانی هستند که همگی بر عملکرد و پایداری سیستم تأثیر می‌گذارند.
	
%	\item \textbf{قیود مأموریت:} فضای مجاز حرکت حول نقطهٔ لاگرانژی~$L_1$ شعاعی معادل 500~km دارد و میزان مصرف پیشران {$J = \int_0^{t_f}\|\mathbf{u}(t)\|\,\mathrm dt$} باید حداقل شود.
	\item \textbf{صورت بازی دیفرانسیلی:} فضاپیما و طبیعت (اغتشاشات) به‌ترتیب به‌عنوان عامل کنترلی و حریفِ مزاحم مدل شده‌است. مسئله به‌عنوان یک بازی مجموع‌صفر \LTRfootnote{Zero-Sum} 
	در افق زمان محدود {$t_f$} فرمول‌بندی شده‌اند.
\end{itemize}

صورت کامل مسأله را می‌توان با یافتن سیاست {$\pi^\star : \mathcal{X}\to \mathcal{U}$} تعریف کرد که معیار بهینه‌سازی هزینه‌ی تجمعی است.

%رشدمی‌یافته
%\[ J(\pi,\omega) = \mathbb{E}\_{\pi,\omega}\Big[\,\alpha_1\,J\_{\text{fuel}} + \alpha_2\,J\_{\text{deviation}}\Big] \]
%را در بدترین سناریوی اغتشاش~$\omega$ کمینه سازد. ضرایب $\alpha_{1,2}$ اهمیت نسبی سوخت و انحراف مسیر را نشان می‌دهد.

%\section{اهداف و نوآوری}


\section{اهداف و نوآوری پژوهش}\label{sec:objectives}

برای پاسخ به مسألهٔ فوق، این رساله سه هدف اصلی و چهار نوآوری کلیدی را دنبال می‌کند.

\subsection*{اهداف}
\begin{enumerate}
	\item توسعهٔ چارچوب یادگیری تقویتی چندعاملی که در آن فضاپیما و طبیعت به‌عنوان دو عامل با اهداف متضاد مدل می‌شوند.
	\item طراحی تابع پاداش چندبُعدی به‌منظور موازنهٔ مصرف سوخت، پایداری موقعیت و نرمی فرمان.
	\item ارزیابی تجربی روی سناریوهای {{CRTBP}} و مقایسه با کنترلگرهای کلاسیک (MPC, $H_\infty$) و تک‌عاملی (DDPG, TD3).
\end{enumerate}

\subsection*{نوآوری‌ها}
\begin{itemize}
	\item \textbf{آموزش متمرکز–اجرا توزیع‌شده (CTDE)}: استفاده از تخمین‌گر مشترک ارزش برای تقابل عامل–حریف و اعمال سیاست غیرمتمرکز در زمان اجرا.
	\item \textbf{تقریب سیاست مقاوم با شبکهٔ دوگان Q}: به‌کارگیری TD3 دوبخشی برای کاهش گرادیان بایاس ناشی از تابع ارزش بیش‌براورد.
	\item \textbf{شکل‌دهی پاداش مبتنی بر انرژی مدار}: معرفی مولفهٔ انرژی ژاکوبی در پاداش به‌منظور تضمین نزدیکی به سطح انرژی نامی~$C_{J^\ast}$.
	\item \textbf{سامانهٔ شبیه‌سازی متن‌باز}: پیاده‌سازی محیط CRTBP در قالب افزونهٔ {OpenAI Gym} با امکان تزریق عدم‌قطعیت و انتشار کد در GitHub.
\end{itemize}
%\section{
%    یادگیری تقویتی
%    }
%

%--------------------------------------------------------------------
\section{یادگیری تقویتی}\label{sec:rl}

یادگیری تقویتی\LTRfootnote{Reinforcement Learning (RL)}
 شاخه‌ای از یادگیریِ ماشین است که در آن توالیِ اقدام‌ها $\boldsymbol{a}_t\in\mathcal{A}$ به‌گونه‌ای انتخاب می‌شود که بازدهِ تجمعیِ آینده بیشینه شود. یک فرایندِ تصمیم‌گیریِ مارکوف\LTRfootnote{Markov Decision Process (MDP)}
  به‌صورت $\langle\mathcal{S},\mathcal{A},p,r,\gamma\rangle$ تعریف می‌شود که در آن:
\begin{itemize}
	\item $\mathcal{S}$:
	 مجموعه‌ی حالات،
	\item $p(\boldsymbol{s}'|\boldsymbol{s},\boldsymbol{a})$:
	 دینامیکِ انتقال،
	\item $r(s,\boldsymbol{a})$:
	 پاداشِ آنی،
	\item $\gamma\in[0,1)$:
	 ضریبِ تنزیل.
\end{itemize}
\noindent
سیاست\LTRfootnote{Policy}
 $\pi(a|s)$ به‌عنوان احتمالِ انتخابِ اقدامِ $\boldsymbol{a}$ در وضعیتِ $\boldsymbol{s}$ بیان می‌شود. هدف، بیشینه‌سازیِ برگشت\LTRfootnote{Return}
 است:
 \begin{equation}
 	G_t=\sum_{k=0}^{\infty}\gamma^k r_{t+k}
 \end{equation}
 
 روش‌های \lr{RL} معمولاً در دو دسته‌ی {ارزش‌محور} (مانند \lr{Q-learning} و \lr{DQN}) و {سیاست‌محور} (مانند \lr{Reinforce}) جای می‌گیرند؛ ترکیبِ این دو به چارچوبِ \lr{Actor–Critic} منتهی می‌شود که در آن، یک بازیگر (\lr{Actor}) سیاست را به‌روزرسانی می‌کند و یک منتقد (\lr{Critic}) ارزش یا \lr{Q} برآورد می‌شود
  \cite{SuttonBarto2018}.

در حضورِ فضاهای پیوسته‌ی حالت–عمل، الگوریتم‌های  \lr{DDPG}، \lr{TD3}، \lr{SAC} و \lr{PPO} با تکیه بر شبکه‌های عصبی به‌عنوان تقریب‌گر توابع، کاراییِ بالایی نشان داده‌اند. در این پژوهش، خانواده‌ی \lr{Actor–Critic} به‌عنوان پایه‌ی توسعه‌ی کنترل‌کننده پیشنهاد شده‌ است و در ادامه، به نسخه‌ی چندعاملیِ آن در بخش~\ref{sec:marl} پیوند داده می‌شود.

\chapter{یادگیری تقویتی چند عاملی}
کاربردهای پیچیده در یادگیری تقویتی نیازمند اضافه کردن چندین عامل\LTRfootnote{Multi-Agent} برای انجام همزمان وظایف مختلف هستند.
با این حال، افزایش تعداد عامل‌ها چالش‌هایی در مدیریت تعاملات میان آن‌ها به همراه دارد.
در این فصل، بر اساس مسئله بهینه‌سازی برای هر عامل، مفهوم تعادل\LTRfootnote{Equilibrium} معرفی شده تا رفتارهای توزیعی چندعاملی را تنظیم کند.
رابطه رقابت میان عامل‌ها در سناریوهای مختلف تحلیل شده و آن‌ها با الگوریتم‌های معمول یادگیری تقویتی چندعاملی ترکیب شده‌اند. بر اساس انواع تعاملات، یک چارچوب نظریه بازی برای مدل‌سازی عمومی در سناریوهای چندعاملی استفاده شده است. با تحلیل بهینه‌سازی و وضعیت تعادل برای هر بخش از چارچوب، سیاست بهینه یادگیری تقویتی چندعاملی برای هر عامل بررسی شده است.




  \section{تعاریف و مفاهیم اساسی }
یادگیری تقویتی چندعاملی\LTRfootnote{Multi-Agent Reinforcement Learning (MARL)} به بررسی چگونگی یادگیری و تصمیم‌گیری چندین عامل مستقل در یک محیط مشترک پرداخته می‌شود. برای تحلیل دقیق و درک بهتر این حوزه، اجزای اصلی آن شامل عامل، سیاست و مطلوبیت\LTRfootnote{Utility} در نظر گرفته می‌شوند که در ادامه به صورت مختصر و منسجم تشریح می‌گردند.

\begin{itemize}
	\item عامل: یک موجودیت مستقل به عنوان عامل تعریف می‌شود که به صورت خودمختار با محیط تعامل کرده و بر اساس مشاهدات رفتار سایر عامل‌ها، سیاست‌هایش انتخاب می‌گردند تا سود حداکثر یا ضرر حداقل حاصل شود. در سناریوهای مورد بررسی، چندین عامل به صورت مستقل عمل می‌کنند؛ اما اگر تعداد عامل‌ها به یک کاهش یابد، \lr{MARL} به یادگیری تقویتی معمولی تبدیل می‌شود.
	
	\item سیاست: برای هر عامل در \lr{MARL}، سیاستی خاص در نظر گرفته می‌شود که به عنوان روشی برای انتخاب اقدامات بر اساس وضعیت محیط و رفتار سایر عامل‌ها تعریف می‌گردد. این سیاست‌ها با هدف به حداکثر رساندن سود و به حداقل رساندن هزینه طراحی شده و تحت تأثیر محیط و سیاست‌های دیگر عامل‌ها قرار می‌گیرند.
	
	\item مطلوبیت: مطلوبیت
	هر عامل بر اساس نیازها و وابستگی‌هایش به محیط و سایر عامل‌ها تعریف شده و به صورت سود منهای هزینه، با توجه به اهداف مختلف محاسبه می‌شود. در سناریوهای چندعاملی، از طریق یادگیری از محیط و تعامل با دیگران، مطلوبیت هر عامل بهینه می‌گردد.
\end{itemize}

در این چارچوب، برای هر عامل در \lr{MARL} تابع مطلوبیت خاصی در نظر گرفته شده و بر اساس مشاهدات و تجربیات حاصل از تعاملات، یادگیری سیاست به صورت مستقل انجام می‌شود تا ارزش مطلوبیت به حداکثر برسد، بدون اینکه مستقیماً به مطلوبیت سایر عامل‌ها توجه شود. این فرآیند ممکن است به رقابت یا همکاری میان عامل‌ها منجر گردد.
 با توجه به پیچیدگی تعاملات میان چندین عامل، تحلیل نظریه بازی‌ها به عنوان ابزاری مؤثر برای تصمیم‌گیری در این حوزه به کار گرفته می‌شود. بسته به سناریوهای مختلف، این بازی‌ها در دسته‌بندی‌های متفاوتی قرار داده شده که در بخش‌های بعدی بررسی خواهند شد.

  % \section{اهمیت یادگیری تقویتی چندعاملی}

یادگیری تقویتی چندعاملی به دلیل قابلیت‌های بالقوه‌اش در مدل‌سازی و حل مسائل پیچیده و پویا، اهمیت زیادی در حوزه‌های مختلف علمی و صنعتی دارد. در این بخش، به بررسی اهمیت \lr{MARL} در زمینه‌های مختلف پرداخته و نقش آن را در توسعه سیستم‌های هوشمند متعدد بررسی می‌کنیم.

\subsubsection{مدل‌سازی سیستم‌های پیچیده و پویا}
یکی از دلایل اصلی اهمیت \lr{MARL}، توانایی آن در مدل‌سازی سیستم‌های پیچیده و پویا است. در بسیاری از کاربردهای واقعی، سیستم‌ها شامل چندین عامل هستند که به صورت همزمان و مستقل به تعامل می‌پردازند. به عنوان مثال، در شبکه‌های ترافیکی، هر خودرو می‌تواند به عنوان یک عامل مستقل عمل کند که نیاز به هماهنگی و تعامل با سایر خودروها برای بهینه‌سازی جریان ترافیک دارد. \lr{MARL} با فراهم کردن چارچوبی برای تعامل و یادگیری میان این عوامل، امکان بهبود کارایی و کاهش ترافیک را فراهم می‌کند.

\subsubsection{کاربرد در رباتیک چندعاملی}
در حوزه رباتیک، سیستم‌های چندعاملی می‌توانند برای انجام وظایف پیچیده‌ای مانند جست‌وجو و نجات، حمل و نقل مواد، و عملیات هماهنگ در محیط‌های غیرقابل پیش‌بینی مورد استفاده قرار گیرند. به عنوان مثال، گروهی از ربات‌های پرنده (دُرون‌ها) می‌توانند با همکاری و تبادل اطلاعات، منطقه‌ای وسیع را برای شناسایی اهداف نظارت کنند یا به سرعت به تغییرات محیطی واکنش نشان دهند. \lr{MARL} در این زمینه بهبود هماهنگی میان ربات‌ها و افزایش کارایی عملیات‌های چندعاملی را ممکن می‌سازد.

\subsubsection{مدیریت منابع در شبکه‌های ارتباطی}
شبکه‌های ارتباطی مدرن نیازمند مدیریت بهینه منابع مانند پهنای باند، انرژی و ظرفیت ذخیره‌سازی هستند. در این راستا، \lr{MARL} می‌تواند به عنوان یک ابزار قدرتمند برای تخصیص بهینه منابع به عوامل مختلف شبکه عمل کند. به عنوان مثال، در شبکه‌های بی‌سیم، هر دستگاه کاربر می‌تواند به عنوان یک عامل مستقل عمل کرده و با یادگیری و تعامل با سایر دستگاه‌ها، نحوه بهینه‌سازی مصرف انرژی و پهنای باند را پیدا کند. این امر منجر به افزایش کارایی شبکه و کاهش هزینه‌های عملیاتی می‌شود.

\subsubsection{توسعه الگوریتم‌های پیشرفته‌تر و قابل اعتمادتر}
یکی دیگر از جنبه‌های مهم \lr{MARL}، فهم و تحلیل تعاملات میان عوامل مختلف است که می‌تواند به توسعه الگوریتم‌های پیشرفته‌تر و قابل اعتمادتر منجر شود. با مطالعه رفتارها و استراتژی‌های مختلف در محیط‌های چندعاملی، پژوهشگران قادر به طراحی الگوریتم‌هایی می‌شوند که نه تنها بهینه عمل می‌کنند بلکه مقاومت بالایی در برابر تغییرات محیطی و رفتارهای غیرمنتظره دارند. این الگوریتم‌ها می‌توانند در شرایط متنوع و پیچیده‌تر به خوبی عمل کنند و از خطاها و ناهنجاری‌های احتمالی جلوگیری نمایند.

\subsubsection{کاربرد در بازی‌های چندعاملی و شبیه‌سازی‌های اقتصادی}
بازی‌های چندعاملی و شبیه‌سازی‌های اقتصادی از دیگر حوزه‌هایی هستند که به شدت از \lr{MARL} بهره‌مند می‌شوند. در بازی‌های استراتژیک چند نفره، \lr{MARL} می‌تواند به بازیگران کمک کند تا استراتژی‌های بهینه‌ای برای رقابت و همکاری با یکدیگر توسعه دهند. همچنین، در شبیه‌سازی‌های اقتصادی، \lr{MARL} می‌تواند به مدل‌سازی و تحلیل رفتارهای بازار و تصمیم‌گیری‌های اقتصادی کمک کند، که این امر به پیش‌بینی دقیق‌تر روندهای اقتصادی و بهبود سیاست‌گذاری‌های مالی منجر می‌شود.

\subsubsection{افزایش قابلیت انعطاف‌پذیری و مقیاس‌پذیری سیستم‌ها}
سیستم‌های چندعاملی معمولاً نیازمند قابلیت انعطاف‌پذیری و مقیاس‌پذیری بالا هستند تا بتوانند با تغییرات محیطی و افزایش تعداد عوامل سازگار شوند. \lr{MARL} با استفاده از الگوریتم‌های توزیع‌شده و یادگیری محلی، امکان توسعه سیستم‌هایی با مقیاس بزرگ و پیچیدگی بالا را فراهم می‌کند. این امر به ویژه در کاربردهایی مانند اینترنت اشیاء\LTRfootnote{Internet of Things (IoT)}، هوش مصنوعی توزیع‌شده و سیستم‌های بزرگ‌مقیاس داده‌های بزرگ\LTRfootnote{Big Data}
 بسیار حائز اهمیت است.

%\subsubsection{نتیجه‌گیری}
%در نهایت، یادگیری تقویتی چندعاملی به عنوان یک ابزار قدرتمند در حل مسائل پیچیده و پویا مطرح است که قابلیت‌های متنوعی در حوزه‌های مختلف علمی و صنعتی ارائه می‌دهد. از مدل‌سازی سیستم‌های پیچیده و رباتیک چندعاملی گرفته تا مدیریت منابع در شبکه‌های ارتباطی و توسعه الگوریتم‌های پیشرفته‌تر، \lr{MARL} نقش کلیدی در پیشرفت تکنولوژی‌های هوشمند و خودکار ایفا می‌کند. با ادامه تحقیقات و بهبود الگوریتم‌های موجود، انتظار می‌رود که کاربردهای \lr{MARL} همچنان گسترش یافته و تاثیرات مثبتی در حوزه‌های مختلف داشته باشد.


    \subsection{بازی‌های جمع صفر}

بازی‌های جمع صفر\LTRfootnote{Zero-Sum}
 یکی از انواع اصلی بازی‌های چندعاملی هستند که در آن سود یک بازیکن به طور مستقیم با ضرر بازیکنان دیگر مرتبط است. در این بازی‌ها، مجموع پاداش‌ها برای همه بازیکنان در هر حالت برابر با صفر است، به این معنی که هر افزایشی در پاداش یکی از بازیکنان منجر به کاهش معادل آن در بازیکنان دیگر می‌شود. این نوع بازی‌ها به خوبی می‌توانند رقابت‌های شدید و استراتژی‌های بهینه را مدل‌سازی کنند.

بازی‌های جمع صفر می‌توانند بر اساس سناریوهای مختلف به دسته‌های متنوعی تقسیم‌بندی شوند. دو دسته اصلی این بازی‌ها عبارتند از بازی‌های ثابت و بازی‌های تکراری.

\begin{itemize}
	\item \textbf{بازی ثابت (\lr{Static Game}):} بازی ثابت ساده‌ترین شکل برای مدل‌سازی تعاملات میان عوامل است. در بازی ثابت، هر عامل تنها یک تصمیم‌گیری واحد را انجام می‌دهد. از آنجایی که هر عامل تنها یک بار عمل می‌کند، تقلب و خیانت غیرمنتظره می‌تواند در این نوع بازی‌ها سودآور باشد. بنابراین، هر عامل نیاز دارد تا به دقت استراتژی‌های سایر عوامل را پیش‌بینی کند تا بتواند به طور هوشمندانه عمل کرده و بیشترین سود ممکن را کسب کند. بازی‌های ثابت معمولاً در سناریوهای رقابتی با تعاملات کوتاه مدت کاربرد دارند.
	
	\item \textbf{بازی تکراری (\lr{Repeated Game}):} بازی تکراری به وضعیتی اشاره دارد که در آن تمام عوامل می‌توانند بر اساس همان وضعیت برای چندین تکرار اقداماتی انجام دهند. سود کلی هر عامل مجموع سودهای تخفیف‌شده برای هر تکرار از بازی است. به دلیل اقدامات مکرر تمام عوامل، تقلب و خیانت در طول تعاملات می‌تواند منجر به مجازات یا انتقام از سوی سایر عوامل در تکرارهای آینده شود. بنابراین، بازی تکراری از رفتارهای مخرب عوامل جلوگیری می‌کند و به طور کلی سود کل برای تمام عوامل را افزایش می‌دهد. بازی‌های تکراری معمولاً در سناریوهای همکاری بلندمدت و تعاملات پویا کاربرد دارند.
\end{itemize}

این دسته‌بندی‌ها به محققان و توسعه‌دهندگان کمک می‌کنند تا بازی‌های چندعاملی را بر اساس ویژگی‌های مختلف آن‌ها شناسایی و تحلیل کنند. در بازی‌های ثابت، تمرکز بر پیش‌بینی دقیق استراتژی‌های دیگر عوامل و اتخاذ بهترین تصمیم در یک لحظه زمانی است. در مقابل، بازی‌های تکراری نیازمند توسعه استراتژی‌های پایدار و قابل اعتماد هستند که نه تنها در تکرار اول بلکه در تکرارهای بعدی نیز موثر باشند.

بازی‌های جمع صفر در یادگیری تقویتی چندعاملی به دلیل سادگی و قابلیت مدل‌سازی دقیق تعاملات رقابتی، به عنوان یک ابزار قدرتمند برای تحلیل و توسعه الگوریتم‌های \lr{MARL} مورد استفاده قرار می‌گیرند. این بازی‌ها امکان بررسی رفتارهای استراتژیک، بهینه‌سازی سیاست‌ها و تحلیل تعادل‌های نش (\lr{Nash Equilibrium}) را فراهم می‌کنند که در نهایت به بهبود عملکرد سیستم‌های چندعاملی منجر می‌شود.

\paragraph{مثال‌ها و کاربردها}
یکی از مثال‌های معروف بازی‌های جمع صفر، بازی شطرنج است که در آن هر حرکت یک بازیکن مستقیماً به نفع یا ضرر بازیکن دیگر است. سایر مثال‌ها شامل بازی‌های استراتژیک مانند \lr{Poker} و \lr{Go} می‌باشند که در آن‌ها تعاملات رقابتی میان بازیکنان به طور کامل با اصول بازی‌های جمع صفر مطابقت دارند.

در حوزه‌های عملی، بازی‌های جمع صفر می‌توانند برای مدل‌سازی رقابت‌های بازار، مذاکرات اقتصادی و حتی تعاملات میان ربات‌های خودران در محیط‌های رقابتی مورد استفاده قرار گیرند. این کاربردها به محققان امکان می‌دهند تا الگوریتم‌هایی طراحی کنند که قادر به بهینه‌سازی عملکرد در شرایط رقابتی و متغیر باشند.

\paragraph{چالش‌ها و فرصت‌ها}
یکی از چالش‌های اصلی در بازی‌های جمع صفر، پیش‌بینی دقیق رفتارهای رقبا و اتخاذ تصمیم‌های بهینه در مواجهه با استراتژی‌های متغیر آن‌ها است. همچنین، در بازی‌های تکراری، ایجاد تعادل‌های پایدار و جلوگیری از رفتارهای مخرب به عنوان یک چالش مهم مطرح است. با این حال، این چالش‌ها فرصت‌های قابل توجهی برای توسعه الگوریتم‌های پیشرفته و افزایش قابلیت‌های یادگیری تقویتی چندعاملی فراهم می‌کنند که می‌توانند در شرایط پیچیده‌تر و پویا نیز عملکرد مطلوبی داشته باشند.

    \section{تعادل نش}

    %\subsection{بازی مجموع صفر}
%
%بازی‌های مجموع صفر\LTRfootnote{Zero-Sum Games}
% دسته‌ای از بازی‌ها هستند که در آن‌ها تابع ارزش یک بازیکن دقیقاً برابر با ضرر بازیکن دیگر است. به عبارت دیگر، مجموع ارزشهای همه بازیکنان در هر مرحله صفر است.
%
%
%
%\begin{itemize}
%	\item تعریف بازی مجموع صفر:
%	در یک بازی دو نفره، اگر تابع ارزش بازیکن اول (\( V_1^{(\pi_1 ,\pi_2)}(s)
%	\)) و بازیکن دوم (\( V_2^{(\pi_1 ,\pi_2)}(s)
%	\)) به‌گونه‌ای باشد که برای هر مجموعه سیاست
%	\( (\pi_1, \pi_2) \) 
%به صورت زیر باشد را یک بازی مجموع صفر نامیده می‌شود.
%	\begin{equation}\label{eq:game_v}
%		V_1^{(\pi_1 ,\pi_2)}(s) + V_2^{(\pi_1 ,\pi_2)}(s) = 0 \to V_1^{(\pi_1 ,\pi_2)}(s) = -V_2^{(\pi_1 ,\pi_2)}(s)
%%		V_1^{(\pi_1 ,\pi_2)}(s) =- V_2^{(\pi_1, \pi_2)}(s)
%	\end{equation}
%	\item سیاست بهینه در بازی مجموع صفر:
%	در بازی‌های مجموع صفر، سیاست بهینه هر بازیکن، انتخابی است که  تابع ارزش خود را در برابر بهترین پاسخ حریف به حداکثر برساند. این سیاست اغلب به تعادل نش منجر می‌شود. سیاست بهینه دو بازیکن در بازی مجموع صفر با تابع  ارزش معادله
%	\eqref{eq:game_v}
%	به صورت زیر است.
%	
%	\begin{align}
%		V_1^*(s) = \max_{\pi_1} \min_{\pi_2} V_1^{(\pi_1 ,\pi_2)}(s) \\
%		V_2^*(s) = \max_{\pi_2} \min_{\pi_1} V_2^{(\pi_1 ,\pi_2)}(s)
%	\end{align}
%	
%	
%\end{itemize}


















\subsection{بازی مجموع صفر}

بازی‌های مجموع صفر\LTRfootnote{Zero-Sum Games}
دسته‌ای از بازی‌ها هستند که در آن‌ها تابع ارزش یک بازیکن دقیقاً برابر با ضرر بازیکن دیگر است؛ بنابراین، مجموع ارزش‌های همهٔ بازیکنان در هر مرحله صفر خواهد بود.

\begin{itemize}
	%------------------------------------
	\item \textbf{تعریف بازی مجموع صفر:}
	
	در یک بازی دو نفره، اگر تابع ارزشِ حالت (value) بازیکن اوّل 
	\(V_1^{(\pi_1 ,\pi_2)}(s)\)
	و بازیکن دوم 
	\(V_2^{(\pi_1 ,\pi_2)}(s)\)
	برای هر مجموعه سیاست 
	\((\pi_1,\pi_2)\)
	به‌گونه‌ای باشند که:
	\begin{equation}\label{eq:game_v}
		V_1^{(\pi_1 ,\pi_2)}(s) + V_2^{(\pi_1 ,\pi_2)}(s) = 0 
		\;\;\Longrightarrow\;\;
		V_1^{(\pi_1 ,\pi_2)}(s) = -\,V_2^{(\pi_1 ,\pi_2)}(s),
	\end{equation}
	آنگاه آن بازی را \emph{بازی مجموع صفر} می‌نامیم.
	
	به‌طور مشابه، اگر تابع ارزش–عمل برای دو بازیکن را با
	\(Q_1^{(\pi_1,\pi_2)}(s,a_1,a_2)\)
	و
	\(Q_2^{(\pi_1,\pi_2)}(s,a_1,a_2)\)
	نشان دهیم، باید برقرار باشد:
	\begin{equation}\label{eq:game_q}
		Q_1^{(\pi_1,\pi_2)}(s,a_1,a_2) + 
		Q_2^{(\pi_1,\pi_2)}(s,a_1,a_2) = 0
		\;\;\Longrightarrow\;\;
		Q_1^{(\pi_1,\pi_2)}(s,a_1,a_2) = -\,Q_2^{(\pi_1,\pi_2)}(s,a_1,a_2).
	\end{equation}
	%------------------------------------
	
	\item \textbf{سیاست بهینه در بازی مجموع صفر:}
	
	در این بازی‌ها، هر بازیکن سیاستی را برمی‌گزیند که تابع ارزش خود را
	در برابر بهترین پاسخِ حریف بیشینه کند؛ این انتخاب در نهایت به
	تعادل نش منجر می‌شود.
	
	به‌صورت تابع ارزشِ حالت:
	\begin{align}
		V_1^*(s) &= \max_{\pi_1}\,\min_{\pi_2} \;
		V_1^{(\pi_1 ,\pi_2)}(s), \\
		V_2^*(s) &= \max_{\pi_2}\,\min_{\pi_1} \;
		V_2^{(\pi_1 ,\pi_2)}(s).
	\end{align}
	
	و به‌صورت تابع ارزش–عمل:
	\begin{align}
		Q_1^*(s,a_1,a_2) &= \max_{\pi_1}\,\min_{\pi_2} \;
		Q_1^{(\pi_1 ,\pi_2)}(s,a_1,a_2), \\
		Q_2^*(s,a_1,a_2) &= \max_{\pi_2}\,\min_{\pi_1} \;
		Q_2^{(\pi_1 ,\pi_2)}(s,a_1,a_2).
	\end{align}
\end{itemize}



    \section{ گرادیان سیاست عمیق قطعی 
در بازی‌های دو­عاملیِ مجموع‌­صفر
}
% insert_edit_into_file("/Users/Ali/Documents/BAI/Master/master-thesis/Report/Chapters/MARL/MADDPG.tex", r"\section{ گرادیان سیاست عمیق قطعی 
% در بازی‌های دو­عاملیِ مجموع‌­صفر
% }", """
% \section{ گرادیان سیاست عمیق قطعی 
% در بازی‌های دو­عاملیِ مجموع‌­صفر
% }\label{sec:MADDPG}

گرادیان سیاست عمیق قطعی چند­عاملی\LTRfootnote{Multi-Agent Deep Deterministic Policy Gradient (MADDPG)}
توسعه‌ای از الگوریتم \lr{DDPG} برای محیط‌های چند­عاملی است. در این بخش، به بررسی این الگوریتم در چارچوب بازی‌های دو­عاملیِ مجموع­‌صفر می‌پردازیم که در آن مجموع پاداش‌های دو عامل همواره صفر است (آنچه یک عامل به دست می‌آورد، عامل دیگر از دست می‌دهد).

\subsection{چالش‌های یادگیری تقویتی در محیط‌های چند­عاملی}

در محیط‌های چند­عاملی، سیاست هر عامل مدام در حال تغییر است، که باعث می‌شود محیط از دید هر عامل غیرایستا\LTRfootnote{Non-stationary} شود. این مسئله چالش بزرگی برای الگوریتم‌های یادگیری تقویتی تک‌عاملی مانند \lr{DDPG} ایجاد می‌کند، زیرا فرض ایستایی محیط را نقض می‌کند.

\lr{MADDPG} با استفاده از رویکرد آموزش متمرکز، اجرای غیرمتمرکز\LTRfootnote{Centralized Training, Decentralized Execution} این مشکل را حل می‌کند. در این رویکرد، هر عامل در زمان آموزش به اطلاعات کامل محیط دسترسی دارد، اما در زمان اجرا تنها از مشاهدات محلی خود استفاده می‌کند.

\subsection{معماری \lr{MADDPG} در بازی‌های مجموع­‌صفر}

در یک بازی دو­عاملیِ مجموع­‌صفر، دو عامل با نمادهای 1 و 2 نشان داده می‌شوند. هر عامل دارای شبکه‌های منحصر به فرد خود است:

\begin{itemize}
    \item \textbf{شبکه‌های بازیگر:} $\mu_{\theta_1}(o_1)$ و $\mu_{\theta_2}(o_2)$ که مشاهدات محلی $o_1$ و $o_2$ را به اعمال $a_1$ و $a_2$ نگاشت می‌کنند.
    \item \textbf{شبکه‌های منتقد:} $Q_{\phi_1}(o_1, a_1, a_2)$ و $Q_{\phi_2}(o_2, a_2, a_1)$ که ارزش حالت-عمل را با توجه به مشاهدات و اعمال تمام عامل‌ها تخمین می‌زنند.
    \item \textbf{شبکه‌های هدف:} مشابه \lr{DDPG}، برای پایدار کردن آموزش از شبکه‌های هدف استفاده می‌شود.
\end{itemize}

در بازی‌های مجموع­‌صفر، پاداش‌ها رابطه $r_1 + r_2 = 0$ دارند که در آن $r_1$ و $r_2$ پاداش‌های دریافتی عامل‌ها هستند. در نتیجه، $r_2 = -r_1$ است که نمایانگر تضاد کامل منافع بین عامل‌هاست.

\subsection{آموزش \lr{MADDPG} در بازی‌های مجموع­‌صفر}

فرایند آموزش \lr{MADDPG} برای بازی‌های مجموع­‌صفر به شرح زیر است:

\subsubsection{یادگیری تابع \lr{Q}}

برای هر عامل $i \in \{1, 2\}$، تابع \lr{Q} با کمینه کردن خطای میانگین مربعات بلمن به‌روزرسانی می‌شود:

\begin{equation}
    L(\phi_i, \mathcal{D}) = \underset{(\boldsymbol{o}, \boldsymbol{a}, r_i, \boldsymbol{o}', d) \sim \mathcal{D}}{\mathrm{E}}\left[ 
    \Bigg( Q_{\phi_i}(o_i, a_1, a_2) - y_i \Bigg)^2
    \right]
\end{equation}

که در آن $\boldsymbol{o} = (o_1, o_2)$ بردار مشاهدات، $\boldsymbol{a} = (a_1, a_2)$ بردار اعمال، و $y_i$ هدف برای عامل $i$ است:

\begin{equation}
    y_i = r_i + \gamma (1 - d) Q_{\phi_{i,\text{targ}}}(o_i', \mu_{\theta_{1,\text{targ}}}(o_1'), \mu_{\theta_{2,\text{targ}}}(o_2'))
\end{equation}

توجه کنید که منتقد هر عامل به اعمال همه عامل‌ها دسترسی دارد، اما در بازی‌های مجموع­‌صفر، عامل شماره 2 جهت مخالف هدف عامل 1 را دنبال می‌کند.

\subsubsection{یادگیری سیاست}

سیاست هر عامل با بیشینه کردن تابع \lr{Q} مربوط به آن عامل به‌روزرسانی می‌شود:

\begin{equation}
    \max_{\theta_i} \underset{\boldsymbol{o} \sim \mathcal{D}}{\mathrm{E}}\left[ Q_{\phi_i}(o_i, \mu_{\theta_i}(o_i), \mu_{\theta_{-i}}(o_{-i})) \right]
\end{equation}

که در آن $-i$ نشان‌دهنده عامل مقابل است. با توجه به ماهیت بازی مجموع­‌صفر، هر عامل تلاش می‌کند تا مطلوبیت خود را افزایش دهد، در حالی که مطلوبیت عامل دیگر به طور همزمان کاهش می‌یابد.

\subsubsection{شبکه‌های هدف و بافر تجربه}

مشابه \lr{DDPG}، برای پایدار کردن آموزش، شبکه‌های هدف با میانگین‌گیری پولیاک به‌روزرسانی می‌شوند:

\begin{align*}
    \phi_{i,\text{targ}} &\leftarrow \rho \phi_{i,\text{targ}} + (1 - \rho) \phi_i \\
    \theta_{i,\text{targ}} &\leftarrow \rho \theta_{i,\text{targ}} + (1 - \rho) \theta_i
\end{align*}

همچنین، از یک بافر تکرار بازی مشترک برای ذخیره تجربیات استفاده می‌شود که شامل وضعیت‌ها، اعمال و پاداش‌های همه عامل‌هاست.

\subsection{اکتشاف در \lr{MADDPG}}

اکتشاف در \lr{MADDPG} مشابه \lr{DDPG} است، اما برای هر عامل به طور جداگانه اعمال می‌شود. در طی آموزش، به اعمال هر عامل نویز اضافه می‌شود:

\begin{equation}
    a_i = \text{clip}(\mu_{\theta_i}(o_i) + \epsilon_i, a_{\text{Low}}, a_{\text{High}})
\end{equation}

که در آن $\epsilon_i$ نویز اضافه شده به عامل $i$ است.

\subsection{شبه‌کد \lr{MADDPG} برای بازی‌های دو­عاملیِ مجموع­‌صفر}

\begin{algorithm}[H]
    \caption{گرادیان سیاست عمیق قطعی چند­عاملی برای بازی‌های مجموع­‌صفر}\label{alg:MADDPG}
    \begin{algorithmic}[1]
        \ورودی پارامترهای اولیه سیاست عامل‌ها $(\theta_1, \theta_2)$، پارامترهای تابع \lr{Q} $(\phi_1, \phi_2)$، بافر تکرار بازی خالی $(\mathcal{D})$
        \State پارامترهای هدف را برابر با پارامترهای اصلی قرار دهید: $\theta_{i,\text{targ}} \leftarrow \theta_i$, $\phi_{i,\text{targ}} \leftarrow \phi_i$ برای $i \in \{1, 2\}$
        
        \While{همگرایی رخ دهد}
            \State \parbox[t]{\dimexpr\linewidth-\algorithmicindent}{
            مشاهدات $(o_1, o_2)$ را دریافت کنید
            \strut}
            \State \parbox[t]{\dimexpr\linewidth-\algorithmicindent}{
            برای هر عامل $i$، عمل $a_i = \text{clip}(\mu_{\theta_i}(o_i) + \epsilon_i, a_{\text{Low}}, a_{\text{High}})$ را انتخاب کنید، به‌طوری که $\epsilon_i \sim \mathcal{N}$ است
            \strut}
            \State اعمال $(a_1, a_2)$ را در محیط اجرا کنید
            \State \parbox[t]{\dimexpr\linewidth-\algorithmicindent}{
            مشاهدات بعدی $(o_1', o_2')$، پاداش‌ها $(r_1, r_2=-r_1)$ و سیگنال پایان $d$ را دریافت کنید
            \strut}
            \State تجربه $(o_1, o_2, a_1, a_2, r_1, r_2, o_1', o_2', d)$ را در بافر $\mathcal{D}$ ذخیره کنید
            \State اگر $d=1$ است، وضعیت محیط را بازنشانی کنید
            
            \If{زمان به‌روزرسانی فرا رسیده است}
                \For{هر تعداد به‌روزرسانی}
                    \State یک دسته تصادفی از تجربیات، $B = \{(\boldsymbol{o}, \boldsymbol{a}, r_1, r_2, \boldsymbol{o}', d)\}$، از $\mathcal{D}$ نمونه‌گیری کنید
                    \State \parbox[t]{\dimexpr\linewidth-\algorithmicindent}{
                    اهداف را محاسبه کنید:
                    \begin{align*}
                        y_1 &= r_1 + \gamma (1-d) Q_{\phi_{1,\text{targ}}}(o_1', \mu_{\theta_{1,\text{targ}}}(o_1'), \mu_{\theta_{2,\text{targ}}}(o_2')) \\
                        y_2 &= r_2 + \gamma (1-d) Q_{\phi_{2,\text{targ}}}(o_2', \mu_{\theta_{2,\text{targ}}}(o_2'), \mu_{\theta_{1,\text{targ}}}(o_1'))
                    \end{align*}
                    \strut}
                    \State \parbox[t]{\dimexpr\linewidth-\algorithmicindent}{
                    توابع \lr{Q} را با نزول گرادیان به‌روزرسانی کنید:
                    \begin{align*}
                        \nabla_{\phi_1} \frac{1}{|B|}\sum_{(\boldsymbol{o}, \boldsymbol{a}, r_1, r_2, \boldsymbol{o}', d) \in B} \left( Q_{\phi_1}(o_1, a_1, a_2) - y_1 \right)^2 \\
                        \nabla_{\phi_2} \frac{1}{|B|}\sum_{(\boldsymbol{o}, \boldsymbol{a}, r_1, r_2, \boldsymbol{o}', d) \in B} \left( Q_{\phi_2}(o_2, a_2, a_1) - y_2 \right)^2
                    \end{align*}
                    \strut}
                    
                    \State \parbox[t]{\dimexpr\linewidth-\algorithmicindent}{
                    سیاست‌ها را با صعود گرادیان به‌روزرسانی کنید:
                    \begin{align*}
                        \nabla_{\theta_1} \frac{1}{|B|}\sum_{\boldsymbol{o} \in B}Q_{\phi_1}(o_1, \mu_{\theta_1}(o_1), a_2) \\
                        \nabla_{\theta_2} \frac{1}{|B|}\sum_{\boldsymbol{o} \in B}Q_{\phi_2}(o_2, \mu_{\theta_2}(o_2), a_1)
                    \end{align*}
                    \strut}
                    
                    \State \parbox[t]{\dimexpr\linewidth-\algorithmicindent}{
                    شبکه‌های هدف را به‌روزرسانی کنید:
                    \begin{align*}
                        \phi_{1,\text{targ}} &\leftarrow \rho \phi_{1,\text{targ}} + (1-\rho) \phi_1 \\
                        \phi_{2,\text{targ}} &\leftarrow \rho \phi_{2,\text{targ}} + (1-\rho) \phi_2 \\
                        \theta_{1,\text{targ}} &\leftarrow \rho \theta_{1,\text{targ}} + (1-\rho) \theta_1 \\
                        \theta_{2,\text{targ}} &\leftarrow \rho \theta_{2,\text{targ}} + (1-\rho) \theta_2
                    \end{align*}
                    \strut}
                \EndFor
            \EndIf
        \EndWhile
    \end{algorithmic}
\end{algorithm}

\subsection{مزایای \lr{MADDPG} در بازی‌های مجموع­‌صفر}

\lr{MADDPG} چندین مزیت برای یادگیری در بازی‌های دو­عاملیِ مجموع­‌صفر ارائه می‌دهد:

\begin{itemize}
    \item \textbf{مقابله با غیرایستایی:} با استفاده از منتقدهایی که به اطلاعات کامل دسترسی دارند، مشکل غیرایستایی محیط از دید هر عامل حل می‌شود.
    \item \textbf{همگرایی بهتر:} در بازی‌های مجموع­‌صفر، \lr{MADDPG} معمولاً همگرایی بهتری نسبت به آموزش مستقل عامل‌ها با \lr{DDPG} نشان می‌دهد.
    \item \textbf{یادگیری استراتژی‌های متقابل:} عامل‌ها می‌توانند استراتژی‌های متقابل پیچیده را یاد بگیرند که در آموزش مستقل امکان‌پذیر نیست.
\end{itemize}

در بازی‌های دو­عاملیِ مجموع­‌صفر، این رویکرد به رقابت کامل بین عامل‌ها منجر می‌شود، که هر یک تلاش می‌کند بهترین استراتژی را در برابر استراتژی رقیب پیدا کند.
%    \subsection{ایمنی و مقاومت در یادگیری تقویتی چندعاملی}

در استفاده از یادگیری تقویتی چندعاملی (\lr{Multi-Agent Reinforcement Learning - \lr{MARL}})، مسائل مربوط به ایمنی و مقاومت در برابر اختلالات یکی از چالش‌های اساسی مطرح می‌گردد. به منظور اطمینان از عملکرد قابل اعتماد و ایمن الگوریتم‌های یادگیری تقویتی چندعاملی، نیازمند توسعه روش‌هایی هستیم که بتوانند در مواجهه با رفتارهای غیرمنتظره یا مخرب سایر عوامل، پایداری و ایمنی سیستم را حفظ نمایند. در این بخش، به بررسی مفاهیم ایمنی و مقاومت در \lr{MARL} پرداخته شده و چگونگی افزایش مقاومت الگوریتم‌ها از طریق در نظر گرفتن عوامل به عنوان اختلالات مورد بحث قرار گرفته است.

ایمنی در \lr{MARL} به معنای تضمین این است که تعاملات میان عوامل منجر به نتایج نامطلوب یا خطرناک نشوند. برای دستیابی به ایمنی، روش‌هایی نظیر محدود کردن فضای عملیاتی، اعمال قیود بر سیاست‌های یادگیری و استفاده از الگوریتم‌های مقاوم در برابر خطا به کار گرفته شده‌اند. یکی از رویکردهای موثر در افزایش مقاومت سیستم، فرض کردن یکی از عوامل به عنوان اختلال (\lr{Disturbance}) در محیط است. با این فرض، الگوریتم‌ها قادر خواهند بود تا به گونه‌ای طراحی شوند که در حضور اختلالات احتمالی، عملکرد سیستم همچنان قابل اعتماد باقی بماند.

\subsubsection{فرض کردن اختلال به عنوان عامل}

در محیط‌های چندعاملی، برخی از عوامل ممکن است رفتارهای مخرب یا غیرمنتظره‌ای را از خود نشان دهند که می‌تواند به عملکرد کلی سیستم آسیب برساند. برای مقابله با این مسئله، فرض می‌شود که یک یا چند عامل به عنوان اختلالات در نظر گرفته شوند. این اختلالات می‌توانند به صورت عمدی یا غیرعمدی ایجاد شوند و هدف آن‌ها کاهش کارایی سیستم است. با فرض کردن این اختلالات، الگوریتم‌های \lr{MARL} قادر خواهند بود تا سیاست‌هایی را یاد بگیرند که در مواجهه با این اختلالات نیز عملکرد بهینه و ایمنی را حفظ کنند.

\subsubsection{تعریف مقاومت و ایمنی در \lr{MARL}}

مقاومت در \lr{MARL} به معنای توانایی الگوریتم در حفظ عملکرد مطلوب در حضور اختلالات و تغییرات محیطی است. این مقاومت می‌تواند از طریق طراحی سیاست‌های بهینه که به گونه‌ای تنظیم شده‌اند که تأثیر اختلالات را به حداقل برسانند، به دست آید. به علاوه، ایمنی می‌تواند از طریق تضمین عدم وقوع رفتارهای خطرناک و حفظ تعادل سیستم در مواجهه با رفتارهای مخرب حاصل شود.

\paragraph{تعریف ریاضی مقاومت}

فرض کنید یک محیط چندعاملی با مجموعه‌ای از عوامل \( \mathcal{A} = \{A_1, A_2, \dots, A_n\} \) وجود دارد که در آن یک عامل \( A_d \) به عنوان اختلال تعریف شده است. هدف این است که الگوریتم \lr{MARL} به گونه‌ای طراحی شود که سیاست‌های یادگرفته شده \( \pi = \{\pi_1, \pi_2, \dots, \pi_n\} \) بتوانند عملکرد بهینه را حتی در حضور \( A_d \) حفظ کنند. به طور ریاضی، مقاومت به صورت زیر تعریف می‌شود:
\[
\forall A_i \in \mathcal{A}, \quad \text{اگر } A_d \text{ رفتار مخرب نشان دهد، } \pi_i \text{ باید همچنان به حداکثر رساندن پاداش خود ادامه دهد.}
\]

\paragraph{تعریف ریاضی ایمنی}

ایمنی در \lr{MARL} به معنای اطمینان از این است که سیستم در هیچ حالت خطرناکی وارد نمی‌شود. به طور ریاضی، ایمنی می‌تواند به صورت مجموعه‌ای از قیود تعریف شود که سیاست‌های یادگرفته شده باید آن‌ها را رعایت کنند:
\[
\forall A_i \in \mathcal{A}, \quad \text{ایمنی: } u_i(s_i, s_{-i}) \geq \theta_i \quad \text{برای همه } s_i \in S_i, \ s_{-i} \in S_{-i}
\]
که در آن \( \theta_i \) آستانه‌ای است که برای هر عامل \( A_i \) تعیین شده و نشان‌دهنده حداقل پاداش قابل قبول است.

\subsubsection{روش‌های افزایش مقاومت و ایمنی}

برای افزایش مقاومت و ایمنی در \lr{MARL}، روش‌های متعددی مورد استفاده قرار گرفته‌اند که در زیر به برخی از آن‌ها پرداخته می‌شود:

\begin{itemize}
	\item \textbf{الگوریتم‌های مقاوم در برابر اختلالات:} این الگوریتم‌ها به گونه‌ای طراحی شده‌اند که بتوانند به سرعت با تغییرات محیطی و حضور اختلالات سازگار شوند. به عنوان مثال، الگوریتم‌های مبتنی بر یادگیری تطبیقی که قادر به تغییر سیاست‌های خود در پاسخ به تغییرات محیط هستند.
	
	\item \textbf{فریم‌ورک‌های ایمنی:} چارچوب‌هایی برای تضمین ایمنی در تعاملات چندعاملی طراحی شده‌اند که شامل محدود کردن فضای عملیاتی و اعمال قیود بر سیاست‌های یادگیری است. این فریم‌ورک‌ها معمولاً شامل روش‌هایی برای نظارت و تنظیم رفتار عامل‌ها به منظور جلوگیری از وقوع رفتارهای خطرناک هستند.
	
	\item \textbf{آموزش در حضور اختلالات:} با آموزش الگوریتم‌ها در محیط‌هایی که شامل اختلالات هستند، می‌توان مقاومت الگوریتم‌ها را افزایش داد. این روش به الگوریتم اجازه می‌دهد تا در مواجهه با اختلالات غیرمنتظره، سیاست‌های مقاومتی یاد بگیرد.
	
	\item \textbf{استفاده از اصول نظریه بازی‌ها:} با بهره‌گیری از تعادل‌های نظریه بازی‌ها مانند تعادل نش (\lr{Nash Equilibrium}), می‌توان سیاست‌هایی طراحی کرد که در مواجهه با استراتژی‌های متغیر سایر عوامل، پایداری و ایمنی سیستم حفظ شود.
\end{itemize}

\subsubsection{نمونه‌های کاربردی}

برای نشان دادن کاربردهای عملی \lr{MARL} در افزایش ایمنی و مقاومت، به چند مثال اشاره می‌شود:

\paragraph{سامانه‌های خودران:} در خودروهای خودران چندعاملی، ایمنی یکی از اولویت‌های اصلی است. با استفاده از \lr{MARL} و فرض کردن سایر خودروها به عنوان عوامل یا اختلالات، می‌توان الگوریتم‌هایی توسعه داد که در مواجهه با رفتارهای غیرمنتظره سایر خودروها، ایمن باقی بمانند.

\paragraph{مدیریت انرژی در شبکه‌های هوشمند:} در شبکه‌های انرژی هوشمند، \lr{MARL} می‌تواند برای مدیریت بهینه انرژی در حضور اختلالات مانند خرابی‌ها یا حملات سایبری استفاده شود. الگوریتم‌های مقاوم می‌توانند با تغییرات ناگهانی در تقاضا یا عرضه انرژی سازگار شده و پایداری شبکه را حفظ کنند.

\paragraph{ربات‌های همکاری‌کننده:} در سیستم‌های رباتیک همکاری‌کننده، اطمینان از ایمنی تعاملات میان ربات‌ها حیاتی است. \lr{MARL} می‌تواند برای طراحی سیاست‌های ایمن که در مواجهه با رفتارهای مخرب یا اختلالات داخلی، سیستم را پایدار نگه دارند، به کار رود.

\paragraph{محیط‌های صنعتی:} در محیط‌های صنعتی که شامل چندین عامل نظیر ربات‌ها و ماشین‌آلات است، ایمنی و مقاومت سیستم‌ها از اهمیت بالایی برخوردار است. با استفاده از \lr{MARL}، می‌توان الگوریتم‌هایی توسعه داد که در مواجهه با اختلالات مانند خرابی تجهیزات یا خطاهای انسانی، عملکرد سیستم را حفظ کنند.

\subsubsection{چالش‌ها و فرصت‌ها}

با وجود مزایای متعدد \lr{MARL} در افزایش ایمنی و مقاومت سیستم‌ها، چالش‌هایی نیز در این زمینه وجود دارد:

\begin{itemize}
	\item \textbf{پیچیدگی مدل‌سازی اختلالات:} مدل‌سازی دقیق اختلالات و رفتارهای مخرب می‌تواند پیچیده و زمان‌بر باشد، به ویژه در محیط‌های پیچیده و پویا.
	
	\item \textbf{هماهنگی میان عوامل:} هماهنگی موثر میان عوامل در حضور اختلالات نیازمند الگوریتم‌های پیچیده و کارآمد است که بتوانند به سرعت و بهینه با تغییرات محیطی سازگار شوند.
	
	\item \textbf{تعادل میان کارایی و ایمنی:} حفظ تعادل میان بهینه‌سازی کارایی سیستم و تضمین ایمنی در مواجهه با اختلالات یک چالش بزرگ است که نیازمند طراحی دقیق الگوریتم‌ها است.
	
	\item \textbf{نیاز به داده‌های بزرگ:} برای آموزش الگوریتم‌های مقاوم و ایمن، نیاز به مجموعه‌های داده بزرگ و متنوعی است که شامل انواع اختلالات و رفتارهای مخرب باشند.
\end{itemize}

با این حال، فرصت‌های بزرگی نیز در این زمینه وجود دارد. توسعه الگوریتم‌های پیشرفته \lr{MARL} که بتوانند به طور موثر با اختلالات مواجه شوند، می‌تواند منجر به سیستم‌های هوشمند و خودکار با کارایی و ایمنی بالاتر شود. همچنین، پژوهش‌های ادامه‌دار در زمینه نظریه بازی‌ها و روش‌های مقاوم‌سازی الگوریتم‌های یادگیری تقویتی، پتانسیل بالایی برای بهبود \lr{MARL} و کاربردهای آن فراهم می‌آورد.

\subsubsection{نتیجه‌گیری}

در نهایت، ایمنی و مقاومت در یادگیری تقویتی چندعاملی به عنوان عوامل کلیدی در توسعه سیستم‌های هوشمند و خودکار مطرح می‌گردند. با فرض کردن اختلالات و توسعه الگوریتم‌های مقاوم، می‌توان عملکرد سیستم‌های \lr{MARL} را در مواجهه با چالش‌های مختلف بهبود بخشید. با توجه به اهمیت و کاربردهای گسترده \lr{MARL} در حوزه‌های علمی و صنعتی، تحقیقات در زمینه افزایش ایمنی و مقاومت این الگوریتم‌ها همچنان ادامه خواهد یافت تا به سیستم‌هایی با قابلیت‌های بیشتر و عملکردی پایدارتر دست یابند.

%    \subsection{الگوریتم‌های یادگیری تقویتی چندعاملی}

    
\section{ساختار گزارش}\label{sec:chapter_content}
در \textbf{فصل دوم} مروری انتقادی بر کارهای مرتبط در هدایت پیشران‌کم و یادگیری تقویتی (تک‌عاملی و چندعاملی) ارائه می‌شود. \textbf{فصل سوم} به مبانی یادگیری تقویتی اختصاص داده شده و الگوریتم‌های \lr{DDPG}، \lr{TD3}، \lr{SAC} و \lr{PPO} مرور می‌شوند. در \textbf{فصل چهارم} چارچوب یادگیری تقویتی چندعاملی و رویکرد \lr{CTDE} تشریح می‌شود و پیوند آن با بازی‌های جمع‌صفر و تعادل نش بیان می‌گردد. در \textbf{فصل پنجم} مدل‌سازی محیط آزمایش بر پایه‌ی \lr{CRTBP} ارائه می‌شود. در \textbf{فصل ششم} طراحی عامل‌ها، فضای حالت/عمل، تابع پاداش و جزئیات آموزش توضیح داده می‌شود. در \textbf{فصل هفتم} چارچوب «سخت‌افزار در حلقه» و ارزیابی زمان‌واقعی گزارش می‌شود. سرانجام، در \textbf{فصل هشتم} نتایج، مقایسه با معیارهای مرجع و مسیرهای آینده‌ی پژوهش جمع‌بندی می‌شود.

