حتماً. معادلات کامل لاگرانژ برای سیستم سه‌جسمی محدود دایره‌ای را به‌صورت دقیق در قالب لاتک آماده می‌کنم و توضیح می‌دهم که این معادلات از کجا به‌دست می‌آیند، چگونه با استفاده از آن‌ها نقاط تعادل استخراج می‌شوند، و در نهایت یک جدول شامل مختصات عددی نقاط لاگرانژی (L1 تا L5) برای سامانه زمین-ماه ارائه می‌دهم.

\section{} معادلات لاگرانژ در مسأله‌ی سه‌جسمی محدود دایره‌ای

\section{} تعریف مسئله و چارچوب مرجع چرخان  
در **مسئله‌ی سه‌جسمی محدود دایره‌ای (CRTBP)** دو جرم بزرگ (مثلاً زمین و ماه) در حال گردش دایره‌ای حول مرکز جرم مشترک هستند و جرم سوم که بسیار کوچک و تقریباً بی‌اثر است، تحت تاثیر گرانش آن دو حرکت می‌کند. فرض می‌شود جرم سوم آن‌قدر کوچک است که تاثیری بر حرکت دو جرم اصلی ندارد. برای تحلیل این مسئله، یک دستگاه مختصات چرخان **هم‌دوران** با دو جرم اصلی انتخاب می‌کنیم به‌طوری‌که مرکز مختصات در مرکز جرم سیستم باشد و محور $x$ خط واصل دو جرم و محور $y$ عمود بر آن در صفحه‌ی حرکت باشد (صفحه‌ی مداری دو جرم اصلی). سرعت زاویه‌ای چرخش این دستگاه برابر سرعت مداری دو جرم اصلی (فرضاً $\omega=1$ در دستگاه بی‌بعد) در نظر گرفته می‌شود. واحد طول برابر با فاصله‌ی بین دو جرم اصلی و واحد زمان چنان انتخاب می‌شود که دوره‌ی مداری دو جرم اصلی $2\pi$ (و در نتیجه $\omega=1$) شود. همچنین مجموع جرم‌های دو جرم اصلی برابر ۱ در نظر گرفته می‌شود. بر این اساس اگر $\mu$ نسبت جرمی جرم دوم (کوچک‌تر) باشد ($\mu=\dfrac{m_2}{m_1+m_2}$)، آن‌گاه جرم اول $m_1=1-\mu$ و جرم دوم $m_2=\mu$ خواهد بود. در مبدأ مختصات چرخان (مرکز جرم) موقعیت جرم اول روی محور $x$ برابر $(-\mu,\,0)$ و جرم دوم برابر $(1-\mu,\,0)$ می‌شود.  

در این چارچوب مرجع چرخان، جرم سوم یک **لاگرانژی** (تابع لاگرانژ) دارد که شامل انرژی جنبشی آن و پتانسیل گرانشی دو جرم اصلی به‌همراه پتانسیل موثر مرکزگرا (ناشی از چارچوب غیرلخت) است. با نرمال‌سازی ثابت گرانش به $G=1$، می‌توان **لاگرانژی** سیستم را به صورت زیر نوشت:

$$ 
L \;=\; T - V \;=\; \dfrac{1}{2}\Big(\dot{x}^2+\dot{y}^2+\dot{z}^2\Big)\;-\;\Bigg[-\dfrac{1-\mu}{r_1}\;-\;\dfrac{\mu}{r_2}\;-\;\dfrac{1}{2}(x^2+y^2)\Bigg]~, 
$$

که در آن $T$ انرژی جنبشی ذره‌ی کوچک و $V$ پتانسیل آن است. عبارت داخل کروشه در واقع پتانسیل گرانشی دو جرم اصلی به‌همراه پتانسیل گریز از مرکز (با علامت منفی) است: به‌ترتیب $-\dfrac{1-\mu}{r_1}$ و $-\dfrac{\mu}{r_2}$ پتانسیل گرانشی جرم‌های اول و دوم در نقطه‌ی $(x,y,z)$ ذره‌ی کوچک (با فواصل $r_1$ و $r_2$ تا آن دو جرم)، و $-\dfrac{1}{2}(x^2+y^2)$ پتانسیل سانتریفیوژ معادل در دستگاه چرخان است. بنابراین تابع لاگرانژ را می‌توان به صورت جمع انرژی جنبشی و «پتانسیل موثر» زیر نوشت:

$$
L \;=\; \dfrac{1}{2}\Big(\dot{x}^2+\dot{y}^2+\dot{z}^2\Big)\;+\;(1-\mu)\dfrac{1}{r_1}\;+\;\mu\dfrac{1}{r_2}\;+\;\dfrac{1}{2}(x^2+y^2)~,
$$

که در آن $r_1=\sqrt{(x+\mu)^2+y^2+z^2}$ فاصله‌ی جرم سوم از جرم اول و $r_2=\sqrt{(x-1+\mu)^2+y^2+z^2}$ فاصله از جرم دوم است. اکنون با به‌کارگیری اصل کمینه‌کردن کنش ($\delta S=0$) و نوشتن معادلات اویلر-لاگرانژ برای مختصات $x, y, z$، معادلات حرکت در این دستگاه به‌دست می‌آیند. شرط اویلر-لاگرانژ برای هر مختصه (مثلاً $x$) به شکل $\dfrac{d}{dt}\dfrac{\partial L}{\partial \dot{x}} - \dfrac{\partial L}{\partial x}=0$ نوشته می‌شود. مشتقات لازم را محاسبه می‌کنیم:

- $\displaystyle \dfrac{\partial L}{\partial \dot{x}} = \dot{x}$، پس $\dfrac{d}{dt}\dfrac{\partial L}{\partial \dot{x}} = \ddot{x}$،  
- $\displaystyle \dfrac{\partial L}{\partial x} = (1-\mu)\,\dfrac{\partial}{\partial x}\dfrac{1}{r_1} + \mu\,\dfrac{\partial}{\partial x}\dfrac{1}{r_2} + \dfrac{\partial}{\partial x}\Big(\dfrac{1}{2}x^2\Big)$ 

با توجه به $r_1=\sqrt{(x+\mu)^2+y^2+z^2}$ و $r_2=\sqrt{(x-1+\mu)^2+y^2+z^2}$، مشتق‌های جزئی پتانسیل‌ها عبارتند از:

$$ 
\dfrac{\partial}{\partial x}\dfrac{1}{r_1} = -\dfrac{x+\mu}{r_1^3}~, \qquad 
\dfrac{\partial}{\partial x}\dfrac{1}{r_2} = -\dfrac{x-(1-\mu)}{r_2^3}~, \qquad 
\dfrac{\partial}{\partial x}\Big(\dfrac{1}{2}x^2\Big) = x~. 
$$

بنابراین:

$$ 
\dfrac{\partial L}{\partial x} = -(1-\mu)\dfrac{x+\mu}{r_1^3} \;-\; \mu\,\dfrac{x-(1-\mu)}{r_2^3} \;+\; x~. 
$$

با قرار دادن این در معادله‌ی اویلر-لاگرانژ $\ddot{x} - \dfrac{\partial L}{\partial x}=0$، معادله‌ی $x$ به‌دست می‌آید. به‌طور مشابه برای $y$ و $z$ نیز عمل می‌کنیم. بدین‌ترتیب، **معادلات لاگرانژ (معادلات حرکت)** در دستگاه چرخان برای ذره‌ی سوم به صورت زیر حاصل می‌شوند:

$$ 
\begin{cases}
	\displaystyle \ddot{x} - 2\,\dot{y} \;=\; x \;-\; \dfrac{1-\mu}{r_1^3}(x+\mu) \;-\; \dfrac{\mu}{r_2^3}\Big(x-(1-\mu)\Big)~,  \\[2ex]
	\displaystyle \ddot{y} + 2\,\dot{x} \;=\; y \;-\; \dfrac{1-\mu}{r_1^3}\,y \;-\; \dfrac{\mu}{r_2^3}\,y~,  \\[2ex]
	\displaystyle \ddot{z} \;=\; -\,\dfrac{1-\mu}{r_1^3}\,z \;-\; \dfrac{\mu}{r_2^3}\,z~, 
\end{cases}
$$

که در آن $r_1=\sqrt{(x+\mu)^2+y^2+z^2}$ و $r_2=\sqrt{\big(x-(1-\mu)\big)^2+y^2+z^2}$ همان‌گونه که تعریف شد. این معادلات لاگرانژ حاصل اصل کمینه‌سازی کنش هستند و معادلات کامل حرکت جرم سوم (با جرم ناچیز) را تحت گرانش دو جرم اولیه در حالت دایره‌ای بیان می‌کنند. توجه شود که وجود ترم‌های کورولیس ($2\,\dot{y}$ در معادله $x$ و $2\,\dot{x}$ در معادله $y$) نتیجه‌ی استفاده از دستگاه مرجع چرخان است. همچنین می‌توان این معادلات را به شکل برداری $ \ddot{\mathbf{r}} + 2\,\mathbf{\omega}\times \dot{\mathbf{r}} = \nabla \Omega(\mathbf{r})$ نیز بیان کرد که در آن $\Omega(x,y,z) = \dfrac{1}{2}(x^2+y^2) + \dfrac{1-\mu}{r_1} + \dfrac{\mu}{r_2}$ پتانسیل موثر در دستگاه چرخان است و گرادیان آن نیروهای موثر (گرانشی و گریز از مرکز) را نتیجه می‌دهد.

\section{} نقاط تعادل (نقاط لاگرانژ $L_1$ تا $L_5$)  
منظور از نقطه‌ی تعادل، حالتی است که جرم سوم (ذره‌ی آزمون) در دستگاه مرجع چرخان نسبت به دو جرم اصلی **ساکن** بماند. این شرایط زمانی رخ می‌دهد که سرعت و شتاب جرم سوم در دستگاه چرخان صفر شود. به‌عبارت دیگر در معادلات بالا باید $\dot{x}=\dot{y}=\dot{z}=0$ و $\ddot{x}=\ddot{y}=\ddot{z}=0$ قرار دهیم. با اعمال این شرط به معادلات لاگرانژ، مجموعه‌ای از معادلات جبری به‌دست می‌آید که مختصات نقاط تعادل (موسوم به **نقاط لاگرانژ**) را تعیین می‌کند. با قراردادن $\dot{x}=\dot{y}=0$ و $\ddot{x}=\ddot{y}=0$ در معادلات داریم:

$$ 
\begin{cases}
	\displaystyle 0 \;=\; x \;-\; \dfrac{1-\mu}{r_1^3}(x+\mu) \;-\; \dfrac{\mu}{r_2^3}\Big(x-(1-\mu)\Big)~,  \\[2ex]
	\displaystyle 0 \;=\; y \;-\; \dfrac{1-\mu}{r_1^3}\,y \;-\; \dfrac{\mu}{r_2^3}\,y~,  \\[1ex]
	0 \;=\; -\,\dfrac{1-\mu}{r_1^3}\,z \;-\; \dfrac{\mu}{r_2^3}\,z~. 
\end{cases}
$$

معادله‌ی سوم در بالا نشان می‌دهد یا باید $z=0$ باشد (نقاط تعادل همگی در صفحه‌ی حرکت هستند) یا عبارت داخل آن صفر شود که برای $z\neq0$ تنها در حالت خاصی مانند $\mu=0$ امکان‌پذیر است. بنابراین برای یافتن نقاط تعادل، حرکت جرم سوم را در همان صفحه‌ی مداری در نظر می‌گیریم ($z=0$). در این صورت $r_1$ و $r_2$ در صفحه به ترتیب $\sqrt{(x+\mu)^2+y^2}$ و $\sqrt{\big(x-(1-\mu)\big)^2+y^2}$ خواهند بود. دو معادله‌ی اول را می‌توان به صورت ساده‌تری نوشت:

$$ 
\begin{cases}
	\displaystyle x - \dfrac{1-\mu}{r_1^3}(x+\mu) - \dfrac{\mu}{r_2^3}\Big(x-(1-\mu)\Big) = 0~,  \\[2ex]
	\displaystyle y\,\Big[\,1 - \dfrac{1-\mu}{r_1^3} - \dfrac{\mu}{r_2^3}\Big] = 0~. 
\end{cases}
$$

از معادله‌ی دوم نتیجه می‌شود که یا $y=0$ (نقطه‌ی تعادل روی محور $x$ واقع است) و یا پرانتز دوم صفر باشد (که منجر به رابطه‌ای بین فواصل می‌شود). بر این اساس، نقاط تعادل به دو دسته‌ی کلی تقسیم می‌شوند:

- **نقاط هم‌خط (Collinear)**: این سه نقطه روی خط واصل دو جرم اصلی (محور $x$) واقع شده و بنابراین $y=0$ دارند. این نقاط را به ترتیب $L_1$، $L_2$ و $L_3$ می‌نامند.  
- **نقاط سه‌گوش (Triangular)**: این دو نقطه در صفحه به‌صورت رئوس مثلث متساوی‌الاضلاع با دو جرم اصلی قرار می‌گیرند و دارای $y\neq0$ هستند. این دو نقطه $L_4$ و $L_5$ نام دارند.  

در ادامه، هر دسته را جداگانه بررسی می‌کنیم.

\section{} نقاط لاگرانژ هم‌خط ($L_1$, $L_2$, $L_3$)  
برای نقاط واقع بر محور $x$، شرط $y=0$ را در معادلات تعادل اعمال می‌کنیم. آنگاه معادله‌ی دوم به‌طور خودکار ارضا می‌شود (زیرا $y=0$ آن را صفر می‌کند) و تنها معادله‌ی اول باقی می‌ماند:

$$ 
x - \dfrac{1-\mu}{|x+\mu|^3}(x+\mu) - \dfrac{\mu}{|x-(1-\mu)|^3}\Big(x-(1-\mu)\Big) = 0~,
$$

که در آن به علت قدرمطلق، بسته به ناحیه‌ی قرارگیری $x$، علامت عبارت‌ها مشخص می‌شود. خوشبختانه می‌توان سه ناحیه‌ی متمایز را در نظر گرفت که متناظر با سه ریشه‌ی این معادله هستند:  

- **$L_1$:** بین دو جرم اصلی واقع است. در این حالت $x$ بین $-\mu$ و $1-\mu$ قرار دارد (بین موقعیت‌های جرم اول و جرم دوم). برای $L_1$ فاصله‌ی آن از جرم دوم را با $d_1$ نشان می‌دهیم (پس $x = (1-\mu) - d_1$). این فاصله را می‌توان با حل معادله‌ی بالا به‌دست آورد. معادله‌ی تعادل در این ناحیه را می‌توان پس از ساده‌سازی به صورت یک معادله‌ی درجه پنج (در $x$ یا $d_1$) نوشت که جواب تحلیلی ساده‌ای ندارد و معمولاً به روش عددی (مثلاً روش نیوتن) حل می‌شود. با این وجود، برای $\mu$های کوچک (مثلاً سیستمی مانند خورشید-زمین یا زمین-ماه) می‌توان تخمین خوبی به‌دست آورد. اگر $\mu \ll 1$ باشد (جرم دوم بسیار کوچک‌تر از جرم اول)، آنگاه $L_1$ بسیار نزدیک به جرم دوم خواهد بود و فاصله‌ی آن از جرم دوم (در واحد فاصله‌ی دو جرم اصلی) تقریباً برابر **شعاع کره‌ی هیل** جرم دوم است. شعاع هیل تقریباً $r_h \approx (\mu/3)^{1/3}$ است. بنابراین می‌توان نوشت: 

$$d_1 \approx \Big(\dfrac{\mu}{3}\Big)^{1/3},$$ 

و در نتیجه مختصات $L_1$ تقریباً برابر است با: 

$$x_{L_1} \approx (1-\mu) - (\mu/3)^{1/3}, \qquad y_{L_1}=0.$$ 

(توجه شود که برای دقت بیشتر، در صورت نیاز باید اثر $\mu$ را در قسمت $(1-\mu)$ نیز لحاظ کرد.) این فرمول یک تخمین از موقعیت $L_1$ است. در عمل حل عددی دقیق معادله، مقدار دقیق‌تری برای $x_{L_1}$ به‌دست می‌دهد. این نقطه جایی است که نیروی گرانش دو جرم (یکی کشنده به چپ و دیگری به راست) و نیروی گریز از مرکز دقیقا همدیگر را خنثی می‌کنند. 

- **$L_2$:** بیرون جرم دوم (کوچک) و در امتداد همان محور قرار دارد. این نقطه در سمت راست جرم دوم واقع است ($x > 1-\mu$) و جرم دوم بین آن و جرم اول قرار می‌گیرد. اگر فاصله‌ی $L_2$ از جرم دوم را $d_2$ بگیریم ($x = (1-\mu) + d_2$)، معادله‌ی تعادل در این ناحیه نیز پس از ساده‌سازی یک معادله‌ی درجه پنج برای $d_2$ خواهد بود که باید عددی حل شود. برای $\mu$ کوچک (جرم دوم خیلی کوچک‌تر)، این فاصله نیز تقریباً برابر $(\mu/3)^{1/3}$ به‌دست می‌آید. یعنی: 

$$d_2 \approx \Big(\dfrac{\mu}{3}\Big)^{1/3},$$ 

و بنابراین: 

$$x_{L_2} \approx (1-\mu) + (\mu/3)^{1/3}, \qquad y_{L_2}=0.$$ 

این نقطه در بیرون مدار جرم دوم قرار دارد و تعادل نیروها در آن به این صورت است که نیروی گریز از مرکز به سمت بیرون توسط مجموع نیروی گرانشی دو جرم (هر دو به سمت داخل) بالانس می‌شود. 

- **$L_3$:** بیرون جرم اول (بزرگ) و در امتداد خط محور $x$ در سمت مقابل جرم دوم واقع است ($x < -\mu$). این نقطه در سوی دیگر جرم بزرگ‌تر قرار دارد به‌طوری که جرم اول بین $L_3$ و جرم دوم است. برای $L_3$ نیز معادله‌ی تعادل به روش عددی حل می‌شود؛ و در حد $\mu \ll 1$ (جرم دوم بسیار کوچک)، موقعیت آن اندکی فراتر از مدار جرم اول (جرم بزرگ) است. می‌توان نشان داد برای $\mu$های بسیار کوچک: 

$$x_{L_3} \approx -\,1 - \dfrac{5}{12}\,\mu, \qquad y_{L_3}=0,$$ 

که نشان می‌دهد $L_3$ تقریباً به اندازه‌ی فاصله‌ی دو جرم اصلی دورتر از جرم اول است (حد $\mu \to 0$ منجر به $x=-1$ می‌شود که درست در سمت مقابل جرم دوم و هم‌فاصله با آن است) و با در نظر گرفتن جرم کوچک دوم، اندکی بیش از آن (چند صدم درصد بسته به مقدار $\mu$) فاصله می‌گیرد. به بیان دیگر، گرانش جرم دوم (هرچند کوچک) باعث می‌شود برای حفظ تعادل، جرم سومِ واقع در $L_3$ کمی به جرم بزرگ‌تر نزدیک‌تر شود تا نیروی گریز از مرکز کمتر شده و با کاهش کمی در فاصله، گرانش جرم بزرگ افزایش یابد و تعادل برقرار گردد.  

معمولاً معادله‌ی دقیق مربوط به $L_1$, $L_2$ و $L_3$ را به صورت یک چندجمله‌ای درجه ۵ نسبت به متغیری کمکی بیان می‌کنند (که به **معادله‌ی کوئینتیک لاگرانژ** مشهور است) و سپس آن را با تقریب یا روش‌های عددی حل می‌نمایند. به عنوان مثال، در مراجع کلاسیک نشان داده می‌شود که اگر $\rho = x + \mu$ فاصله‌ی مرکز جرم تا نقطه‌ی تعادل در یک جهت در نظر گرفته شود، معادله‌ی درجه پنج در $\rho$ قابل بیان است. اما در کاربردهای عملی، استفاده از روش‌های عددی سریع‌تر به جواب منجر می‌شود. 

\section{} نقاط لاگرانژ سه‌گوش ($L_4$, $L_5$)  
دو نقطه‌ی تعادل دیگر در صفحه‌ی مدار به‌صورتی قرار می‌گیرند که همراه با دو جرم اصلی تشکیل یک مثلث متساوی‌الاضلاع بدهند. در چنین آرایشی، جرم سوم می‌تواند در حالت تعادل نسبی (هم‌دوران با دو جرم دیگر) باقی بماند. برای این نقاط، واضح است که $y \neq 0$ خواهد بود. بنابراین در معادلات تعادل باید قسمت داخل کروشه (که در معادله‌ی تعادل $y$ ظاهر شد) صفر گردد:

$$1 - \dfrac{1-\mu}{r_1^3} - \dfrac{\mu}{r_2^3} = 0.$$

این رابطه همراه با معادله‌ی تعادل در $x$ باید همزمان ارضا شوند. از آنجا که مثلث متساوی‌الاضلاع است، فاصله‌ی جرم سوم از هر دو جرم اصلی برابر است ($r_1 = r_2$). با توجه به واحد طول نرمال‌شده (فاصله‌ی بین دو جرم اصلی برابر ۱)، این فاصله باید برابر با ۱ (طول ضلع مثلث) باشد. لذا $r_1 = r_2 = 1$. در این صورت، رابطه‌ی فوق به سادگی $1 - (1-\mu) - \mu = 0$ خواهد بود که برقرار است. با استفاده از هندسه‌ی مثلث متساوی‌الاضلاع در دستگاه مختصات تعریف‌شده می‌توان مختصات دقیق $L_4$ و $L_5$ را به‌دست آورد. اگر ضلع پایینی مثلث روی محور $x$ باشد (جرم‌ها روی محور $x$ قرار دارند) و جرم سوم رأس مثلث باشد، آنگاه مختصات آن عبارت‌اند از:

$$ 
x_{L_4} \;=\; x_{L_5} \;=\; \dfrac{1}{2} - \mu~, \qquad 
y_{L_4} \;=\; +\dfrac{\sqrt{3}}{2}~, \qquad 
y_{L_5} \;=\; -\dfrac{\sqrt{3}}{2}~.
$$

به بیان دیگر، در دستگاه باریسنتر چرخان، هر دوی $L_4$ و $L_5$ به اندازه‌ی نیمی از فاصله‌ی دو جرم روی محور $x$ از مرکز جرم فاصله دارند (کمی جابه‌جا‌شده به سمت جرم بزرگ‌تر به اندازه‌ی $\mu$) و مؤلفه‌ی عمودی (محور $y$) آن‌ها برابر $\pm\dfrac{\sqrt{3}}{2}$ است که نشان‌دهنده‌ی زاویه‌ی $60^\circ$ نسبت به محور $x$ می‌باشد. این دو نقطه، در صورت کافی‌بودن نسبت جرم‌ها (بزرگ‌تر بودن قابل توجه جرم اول نسبت به جرم دوم؛ معمولاً $\dfrac{m_1}{m_2} > 24.96$ شرط پایداری است)، **نقاط تعادل پایدار** هستند. به عنوان مثال سیاره‌ی مشتری تعداد زیادی سیارک‌های تروجان در نقاط $L_4$ و $L_5$ خود نسبت به خورشید دارد. در مقابل، سه نقطه‌ی هم‌خط ($L_1,L_2,L_3$) پایداری ذاتی ندارند و اجسام در آن‌ها بدون تصحیح مداری معمولا از نقطه‌ی تعادل دور می‌شوند، با این حال برای تحقیقات فضایی (مانند رصدخانه‌ها یا رله‌های مخابراتی) محبوب‌اند زیرا مدارهای هاله‌ای یا لیساژور حول این نقاط با صرف انرژی اندک قابل حفظ هستند.

\section{} نمونه: نقاط لاگرانژ در سیستم زمین-ماه  
برای سیستم زمین-ماه مقدار $\mu \approx 0.01215$ است (نسبت جرم ماه به مجموع جرم زمین و ماه). در این سیستم با اعمال روابط به‌دست‌آمده می‌توان موقعیت عددی هر پنج نقطه‌ی لاگرانژ را محاسبه کرد. جدول زیر مختصات بی‌بعد (بر حسب فاصله‌ی زمین-ماه به عنوان واحد طول) این نقاط را در دستگاه مختصات تعریف‌شده‌ی باریسنتر (مرکز جرم در مبدأ، زمین در مختصات $(-\mu,0)$ و ماه در $(1-\mu,0)$) نشان می‌دهد. لازم به ذکر است که در این مقیاس، مختصات زمین تقریباً $(-0.01215,\;0)$ و ماه $(0.98785,\;0)$ می‌باشد. همان‌طور که مشاهده می‌شود، $L_1$ حدوداً در 84\% فاصله‌ی زمین-ماه (از سمت زمین) واقع شده و $L_2$ کمی بیرون مدار ماه قرار دارد. نقطه‌ی $L_3$ اندکی آن‌سوی مدار زمین (در جهت مخالف ماه) است. نقاط $L_4$ و $L_5$ نیز به‌ترتیب در بالا و پایین صفحه (در زاویه‌ی $60^\circ$) و تقریباً در فاصله‌ی مساوی از زمین و ماه قرار گرفته‌اند.

| نقطه‌ی لاگرانژ | $x$ (بی‌بعد)     | $y$ (بی‌بعد)    |
| -------------- | -------------- | -------------- |
| $L_1$          | $+0.83692$     | $0$            |
| $L_2$          | $+1.15568$     | $0$            |
| $L_3$          | $-1.00506$     | $0$            |
| $L_4$          | $+0.48785$     | $+0.86603$     |
| $L_5$          | $+0.48785$     | $-0.86603$     |

در جدول بالا علامت مثبت/منفی $x$ نشان‌دهنده‌ی موقعیت نسبت به مرکز جرم (مبدأ مختصات) روی محور زمین-ماه است (مثبت به سمت ماه و منفی به سمت زمین) و محور $y$ عمود بر آن صفحه‌ی مداری در جهت چرخش سیستم تعریف شده است. مقادیر عددی ارائه‌شده نشان می‌دهند که در منظومه‌ی زمین-ماه، $L_1$ تقریباً در فاصلهٔ $0.84$ برابر فاصلهٔ زمین-ماه از زمین قرار دارد (یعنی فاصلهٔ آن از ماه حدود $0.16$ برابر فاصلهٔ زمین-ماه است). به طور مشابه $L_2$ حدود $1.156$ برابر فاصلهٔ زمین-ماه از زمین فاصله دارد (حدود $0.156$ برابر فاصلهٔ زمین-ماه آن‌سوی مدار ماه). نقطه $L_3$ تقریباً در فاصله‌ای برابر با فاصلهٔ زمین-ماه در سمت مقابل (پشت زمین نسبت به ماه) قرار گرفته است. نقاط $L_4$ و $L_5$ نیز تقریباً در مختصات $(0.488,\;\pm0.866)$ واقع شده‌اند که نشان‌دهندهٔ تشکیل مثلث متساوی‌الاضلاع با زمین و ماه می‌باشد. این مقادیر و تحلیلات تأیید می‌کنند که روابط تحلیلی به‌دست‌آمده برای معادلات لاگرانژ و نقاط تعادل، به‌خوبی می‌توانند برای توصیف وضعیت‌های پایدار و ناپایدار یک جرم کوچک در میدان گرانشی دو جرم آسمانی به‌کار روند. 

**مراجع:**  
1. Richard H. Battin, *An Introduction to the Mathematics and Methods of Astrodynamics*, Revised Edition, AIAA Education Series, Eq.(1) for Lagrange’s quintic equation. (معادله درجه پنج لاگرانژ برای محاسبه نقاط $L_1, L_2, L_3$)