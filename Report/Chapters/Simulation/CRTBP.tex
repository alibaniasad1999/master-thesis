\pgfmathsetmacro{\r}{0.8}	
\pgfmathsetmacro{\Phi}{-160}
\pgfmathsetmacro{\Theta}{-90}

\section{مسئله‌ی سه‌جسمیِ محدودِ دایره‌ای (\lr{CRTBP})}\label{sec:crtbp}

دو جرمِ اصلی (زمین
با جرم~$m_{1}$ و ماه با جرم~$m_{2}$) روی مدارهایی دایره‌ای و هم‌صفحه پیرامونِ مرکزِ جرمِ مشترک حرکت می‌کنند. جرمِ سوم (فضاپیما با جرمِ ناچیز $m_{3}$) چنان کوچک فرض می‌شود که تأثیرِ گرانشیِ آن بر حرکتِ دو جسمِ اصلی قابل صرفِ نظر است؛ بدین ترتیب، مسئله‌ی سه‌جسمیِ محدودِ دایره‌ای شکل می‌گیرد.



\begin{table}[H]
	\centering
	\caption{مقادیر عددی برای مسئله سه‌جسمی محدود (سیستم زمین–ماه)}
	\begin{tabular}{|c|c|c|}
		\hline
		پارامتر & توصیف & مقدار عددی \\
		\hline
		$m_1$ & جرم زمین & $5.972 \times 10^{24}\,\mathrm{kg}$ \\
		$m_2$ & جرم ماه & $7.348 \times 10^{22}\,\mathrm{kg}$ \\
		$\mu$ & نسبت جرمی & $0.0121505856$ \\
		$\omega$ & سرعت زاویه‌ای سیستم & $2.6617 \times 10^{-6}\,\mathrm{rad/s}$ \\
		\hline
	\end{tabular}
	\label{tab:params}
\end{table}



دستگاهِ مختصاتِ چرخانی هم‌دوران با دو جرم اصلی انتخاب می‌شود؛ مبدأ در مرکزِ جرمِ سامانه است، محور~$x$ خطِ واصلِ دو جرم و محور~$y$ بر آن عمود (در صفحه‌ی مدارها) است. واحدِ طول برابر فاصله‌ی ثابتِ میان دو جرم و واحدِ زمان چنان تعریف می‌شود که دوره‌ی مداریِ سامانه $2\pi$ (و در نتیجه $\omega=1$) گردد. همچنین جرم‌ها به‌گونه‌ای مقیاس می‌شود که مجموع دو جرم برابر با یک شود:
\begin{equation}
	 m_{1}+m_{2}=1.
\end{equation}
با نسبتِ جرمی
\begin{equation}
	\mu\equiv\frac{m_{2}}{m_{1}+m_{2}},
\end{equation}
داریم $m_{1}=1-\mu$ و $m_{2}=\mu$ و مکانِ دو جرم در دستگاهِ بی‌بُعد به صورت
\begin{equation}
	\mathbf r_{\text{Earth}}=(-\mu,0),\qquad \mathbf r_{\text{Moon}}=(1-\mu,0).
\end{equation}


\begin{figure}[H]
\centering
\begin{tikzpicture}
	% Coordinates
	\coordinate (earth) at (1,2);
	\coordinate (moon) at (8,1);
	\coordinate (earth-point1) at ({\r*cos(\Theta)+1},{\r*sin(\Theta)+2});
	\coordinate (A) at (-.5,.5);
	\coordinate (B) at (8.5,-0.5);
	
	% Earth
	\draw[thick, fill=black!30, draw=black!30
	] (earth) circle (\r);
	\node[inner sep=0pt] (Earth_c) at (earth) {\includegraphics[width=1.8cm]{../Figure/TBP/Earth.png}};
	% Text
	\node[below, shift={(0,-0.8)}] at (earth) {$m_1$};
	\node (a) at (A) {زمین};
	
	% Moon
	\node[circle, inner sep=5.5pt, fill=black!30] (MOON) at (moon) {};
	\node[inner sep=0pt] (moon_c) at (moon) {\includegraphics[width=.5cm]{../Figure/TBP/Moon.png}};
	% Text 
	\node[below, shift={(0,-0.4)}] at (MOON) {$m_2$};
	\node (b) at (B) {ماه};
	
	% Lines
	\draw[-stealth] (a) to[bend left=30] ({\r*cos(\Phi)+1},{\r*sin(\Phi)+2});
	\draw[-stealth] (b) to[bend left=-30] (MOON);
	\draw[dashed, black] (earth) -- (MOON.center);
	
	% center of mass 0.25 from earth
	\coordinate (center) at ($(earth)!0.3!(MOON)$);
	% small circle
	\draw[fill=black] (center) circle (1.5pt) node[below, shift={(0,-0.1)}] {جرم مرکز };
	
	% Calculate direction from Earth to Moon
	\pgfmathsetmacro{\xDiff}{8 - 1} % X difference between Moon and Earth
	\pgfmathsetmacro{\yDiff}{1 - 2} % Y difference between Moon and Earth
	\pgfmathsetmacro{\angle}{atan2(\yDiff,\xDiff)} % Angle of the line
	
	% Add axes at center of mass
	\draw[->, thick] (center) -- ++(\angle:2) node[above, shift={(0,0.2)}] {محور $x$};
	\draw[->, thick] (center) -- ++(\angle+90:2) node[above] {محور $y$};
	
	% add satellite with shift
	\coordinate (satellite) at ($(center)!0.5!(MOON)+(0,2)$);
	\node (satellite) at (satellite) {\faSatellite};
	
	% connect earth to satellite r1
	\draw[-stealth] (earth) -- (satellite) node[pos=0.3, above] {$\vb{r}_1$};   
	% connect moon to satellite r2
	\draw[-stealth] (MOON) -- (satellite) node[pos=0.5, above] {$\vb{r}_2$};
	% connect center of mass to satellite r
	\draw[-stealth] (center) -- (satellite) node[pos=0.5, above] {$\vb{r}$};
	% add line to show satellite is in between
	\node (c) at ($(satellite)+(1.5,0.5)$) {فضاپیما};
	\draw[-stealth] (c) to[bend left=30] (satellite);
	
\end{tikzpicture}
\caption{هندسه‌ی مسئله‌ی سه‌جسمیِ محدود در چارچوبِ چرخان}
\end{figure}



\subsection{لاگرانژ و معادلات حرکت}
با در نظر گرفتن
$G=1$
در حالت بی‌بُعد،
تابعِ لاگرانژِ جرمِ سوم در دستگاهِ چرخان برابر است با\cite{vallado2001fundamentals}
\begin{equation}\label{eq:L_crtbp}
	L=\tfrac12\bigl(\dot x^{2}+\dot y^{2}+\dot z^{2}\bigr)
	+(1-\mu)\,\frac{1}{r_{1}}+\mu\,\frac{1}{r_{2}}
	+\tfrac12\bigl(x^{2}+y^{2}\bigr),
\end{equation}
که در آن
\begin{equation}
	r_{1}=\sqrt{(x+\mu)^{2}+y^{2}+z^{2}},\qquad
	r_{2}=\sqrt{\bigl(x-1+\mu\bigr)^{2}+y^{2}+z^{2}}.
\end{equation}

با به‌کارگیری رابطه‌ی اویلر–لاگرانژ
\begin{equation*}
	\frac{\mathrm d}{\mathrm dt}\frac{\partial L}{\partial \dot q_{i}}-
	\frac{\partial L}{\partial q_{i}}=0,\qquad q_{i}\in\{x,y,z\},
\end{equation*}
معادلاتِ بی‌بُعدِ حرکتِ جرمِ سوم به دست می‌آید:
\begin{align}
	\ddot x-2\dot y &=
	x-\frac{1-\mu}{r_{1}^{3}}(x+\mu)-\frac{\mu}{r_{2}^{3}}\bigl(x-1+\mu\bigr),\\[2pt]
	\ddot y+2\dot x &=
	y-\frac{1-\mu}{r_{1}^{3}}\,y-\frac{\mu}{r_{2}^{3}}\,y,\\[2pt]
	\ddot z &= -\frac{1-\mu}{r_{1}^{3}}\,z-\frac{\mu}{r_{2}^{3}}\,z.
\end{align}
یا به نگاشتِ برداری به‌صورت زیر است.
\begin{equation}
	\ddot{\mathbf r}+2\,\boldsymbol\omega\times\dot{\mathbf r}=\nabla\Omega(\mathbf r),\qquad
	\Omega(x,y,z)=\tfrac12\bigl(x^{2}+y^{2}\bigr)+\frac{1-\mu}{r_{1}}+\frac{\mu}{r_{2}}.
\end{equation}
که در آن $\Omega$ پتانسیلِ مؤثر است و در بخش \ref{sec:lag-points} برای یافتنِ نقاطِ تعادل از شرطِ $\nabla\Omega=0$ استفاده می‌شود.
%ترم‌های کورولیس~$\pm2\,\dot{\vphantom y}$ پیامد استفاده از چارچوبِ چرخان است.
